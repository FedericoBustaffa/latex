\section{Descrizione del dominio}
La piattaforma mette a disposizione la possibilità di creare
\textbf{account} di due tipologie:
\begin{itemize}
	\item \textbf{Temporaneo}: per gli studenti che richiedono
	      una o più delibere (forse).
	\item \textbf{Amministrativo}: per i docenti che gestiscono
	      una o più delibere.
\end{itemize}
Entrambi devono inserire nome, cognome, email e password per
creare un account. Ciò che differenzia gli uni dagli altri è che
lo studente deve inserire la propria matricola mentre il
professore deve inserire un codice.

Ogni studente una volta fatto creato l'account e fatto l'accesso
deve inserire una lista contente gli esami da lui sostenuti,
qualunque sia il tipo di delibera richiesta. Questo è necessario
poiché ogni \textbf{delibera} avrà bisogno, in un modo o
nell'altro, di sapere quali esami sono stati sostenuti per far
sì che la commisione riesca a decidere quali esami o attività
extracurriculari debbano essere convalidate.

Ogni \textbf{esame} inserito dallo studente si compone di un
codice esame (univoco per ogni esame), il nome dell'esame, i
crediti, la tipologia (obbligatorio, complementare oppure a
scelta) e il link al programma del corso. Ogni studente
specifica il \textbf{voto} per ogni esame inserito nella lista.

Ogni studente quando crea l'account specifica il
\textbf{corso di studio} che frequenta, il quale è composto da
un codice (univoco per ogni corso di studio), il nome,
l'università, il dipartimento, i crediti totali previsti da
quel corso di studio per conseguire la laurea, la somma dei
crediti degli esami obbligatori, la somma dei crediti degli
esami complementari e la somma dei crediti degli esami a libera
scelta.

Le \textbf{delibere} sono tutte identificate univocamente da un
codice e possono essere di quattro tipologie differenti:
\begin{itemize}
	\item \textbf{Trasferimento}: per gli studenti che
	      provengono da un'università diversa da quella di Pisa.
	\item \textbf{Abbreviazione della carriera}: per gli
	      studenti che hanno già conseguito un titolo.
	\item \textbf{Ricongiugimento dopo inattività}: per gli
	      studenti che hanno effettuato la rinuncia agli studi
	      (credo) e si riscrivono. L'ordinamento potrebbe essere
	      cambiato ed è quindi necessario specificare l'anno in
	      cui è stata effettuata la rinuncia.
	\item \textbf{Riconoscimento dei crediti}: per gli studenti
	      che hanno svolto attività extracurriculari che possono
	      valere come esami e dunque hanno un loro peso in
	      crediti.
\end{itemize}
Come sarà chiaro fra poco è necessario che ogni delibera abbia
anche una data in cui questa è stata richiesta.

Quando uno o più esami vengono convalidati dalla commissione
è quest'ultima ha modificare la base di dati indicando che lo
studente ha sostenuto un nuovo esame (o più esami in caso un
esame venga convertito in più esami). Sarà sempre la
commissisone a inserire il voto del nuovo esame convalidato.

