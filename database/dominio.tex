\section{Descrizione del dominio}
Il sistema ha il compito di gestire la vendita online di occhiali. La prima entità rappresenta
gli \textbf{utenti} registrati nel sistema e i loro dati. Quello che interessa principalmente
è l'indirizzo email, diverso per ogni utente.

Ogni utente può visualizzare i vari \textbf{modelli} di occhiali, ciascuno dei quali possiede
delle caratteristiche come la categoria, il marchio, il materiale, prezzo, quantità disponibile
e così via.

L'utente può inoltre applicare delle \textbf{personalizzazioni} al modello che sta visualizzando
specificando colore e forma di lenti e montatura tra quelli disponibili per quel modello. Ogni
modello ha almeno una personalizzazione possibile e una viene applicata di default quando viene
visualizzato.

Ogni utente \emph{modifica} un \textbf{carrello}, al quale è possibile \emph{aggiungere}, o
\emph{rimuovere} modelli di occhiali con la relativa personalizzazione e specificando la quantità
(con relativa personalizzazione e quantità). L'utente può inoltre \emph{prorogare}, se lo desidera,
il tempo in cui un certo prodotto può rimanere nel carrello.

Ogni carrello può quindi contenere più modelli e, nel caso l'utente sia interessato ad acquistare
più pezzi di un modello con la stessa personalizzazione, è sufficiente indicare la quantità
desiderata. Abbiamo quindi che ogni elemento del carrello è composto dal modello, una sua
personalizzazione e il numero di pezzi che si è interessati ad acquistare.

Quando un elemento viene inserito nel carrello gli viene associato il timestamp del momento in cui
è stato inserito e un valore booleano che indica la volontà di prorogare il mantenimento nel
carrello o meno.

L'utente può \emph{effettuare} un \textbf{ordine}, acquistando ciò che ha nel carrello.
Ogni ordine contiene le informazioni relative all'acquisto come il metodo di pagamento, la data,
lo stato e la data in cui è avvenuta la consegna.

L'utente è anche in grado di \emph{scrivere} una \textbf{recensione} se lo desidera, dando una
valutazione e scrivendo un commento. Ogni recensione \emph{riguarda} un solo modello di occhiali.

Abbiamo infine un'entità \textbf{responsabile} con cui ci riferiamo alle figure che sono in
grado di \emph{gestire} vari tipi di \textbf{promozione}, le quali possono essere di 3 tipologie:
\begin{itemize}
	\item \textbf{Sconti}: quando si vuole applicare una certa percentuale di sconto su uno o più
	      modelli.
	\item \textbf{Offerte}: sono della tipologia in cui se si fanno almeno un certo numero di
	      acquisti si riceve qualcosa in omaggio o un sconto sul totale.
	\item \textbf{Codice promozionale}: sono codici che l'utente può usare in un secondo momento
	      per ottenere vantaggi su futuri acquisti.
\end{itemize}
Non ci possono essere promozioni generiche ma solo di una di queste tre tipologie e ognuna riguarda
uno o più modelli di occhiali.