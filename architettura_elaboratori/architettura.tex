\documentclass[11pt]{report}

% --------------- PACKAGES ---------------
\usepackage[utf8]{inputenc}
\usepackage[T1]{fontenc}
\usepackage[italian]{babel}
\usepackage[hidelinks]{hyperref}

\hypersetup{
	colorlinks=true,
	linkcolor=blue
}

% --------------- STYLE ---------------
\usepackage[margin=1.25in]{geometry}
\usepackage[most]{tcolorbox}

% Font
\usepackage{sansmath}

\renewcommand{\familydefault}{\sfdefault}
\sansmath

% page style
\usepackage{fancyhdr}
\usepackage[Sonny]{fncychap}

\pagestyle{fancy}
\setlength{\headheight}{15pt}
\rhead{\thepage}
\cfoot{\thepage}

% --------------- MATH ---------------
\usepackage{amsmath}
\usepackage{amssymb}
\usepackage{amsthm}
\usepackage{amsfonts}
\usepackage{mathtools}
\usepackage{mdframed}

\newcommand{\N}{\mathbb{N}}
\newcommand{\Z}{\mathbb{Z}}
\newcommand{\R}{\mathbb{R}}
\newcommand{\C}{\mathbb{C}}
\newcommand{\E}{\mathbb{E}}

\newcommand{\F}{\mathcal{F}}

\DeclareMathOperator{\Var}{Var}
\DeclareMathOperator{\Cov}{Cov}

% Boxes for theorem, definitions and examples
\newtheoremstyle{math_box}
{0pt}
{0pt}
{\normalfont}
{}
{\color{orange}}
{\;}
{0.25em}
{\thmname{\textbf{#1}}\thmnumber{ \textbf{#2}}{\color{black}\thmnote{\textbf{ -- #3.}}}}

\newmdenv[
	rightline=false,
	leftline=true,
	topline=false,
	bottomline=false,
	linecolor=orange!40,
	innerleftmargin=5pt,
	innerrightmargin=5pt,
	innertopmargin=0pt,
	innerbottommargin=0pt,
	leftmargin=0cm,
	rightmargin=0cm,
	linewidth=3pt
]{dBox}

\newmdenv[
	rightline=false,
	leftline=false,
	topline=false,
	bottomline=false,
	backgroundcolor=orange!15,
	innerleftmargin=5pt,
	innerrightmargin=5pt,
	innertopmargin=5pt,
	innerbottommargin=5pt,
	leftmargin=0cm,
	rightmargin=0cm,
]{pBox}

\theoremstyle{math_box}
\newtheorem{theoremeT}{Teorema}[section]
\newtheorem{definitionT}{Definizione}[section]
\newtheorem{propositionT}{Proposizione}[section]
\newtheorem{corollary}{Corollario}[section]
\newtheorem{lemma}{Lemma}[section]
\newtheorem{observation}{Osservazione}[section]
\newtheorem{exampleT}{Esempio}[subsection]

\newenvironment{theorem}{\begin{pBox}\begin{theoremeT}}{\end{theoremeT}\end{pBox}}
\newenvironment{definition}{\begin{dBox}\begin{definitionT}}{\end{definitionT}\end{dBox}}
\newenvironment{proposition}{\begin{pBox}\begin{propositionT}}{\end{propositionT}\end{pBox}}
\newenvironment{example}{\begin{dBox}\begin{exampleT}}{\end{exampleT}\end{dBox}}

\usepackage{tikz, pgfplots, pgf-pie}
\usepackage{caption, subcaption}
\usepackage{tikz}
\usepackage{scalerel}
\usepackage{pict2e}
\usepackage{tkz-euclide}
\usepackage{pgfplots, pgfplotstable, pgf-pie}

\usetikzlibrary{calc}
\usetikzlibrary{patterns, arrows}
\usetikzlibrary{shadows}
\usetikzlibrary{external}

\pgfplotsset{compat=newest}
\usepgfplotslibrary{statistics, fillbetween}

\usepackage{minted}
\usepackage{diagbox}
\usepackage[label=corner]{karnaugh-map}

\definecolor{minted_bg}{rgb}{0.9, 0.9, 0.9}
\usemintedstyle{borland}

\setminted[c]{
	tabsize=4,
	% linenos=true,
	bgcolor=minted_bg,
	fontsize=\small,
	mathescape=true
}

\setminted[gas]{
	tabsize=4,
	% linenos=true,
	bgcolor=minted_bg,
	fontsize=\small,
	mathescape=true
}

\title{Architettura degli elaboratori}
\author{Federico Bustaffa}
\date{13/11/2023}

\begin{document}

\maketitle
\tableofcontents

\chapter{Introduzione}
Tutti i calcolatori moderni possono essere schematizzati semplicemente tramite il cosiddetto
\textbf{modello Von Neumann} in cui abbiamo tre componenti principali: \textbf{memoria},
\textbf{processore} e canali di \textbf{I/O}.

\begin{center}
	\begin{tikzpicture}[scale=1.5]
		\node[draw] (mem) at (0, 0) {Memoria};
		\node[draw] (cpu) at (0, -1.5) {CPU};
		\node[draw] (io) at (2, -1.5) {I/O};

		\draw[<->] (mem) -- (cpu);
		\draw[<->] (cpu) -- (io);
	\end{tikzpicture}
\end{center}

Come possiamo vedere dalla figura i collegamenti tra le varie entità sono bidirezionali. Il
processore (ma anche la memoria) è collegato ai canali di I/O e il collegamento che c'è tra memoria
e processore è chiamato \textbf{Von Neumann bottleneck}. Quello che accade tra memoria e processore
è, grosso modo, quello che viene descritto dal seguente pseudocodice ed è denominato ciclo di
\textbf{fetch-decode-execute}.

\begin{minted}{c}
while (true) {
	istr = M[PC]
	decode(istr)
	res = exec(istr)
	update(PC)
	writeback(res)
	interrupt_handling()
}
\end{minted}

In pratica viene estratta dalla memoria l'istruzione puntata da un \textbf{Program Counter}, la
si decodifica e la si esegue. In seguito il Program Counter viene aggiornato e i risultati vengono
consolidati nei registri della CPU oppure in memoria.

Per ognuna delle istruzioni eseguite sul canale presente tra processore e memoria vengono svolte
le seguenti operazioni
\begin{enumerate}
	\item Il processore manda un indirizzo (PC) dove andare a prendere l'istruzione.
	\item La memoria risponde con un'istruzione.
	\item Il processore esegue e durante l'esecuzione può richiedere la lettura o scrittura dalla
	      memoria.
\end{enumerate}
Questo processo avviene molto spesso e rende il traffico sul canale molto intenso creando per
l'appunto un \emph{collo di bottiglia}. Questo è dovuto alla rapida evoluzione dei processori e
alla non altrettanto rapida evoluzione della memoria, la quale ha una velocità nel fornire le
istruzioni minore della velocità che impiega il processore ad eseguirle e richiederne altre.

Per misurare le performance di un processore siamo abituati a ragionare in termini di
\textbf{cicli di clock}. Dato che, approssimativamente, possiamo dire che un ciclo di clock
corrisponde ad un'istruzione eseguita, se ad esempio abbiamo un processore a 1 GHz, questo riuscirà
ad eseguire un'istruzione in 1 nanosecondo.

Durante il corso verrano trattate varie architetture, dalle prime progettate alle più recenti,
cercando di capire come funzionano e i loro pregi e difetti. Tra quelle che tratteremo elenchiamo
\begin{itemize}
	\item \textbf{Single cycle}: modello in cui un ciclo fetch-execute viene effettuato in un ciclo
	      di clock.
	\item \textbf{Pipeline}: modello in cui si hanno componenti distinte per le operazioni di fetch,
	      decode ed execute. In questo modo ogni componente continua a svolgere il proprio lavoro
	      parallelamente alle altre. Una volta arrivato a regime questo processore riesce a
	      eseguire un'istruzione all'inizio di ogni ciclo di clock e non alla fine (come nel
	      modello single cycle).
\end{itemize}

\include{2-binaria_booleana/2.1-aritmetica_binaria}
\section{Circuiti logici}
Nel corso non andremo a trattare l'ultimo livello di astrazione, ossia quello più basso, ma andremo
a trattare lo strato soprastante, che tramite delle \textbf{porte logiche} e delle operazioni
aritmetiche binarie riesce a rappresentare quello che succede. Le tre operazioni implmentate dalle
porte logiche sono
\begin{itemize}
	\item \verb|AND(x,y)|: 1 se \verb|x=y=1|, 0 altrimenti.
	\item \verb|OR(x,y)|: 0 se \verb|x=y=0|, 1 altrimenti.
	\item \verb|NOT(x)|: 1 se \verb|x=0|, 0 se \verb|x=1|.
\end{itemize}
Altro strumento utile per capire meglio come funzionano tali porte e per vedere come funzionano
altre porte che risultano essere una combinazione di esse, sono le \textbf{tabelle di verità}.
Nelle tabelle di verità immettiamo tutti i possibili valori di input e calcoliamo i relativi output.
Per esempio, la tabella di verità di una porta logica \verb|AND| è la seguente
\begin{center}
	\begin{tabular}{c c | c}
		x & y & z \\ \hline
		0 & 0 & 0 \\
		0 & 1 & 0 \\
		1 & 0 & 0 \\
		1 & 1 & 1
	\end{tabular}
\end{center}
Per una questione legata alla circuiteria sottostante e alla leggi fisiche che regolano il
funzionamento dei transistor, il numero di ingressi delle porte è, in genere, al più 8 poiché
averne di più introduce troppo ritardo nell'elaborazione dei segnali.

Quello che useremo d'ora in poi saranno delle funzioni che hanno un certo numero di ingressi e
uscite booleani che realizzeremo come \textbf{reti combinatorie}, ossia composizioni di porte
\verb|AND|, \verb|OR| e \verb|NOT| a seconda delle necessità.

Supponiamo ad esempio di voler calcolare il numero di bit a 1 su 2 ingressi, ciascuno da 1 bit. In
questo caso i possibili valori di output sono 3 (0, 1 e 2) e abbiamo quindi bisogno di un numero
di uscite pari a $\lceil \log_2 (3) \rceil = 2$ uscite. La tabella di verità del nostro circuito
avrà la seguente tabella di verità
\begin{center}
	\begin{tabular}{c c | c c}
		$x_0$ & $x_1$ & $z_0$ & $z_1$ \\ \hline
		0     & 0     & 0     & 0     \\
		0     & 1     & 0     & 1     \\
		1     & 0     & 0     & 1     \\
		1     & 1     & 1     & 0
	\end{tabular}
\end{center}
Per trovare il circuito desiderato c'è una procedura standard, la quale utilizza il fatto che un
\verb|AND| logico corrisponde al prodotto tra due numeri mentre l'\verb|OR| logico corrisponde alla
somma:
\begin{enumerate}
	\item Per ogni riga in cui una delle colonne d'uscita presenta almeno un 1 mettiamo in
	      \verb|AND| gli ingressi, negandoli se uguali a 0.
	\item Per ogni colonna si mettono in \verb|OR| tutti i risultati ottenuti al passo precedente.
\end{enumerate}
Nel nostro caso la colonna $z_0$ ha un 1 sull'ultima riga e i relativi valori di $x_0$ e $x_1$ sono
entrambi 1 quindi possiamo dire che
\[ z_0 = x_0 \cdot x_1 \]
ossia
\begin{center}
	\verb|z0 = AND(x0, x1)|
\end{center}
Per quanto riguarda invece la colonna $z_1$ abbiamo due 1 e in corrispondenza della seconda e terza
riga. Ma in entrambi i casi uno dei due valori in ingresso è 0 e l'altro è 1 e dunque il risultato
finale è
\[ z_1 = \bar{x_0} \cdot x_1 + x_0 \cdot \bar{x_1} \]
ossia
\begin{center}
	\verb|z1 = OR(AND(NOT(x0), x1), AND(x0, NOT(x1)))|
\end{center}
Il circuito logico che ne deriva è il seguente

% \begin{tikzpicture}[label distance=2mm]

% 	\node (x3) at (0,0) {$x_3$};
% 	\node (x2) at (1,0) {$x_2$};
% 	\node (x1) at (2,0) {$x_1$};
% 	\node (x0) at (3,0) {$x_0$};

% 	\node[not gate US, draw, rotate=-90] at ($(x2)+(0.5,-1)$) (Not2) {};
% 	\node[not gate US, draw, rotate=-90] at ($(x1)+(0.5,-1)$) (Not1) {};
% 	\node[not gate US, draw, rotate=-90] at ($(x0)+(0.5,-1)$) (Not0) {};

% 	\node[or gate US, draw, logic gate inputs=nnn] at ($(x0)+(2,-2)$) (Or1) {};
% 	\node[or gate US, draw, logic gate inputs=nnnn] at ($(Or1)+(0,-1)$) (Or2) {};
% 	\node[or gate US, draw, logic gate inputs=nnn] at ($(Or2)+(0,-1)$) (Or3) {};
% 	\node[xor gate US, draw, logic gate inputs=nn] at ($(Or3)+(0,-1)$) (Xor1) {};
% 	\node[and gate US, draw, logic gate inputs=nn, anchor=input 1] at ($(Or3.output)+(1,0)$) (And1) {};
% 	\node[nor gate US, draw, logic gate inputs=nn, anchor=input 1] at ($(Or2.output -| And1.output)+(1,0)$) (Nor1) {};
% 	\node[and gate US, draw, logic gate inputs=nn, anchor=input 1] at ($(Or1.output -| Nor1.output)+(1,0)$) (And2) {};

% 	\foreach \i in {2,1,0}
% 		{
% 			\path (x\i) -- coordinate (punt\i) (x\i |- Not\i.input);
% 			\draw (punt\i) node[branch] {} -| (Not\i.input);
% 		}
% 	\draw (x3) |- (Or2.input 1);
% 	\draw (x3 |- Or1.input 1) node[branch] {} -- (Or1.input 1);
% 	\draw (x2) |- (Xor1.input 1);
% 	\draw (x2 |- Or3.input 1) node[branch] {} -- (Or3.input 1);
% 	\draw (Not2.output) |- (Or2.input 2);
% 	\draw (x1) |- (Or3.input 2);
% 	\draw (x1 |- Or1.input 2) node[branch] {} -- (Or1.input 2);
% 	\draw (Not1.output) |- (Xor1.input 2);
% 	\draw (Not1.output |- Or2.input 3) node[branch] {} -- (Or2.input 3);
% 	\draw (x0) |- (Or2.input 4);
% 	\draw (Not0.output) |- (Or3.input 3);
% 	\draw (Not0.output |- Or1.input 3) node[branch] {} -- (Or1.input 3);
% 	\draw (Or3.output) -- (And1.input 1);
% 	\draw (Xor1.output) -- ([xshift=0.5cm]Xor1.output) |- (And1.input 2);
% 	\draw (Or2.output) -- (Nor1.input 1);
% 	\draw (And1.output) -- ([xshift=0.5cm]And1.output) |- (Nor1.input 2);
% 	\draw (Or1.output) -- (And2.input 1);
% 	\draw (Nor1.output) -- ([xshift=0.5cm]Nor1.output) |- (And2.input 2);
% 	\draw (And2.output) -- ([xshift=0.5cm]And2.output) node[above] {$f_1$};

% \end{tikzpicture}


\begin{center}
	\begin{circuitikz}
		% gate
		\node[and port] (and1) at (3.5, -1) {};
		\node[and port] (and2) at (3.5, -2.5) {};
		\node[and port] (and3) at (3.5, -4) {};
		\node[or port] (or) at (5.5, -3.25) {};

		% connessioni
		\draw (0, 0) node[label=above:$x_0$] {} to[short, -*] (0, 52 |- and1.in 1) -- (and1.in 1);
		\draw (0, 52 |- and1.in 1) to[short, -*] (0, 52 |- and2.in 1) to[short, -o] (and2.in 1);
		\draw (0, 52 |- and2.in 1) to[short, -*] (0, 52 |- and3.in 1) -- (and3.in 1);

		\draw (1, 0) node[label=above:$x_1$] {} to[short, -*] (1, 52 |- and1.in 2) -- (and1.in 2);
		\draw (1, 52 |- and1.in 2) to[short, -*] (1, 52 |- and2.in 2) -- (and2.in 2);
		\draw (1, 52 |- and2.in 2) to[short, -*] (1, 52 |- and3.in 2) to[short, -o] (and3.in 2);

		\draw (and2.out) |- (or.in 1);
		\draw (and3.out) |- (or.in 2);

		\draw (and1.out) -- (6.5, 52 |- and1.out) node[label=above:$z_0$] {};
		\draw (or.out) -- (6.5, 52 |- or.out) node[label=above:$z_1$] {};
	\end{circuitikz}
\end{center}
sul quale è possibile provare ad inserire vari input di $x_0$ e $x_1$ per verificarne la
correttezza.

Supponiamo ora di dover scegliere uno tra due ingressi possibili a seconda di un ingresso di
controllo regolato da un \textbf{multiplexer} che ha una forma di questo tipo
\begin{center}
	\begin{circuitikz}
		\draw[thick] (0, -1) -- (0, 1) -- (1, 0.5) -- (1, -0.5) -- cycle;
		\draw (-1, 0.5) node[label=left:$x_0$] {} -- (0, 0.5);
		\draw (-1, -0.5) node[label=left:$x_1$] {} -- (0, -0.5);
		\draw (0.5, 1.5) node[label=above:$c$] {} -- (0.5, 0.75);
		\draw (1, 0) -- (2, 0) node[label=right:$z$] {};
	\end{circuitikz}
\end{center}
Di fatto dobbiamo implementare un circuito che da come risultato il valore di $x_0$ quando $c=0$ e
da come risultato il valore di $x_1$ quando $c=1$.

In questo caso abbiamo tre ingressi e un'uscita, dovremmo quindi scrivere una tabella di verità con
8 righe, ma dato che uno dei valori viene scartato a seconda del valore di $c$ il risultato è una
tabella più compatta.

Avere una tabella più compatta significa anche avere un circuito più compatto e con meno componenti.
Questo si traduce in un minor numero di nodi di calcolo e quindi una computazione più veloce, ma
anche in un minor consumo di energia e minor bisogno di spazio sul processore.
\begin{center}
	\begin{tabular}{c c c | c}
		$x_0$ & $x_1$ & $c$ & $z$ \\ \hline
		0     & -     & 0   & 0   \\
		1     & -     & 0   & 1   \\
		-     & 0     & 1   & 0   \\
		-     & 1     & 1   & 1
	\end{tabular}
\end{center}
Svolgiamo lo stesso procedimento di prima e ricaviamo un circuito di questo tipo
\begin{center}
	\begin{circuitikz}
		\node[and port] (and1) at (3.5, 0.75) {};
		\node[and port] (and2) at (3.5, -0.75) {};
		\node[or port] (or) at (5.5, 0) {};

		% connessioni
		\draw (0, 1.5) node[label=above:$x_0$] {} to[short, -*] (0, 52 |- and1.in 1) -- (and1.in 1);

		\draw (0.5, 1.5) node[label=above:$x_1$] {} to[short, -*] (0.5, 52 |- and2.in 1) -- (and2.in 1);

		\draw (1, 1.5) node[label=above:$c$] {} to[short, -*] (1, 52 |- and1.in 2) to[short, -o] (and1.in 2);
		\draw (1, 52 |- and1.in 2) to[short, -*] (1, 52 |- and2.in 2) -- (and2.in 2);

		\draw (and1.out) |- (or.in 1);
		\draw (and2.out) |- (or.in 2);

		\draw (or.out) -- (6.5, 52 |- or.out) node[label=above:$z$] {};
	\end{circuitikz}
\end{center}
che calcola esattamente
\[ z = x_0 \cdot \bar{c} + x_1 \cdot c \]
ossia il valore del canale scelto dal multiplexer.
\section{Algebra di Boole}
Per introdurre l'\textbf{algebra di Boole} introduciamo i seguenti \textbf{assiomi} che determinano
il comportamento dell'alfabeto $\{0, 1\}$ in relazione a delle operazioni di base che possiamo
fare con i suoi elementi.
\begin{gather*}
	a = 0 \implies a \neq 1 \quad \land \quad a = 1 \implies a \neq 0 \\
	a = 0 \implies \bar{a} = 1 \quad \land \quad a = 1 \implies \bar{a} = 0 \\
	0 \cdot 1 = 0 \quad \land \quad 1 \cdot 0 = 0 \\
	0 \cdot 0 = 0 \quad \land \quad 1 \cdot 1 = 1 \\
	0 + 1 = 1 \quad \land \quad 1 + 0 = 1 \\
	0 + 0 = 0 \quad \land \quad 1 + 1 = 1
\end{gather*}
Da questi deduciamo anche che
\begin{gather*}
	A \cdot 1 = A \quad \land \quad A + 0 = A \\
	A \cdot 0 = 0 \quad \land \quad A + 1 = 1 \\
	A \cdot A = A \quad \land \quad A + A = A \\
	A \cdot \bar{A} = 0 \quad \land \quad A + \bar{A} = 1 \\
	\bar{\bar{A}} = A
\end{gather*}
Le altre proprietà fondamentali per le operazioni di \verb|AND| e \verb|OR| nell'algebra booleana
sono
\begin{itemize}
	\item \textbf{Commutatività} per l'\verb|AND|: $A \cdot B = B \cdot A$
	\item \textbf{Commutatività} per l'\verb|OR|: $A + B = B + A$
	\item \textbf{Distributività}: $A \cdot (B + C) = A \cdot B + A \cdot C$ e la formula duale
	      $A + (B \cdot C) = (A \cdot B) + (A \cdot C)$
	\item \textbf{De Morgan}: $\overline{A \cdot B} = \bar{A} + \bar{B}$ e la formula duale
	      $\overline{A + B} = \bar{A} \cdot \bar{B}$
\end{itemize}
Con queste proprietà è possibile semplificare alcune delle formule generate da alcune tabelle di
verità come abbiamo fatto nel caso del multiplexer. Supponiamo che per un qualche motivo otteniamo
una funzione di $a$, $b$ e $c$ tale che
\[ f(a,b,c) = \bar{a} \bar{b} \bar{c} + a \bar{b} \bar{c} + a \bar{b} c \]
Se volessimo implementare questa formula tramite un circuito avremmo bisogno di tre porte
\verb|AND3| e di 1 porta \verb|OR3|. Usando le proprietà possiamo ottenere
\[
	\bar{a} \bar{b} \bar{c} + a \bar{b} \bar{c} + a \bar{b} c
	= \bar{b} \bar{c} (\bar{a} + a) + a \bar{b} c
	= \bar{b} \bar{c} + a \bar{b} c
\]
Passando così ad una formula che ci permette di implementare un circuito tramite due porte
\verb|AND3| e una porta \verb|OR3|. Proviamo un altro modo di procedere
\begin{align*}
	\bar{a} \bar{b} \bar{c} + a \bar{b} \bar{c} + a \bar{b} c
	 & = \bar{a} \bar{b} \bar{c} + a \bar{b} \bar{c} + a \bar{b} \bar{c} + a \bar{b} c         \\
	 & = \bar{b} \bar{c} (\bar{a} + a) + a \bar{b} (c + \bar{c}) = \bar{b} \bar{c} + a \bar{b}
\end{align*}
ottenendo così la possibilità di implementare un circuito tramite due porte \verb|AND2| e una porta
\verb|OR2|. Come possiamo vedere, a seconda di come usiamo queste proprietà, è possibile diminuire
notevolmente la dimensione dei circuiti e quindi la complessità di ciò che stiamo calcolando.
\begin{center}
	\begin{circuitikz}
		% gates
		\node[and port] (and1) at (3.5, 1) {};
		\node[and port] (and2) at (3.5, -1) {};
		\node[or port] (or) at (5.5, 0) {};

		\draw (0, 2) node[label=above:$a$] {} to[short, -*] (0, 52 |- and2.in 1) -- (and2.in 1);
		\draw (0.5, 2) node[label=above:$b$] {} to[short, -*] (0.5, 52 |- and1.in 1) to[short, -o] (and1.in 1);
		\draw (0.5, 52 |- and1.in 1) to[short, -*] (0.5, 52 |- and2.in 2) to[short, -o] (and2.in 2);
		\draw (1, 2) node[label=above:$c$] {} to[short, -*] (1, 52 |- and1.in 2) to[short, -o] (and1.in 2);

		\draw (and1.out) |- (or.in 1);
		\draw (and2.out) |- (or.in 2);
	\end{circuitikz}
\end{center}
A questo punto sarebbe possibile semplificare ulteriormente la formula raccogliendo $\bar{b}$ e
implementando il circuito descritto da
\[ \bar{b} \cdot (\bar{c} + a) \]
ma questo introduce un problema in quanto il circuito generato è asimmetrico, ossia i segnali in
ingresso non attraversano tutti lo stesso numero di porte come possiamo vedere in figura
\begin{center}
	\begin{circuitikz}
		\node[or port] (or) at (3.5, 1) {};
		\node[and port] (and) at (5.5, 0) {};

		\draw (0, 2) node[label=above:$a$] {} to[short, -*] (0, 52 |- or.in 1) -- (or.in 1);
		\draw (0.5, 2) node[label=above:$b$] {} to[short, -*] (0.5, 52 |- and.in 2) to[short, -o] (and.in 2);
		\draw (1, 2) node[label=above:$c$] {} to[short, -*] (1, 52 |- or.in 2) to[short, -o] (or.in 2);

		\draw (or.out) |- (and.in 1);
	\end{circuitikz}
\end{center}
Questo si traduce in un intervallo di tempo in cui la porta \verb|AND| riceve, da una parte il
vecchio segnale trasmesso dalla porta \verb|OR| prodotto al calcolo precedente, dall'altra l'ultimo
segnale prodotto dall'ingresso $b$.

Fino a che la porta \verb|OR| non finisce di elaborare i segnali in arrivo da $a$ e $c$ la porta
\verb|AND| potrebbe produrre risultati errati, dovuti a quello che viene chiamato \textbf{glitch}.

\subsection{Mappe di Karnaugh}
Come abbiamo appena visto, non sempre ridurre la complessità della nostra formula in modo
\emph{monotòno} ci porta alla migliore ottimizzazione. A volte conviene aumentare la complessità
per poi giungere ad un modello migliore.

Le \textbf{mappe di Karnaugh} forniscono un metodo grafico per riuscire a semplificare le formule
booleane senza però garantire la miglior minimizzazione di quest'ultime. Nell'esempio di prima
abbiamo una funzione booleana con la seguente tabella di verità
\begin{center}
	\begin{tabular}{c c c | c}
		$a$ & $b$ & $c$ & $f(a,b,c)$ \\ \hline
		0   & 0   & 0   & 1          \\
		0   & 0   & 1   & 0          \\
		0   & 1   & 0   & 0          \\
		0   & 1   & 1   & 0          \\
		1   & 0   & 0   & 1          \\
		1   & 0   & 1   & 1          \\
		1   & 1   & 0   & 0          \\
		1   & 1   & 1   & 0
	\end{tabular}
\end{center}
Da questa tabella possiamo ricavare una mappa di Karnaugh prendendo tutti i possibili valori di $a$
e mettendoli nella prima colonna e poi prendendo tutti i possibili valori della coppia $bc$ e
mettendoli sulla prima riga, disponendoli in modo che ogni valore differisca dal precedente al più
di un bit.

Il nostro obbiettivo è quello di individuare i quadrati o rettangoli contenenti un numero di 1 pari
ad una potenza di 2 e raggrupparli. Per tale raggruppamento è possibile
\begin{itemize}
	\item Uscire dalla tabella e rientrare dall'altra parte se ho degli 1 agli estremi.
	\item Includere degli 1 già raccolti in un precedente raggruppamento.
\end{itemize}
Nel nostro caso abbiamo due rettangoli da due 1: il primo verticale che prende la prima colonna per
intero e il secondo orizzontale che prende la prima metà della seconda riga.
\begin{center}
\begin{karnaugh-map}[4][2][1][$c$][$b$][$a$]
\maxterms{1, 2, 3, 6, 7}
\minterms{0, 4, 5}
\implicant{0}{4}
\implicant{4}{5}
\end{karnaugh-map}
\end{center}
A questo punto siamo
in grado di semplificare la formula di partenza
\begin{enumerate}
	\item Mettendo in \verb|AND| le variabili facenti parte dello stesso raggruppamento che
	      rimangono costanti e negando quelle con valore 0.
	\item Sommando tra di loro i raggruppamenti.
\end{enumerate}
Otteniamo così la formula ottenuta in precedenza con le proprietà dell'algebra booleana
\[ \bar{b} \bar{c} + a \bar{b} \]
in modo meccanico. Il primo termine della somma è ottenuto prendendo in considerazione il
raggruppamento verticale di 1 e considerando che $b$ e $c$ non variano ed essendo a 0 vengono
negati. Il secondo termini si ottiene similmente notando che $a$ e $b$ sono la parte costante del
raggruppamento ed inoltre $b$ è a 0 e dunque deve essere negato.

Prendiamo ora come esempio un \textbf{sommatore} di 2 bit con riporto, il cui funzionamento dipende
da tre parametri di ingresso: $x_1$ e $x_2$ i bit che vogliamo sommare e $r_0$ il possibile riporto
da aggiungere. Abbiamo inoltre due uscite: il risultato $s$ della somma e il possibile riporto $r_1$
generato da essa.
\begin{center}
	\begin{tikzpicture}
		\draw[thick] (0, 0) rectangle (2, 1.5);

		\draw (0.5, 2) node[label=above:$x_1$] {} to[short, o-] (0.5, 1.5);
		\draw (1.5, 2) node[label=above:$x_2$] {} to[short, o-] (1.5, 1.5);
		\draw (2.5, 0.75) node[label=right:$r_0$] {} to[short, o-] (2, 0.75);
		\draw (0, 0.75) -- (-0.5, 0.75) -- (-0.5, -0.75) node[label=below:$r_1$] {};
		\draw (1, 0) -- (1, -0.75) node[label=below:$s$] {};
	\end{tikzpicture}
\end{center}
In questo caso la tabella di verità di tale oggetto è
\begin{center}
	\begin{tabular}{c c c | c | c }
		$x_1$ & $x_2$ & $r_0$ & $s$ & $r_1$ \\ \hline
		0     & 0     & 0     & 0   & 0     \\
		0     & 0     & 1     & 1   & 0     \\
		0     & 1     & 0     & 1   & 0     \\
		0     & 1     & 1     & 0   & 1     \\
		1     & 0     & 0     & 1   & 0     \\
		1     & 0     & 1     & 0   & 1     \\
		1     & 1     & 0     & 0   & 1     \\
		1     & 1     & 1     & 1   & 1
	\end{tabular}
\end{center}
Le mappe di Karnaugh per $s$ ed $r_1$ risultano le seguenti
\begin{figure}[h!]
\centering
\begin{subfigure}[b]{0.4\textwidth}
\centering
\begin{karnaugh-map}[4][2][1][$c$][$b$][$a$]
\minterms{1,2,4,7}
\maxterms{0,3,5,6}
\implicant{1}{1}
\implicant{2}{2}
\implicant{4}{4}
\implicant{7}{7}
\end{karnaugh-map}
\end{subfigure}
\begin{subfigure}[b]{0.4\textwidth}
\begin{karnaugh-map}[4][2][1][$c$][$b$][$a$]
\minterms{3,5,7,6}
\maxterms{0,1,2,4}
\implicant{3}{7}
\implicant{7}{6}
\implicant{5}{6}
\end{karnaugh-map}
\end{subfigure}
\end{figure}

Da tali mappe di Karnaugh ricaviamo le seguenti formule per $s$ ed $r_1$
\begin{align*}
	s   & = r_0 \bar{x_1} \bar{x_2} + \bar{r_0} \bar{x_1} x_2 + r_0 x_1 x_2 + \bar{r_0} x_1 \bar{x_2} \\
	r_1 & = x_1 x_2 + r_0 x_2 + r_0 x_1
\end{align*}
Di seguito raffiguriamo il circuito ricavato dalla formula per $r_1$.
\begin{center}
	\begin{circuitikz}
		\node[and port] (and1) at (3.5, 1.5) {};
		\node[and port] (and2) at (3.5, 0) {};
		\node[and port] (and3) at (3.5, -1.5) {};
		\node[or port, number inputs=3] (or) at (5.5, 0) {};

		\draw (0, 2) node[label=above:$x_1$] {} to[short, -*] (0, 52 |- and1.in 1) -- (and1.in 1);
		\draw (0, 52 |- and1.in 1) to[short, -*] (0, 52 |- and3.in 2) -- (and3.in 2);

		\draw (0.5, 2) node[label=above:$x_2$] {} to[short, -*] (0.5, 52 |- and1.in 2) -- (and1.in 2);
		\draw (0.5, 52 |- and1.in 2) to[short, -*] (0.5, 52 |- and2.in 2) -- (and2.in 2);

		\draw (1, 2) node[label=above:$r_0$] {} to[short, -*] (1, 52 |- and2.in 1) -- (and2.in 1);
		\draw (1, 52 |- and2.in 1) to[short, -*] (1, 52 |- and3.in 1) -- (and3.in 1);

		\draw (and1.out) -- (or.in 1);
		\draw (and2.out) -- (or.in 2);
		\draw (and3.out) -- (or.in 3);
		\draw (or.out) --++ (0.5, 0) node[label=right:$r_1$] {};
	\end{circuitikz}
\end{center}
Per riassumere possiamo usare sia le regole e gli assiomi dell'algebra booleana per semplificare le
formule ma questo potrebbe portarci sia alla minima forma possibile sia ad un'espressione più
complessa. Con le mappe di Karnaugh non abbiamo la certezza di ottenere la miglior minimizzazione
ma ci offre un modo meccanico per ridurre la complessità.

% \include{3-reti_logiche/reti_logiche}
% \include{4-firmware/firmware}
% \include{5-assembler/assembler}
% \include{6-microarchitettura/microarchitettura}
% \include{7-memoria/memoria}
% \include{8-IO/gestione_IO}
% \include{9-avanzate/architetture_avanzate}

% \begin{figure}[!h]\centering
% 	\begin{circuitikz}
% 		% logic gates
% 		\node[and port] (and1) at (0, 2)  {};
% 		\node[or port] (or) at (0, 0)  {};
% 		\node[and port]	(and2) at (0, -2) {};

% 		\node[xnor port] (xnor) at (2.5, 1) {};
% 		\node[not port] (not) at (2.5, -1) {};

% 		\node[xor port] (xor) at (5, 0) {};

% 		% connections
% 		\draw (and1.out) |- (xnor.in 1);
% 		\draw (or.out)   |- (xnor.in 2);
% 		\draw (and2.out) |- (not.in);

% 		\draw (xnor.out) |- (xor.in 1);
% 		\draw (not.out)  |- (xor.in 2);
% 	\end{circuitikz}
% \end{figure}

% \begin{tabular}{c|c}
% 	Pari & Dispari \\ \hline
% 	0    & 1       \\
% 	2    & 3       \\
% 	4    & 5       \\
% 	6    & 7       \\
% 	8    & 9
% \end{tabular}

\end{document}
