\documentclass[11pt]{report}

% --------------- PACKAGES ---------------
\usepackage[utf8]{inputenc}
\usepackage[T1]{fontenc}
\usepackage[italian]{babel}
\usepackage[hidelinks]{hyperref}

\hypersetup{
	colorlinks=true,
	linkcolor=blue
}

% --------------- STYLE ---------------
\usepackage[margin=1.25in]{geometry}
\usepackage[most]{tcolorbox}

% Font
\usepackage{sansmath}

\renewcommand{\familydefault}{\sfdefault}
\sansmath

% page style
\usepackage{fancyhdr}
\usepackage[Sonny]{fncychap}

\pagestyle{fancy}
\setlength{\headheight}{15pt}
\rhead{\thepage}
\cfoot{\thepage}

% --------------- MATH ---------------
\usepackage{amsmath}
\usepackage{amssymb}
\usepackage{amsthm}
\usepackage{amsfonts}
\usepackage{mathtools}
\usepackage{mdframed}

\newcommand{\N}{\mathbb{N}}
\newcommand{\Z}{\mathbb{Z}}
\newcommand{\R}{\mathbb{R}}
\newcommand{\C}{\mathbb{C}}
\newcommand{\E}{\mathbb{E}}

\newcommand{\F}{\mathcal{F}}

\DeclareMathOperator{\Var}{Var}
\DeclareMathOperator{\Cov}{Cov}

% Boxes for theorem, definitions and examples
\newtheoremstyle{math_box}
{0pt}
{0pt}
{\normalfont}
{}
{\color{orange}}
{\;}
{0.25em}
{\thmname{\textbf{#1}}\thmnumber{ \textbf{#2}}{\color{black}\thmnote{\textbf{ -- #3.}}}}

\newmdenv[
	rightline=false,
	leftline=true,
	topline=false,
	bottomline=false,
	linecolor=orange!40,
	innerleftmargin=5pt,
	innerrightmargin=5pt,
	innertopmargin=0pt,
	innerbottommargin=0pt,
	leftmargin=0cm,
	rightmargin=0cm,
	linewidth=3pt
]{dBox}

\newmdenv[
	rightline=false,
	leftline=false,
	topline=false,
	bottomline=false,
	backgroundcolor=orange!15,
	innerleftmargin=5pt,
	innerrightmargin=5pt,
	innertopmargin=5pt,
	innerbottommargin=5pt,
	leftmargin=0cm,
	rightmargin=0cm,
]{pBox}

\theoremstyle{math_box}
\newtheorem{theoremeT}{Teorema}[section]
\newtheorem{definitionT}{Definizione}[section]
\newtheorem{propositionT}{Proposizione}[section]
\newtheorem{corollary}{Corollario}[section]
\newtheorem{lemma}{Lemma}[section]
\newtheorem{observation}{Osservazione}[section]
\newtheorem{exampleT}{Esempio}[subsection]

\newenvironment{theorem}{\begin{pBox}\begin{theoremeT}}{\end{theoremeT}\end{pBox}}
\newenvironment{definition}{\begin{dBox}\begin{definitionT}}{\end{definitionT}\end{dBox}}
\newenvironment{proposition}{\begin{pBox}\begin{propositionT}}{\end{propositionT}\end{pBox}}
\newenvironment{example}{\begin{dBox}\begin{exampleT}}{\end{exampleT}\end{dBox}}

\usepackage{tikz, pgfplots, pgf-pie}
\usepackage{caption, subcaption}
\usepackage{tikz}
\usepackage{scalerel}
\usepackage{pict2e}
\usepackage{tkz-euclide}
\usepackage{pgfplots, pgfplotstable, pgf-pie}

\usetikzlibrary{calc}
\usetikzlibrary{patterns, arrows}
\usetikzlibrary{shadows}
\usetikzlibrary{external}

\pgfplotsset{compat=newest}
\usepgfplotslibrary{statistics, fillbetween}

\ctikzset{logic ports=ieee}

\title{Architettura degli elaboratori}
\author{Federico Bustaffa}
\date{13/11/2023}

\begin{document}

\maketitle
\tableofcontents

\chapter{Introduzione}
Tutti i calcolatori moderni possono essere schematizzati semplicemente tramite il cosiddetto
\textbf{modello Von Neumann} in cui abbiamo tre componenti principali: \textbf{memoria},
\textbf{processore} e canali di \textbf{I/O}.

\begin{center}
	\begin{tikzpicture}[scale=1.5]
		\node[draw] (mem) at (0, 0) {Memoria};
		\node[draw] (cpu) at (0, -1.5) {CPU};
		\node[draw] (io) at (2, -1.5) {I/O};

		\draw[<->] (mem) -- (cpu);
		\draw[<->] (cpu) -- (io);
	\end{tikzpicture}
\end{center}

Come possiamo vedere dalla figura i collegamenti tra le varie entità sono bidirezionali. Il
processore (ma anche la memoria) è collegato ai canali di I/O e il collegamento che c'è tra memoria
e processore è chiamato \textbf{Von Neumann bottleneck}. Quello che accade tra memoria e processore
è, grosso modo, quello che viene descritto dal seguente pseudocodice ed è denominato ciclo di
\textbf{fetch-decode-execute}.

\begin{minted}{c}
while (true) {
	istr = M[PC]
	decode(istr)
	res = exec(istr)
	update(PC)
	writeback(res)
	interrupt_handling()
}
\end{minted}

In pratica viene estratta dalla memoria l'istruzione puntata da un \textbf{Program Counter}, la
si decodifica e la si esegue. In seguito il Program Counter viene aggiornato e i risultati vengono
consolidati nei registri della CPU oppure in memoria.

Per ognuna delle istruzioni eseguite sul canale presente tra processore e memoria vengono svolte
le seguenti operazioni
\begin{enumerate}
	\item Il processore manda un indirizzo (PC) dove andare a prendere l'istruzione.
	\item La memoria risponde con un'istruzione.
	\item Il processore esegue e durante l'esecuzione può richiedere la lettura o scrittura dalla
	      memoria.
\end{enumerate}
Questo processo avviene molto spesso e rende il traffico sul canale molto intenso creando per
l'appunto un \emph{collo di bottiglia}. Questo è dovuto alla rapida evoluzione dei processori e
alla non altrettanto rapida evoluzione della memoria, la quale ha una velocità nel fornire le
istruzioni minore della velocità che impiega il processore ad eseguirle e richiederne altre.

Per misurare le performance di un processore siamo abituati a ragionare in termini di
\textbf{cicli di clock}. Dato che, approssimativamente, possiamo dire che un ciclo di clock
corrisponde ad un'istruzione eseguita, se ad esempio abbiamo un processore a 1 GHz, questo riuscirà
ad eseguire un'istruzione in 1 nanosecondo.

Durante il corso verrano trattate varie architetture, dalle prime progettate alle più recenti,
cercando di capire come funzionano e i loro pregi e difetti. Tra quelle che tratteremo elenchiamo
\begin{itemize}
	\item \textbf{Single cycle}: modello in cui un ciclo fetch-execute viene effettuato in un ciclo
	      di clock.
	\item \textbf{Pipeline}: modello in cui si hanno componenti distinte per le operazioni di fetch,
	      decode ed execute. In questo modo ogni componente continua a svolgere il proprio lavoro
	      parallelamente alle altre. Una volta arrivato a regime questo processore riesce a
	      eseguire un'istruzione all'inizio di ogni ciclo di clock e non alla fine (come nel
	      modello single cycle).
\end{itemize}
\include{reti_logiche/reti_logiche}
\include{firmware/firmware}
\include{assembler/assembler}
\include{microarchitettura/microarchitettura}
\include{memoria/memoria}
\include{IO/gestione_IO}
\include{architetture_avanzate/architetture_avanzate}

% \begin{figure}[!h]\centering
% 	\begin{circuitikz}
% 		% logic gates
% 		\node[and port] (and1) at (0, 2)  {};
% 		\node[or port] (or) at (0, 0)  {};
% 		\node[and port]	(and2) at (0, -2) {};

% 		\node[xnor port] (xnor) at (2.5, 1) {};
% 		\node[not port] (not) at (2.5, -1) {};

% 		\node[xor port] (xor) at (5, 0) {};

% 		% connections
% 		\draw (and1.out) |- (xnor.in 1);
% 		\draw (or.out)   |- (xnor.in 2);
% 		\draw (and2.out) |- (not.in);

% 		\draw (xnor.out) |- (xor.in 1);
% 		\draw (not.out)  |- (xor.in 2);
% 	\end{circuitikz}
% \end{figure}

% \begin{tabular}{c|c}
% 	Pari & Dispari \\ \hline
% 	0    & 1       \\
% 	2    & 3       \\
% 	4    & 5       \\
% 	6    & 7       \\
% 	8    & 9
% \end{tabular}

\end{document}
