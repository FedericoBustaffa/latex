\section{Automi}
Questi registri ci permettono di implementare degli \textbf{automi} che ad esempio ci permettono
di riconoscere sottostringhe di un certo tipo. Se ad esempio avessimo un vocabolario $\{a, b, c\}$
e volessimo implementare un automa in grado di riconoscere la presenza di sottosequenze del tipo
$abb$ avremmo un qualcosa di questo tipo
\begin{center}
	\begin{tikzpicture}[
			->, >=Stealth,
			node distance=3cm,
			main node/.style={circle, draw, thick, font=\Large}
		]
		\node[state, main node] (init) {Init};
		\node[state, main node] (A) [below left of=init] {A};
		\node[state, main node] (AB) [below right of=init] {AB};

		\draw
		(init) edge node[->, below right, black] {a/0} (A)
		(init) edge[loop above] node{b,c/0} (init)
		(A) edge[above right] node{b/0} (AB)
		(A) edge[loop left] node{a/0} (A)
		(A) edge[bend left] node[above left, black] {c/0} (init)
		(AB) edge[bend right] node[above right, black] {b/1, c/0} (init)
		(AB) edge[bend left] node[below, black] {a/0} (A);
	\end{tikzpicture}
\end{center}
Quello che ora vorremmo essere in grado di fare è implementarlo su un circuito. Per farlo dobbiamo
soddisfare 4 requisiti fondamentali, i quali prevedono la definizione di
\begin{enumerate}
	\item Una rappresentazione delle stringhe in ingresso.
	\item Una rappresentazione per gli stati.
	\item Una funzione che, dato uno stato e un valore (un carattere nel nostro caso), restituisce
	      un valore d'uscita significativo.
	\item Una funzione che, dato uno stato e un valore, restituisce il prossimo stato in cui
	      passare.
\end{enumerate}
Iniziamo con il fornire una rappresentazione delle stringhe in ingresso
\begin{center}
	\begin{tabular}{c | c}
		a & 00 \\ \hline
		b & 01 \\ \hline
		c & 11
	\end{tabular}
\end{center}
Similmente la rappresentazione dei vari stati sarà
\begin{center}
	\begin{tabular}{c | c}
		Init & 00 \\ \hline
		A    & 01 \\ \hline
		AB   & 11
	\end{tabular}
\end{center}
Per definire la funzione in grado di dirci se abbiamo riconosciuto la sequenza oppure no abbiamo
bisogno di di due bit di stato $S_1$ ed $S_0$ e poi di due bit che rappresentano gli ingressi $I_1$
e $I_0$. Non stiamo a scrivere la tabella di verità deducibile dallo schema dell'automa stesso
ma alla fine ricaviamo che l'uscita $Z$ è definita da
\[ Z = S_1 \cdot S_0 \cdot \bar{I}_1 \cdot I_0 \]
In ultima battuta definiamo la funzione che ci permette di dedurre il prossimo stato in cui ci
troveremo basandosi sullo stato attuale e sugli ingressi.
\begin{align*}
	S_1' = & \bar{S}_1 S_0 	\bar{I}_1 I_0                                                    \\
	S_0' = & \bar{S}_1 		\bar{S}_0 \bar{I}_1 \bar{I}_0 + \bar{S}_1  S_0 \bar{I}_1 \bar{I}_0 +
	\bar{S}_1  S_0 \bar{I}_1 I_0 + S_1 S_0 \bar{I}_1 \bar{I}_0
\end{align*}
Non ci rimane che implementare il nostro automa tramite un circuito composto da due moduli che
calcolano rispettivamente $Z$ ed $S'$ e da un registro in grado di memorizzare 2 bit di stato.
\begin{center}
	\begin{tikzpicture}[->, >=Stealth]
		\draw[thick] (-2.5, 0) rectangle (-1, 1) (-1.75, 0.5) node{$S'$};
		\draw[thick] (0.5, 0) rectangle (1.5, 1) (1, 0.5) node{R2};
		\draw[thick] (3, 0) rectangle (4.5, 1) (3.75, 0.5) node{$Z$};

		\draw (-4, 0.5) node[label=left:$I$] {} -- (-2.5, 0.5);
		\draw (-3.25, 0.5) to[short, *-] (-3.25, 1.5) -- (3.75, 1.5) -- (3.75, 1);

		\draw (-1, 0.5) -- (0.5, 0.5);
		\draw (1.5, 0.5) -- (3, 0.5);
		\draw (2.25, 0.5) to[short, *-] (2.25, -0.5) -- (-1.75, -0.5) -- (-1.75, 0);
		\draw (4.5, 0.5) -- (5.5, 0.5);
		\draw (1.25, -1) node[label=right:EN] {} -- (1.25, 0);
		\draw[dashed] (0.75, -1) node[label=left:clock] {} -- (0.75, 0);
	\end{tikzpicture}
\end{center}
Questa che vediamo è detta \textbf{rete sequenziale di Mealy} in quanto implementa un automa di
Mealy. Se non avessimo la diramazione dell'ingresso $I$ che va in $Z$ avremmo parlato di
\textbf{rete sequenziale di Moore}, la quale implementa un automa di Moore che è definito in
maniera leggermente diversa.

\subsection{Sincronizzazione}
Un altro tipo di rete non prevede l'utilizzo del registro R2 ma così fancendo si rende instabile
tutto il circuito per motivi non di nostro interesse. In realtà anche la versione sopra descritta
dovrebbe essere resa stabile tramite un \textbf{sincronizzatore} che può essere sia un componente
specializzato sia un registro come quelli che abbiamo già descritto.
\begin{center}
	\begin{tikzpicture}[->, >=Stealth]
		\draw[thick] (-5.5, 0) rectangle (-4, 1) (-4.75, 0.5) node{Synch};
		\draw[thick] (-2.5, 0) rectangle (-1, 1) (-1.75, 0.5) node{$S'$};
		\draw[thick] (0.5, 0) rectangle (1.5, 1) (1, 0.5) node{R2};
		\draw[thick] (3, 0) rectangle (4.5, 1) (3.75, 0.5) node{$Z$};

		\draw (-6.5, 0.5) node[label=left:$I_1$] {} -- (-5.5, 0.5);
		\draw (-4, 0.5) -- (-2.5, 0.5);
		\node[label=below:$I_2$] at (-3.25, 0.5) {};
		\draw (-3.25, 0.5) to[short, *-] (-3.25, 1.5) -- (3.75, 1.5) -- (3.75, 1);

		\draw (-1, 0.5) -- (0.5, 0.5);
		\draw (1.5, 0.5) -- (3, 0.5);
		\draw (2.25, 0.5) to[short, *-] (2.25, -0.5) -- (-1.75, -0.5) -- (-1.75, 0);
		\draw (4.5, 0.5) -- (5.5, 0.5);
		\draw (1.25, -1) node[label=right:EN] {} -- (1.25, 0);
		\draw[dashed] (-4.75, -1) to[short, *-] (-4.75, 0);
		\draw[dashed] (-6, -1) node[label=left:clock] {} -- (0.75, -1) -- (0.75, 0);
	\end{tikzpicture}
\end{center}
Quello che vogliamo è che il valore degli ingressi sia stabile sul \emph{fronte di salita} del
ciclo di clock. Il blocco Synch serve proprio a stabilizzare e sincronizzare gli ingressi variandoli solo
quando il clock va alto.
