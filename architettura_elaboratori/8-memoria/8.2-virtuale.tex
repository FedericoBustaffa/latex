\section{Memoria virtuale}
Come dovrebbe essere chiaro, quando compiliamo un programma questo possiede alcune informazioni
necessarie per la sua esecuzione. Tra queste abbiamo gli indirizzi ai quali è possibile trovare
istruzioni e dati del nostro eseguibile.

Quando eseguiamo un programma questo viene caricato in memoria principale utilizzando tali indirzzi
e sempre tramite gli stessi indirizzi è possibile andare a reperire, all'occorrenza, istruzioni e
dati.

Ovviamente gli indirizzi che un programma si porta dietro non sono assoluti, non andiamo quindi a
specificare l'esatto indirizzo in memoria fisica poiché si dovrebbe ricompilare il programma ogni
volta per ottenere sempre nuovi indirizzi.

Quello che si fa è assegnare a istruzioni e dati del nostro programma degli
\textbf{indirizzi virtuali}. Per farla semplice, se il nostro programma ha $n$ istruzioni, la prima
istruzione ha indirizzo 0 e l'ultima ha indirizzo $n-1$.

Dato che però tutti i programmi sono compilati in questo modo, abbiamo bisogno di un modo per
tradurre indirizzi virtuali (o logici) in \textbf{indirizzi fisici}.

\subsection{Paginazione}
Per la precisione non andiamo a caricare in memoria le istruzioni una per una, ma andiamo a caricare
delle \textbf{pagine}, ossia porzioni tutte uguali (in genere 4KB) del nostro codice in memoria,
anch'essa divisa in pagine della stessa dimensione.

Quando vogliamo caricare una pagina in memoria traduciamo l'indirizzo della prima istruzione della
pagina in un indirizzo fisico della memoria principale, basandoci sulle pagine libere di
quest'ultima.

La corrispondenza tra indirizzo virtuale e indirizzo fisico viene memorizzata in una tabella detta
\textbf{TLB} (Translation Lookaside Buffer), di cui è provvisto ogni processo che carichiamo in
memoria. Questo metodo ci permette di allocare pagine in memoria principale senza che esse siano
contigue.

Possiamo supporre che il contenuto della TLB rimanga tale fin tanto che il programma è in
esecuzione. Parti del suo contenuto sono però lette e scritte di continuo sulle varie cache
all'interno della CPU.

In particolare abbiamo una cache associativa che effettua