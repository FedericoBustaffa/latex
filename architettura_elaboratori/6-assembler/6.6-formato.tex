\section{Formato delle istruzioni}\label{sz: formato}
Andiamo ora a trattare come viene gestito il \textbf{formato} delle istruzioni assembler in memoria,
più precisamente il formato delle istruzioni nei file \verb|.o|, ossia i \textbf{file oggetto}.

In generale possiamo dire che tali file sono composti da una parola per istruzione e un po' di
memoria, eventualmente inizializzata, che contiene informazioni relative alle variabili globali.

I formati delle istruzioni rispetta i principi di progettazione RISC e possiamo dividerli in tre
gruppi in base alle tre famiglie principali di istruzioni. Tratteremo il formato a 32 bit e come
vedremo avremo dei bit che indicheranno sempre la solita cosa per ogni tipo di istruzione:
\begin{itemize}
	\item I primi 4 bit indicano la \textbf{condizione} secondo cui l'istruzione deve essere
	      eseguita. Sono i bit che riguardano l'esecuzione delle istruzioni condizionali.
	\item I 2 bit che seguono ci dicono il tipo dell'istruzione, ovvero a quale famiglia di
	      istruzioni questa appartiene.
\end{itemize}
Vediamo ora come variano i bit successivi e il loro significato in base all'appartenza ad una certa
famiglia di istruzioni.
\begin{itemize}
	\item \textbf{Istruzioni operative}: per quanto riguarda le istruzioni operative abbiamo
	      \begin{itemize}
		      \item 6 bit che identificano l'istruzione. Del primo di questi 6 bit parleremo più
		            avanti, i seguenti 4 bit sono il vero e proprio identificatore dell'istruzione
		            e l'ultimo bit \verb|S| indica la necessità o meno di settare i flag nei primi
		            4 bit di cui abbiamo parlato in precedenza
		      \item 4 bit che indicano il registro \textbf{sorgente}.
		      \item 4 bit che indicano il registro \textbf{destinazione}.
		      \item I 12 bit rimanenti ci dice come è fatto l'ultimo operando.
	      \end{itemize}
	\item \textbf{Istruzioni di memoria}: la formattazione nel caso di istruzioni per la memoria è
	      leggermente più complessa ma possiamo identificare comunque
	      \begin{itemize}
		      \item 6 bit che ci danno informazioni sull'istruzione di cui 1 bit per capire se
		            stiamo eseguendo una \verb|load| o una \verb|store|, 1 bit ci dice se stiamo
		            scrivendo/leggendo 1 byte o una parola e altri bit ci danno informazioni su
		            come viene fatto l'indirizzamento.
		      \item 4 bit che indicano il registro \textbf{sorgente}.
		      \item 4 bit che indicano il registro \textbf{destinazione}.
		      \item I 12 bit rimanenti ci dice come è fatto l'ultimo operando.
	      \end{itemize}
	\item \textbf{Istruzioni di salto}: in questo caso la formattazione è più semplice rispetto
	      alle istruzioni operative e di memoria in quanto
	      \begin{itemize}
		      \item 2 bit identificano l'istruzione: \verb|b| o \verb|bl|.
		      \item I restanti 24 bit indicano l'offset per muovere il program counter all'interno
		            del nostro programma.
	      \end{itemize}
\end{itemize}
L'assemblatore non fa altro che prendere del codice ASCII e tradurlo in questo formato binario. Per
vedere il formato delle istruzioni è possibile compilare il nostro programma assembler con opzione
\verb|-c| e poi eseguire il comando \verb|objdump -d|:
\begin{minted}{bash}
$ gcc -c file.s
$ objdump -d file.o	
\end{minted}
Supponiamo che il nostro \verb|file.s| sia scritto in questo modo
\begin{minted}{gas}
main:
	b fine
	bl fine
fine:
	mov pc, lr
\end{minted}
l'output del comando \verb|objdump| sarà il seguente
\begin{minted}{bash}
00000000 <main>:
	0:	ea000000	b	8 <fine>
	4:	ebffffff	bl	8 <fine>

00000008 <fine>:
	8:	e1a0f00e	mov	pc, lr
\end{minted}
Il codice esadecimale prima del nome dell'istruzione può essere convertito in binario per
verificare che effettivamente corrisponde al formato binario che abbiamo descritto in precedenza.
Per esempio, nel caso dell'istruzione \verb|b| abbiamo
\begin{center}
	\verb|ea000000 = 11101010 00000000 00000000 00000000|
\end{center}
Come possiamo vedere i primi 4 bit sono \verb|1110|, che si traduce in un'istruzione incondizionata,
i 2 bit che seguono sono \verb|10|, i quali ci dicono la famiglia dell'istruzione, in questo caso
istruzione di salto. Gli ultimi due bit di interesse ci dicono qual'è l'operazione in questione,
in questo caso \verb|10| equivale esattamente ad una \verb|b|.

Il motivo per il quale gli ultimi 24 bit sono a 0 è che l'offset viene espresso in numero di parole
per giungere all'etichetta desiderata, partendo dal valore corrente del program counter, che però
una volta giunto all'istruzione \verb|b| non ha più l'indirizzo di \verb|b| al suo interno ma
contiene l'indirizzo di \verb|b| più 8 che nel nostro caso è proprio 8, ovvero l'indirizzo
dell'etichetta \verb|fine|. Per quanto riguarda l'operazione \verb|bl| l'offset vale \verb|ffffff|,
ossia -1.

\subsection{Campo SRC2}
Il campo \verb|SRC2| è l'ultimo campo da 12 bit delle istruzioni operative e merita un discorso un
po' più approfondito. Per capire questi 12 bit dobbiamo andare a considerare il campo \verb|I|,
rappresentato tramite il primo dei 6 bit \emph{operativi} che abbiamo tralasciato in precedenza.

Quando \verb|I = 0| i 12 bit si dividono in un primo campo \verb|rot| da 4 bit che, una volta
moltiplicato per 2, indica il numero di \emph{rotazioni} a destra da effettuare sugli ultimi 8 bit
per ottenere il risultato immediato.

Se invece \verb|I = 1| dividiamo in due ulteriori casistiche, in ognuna delle quali gli ultimi 4
bit rappresentano un registro e
\begin{itemize}
	\item Nel primo caso abbiamo
	      \begin{itemize}
		      \item 5 bit per indicare una costante.
		      \item 2 bit che ci dicono come e di quanto fare lo shift.
		      \item Un bit a 0 che ci dice come interpretare gli ultimi 4 bit.
	      \end{itemize}
	\item Nel secondo caso abbiamo
	      \begin{itemize}
		      \item 4 bit che indicano un registro.
		      \item 1 bit a 0.
		      \item 2 bit che ci dicono come e di quanto fare lo shift.
		      \item Un bit a 0 che ci dice come interpretare gli ultimi 4 bit.
	      \end{itemize}
\end{itemize}
Il campo \verb|SRC2| ci serve quindi per effettuare operazioni più complesse e serve
all'assemblatore per capire se il terzo operando è un registro, e nel caso come trattarlo, oppure
se è una costante.

\subsection{Istruzioni SIMD}
Il set di istruzioni \textbf{SIMD} (Single Instruction Multiple Data) permettono di utilizzare un
registro dividendolo in più parti indipendenti l'una dall'altra per effettuare operazioni multiple
in un singolo ciclo di clock.

Possiamo infatti effettuare due somme contemporaneamente prendendo due registri divisi a metà,
sommando alla prima metà del primo registro la prima metà del secondo e registro e poi facendo lo
stesso con la seconda metà dei registri. Il risultato sarà memorizzato in un registro, anch'esso
diviso a metà e contenente due risultati di due somme totalmente indipendenti.

Ovviamente nel caso la parte meno significativa dei registri produca un riporto, questo non va ad
influenzare la metà più significativa del registro dove memorizziamo il risultato.

Fino ad ora abbiamo sempre usato registri ad 32 bit ma in ARM sono comunque presenti dei registri
a 64 bit, che altro non sono che l'unione di due registri a 32 bit, con i quali è possibile
effettuare queste operazioni avendo lo stesso numero di bit disponibili per la rappresentazione.

Nel caso in cui volessimo effettuare le stesse operazioni con registri da 32 bit dobbiamo tenere di
conto che avremo a disposizione solo 16 bit per rappresentare il risultato.

\subsection{Istruzioni THUMB}
Le istruzioni \textbf{THUMB} fanno parte di un set di istruzioni più compatto e la cui
rappresentazione avviene su 16 bit.

Sono istruzioni utili per contrarre la forma del codice che scriviamo, come per esempio quando si
vuole effettuare l'incremento di una variabile per una costante
\begin{minted}{gas}
add r0, r0, #1
\end{minted}
Le istruzioni THUMB permettono di considerare il registro \verb|r0| (in questo caso) come registro
sia sorgente che destinazione. Il codice risultante è il seguente
\begin{minted}{gas}
add r0, #1
\end{minted}
Tali istruzioni permettono di leggere due istruzioni alla volta in quanto sono presenti su un unico
registro da 32 bit ma sono più limitate in quanto si rinuncia ad operazioni condizionali,
operazioni in cui si settano i flag e i bit a disposizione per gli immediati (campo \verb|SRC2|)
sono di meno.

In compenso abbiamo meno traffico per la lettura e scrittura delle memoria da e verso il processore
e dunque un \emph{fetch} più rapido e un'esecuzione più agile. Sono istruzioni molto utili per
quanto riguarda il mondo dei microcontrollori che hanno capacità hardware molto limitate.
