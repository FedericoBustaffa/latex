\section{Direttive}
In assembler possiamo utilizzare delle \textbf{direttive} da dare all'assemblatore per svolgere
determinati compiti e per suddividere il codice in sezioni adibite a compiti differenti, a noi
interessano due direttive principali:
\begin{itemize}
	\item \verb|.data|: definisce una sezione di dati, ovvero con sole direttive che riservano aree
	      di memoria. All'interno della sezione .data tipicamente troviamo direttive
	      \begin{itemize}
		      \item \verb|.word|: seguita da valori separati da virgole, riserva un'area di
		            memoria di tante parole quanti sono i valori che seguono, inizializzata con i
		            valori che seguono la .word.
		      \item \verb|.byte|: seguita da valori separati da virgole, riserva un'area di
		            memoria di tanti byte quanti sono i valori che seguono, inizializzata con i
		            valori che seguono la .byte.
		      \item \verb|.fill|: seguita da un numero intero (che deve essere un multiplo di 4),
		            riserva un'area di memoria di tanti byte quanto vale il parametro intero.
		      \item \verb|.string|: seguita da una stringa fra virgolette, riserva un'area di
		            memoria sufficiente a contenere i caratteri della stringa seguiti da un
		            carattere con codice 0.
	      \end{itemize}
	      Tutte queste direttive sono normalmente precedute da un'etichetta, che può essere
	      utilizzata per reperirne l'indirizzo base.
	\item \verb|.text|: definisce un'area di codice. All'interno della sezione troviamo normalmente
	      una o più direttive
	      \begin{itemize}
		      \item \verb|.global <etichetta>|: che denota i simboli che devono essere resi noti al
		            debugger.
	      \end{itemize}
	      oltre ovviamente al codice assembler vero e proprio.
\end{itemize}
Queste direttive ci permettono ad esempio di allocare oggetti in memoria e poi utilizzare ad
esempio il comando \verb|ldr| che carica un registro con il valore di un'etichetta.