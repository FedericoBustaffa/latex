\section{Reti sequenziali}
Le \textbf{reti sequenziali} implementano di fatto delle \textbf{macchine a stati finiti} o
\textbf{automi}, i quali hanno bisogno di una componente di \textbf{memoria} che ci permetta di
creare uno \textbf{stato} all'interno della nostra rete.

Il primo oggetto di cui andiamo a parlare è chiamato \textbf{latch SR} dove SR sta per
\emph{set reset} e permette di memorizzare un bit.
\begin{center}
	\begin{circuitikz}
		\node[nor port] (or1) at (0, 1) {};
		\node[nor port] (or2) at (0, -1) {};

		\draw (-2, 52 |- or1.in 1) node[label=left:$R$] {} -- (or1.in 1);
		\draw (-2, 52 |- or2.in 2) node[label=left:$S$] {} -- (or2.in 2);

		\draw (or2.in 1) --++ (0, 0.5) -- (0.5, 0.5) |- (or1.out);
		\draw (or1.in 2) --++ (0, -0.5) -- (0.5, -0.5) |- (or2.out);

		\draw (0.5, 52 |- or1.out) to[short, *-] (1.5, 52 |- or1.out) node[label=right:$Q$]{};
		\draw (0.5, 52 |- or2.out) to[short, *-] (1.5, 52 |- or2.out) node[label=right:$\bar{Q}$]{};
	\end{circuitikz}
\end{center}
