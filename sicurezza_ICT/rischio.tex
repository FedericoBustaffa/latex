\chapter{Valutazione del rischio}
Il meccanismo di sicurezza di un sistema informatico viene progettato in base ad alcuni parametri utili a valutare
quanto tale sistema sia "\emph{a rischio}". Come vedremo tra poco, il \textbf{fattore di rischio}, è dato da un
insieme di altri fattori che toccano vari aspetti del sistema stesso.

\section{Parametri}
Tra i parametri di cui parlavamo prima abbiamo
\begin{itemize}
	\item Impatti di un'intrusione di un attaccante $A$ \[ Imp_A \]
	\item Probabilità che $A$ tenti l'intrusione \[ P(Int_A) \]
	\item Probabilità di successo dell'intrusione di $A$ \[ P(Succ_A \mid Int_A) \]
	\item Probabilità che $A$ provochi l'impatto di $A$ \[ P(Imp_A) = P(Succ_A \mid Int_A) \times P(Int_A) \]
	\item Rischio dovuto ad $A$ \[ F(Imp_A, P(Imp_A)) \]
\end{itemize}
L'attaccante analizza i sistemi e attacca il sistema che massimizza la probabilità di successo.

\subsection{Valutazione dei parametri}
Per riuscire a dare un valore ai parametri appena elencati si possono usare varie tecniche a seconda del parametro
che stiamo valutando.
\begin{itemize}
	\item La probabilità che l'intrusione avvenga con successo, $P(Succ_A \mid Int_A)$, può essere stimata mediante
	      meccanismi di emulazione dell'avversario (anche ripetute) oppure tramite Mitre Attack Matrix.
	\item La probabilità $P(Int_A)$, che l'attaccante $A$ tenti l'attacco, è molto difficile da valutare ed è per
	      questo che non si ha un valore preciso ma un \textbf{intervallo}. Esistono due framework per la valutazione
	      di questo intervallo: \textbf{DREAD} e \textbf{OWASP}.
\end{itemize}

\subsubsection{DREAD}
Il framework DREAD considera 5 elementi per ordinare le varie intrusioni possibili:
\begin{itemize}
	\item \textbf{Damage}
	\item \textbf{Reproducibility}
	\item \textbf{Exploitability}
	\item \textbf{Affected users}
	\item \textbf{Discoverability}
\end{itemize}
\emph{Damage} e \emph{Affected users} valutano l'impatto che l'intrusione avrebbe, mentre gli altri tre parametri ci
dicono quanto è probabile che l'intrusione avvenga.

Ogni parametro viene valutato in un intervallo da 0 a 10 e la loro somma ci fornisce la valutazione.

\subsubsection{OWASP}
Modello più sofisticato che valuta minaccia e sistema separatemente. La minaccia è valutata in base a
\begin{itemize}
	\item \textbf{Livello di abilità}
	\item \textbf{Motivazione}
	\item \textbf{Opportunità e possibilità}
	\item \textbf{Dimensione della minaccia}
\end{itemize}
Il sistema viene invece valutato in base a
\begin{itemize}
	\item \textbf{Facilità con cui viene scoperto}
	\item \textbf{Costruzione di exploit}
	\item \textbf{Conoscenza delle vulnerabilità da parte della minaccia}
	\item \textbf{Rilevamento di exploit}
\end{itemize}
Questa seconda valutazione è applicabile se conosciamo interamente il sistema e non se consideriamo la singola
vulnerabilità.

Ovviamente per riuscire a valutare correttamente il sistema lo si deve osservare dal punto di vista dell'attaccante
cercando di individuare ciò di cui è a conoscenza e cosa invece non conosce.

\section{Calcolo del rischio}
Una volta compiuta la fase di valutazione è possibile trasformare anche $P(Succ_A \mid Int_A)$ in una valutazione
semiquantitativa della probabilità di intrusione ottenendo la \textbf{matrice di valutazione di rischio}.

Si tratta di una matrice bidimensionale che mette insieme la probabilità che un attacco avvenga e la probabilità che
questo abbia successo.