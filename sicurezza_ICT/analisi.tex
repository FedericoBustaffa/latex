\chapter{Analisi e gestione del rischio}
L'approccio incondizionale comporta costi molto elevati e spesso inaccettabili, inoltre richiede una quantità di lavoro
enorme e spesso inutile.

Con l'approccio condizionale invece si cerca di capire quali componenti del sistema si possono difendere e soprattutto
quali componenti \emph{conviene} difendere.

Per capirlo è necessaria un'\textbf{analisi del rischio} con la quale si cerca di individuare la tipologia di attacco
più probabile in relazione al sistema che stiamo cercando di proteggere.

\section{Tipologie di analisi}
L'analisi del rischio si divide in diverse sottocategorie più specifiche che ci permettono di individuare in modo più
mirato eventuali problemi di diversa natura. In particolare parliamo di
\begin{itemize}
	\item Analisi delle risorse da proteggere (asset)
	\item Analisi delle minacce
	\item Analisi delle vulnerabilità
	\item Analisi degli attacchi
	\item Analisi degli impatti
	\item Individuazione del rischio, rischio accettabile ed introduzione di contromisure
\end{itemize}

\subsection{Analisi delle risorse}
In questa fase si cerca di individuare un insieme di oggetti ed alcune proprietà di sicurezza. Si definisce in seguito
una politica su questi oggetti in termini delle proprietà precedenti come diritti di lettura, scrittura, esecuzione
e così via.

\subsection{Analisi delle minacce}
In questa fase cerchiamo di capire chi è interessato ad attaccare il nostro sistema per rubare o modificare informazioni
o per impedire agli utenti di utilizzare il sistema.

Le possibili minacce possono arrivare sia da attaccanti che vogliono violare il sistema sia da eventi naturali che
potrebbero in qualche modo comprometterne l'integrità (in questo caso parliamo di \emph{safety}).

\subsubsection{Safety e Security}
Come anticipato, quando ci riferiamo alla capacità di un sistema di resistere ad eventi di origine non umana e casuale
come terremoti, crolli e così via, parliamo di \textbf{safety}.

Parliamo invece di \textbf{security} quando ci riferiamo alla capacità del sistema di resistere ad attacchi umani con
uno scopo malizioso per raggiungere un obbiettivo.

\subsection{Analisi delle vulnerabilità}
In questo caso cerchiamo di individuare quali sono le vulnerabilità che permettono ad un attaccante di ottenere, in un 
certo numero di passi, accesso alle risorse di suo interesse.