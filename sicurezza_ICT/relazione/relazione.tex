\documentclass[11pt, a4paper]{article}

\pdfpagewidth\paperwidth
\pdfpageheight\paperheight

\usepackage[utf8]{inputenc}
\usepackage[T1]{fontenc}
\usepackage[italian]{babel}
\usepackage[hidelinks]{hyperref}

\title{Rischi legati alla blockchain}
\author{Federico Bustaffa}
\date{02/12/2022}

\begin{document}

\maketitle
\tableofcontents

\section{Blockchain}
Per prima cosa facciamo chiarezza sul funzionamento di una \textbf{blockchain} a prescindere dall'ambito per il
quale essa viene utilizzata. Che si tratti di criptovalute, sistemi di sicurezza, meccanismi di autenticazione o
autorizzazione, la blockchain è una struttura dati con proprietà crittografiche in grado di garantire grande
sicurezza ma soprattutto \textbf{consistenza} dei dati.

\subsection{Struttura}
La blockchain altro non è che una \textbf{lista concatenata} di blocchi contenenti delle informazioni. Ogni blocco
possiede un \textbf{header} e un \textbf{corpo}.

Nell'header sono contenute informazioni come riferimenti al blocco precedente e altri dati fondamentali per
il mantenimento dell'integrità della struttura di cui parleremo tra poco.

Nel corpo sono invece contenuti i dati veri e propri che si intende proteggere, nel caso di Bitcoin, il corpo di
un blocco contiene transazioni.

\subsubsection{Header}
Un generico header della blockchain contiene quattro campi:
\begin{itemize}
	\item \textbf{Timestamp}: informazione temporale che indica il momento di creazione del blocco.
	\item \textbf{Nonce}: si tratta di un valore numerico di cui parleremo più avanti, fondamentale per il
	      funzionamento di tutto il meccanismo.
	\item \textbf{Merkle Root}: la radice di un albero di Merkle costruito calcolando l'hash del contenuto del
	      corpo di un blocco. Fondamentale per verificare l'integrità dei dati.
	\item \textbf{Previous Block Hash}: si tratta del valore hash calcolato sull'header del blocco precedente.
\end{itemize}

\subsubsection{Corpo}
Nel corpo del blocco vi sono semplicemente le informazioni di interesse della blockchain, come già detto, nel caso
di Bitcoin, il corpo di un blocco contiene delle transazioni in bitcoin mentre in altri casi potrebbe contenere
dati utili per autenticare o autorizzare un certo utente ad accedere a un determinato servizio.

\subsection{Funzionamento}
Perché un sistema basato su blockchain funzioni c'è bisogni di un sistema \textbf{P2P} in cui ogni nodo
rappresenta un utente, il quale possiede una copia della blockchain.

Ogni utente del sistema \emph{compete} con altri utenti per riuscire ad aggiungere un blocco alla catena (più
avanti vedremo nello specifico in cosa consiste questa competizione). Nel caso si riesca a vincere tale
competizione si ha diritto a creare ed aggiungere un nuovo blocco.

In realtà, prima di aggiungere un nuovo blocco, questo dev'essere validato da tutti gli altri utenti del
sistema. Come sappiamo, ogni blocco della catena possiede un header contenente dei valori, come la radice
dell'albero di Merkle, la quale viene calcolata sulla base dei valori di tutti i blocchi precedenti della
catena.

\subsubsection{Competizione}
L'idea alla base della blockchain è quella di mettere in \textbf{competizione} i vari utenti del sistema facendogli
risolvere un problema \emph{computazionalmente complesso}.

Il problema consiste nel trovare un certo valore, \textbf{nonce}, di cui abbiamo parlato prima, tale che la funzione
hash (per esempio SHA256), applicata a tale valore, concatenato con l'header del blocco precedente dia come
risultato un valore inferiore ad un \textbf{difficulty target}.

Se si riesce a trovare tale valore per primi si ha diritto ad appendere un nuovo blocco in fondo alla catena e si
viene ricompensati per il tempo macchina messo a disposizione.

\end{document}