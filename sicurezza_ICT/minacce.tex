\chapter{Descrizione di una minaccia}
Riuscire a descrivere un attaccante è diventato un compito di sempre maggiore importantanza, ecco perché, a tal fine,
sono nati strumenti come la \textbf{Mitre Attack Matrix} costruita in questo modo:
\begin{itemize}
	\item Ogni colonna indica una \textbf{tattica}, ossia un obbiettivo a breve termine durante un'intrusione.
	\item Ogni riga indica una \textbf{tecnica}, ovvero il come viene raggiunto un obbiettivo a breve termine. Ogni
	      tattica è composta di più tecniche per riuscire a raggiungere l'obbiettivo.
	\item L'implementazione dettagliata di una tecnica è chiamata \textbf{procedura}.
\end{itemize}
Questa matrice permette un'emulazione molto dettagliata di un avversario ma è carente sul tipo di strategia utilizzata,
ossia sull'ordine con cui vengono effettuate le varie azioni.

\section{Attacchi al cloud}
Secondo il MITRE le intrusioni in sistemi cloud, di solito, non comprendono malware perché il provider è in grado di
rilevarlo.

Ecco che in questo ambito sono nati nuovi comportamenti possibili come furto di credenziali, creazione di nuove
macchine virtuali. Un'operazione come l'\textbf{esfiltrazione} può essere implementata come trasferimento di dati
da un account all'altro.

\section{Infrastruttura di comando e controllo}
La matrice citata in precedenza trascura azioni dell'attaccante per creare una propria infrastruttura di comando e
controllo che può servire sia per lanciare i primi passi di intrusione che per interagire con il sistema attaccato
quando intrusione o persistenza hanno successo.

Le azioni per creare tale infrastruttura non riguardano il sistema target e quindi non è possibile tenerne conto per
rilevare e sconfiggere un'intrusione. La sua creazione è semplificata se esiste un grande numero di sistemi a bassa
sicurezza per riuscire a creare facilmente su ognuno di essi una botnet.