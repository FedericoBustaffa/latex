\chapter{Contromisure}
Una prima classificazione delle contromisure:
\begin{itemize}
	\item \textbf{Proattive}:
	\item \textbf{Dinamiche}:
	\item \textbf{Reattive}:
\end{itemize}
Il processo per l'eliminazione delle vulnerabilità prevede i seguenti passi:
\begin{enumerate}
	\item \textbf{Prevenire}
	\item \textbf{Resistere}
	\item \textbf{Scoprire}
	\item \textbf{Ripristinare}
	\item \textbf{Reagire}
\end{enumerate}

\section{Robustezza e resilienza}
Due punti cardine della sicurezza sono \textbf{robustezza} e \textbf{resilienza}.

Per robustezza si intende la capacità di un sistema di \emph{respingere} le intrusioni.

Una volta che le intrusioni hanno successo le prestazioni (in termini di sicurezza) cala. La capacità di un sistema
di ripristinare la robustezza dopo che questo viene violato è la resilienza.

\subsection{Ridondanza ed eterogeneità}
Robustezza e resilienza si basano su \textbf{ridondanza} ed \textbf{eterogeneità}.

La ridondanza viene si ottiene sovradimensionando il sistema con componenti \emph{inutili} ai fini di funzionamento
del sistema ma che sono molto utili nel momento in cui il sistema viene violato per ripristinarlo.

La ridondanza può essere di vario tipo:
\begin{itemize}
	\item \textbf{Ridondanza fredda}: componenti inattive che vengono attivate solo quando il sistema viene attaccato
	      per riuscire a ripristinare lo stato sicuro.
	\item \textbf{Ridondanza calda}: componenti sempre attive in modo da riuscire a sopportare meglio gli attacchi.
	\item \textbf{Triple Modular Redundancy}: ridondanza calda dove tre copie di uno stesso modulo prendono lo stesso
	      input ed eseguono la stessa computazione \emph{votando} infine il risultato. Aumenta la safety ma non sempre
	      la security.
\end{itemize}

Per eterogeneità si intende l'uso di componenti diversi per evitare la caduta del sistema su una singola vulnerabilità.

\section{Sistema minimale}
Una buona pratica per la sicurezza è quella di costruire un sottoinsieme del sistema, formato da componenti
particolarmente robusti e spesso eterogenei rispetto a quelli del sistema principale.

Il sistema minimale è quello da cui parte il ripristino del normale funzionamento del sistema.

La violazione del sistema minimale comporta l'incapacità di ripristinare il sistema dopo un attacco.

\section{Meccanismi per contromisure}
L'implementazione di contromisure è basata di meccanismi condivisi i quali devono essere molto robusti dato che una
loro vulnerabilità potrebbe influenzare molte contromisure del sistema.

Tali meccanismi sono implementati su un security kernel che fa parte del TCB e che gestisce
\begin{itemize}
	\item Identificazione
	\item Autenticazione
	\item Diritti
\end{itemize}

\subsection{Autenticazione}
Per implementare un buon meccanismo di autenticazione si fa sempre riferimento alla tripla
\begin{center}
	\verb|< soggetto, oggetto, operazione >|
\end{center}
la quale associa ad un soeggetto un certo diritto.

Possiamo associare due tipi di controlli a questa tupla
\begin{itemize}
	\item Controlli sull'identità del soggetto.
	\item Controlli sul possesso del diritto \verb|< oggetto, operazione >|.
\end{itemize}

Il controllo sul possesso dei diritti dei vari utenti e la loro gesione è compito del sistema operativo. Anche
l'autenticazione è compito del sistema operativo ma ora esistono componenti specializzati per l'autenticazione che in
seguito forniscono un token che certificano in qualche modo l'avvenuta autenticazione.

\subsubsection{Tipi di autenticazione}
Esistono tre tipi principali di autenticazione:
\begin{itemize}
	\item \textbf{Debole statica}: include password e altre tecniche soggette ad attacchi che cercano di riprodurre
	      le sequenze di autenticazione.
	\item \textbf{Debole non statica}: fa uso di meccanismi crittografici per creare password utilizzabili per una
	      singola sessione. Può essere compromessa da attacchi di tipi \emph{session hijacking}, ossia dal
	      \emph{furto} della sessione.
	\item \textbf{Forte}: utilizza forti meccanismi crittografici per prevenire i principali problemi legati
	      all'autenticazione debole.
\end{itemize}

\subsubsection{Meccanismi di autenticazione}
Tra i meccanismi di autenticazione più comuni abbiamo:
\begin{itemize}
	\item L'utilizzo di un qualcosa che solo l'utente conosce, come una frase o un numero. Per difendersi da possibili
	      attacchi in genere si salva un hash di queste informazioni e non l'informazione in chiaro. Un altro metodo
	      consiste nell'uso di database diversi per impedire attacchi che precalcolano l'hash.
	\item L'utilizzo di qualcosa che solo l'utente possiede, come una chiave di cifratura, una scheda magnetica o un
	      meccanismo di autenticazione. Con questo metodo di autenticazione si può
	      \begin{itemize}
		      \item Generare il prossima bit di una sequenza pseudocasuale
		      \item Applicare una funzione condivisa ad un valore ricevuto dal server e restituire il risultato.
	      \end{itemize}
	\item L'utilizzo di parametri biometrici. Questo tipo di autenticazione porta con sé problemi di falsi
	      positivi/negativi, attacchi alla trasmissione di informazioni biometriche e difficoltà nel cambiare la
	      password.
\end{itemize}

\subsubsection{Kerberos}