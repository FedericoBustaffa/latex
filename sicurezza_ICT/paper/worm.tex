\documentclass{beamer}

\usepackage[utf8]{inputenc}

\title{Rilevamento di worm polimorfi}
\author{Federico Bustaffa}
\date{10/12/2022}

\begin{document}

\maketitle

\begin{frame}{Worm polimorfi}
	\only<1>{
		L'attacco tramite \textbf{worm polimorfi}, in particolare quelli di tipo zero-day, rappresenta uno dei maggiori problemi di sicurezza
		informatica:
		\begin{itemize}
			\item  Rapidità di propagazione da un host all'altro
			\item  Capacità di auto-modifica
			\item  Cifratura del proprio corpo tramite chiavi diverse
			\item  Uso di exploit zero-day per violare i sistemi sfruttando vulnerabilità non ancora corrette
		\end{itemize}
	}

	\onslide<2->{
		Utilizzando tecniche di modifica del proprio corpo, un singolo worm possiede diverse possibili firme, il che rende difficile
		ottenere una loro fingerprint.
	}

	\onslide<3>{
		Questo fa sì che i meccanismi di rilevamento tradizionali e quelli basati su firma non riescano ad individuare la minaccia.
	}

\end{frame}

\end{document}