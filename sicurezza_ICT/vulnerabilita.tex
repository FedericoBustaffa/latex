\chapter{Vulnerabilità e attacchi}
Iniziamo in questo capitolo a parlare più nello specifico di \textbf{vulnerabilità} e \textbf{attacchi}.

\section{Buffer overflow}
Uno degli esempi più comuni è il \textbf{buffer overflow}, dove il \textbf{buffer} è una qualunque zona di memoria
in cui è possibile memorizzare informazioni e l'\textbf{overflow} è un'operazione che consiste nell'andare a scrivere,
in quella zona di memoria, più informazioni di quante essa ne possa contenere.

Se il software è privo di controlli è possibile trasmettere del codice eseguibile (\textbf{code injection}) ed
eseguirlo. Questo tipo di attacco è molto complesso ma \emph{automatizzabile} e può garantire il pieno controllo del
sistema.

Il meccanismo di base dell'attacco si basa sulla copia di una stringa in una struttura dati: una volta capito quali
caratteristiche deve avere la stringa l'attacco è molto semplice e dato che è automatizzabile potrebbe essere condotto
da chiunque.

\subsection{Memoria di un processo}
Per capire il buffer overflow è fondamentale capire la struttura della memoria di un processo, la quale è organizzata
in tre segmenti: \emph{text}, \emph{data} e \emph{stack}.
\begin{itemize}
	\item Il \textbf{text segment} ha dimensione fissa, memorizza il codice del programma ed è in sola lettura. Ogni
	      tentativo di scrittura provoca un \emph{segmentation fault}.
	\item Il \textbf{data segment} memorizza variabili statiche e dinamiche.
	\item Lo \textbf{stack segment} memorizza i dati per gestire \emph{call} e \emph{return} di funzioni.
\end{itemize}

\subsection{Stack overflow}
Un tipico esempio è quello di attacco \textbf{stack overflow}, con il quale si sfruttano linguaggi di basso livello
come il C per poter scrivere codice in memoria.

Ciò che vogliamo sfruttare, in questo caso, è il fatto che quando vogliamo copiare una stringa di una certa dimensione
in un buffer di dimensione più piccola, si verificherà un \emph{segmentation fault} e lo stack del processo viene
riscritto con la parte di stringa che non è entrata nel buffer all'indirizzo di ritorno del processo stesso.

Se nella stringa inseriamo del codice e stimiamo correttamente la grandezza del buffer, tramite questa tecnica, è
possibile scrivere in memoria un programma ed eseguirlo.
