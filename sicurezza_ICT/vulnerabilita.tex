\chapter{Vulnerabilità}
Iniziamo in questo capitolo a parlare più nello specifico di \textbf{vulnerabilità} e \textbf{attacchi}.

\section{Buffer overflow}
Uno degli esempi più comuni è il \textbf{buffer overflow}, dove il \textbf{buffer} è una qualunque zona di memoria
in cui è possibile memorizzare informazioni e l'\textbf{overflow} è un'operazione che consiste nell'andare a scrivere,
in quella zona di memoria, più informazioni di quante essa ne possa contenere.

Se il software è privo di controlli è possibile trasmettere del codice eseguibile (\textbf{code injection}) ed
eseguirlo. Questo tipo di attacco è molto complesso ma \emph{automatizzabile} e può garantire il pieno controllo del
sistema.

L'attacco si basa sull'operazione di copia di una stringa in una struttura dati: una volta capito quale caratteristiche
deve avere la stringa l'attacco è molto semplice e dato che è automatizzabile potrebbe essere condotto da chiunque.

\subsection{Memoria di un processo}
Per capire il buffer overflow è fondamentale capire la struttura della memoria di un processo, la quale è organizzata
in tre segmenti: \emph{text}, \emph{data} e \emph{stack}.
\begin{itemize}
	\item Il \textbf{text segment} ha dimensione fissa, memorizza il codice del programma ed è in sola lettura. Ogni
	      tentativo di scrittura provoca un \emph{segmentation fault}.
	\item Il \textbf{data segment} memorizza variabili statiche e dinamiche.
	\item Lo \textbf{stack segment} memorizza i dati per gestire \emph{call} e \emph{return} di funzioni.
\end{itemize}

\subsection{Stack overflow}
Un tipico esempio di buffer overflow è quello dello \textbf{stack overflow}, con il quale si sfruttano linguaggi di
basso livello come il C per poter scrivere codice in memoria da locale.

Questo tipo di attacco sfrutta il fatto che scrivere una stringa di una certa dimensione in un buffer di dimensione
inferiore, provoca un \emph{segmentation fault}. Il risultato finale è aver sovrascritto, con la parte di stringa che
non è entrata nel buffer, lo stack del processo all'indirizzo di ritorno del processo stesso.

Di norma, se l'indirizzo di ritorno del processo è valido, non viene sollevata alcuna eccezione e l'esecuzione continua
eseguendo l'istruzione a quell'indirizzo. L'attacco sfrutta l'assenza di controlli per sostituire l'indirizzo di ritorno
con un puntatore al codice iniettato.

In genere si prova ad attaccare procedure al livello di kernel del sistema operativo dato che queste sono eseguite
in stato \emph{root} e, nel caso l'attacco riuscisse, si ottiene una shell con privilegi di amministratore.

Se sappiamo come scrivere la stringa in modo da cancellare l'indirizzo di ritorno e sostituirlo con un puntatore al
codice iniettato si ottiene un attacco automatizzabile molto semplice da eseguire.

\subsection{Contromisure}
Per ovviare al problema degli attacchi buffer overflow ci sono varie contromisure:
\begin{itemize}
	\item \textbf{Meccanismo di tipi forte} con conseguente aumento di overhead nell'esecuzione del software.
	\item \textbf{ASLR}: generazione casuale degli indirizzi dei vari segmenti così da impedire all'attaccante di
	      conoscere staticamente i vari indirizzi.
	\item \textbf{Canary}: inserzione di un valore casuale prima dell'indirizzo di ritorno. Tale valore viene generato
	      ad ogni invocazione di funzione e prima che la funzione ritorni si verifica che tale valore non sia stato
	      cambiato.
	\item \textbf{Stack non eseguibile}: si costruisce uno stack che contiene solo variabili e non codice.
	\item \textbf{Shadow stack}: si crea una copia dello stack dove verranno inseriti solo gli indirizzi di ritorno
	      e non le variabili o le istruzioni e prima di ritornare si confrontano i valori dello stack copia con quello
	      originale e nel caso in cui non coincidano potrebbe essere segno di un attacco di tipo buffer overflow.
\end{itemize}

\subsubsection{Vulnerabilità}
Le vulnerabilità di un software in C che permettono questo attacco sono:
\begin{itemize}
	\item Il mancato controllo sulla dimensioni delle variabili
	\item Cattivo sistema di tipi
	\item Non c'è controllo su operazioni di memoria errate
	\item Errata dimensione dello stack
\end{itemize}

\section{Vulnerabilità strutturali TCP/IP}
Quando è nato lo stack TCP/IP, uno degli obbiettivi era costruire una rete che sopravvivesse ad attacchi fisici
distruttivi per i componenti.

Esistono dei metodi per capire se un nodo in rete è ancora attivo e raggiungibile ma non è possibile capire l'origine
delle comunicazioni.

Per controllare se un nodo \verb|B| è attivo, un nodo \verb|A| potrebbe inviare un messaggio di \verb|ECHO|, a cui
\verb|B| risponde con lo stesso messaggio.

Per verificare con un solo messaggio se più nodi sono attivi è possibile specificare un indirizzo parziale per
\verb|ECHO| in modo da realizzare una comunicazione di tipo uno a molti.

Con il protocollo IPv4 non esiste un controllo sui campi di un pacchetto dunque un utente potrebbe spacciarsi per un
altro senza problemi.

\subsection{Distributed denial of service}
Supponiamo di avere una rete \verb|R| di 1000 nodi, con un indirizzo parziale \verb|X| comune a tutti i nodi in
\verb|R|. A questo punto un attaccante \verb|A| potrebbe condurre un attacco di questo tipo:
\begin{enumerate}
	\item \verb|A| manda un messaggio di \verb|ECHO| all'indirizzo parziale \verb|X| dicendo di essere \verb|B|.
	\item Tutti i nodi di \verb|R| rispondono a \verb|B| con il messaggio di \verb|ECHO|.
	\item \verb|B| per un certo periodo di tempo non riesce a comunicare perché le sue linee di comunicazione sono
	      intasate.
\end{enumerate}
L'effetto di questo attacco si può amplificare aumentando il numero di nodi che hanno il ruolo di \verb|A| (nodi
\emph{zombie}).

Difendersi da questo attacco è molto difficile dato che \verb|B| non ha modo di capire che l'attacco è in preparazione.

\subsection{Contromisure}
Per quanto riguarda questi attacchi l'unica soluzione possibile è l'autenticazione del mittente che instaura la
connessione e invia il pacchetto tramite IPv6.

\section{Classificazione delle vulnerabilità}
In generale non esiste un metodo unico per classificare una vulnerabilità. Esistono diverse tassonomie, ognuna delle
quali in grado di classificarle in modo più o meno accurato in relazione al problema che si vuole risolvere.

Un possibile metodo di classificazione di una vulnerabilità funziona in base al \emph{dove} questa si trova. Nello
specifico si parla di vulnerabilità
\begin{itemize}
	\item \textbf{Procedurale} quando le azioni eseguite non sono ben definite.
	\item \textbf{Organizzative} quando le azioni sono ben definite ma gli utenti le eseguono male.
	\item \textbf{Strumentali} (sistema operativo, compilatori, supporti a tempo di esecuzione e così via) quando le
	      azioni sono ben definite ed eseguite ma con cattivi strumenti.
\end{itemize}

Un'altra possibile classificazione, spesso non rigorosa ed ambigua, è quella relativa agli strumenti che divide le
vulnerabilità come segue:
\begin{itemize}
	\item Di \textbf{specifica}: uno strumento è più generale di quanto richiesto.
	\item \textbf{Implementative}: errore nel codice di uno strumento.
	\item \textbf{Strutturali}: uno o più strumenti vengono integrati in unico sistema causando comportamenti anomali.
\end{itemize}

\section{Ciclo di vita delle vulnerabilità}
Il ciclo di vita di una vulnerabilità segue queste fasi:
\begin{enumerate}
	\item Nasce quando qualcuno commette un errore.
	\item Diventa nota quando qualcuno la scopre.
	\item Diventa pubblica quando la sua presenza viene rivelata.
	\item Inizio di due attività parallele:
	      \begin{itemize}
		      \item Ricerca di un modo per eliminarla.
		      \item Ricerca di un exploit per sfruttarla.
	      \end{itemize}
	\item In base a chi arriva prima potrebbe diventare una vulnerabilità sfruttata.
	\item Quando è disponibile un modo per eliminarla deve essere applicato.
\end{enumerate}

\subsection{Patch}
Le \textbf{patch} del software, ovvero l'eliminazione di una o più vulnerabilità, solitamente sono file eseguibili e
non codice sorgente, la cui esecuzione apporta modifiche o sostituisce i file binari già presenti.

Le patch possono essere anche essere rilasciate sotto forma di file di testo contenenti le differenze tra le varie
versioni, sarà poi compito del gestore di versioni (CVS) generare la vera e propria patch in locale. Si tratta di una
soluzione solitamente usata da sistemi open source in cui è l'utente a dover compilare i nuovi sorgenti.

Una nuova patch potrebbe però introdurre problematiche che richiedono una \textbf{regressione} della versione software.

Occorre quindi un \textbf{test di regressione} in grado di verificare il corretto funzionamento dopo
\begin{itemize}
	\item L'implementazione di una nuova funzionalità
	\item La correzione di eventuali bug
\end{itemize}
Per garantire la qualità del software è essenziale verificare che il codice aggiunto non pregiudichi le funzionalità
esistenti.

\section{Ricerca delle vulnerabilità}
Per effettuare una ricerca delle vulnerabilità occorre distinguerle in tre categorie:
\begin{itemize}
	\item \textbf{Note}: pubbliche in moduli noti (molto comuni).
	\item \textbf{Non pubbliche}: scoperte ma non pubblicate.
	\item \textbf{Zero day}: scoperte e non pubblicate per essere utilizzate nelle intrusioni (molto rare).
\end{itemize}

Ogni sistema può essere descritto come una composizione di moduli standard e specializzati. I moduli standard sono
affetti da vulnerabilità note mentre quelle di moduli specializzati non lo sono. Una strategia efficace di ricerca
delle vulnerabilità opera in quest'ordine:
\begin{enumerate}
	\item Ricerca di vulnerabilità in moduli standard.
	\item Ricerca di vulnerabilità in moduli specializzati.
	\item Ricerca di vulnerabilità strutturali nate dalla composizione delle due tipologie di modulo.
\end{enumerate}

\subsection{Inventario}
Un passo fondamentale per trovare le vulnerabilità di un sistema è conoscere i moduli standard che utilizza, a tal
fine occorre un \textbf{inventario} dei moduli. Costruire l'inventario non è banale e spesso rappresenta il problema
di sicurezza più complesso ed è anche per questo che esistono strumenti informatici con l'unico fine di costruire
inventari.

Una volta costruito l'inventario è necessario produrre del softare cosiddetto \textbf{bill of materials}, il quale
richiede inventario e provenienza dei vari moduli.

\subsection{Vulnerability scanner}
Il \textbf{vulnerability scanner} è uno strumento che permette di costruire un inventario e conoscere le vulnerabilità
note dei moduli dei vari nodi.

Lo strumento opera su un certo sottoinsieme di nodi e fornisce i moduli di ogni nodo che interagiscono con gli altri
moduli e le vulnerabilità di questi moduli. In particolare, sfrutta il fatto che ogni modulo opera su una particolare
porta del nodo e invia pacchetti sulla porta e dall'analisi delle risposte deduce quale modulo sta operando. Una volta
riconosciuto il modulo ricerca le sue vulnerabilità note in appositi database.

\subsubsection{Fingerprinting attivo}
Questo tipo di \emph{vulnerability scanner} opera inviando, sulle porte note (es. 80), dei pacchetti che violano lo
standard del protocollo.

Tutte le implementazioni del servizio rispondono allo stesso modo a pacchetti che rispettano il protocollo, rispondono
in modo diverso a pacchetti lo violano.

Conoscendo le risposte a richieste non standard si possono identificare i moduli che implementano un servizio su una
porta nota.

\subsubsection{Fingerprinting passivo}
In questo caso si opera senza interferire con il sistema da analizzare raccogliendo ed analizzando pacchetti in rete
in maniera trasparente.

In base ad alcune caratteristiche dei pacchetti si riesce a dedurre i moduli che li hanno prodotti.

Rinunciare all'approccio attivo rende il procedimento molto meno invasivo ma i pacchetti da analizzare sono molti di
più, si ha quindi un aumento del tempo di scanning.

\subsubsection{Possibili errori}
Un \emph{vulnerability scanner} potrebbe riconoscere il modulo che implementa un certo servizio e accedere ad un
database con le relative vulnerabilità che potrebbero essere state patchate. In questo caso si ha un \textbf{falso
	positivo}.

Si possono avere anche casi di \textbf{falso negativo} quando una vulnerabilità non viene segnalata dato che non
appartiene al database usato ma è comunque presente nel sistema.

\subsection{Matrice di confusione}
La \textbf{matrice di confusione} è uno strumento nato in medicina per analizzare fenomeni complessi, i quali non
possono essere risolti utilizzando un metodo booleano.

In medicina quando si vuole sapere se un paziente ha una certa malattia si effettuano dei test, i quali vengono
condotti in base ai sintomi. Il risultato del test può essere positivo o negativo in entrambi i casi si deve sempre
considerare la possibilità di errore del test (falso positivo/negativo).
\begin{center}
	\begin{tabular}{ c c | c c }
		                                      &   & \multicolumn{2}{ c }{ Caratteristica }                  \\
		                                      &   & F                                      & T              \\
		\hline
		\multirow{2}{2.5 cm}{ Vulnerabilità } & F & vero negativo                          & falso positivo \\
		                                      & T & falso negativo                         & vero postivo
	\end{tabular}
\end{center}
In ambito informatico si compie un procedimento analogo per riuscire a capire quali moduli producono un certo servizio
in base alla risposta che il nodo su cui facciamo il test ci fornisce.

\section{Sistema di misurazione delle vulnerabilità}
Si tratta di uno strumento per \emph{misurare} la pericolosità delle vulnerabilità assegnando loro un punteggio da 0 a
10: più alto è il punteggio e più pericolosa è quella vulnerabilità.

Come già detto, valutare una vulnerabilità indipendentemente dal sistema non ha alcun senso poiché potrebbe essere
critica per alcuni sistemi ma assolutamente innocua per altri.

Il punteggio viene assegnato in base a tre parametri:
\begin{itemize}
	\item \textbf{Base}: caratteristica intrinseca e fondamentale indipendente dal tempo e dall'ambiente.
	\item \textbf{Temporale}: caratteristica che cambia nel tempo ma è indipendente dall'ambiente.
	\item \textbf{Ambientale}: caratteristica specifica e solo per un particolare ambiente.
\end{itemize}

\subsection{Vettore di accesso}
Il \textbf{vettore di accesso} indica il metodo di accesso necessario per sfruttare la vulnerabilità. Può essere
\begin{itemize}
	\item \textbf{Fisico}: è necessario avere accesso all'hardware su cui è implementato il sistema.
	\item \textbf{Locale}: è necessario avere accesso ad un account del sistema.
	\item \textbf{Adicaente}: è necessario avere accesso ad una sottorete.
	\item \textbf{Network}: è possibile arrivare al sistema da un qualsiasi nodo della rete.
\end{itemize}
Quando il vettore di accesso è fisico la sicurezza è massima mentre un vettore di accesso di tipo network crea più
problemi.

\subsection{Complessità dell'attacco}
Indica quanto è \emph{difficile} effettuare l'attacco: più l'attacco è complesso più il nostro sistema è sicuro, al
contrario, un attacco semplice da eseguire è sintomo di un sistema vulnerabile.

\subsection{Privilegi richiesti e interazione utente}
Indica quali diritti sono necessari per eseguire l'attacco
\begin{itemize}
	\item \textbf{Nessuno}: l'attacco si può eseguire senza alcun diritto sul sistema.
	\item \textbf{Basso}: in genere si parla di avere accesso a privilegi base che un qualsiasi utente del sistema
	      possiede.
	\item \textbf{Alto}: quando è necessario avere diritti su componenti critiche del sistema, ad esempio essere
	      amministratore.
\end{itemize}
e se è necessaria l'interazione dell'utente come nel caso del \emph{phishing}.

\subsection{Livello dell'exploit}
Questa parametro indica se esiste del codice per sfruttare una vulnerabilità e se questo è affidabile o meno.

\section{Numero di vulnerabilità}
Il numero di vulnerabilità \emph{note} di un software non è indice della qualità di quest'ultimo. In generale possiamo
dire che tale numero è direttamente proporzionale al numero \emph{totale} di vulnerabilità del sistema e al numero di
utenti che ricercano vulnerabilità in quel sistema.

Il numero di ricercatori è proporzionale al valore di una certa vulnerabilità in quanto una vulnerabilità di un
prodotto poco usato non interessa a nessuno. Il fatto che ci siano tanti ricercartori è sintomo di popolarità di un
determinato software.

\subsection{Scoperta di vulnerabilità}
Spesso, per scoprire una vulnerabilità, è necessario uno strumento di \emph{reverse engineering} per passare da
eseguibile ad un versione di controllo.

Una volta effettuata la conversione si può procedere con un'analisi manuale o tramite strumenti automatizzati di
\begin{itemize}
	\item \textbf{Analisi statica} (\textbf{SAST}): sono strumenti che dipendono dal linguaggio di programmazione
	      utilizzato e dal contesto in cui si trova il sistema.
	\item \textbf{Analisi dinamica} (\textbf{Fuzzing}): basato sulla generazione e trasmissione di valori errati come
	      input del programma.
\end{itemize}

\subsubsection{SAST}
Tra i punti di forza del SAST abbiamo
\begin{itemize}
	\item La scalabilità poiché permette di analizzare grandi quantità di codice a basso costo.
	\item L'identificazione di un sottoinsieme di classi di vulnerabilità quali buffer e stack overflow o SQL
	      injection.
	\item I risultati aiutano gli sviluppatori ad individuare le istruzioni coinvolte.
\end{itemize}
Le debolezze di questo sistema sono invece
\begin{itemize}
	\item La difficoltà nell'implementazione di analisi per la ricerca di alcune vulnerabilità come
	      \begin{itemize}
		      \item Problemi di autenticazione
		      \item Gestione errata dei diritti
		      \item Uso non sicuro della crittografia
	      \end{itemize}
	\item Potenza limitata in quanto possono individuare basse percentuali di vulnerabilità.
	\item Elevato numero di false positive.
	\item Incapace di identificare problemi di sicurezza dovuti alla configurazione che non possono essere scoperti
	      dal codice.
	\item Spesso non analizzano librerie.
\end{itemize}

\subsubsection{Fuzzing}
L'idea di base del \textbf{fuzzing} è inviare al software degli input errati o malformati cercando di causare il blocco
dell'applicazione o un suo malfunzionamento, evidenziando così una vulnerabilità non ancora nota.

Per riuscire a fare del fuzzing in maniera efficiente è necessario automatizzare il processo vista la grande quantità
di dati da fornire in input.

\section{Fuzzing}
Entriamo ora nello specifico di questa tecnica largamente utilizzata per la ricerca di vulnerabilità all'interno di un
sistema. In generale possiamo dire che gli strumenti di fuzzing sono il risultato di tre moduli:
\begin{itemize}
	\item Il \textbf{generatore} di input malformati.
	\item Il \textbf{distributore dei dati} verso il modulo target.
	\item Un \textbf{componente di monitoraggio} per gestire eventuali errori e scoprire il percorso del dato in input
	      nel modulo.
\end{itemize}
Abbiamo tre tipologie principali di fuzzing:
\begin{itemize}
	\item \textbf{Application}: testa funzioni come pulsanti e campi di immissione dei programmi grafici o le opzioni
	      dei programmi della riga di comando.
	\item \textbf{Protocol}: verifica il comportamento del programma in caso di invio di dati di formato errato.
	\item \textbf{File format}: genera file errati o di formato non atteso e li manda in input al software.
\end{itemize}

\subsection{Scanning}
Torniamo indietro alla fase di scanning, eseguita sia dall'attaccante che dal difensore. Ovviamente il difensore è
interessato a sapere se qualcuno sta cercando vulnerabilità nel suo sistema, ecco perché ogni scanner permette di
decidere la frequenza con cui si inviano i messaggi ad un certo nodo ed il numero di nodi analizzati in parallelo.

La probabilità di un attaccante di essere scoperto è tanto minore quanto minore è la frequenza di messaggi inviati e
il numero di nodi analizzati.

\subsubsection{Client scanning}
Ci sono anche casi di \textbf{client scanning} in cui è un server che esegue uno scanning su un client che prova a
connettersi per capire come esso sia configurato, quali sono eventuali vulnerabilità ecc.

Questa pratica è considerata molto ambigua in quanto, a volte, permette di entrare in possesso di informazioni
personali dell'utente andando a comportare importanti problemi di privacy.

\subsection{Classi di fuzzer}
Un \textbf{fuzzer} può essere classificato in diversi modi:
\begin{itemize}
	\item Gli input possono essere generati da zero o modificando quelli esistenti:
	      \begin{itemize}
		      \item \textbf{Generation based}
		      \item \textbf{Mutation based}
	      \end{itemize}
	\item Può conoscere la struttura dell'input in modo da modificare punti specifici o violando in uno o più punti
	      la struttura:
	      \begin{itemize}
		      \item \textbf{Grammar based}
		      \item \textbf{Not grammar based}
	      \end{itemize}
	\item Può conoscere a diversi livelli la struttura del programma ed utilizzarla per garantire di aver testato tutti
	      i cammini:
	      \begin{itemize}
		      \item \textbf{White box}
		      \item \textbf{Black box}
		      \item \textbf{Gray box}
	      \end{itemize}
	\item Si può applicare al codice o ad un parser:
	      \begin{itemize}
		      \item \textbf{Text fuzzing}
	      \end{itemize}
\end{itemize}

\subsection{Tainting analysis}
La \textbf{tainting analysis} è un'analisi statica preliminare che va ad associare ad un certo input le istruzioni che
andranno a lavorare su quello specifico input.

Questa pratica permette scoprire quali istruzioni ricevono un input e quindi di calcolare l'insieme di variabili di
un possibile buffer overflow. In genere restituisce un insieme più grande di quello effettivo (falsi positivi).

\subsection{Regole euristiche}
Per effettuare un buon fuzzing si dovrebbe sempre tenere a mente queste regole:
\begin{itemize}
	\item La conoscenza del formato dell'input è molto utile.
	\item Fuzzer generazionali sono migliori di fuzzer casuali.
	\item Usare diversi tipi di fuzzer permette di trovare diversi tipi di vulnerabilità.
	\item Più è ampio il tempo di esecuzione di un fuzzer maggiore è la probabilità di trovare vulnerabilità.
	\item Il processo deve essere guidato.
	\item Capire quando il fuzzer non riesce a sollecitare istruzioni non ancora utilizzate.
\end{itemize}

\section{Vulnerabilità strutturali}
Per scoprire questo tipo di vulnerabilità non esistono strumenti in grado di considerare un insieme di moduli
cooperanti. Esistono fuzzer in grado di monitorare moduli esterni ma non in grado di operare su più moduli
contemporaneamente. In generale, le cause di tali vulnerabilità possono essere molteplici:
\begin{itemize}
	\item \textbf{Input non controllati}: il fallimento nel ricevente di un valore indica la mancanza di controlli su
	      valori ricevuti ma anche la mancanza di controlli dalla parte di chi invia il valore.
	\item \textbf{Mancata autenticazione}: il sistema non compie un'autenticazione dei canali in ingresso, ossia non
	      si preoccupa di chi può inviare messaggi su un certo canale o porta.
	\item \textbf{Delega dei controlli}: in alcuni casi si delegano i controlli ad un altro modulo permettendo così
	      l'invio di dati manipolati nel momento in cui manca un sistema di autenticazione.
	\item \textbf{Mancata valutazione dei messaggi}: non si controlla se i messaggi sono: ripetuti, mancanti o inviati
	      in ordine diverso da quello atteso.
	\item \textbf{Scambio di informazioni non protette}: uno scambio non protetto può permettere accesso ad informazioni
	      interne ad un modulo e ad una catena di attacchi che sfruttano l'informazione acquisita.
\end{itemize}
