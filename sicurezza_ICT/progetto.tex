\chapter{Verifica del progetto di sistema}
In questo capitolo andiamo a vedere come un sistema dovrebbe essere progettato per limitare al minimo le
vulnerabilità.

\section{Principi di Saltzer \& Schroeder}
Nel 1975, Saltzer \& Schroeder hanno definito dei principi che permettono di scoprire vulnerabilità non legati ad
errori di programmazione:
\begin{enumerate}
	\item \textbf{Economia dei meccanismi}: l'implementazione dei sistemi di
	      sicurezza devono essere semplici e compatti.
	\item \textbf{Fail-safe Default}: i meccanismi di protezione dovrebbero
	      vietare l'esecuzione di qualsiasia azione in assenza di diritti
	\item \textbf{Mediazione completa}: il meccanismo di protezione dovrebbe
	      controllare l'accesso ogni volta ad ogni oggetto.
	\item \textbf{Open Design}: il sistema deve rimanere sicuro finché
	      l'attaccante non scopre la chiave di cifratura.
	\item \textbf{Privilegio di separazione}: il meccanismo di protezione
	      dovrebbe permettere l'accesso tramite più di un pezzo di
	      informazione.
	\item \textbf{Privilegio minimo}: ogni processo dovrebbe essere eseguito
	      con il numero minimo di diritti.
	\item \textbf{Meccanismo comune}: si deve ridurre al minimo la
	      condivisione di informazioni tra utenti. Ogni canale di
	      condivisione può essere causa di vulnerabilità.
	\item \textbf{Psychological Acceptability}: Il meccanismo di protezione dovrebbe essere facile da utilizzare
	      per l'utente finale.
\end{enumerate}
Ogni sistema deve stabilire un compromesso tra verifica di questi principi e prestazioni, la regola fondamentale è
questa: possono mancare solo controlli che peggiorano delle prestazioni di interesse, un controllo che non peggiora
le prestazioni deve essere presente.

\section{Fallimento sicuro}
Il sistema deve essere strutturato in modo che un certo soggetto debba perforza avere un \emph{ragione} per compiere
una determinata azione su un certo oggetto.

Se non c'è ragione per un soggetto di compiere una determinata azione non deve poterla compiere. Se non si è in grado
di spiegare perché un soggetto possa compiere un'operazione si sta violando il principio del privilegio minimo.

\section{Mediazione completa}
Recentemente sono stati aggiunti dei supporti hardware più veloci ed efficienti delle architetture a capability.

La memoria virtuale viene partizionata in zone ed ogni zona viene dedicata ad un certo tipo di dato e nessuna zona
di memoria può essere utilizzata da tipi di dato diversi.

Alcuni bit di ogni puntatore di memoria (\textbf{tag}) sono dedicati ad informazioni sul tipo di dato puntato e ad
ogni accesso si controlla che il tipo del puntatore sia coerente con quello dell'area di memoria puntata.