\chapter{Rotazioni in 3D}\label{3D}
Tutto ci\`o che \`e stato detto fin ad ora per le trasformazioni nello spazio bidimensionale si pu\`o estendere allo
spazio tridimensionale. Per farlo ci baster\`a aggiungere una riga e una colonna alla matrice di trasformazione per lo
spazio bidimensionale.

Mentre per traslazione e scalatura le operazioni sono analoghe allo spazio bidimensionale, per la rotazione possiamo
introdurre qualche strumento in pi\`u.

\section{Rotazione intorno agli assi standard}
Nello spazio tridimensionale \`e necessario determinare intorno a quale dei tre assi ruotare. Quel che abbiamo fatto
fino ad ora nello spazio bidimensionale era ruotare la nostra geometria intorno all'asse $z$.

D'ora in poi le matrici di rotazione cambieranno in questo modo:
\begin{itemize}
	\item Rotazione intorno all'asse $x$
	      \[
		      R_x(\theta) = \begin{bmatrix}
			      1 & 0            & 0             & 0 \\
			      0 & \cos(\theta) & -\sin(\theta) & 0 \\
			      0 & \sin(\theta) & \cos(\theta)  & 0 \\
			      0 & 0            & 0             & 1
		      \end{bmatrix}
	      \]
	\item Rotazione intorno all'asse $y$
	      \[
		      R_x(\theta) = \begin{bmatrix}
			      \cos(\theta) & 0 & -\sin(\theta) & 0 \\
			      0            & 1 & 0             & 0 \\
			      \sin(\theta) & 0 & \cos(\theta)  & 0 \\
			      0            & 0 & 0             & 1
		      \end{bmatrix}
	      \]
	\item Rotazione intorno all'asse $z$
	      \[
		      R_x(\theta) = \begin{bmatrix}
			      \cos(\theta) & -\sin(\theta) & 0 & 0 \\
			      \sin(\theta) & \cos(\theta)  & 0 & 0 \\
			      0            & 0             & 1 & 0 \\
			      0            & 0             & 0 & 1
		      \end{bmatrix}
	      \]
\end{itemize}
In questo caso si fa riferimento ad una generica matrice di trasformazione alla quale viene applicata solo la rotazione.
Infatti si ha una matrice che comprende anche un ipotetico vettore di traslazione.

\section{Rotazione intorno ad un asse generico}
Per ottenere una rotazione attorno ad un asse generico abbiamo vari metodi a disposizione. Di seguito due dei pi\`u
elementari:
\begin{itemize}
	\item Il primo procedimento \`e analogo a quello nello spazio bidimensionale.
	      \begin{enumerate}
		      \item Considero un frame tale per cui l'asse $z$ \`e l'asse attorno al quel voglio far ruotare la mia
		            figura.
		      \item Applico una trasformazione che porti tale frame a combaciare col frame canonico.
		      \item Effettuo la rotazione intorno all'asse $z$ del frame canonico.
		      \item Riporto il frame nella sua posizione iniziale.
	      \end{enumerate}
	\item Il secondo procedimento usa la formula di Rodriguez per ottenere il punto $p'$
	      \[ p' = \cos(\theta) p + (1 - \cos(\theta))(p \cdot r) r + \sin(\theta)(r \times p) \]
\end{itemize}

\section{Quaternioni}
Un altro strumento che ci permette ci effettuare rotazioni in 3D \`e il \textbf{quaternione}. Il quaternione \`e
un'estensione del numero complesso che abbiamo in 2D.

Se ci pensiamo, un numero complesso \`e esprimibile in coordinate polari e come abbiamo gi\`a visto le coordinate polari
identificano un punto sul piano bidimensionale.

Ci\`o che abbiamo visto per le rotazioni in 2D si pu\`o perfettamente ricondurre ad operazioni con numeri complessi.

Per estendere il discorso allo spazio tridimensionale prima dobbiamo chiarire cosa sia un quaternione. Si tratta di un
numero complesso composto da una parte reale e tre parti immaginarie. La rappresentazione \`e analoga ai normali numeri
complessi.
\[ q = w + ix + jy + kz \]
dove $i, j, k$ sono unit\`a immaginarie come lo era $i$ per i numeri complessi in 2D. Godono dunque della stessa
propriet\`a:
\[ i^2 = j^2 = k^2 = -1 \]
ma vale anche
\[ i^2 = j^2 = k^2 = ijk \]
I quaternioni possono anche essere scritti in questo modo:
\[ q = (w, x, y, z) = (s, v) \]
dove $s$ \`e un numero reale e $v$ \`e un vettore in $\mathbb{R}^3$.

Le parti immaginarie dei quaternioni inoltre, se moltiplicate fra loro, danno questi risultati:
\[
	\begin{array}{lr}
		ij = k, & ji = -k \\
		jk = i, & kj = -i \\
		ki = j, & ik = -j
	\end{array}
\]
\subsection{Operazioni con i quaternioni}
Definiamo ora le principali operazioni applicabili ai quaternioni. Useremo la notazione
\[ q = (s, v) \]
per indicare un quaternione.
\begin{itemize}
	\item \textbf{Somma}: $q_1 + q_2 = (s_1 + s_2, v_1 + v_2)$
	\item \textbf{Prodotto}: $q_1 q_2 = (s_1 s_2 - v_1 v_2, s_1 v_2 + s_2 v_1 + v_1 \times v_2)$
	\item \textbf{Coniugio}: $\overline{q} = (s, -v)$
	\item \textbf{Norma}: $\| q \| = \sqrt{\overline{q} q} = \sqrt{s^2 + v^2}$
	\item \textbf{Iverso}: $q^{-1} = \displaystyle\frac{\overline{q}}{|q|^2}$
\end{itemize}

\subsection{Rotazione con i quaternioni}
Ora abbiamo tutto ci\`o che serve per definire l'operazione di rotazione usando i quaternioni.

Dato un angolo $\theta$ e un vettore $v$, il quaternione
\[
	q = \begin{pmatrix} \cos{\displaystyle\frac{\theta}{2}},
		 & \sin{\displaystyle\frac{\theta}{2} \cdot v}\end{pmatrix}
\]
rappresenta una rotazione intorno al vettore $v$ di un angolo $\theta$.

Per riuscire ad applicare tale rotazione dobbiamo introdurre i cosiddetti \textbf{quaternioni puri}. Un quaternione \`e
detto puro se la parte reale \`e uguale a 0.
\[ p = (0, v) \]
Se ora interpretiamo $p$ come un punto (che \`e l'origine degli assi traslata di un vettore $v$) e $q$ come il quaternione
unitario
\[
	q = \begin{pmatrix}
		\cos{\displaystyle\frac{\theta}{2}},
		 & \sin{\displaystyle\frac{\theta}{2} \cdot v}
	\end{pmatrix}
\]
allora
\[
	p' = q p \overline{q} =
	\begin{pmatrix}
		\cos{\displaystyle\frac{\theta}{2}}, & \sin{\displaystyle\frac{\theta}{2} \cdot v}
	\end{pmatrix}
	\begin{pmatrix}
		0, & v
	\end{pmatrix}
	\begin{pmatrix}
		\cos{\displaystyle\frac{\theta}{2}}, & -\sin{\displaystyle\frac{\theta}{2} \cdot v}
	\end{pmatrix}
\]
dove $p'$ equivale a $p$ ruotato attorno a $v$ di un angolo $\theta$.

\subsection{Conversioni}
A partire dal quaternione $q = (s, v)$ di rotazione posso anche ricavare l'angolo di rotazione e il raggio in questo modo:
\begin{gather*}
	\theta = 2 \arctan 2 (s, \| v \|) \\
	r = \frac{v}{\| v \|}
\end{gather*}
oppure posso scrivere la relativa matrice di rotazione
\[
	R_q = \begin{bmatrix}
		1 - 2 (q_y^2 + q_z^2) & 2 (q_x q_y - q_z q_w) & 2 (q_x q_z + q_w q_y) \\
		2 (q_x q_y + q_w q_z) & 1 - 2 (q_x^2 + q_z^2) & 2 (q_y q_z - q_w q_x) \\
		2 (q_x q_z - q_w q_y) & 2 (q_y q_z + q_w q_x) & 1 - 2 (q_x^2 + q_y^2)
	\end{bmatrix}
\]

\subsection{Propriet\`a per la rotazione}
I quaternioni hanno una propriet\`a molto utile che li rende tra i candidati migliori per effettuare rotazioni complesse.

Il prodotto di due quaternioni unitari \`e a sua volta un quaternione unitario. Dunque possiamo
concatenare un qualsiasi numero di rotazioni e il risultato sar\`a ancora una rotazione.

Per esempio potrei concatenare i quaternioni in questo modo
\[ q = q_1 \cdot q_2 \cdot q_3 \cdot q_4 \]
e la rotazione
\[
	q_4 \cdot q_3 \cdot q_2 \cdot q_1 \cdot p \cdot
	\overline{q_1} \cdot \overline{q_2} \cdot \overline{q_3} \cdot \overline{q_4}
\]
equivale alla rotazione
\[ q \cdot p \cdot \overline{q} \]

In generale possiamo preferire i quaternioni per effettuare rotazioni perch\'e
\begin{itemize}
	\item Sono pi\`u veloci da moltiplicare rispetto alle matrici.
	\item Hanno una notazione pi\`u compatta per comporre pi\`u rotazioni.
	\item Sono numericamente pi\`u stabili per piccole variazioni.
	\item Possono essere convertiti in matrice e angoli di rotazione.
	\item Sono ottimi per l'interpolazione tra rotazioni.
\end{itemize}

\section{Interpolare tra pi\`u rotazioni}
Una rotazione \`e una trasformazione affine ed \`e anche vista come un frame ortonormale centrato nell'origine.
Interpolare tra pi\`u rotazioni significa interpolare tra pi\`u frame.

L'interpolazione lineare non funziona per le matrici. Questo perch\'e il risultato dell'interpolazione tra due rotazioni
non \`e una matrice di rotazione. Possiamo normalizzare il risultato ma questo funziona solo per piccole rotazioni.

Esiste tuttavia un metodo per compiere un'interpolazione lineare ma la rotazione non sar\`a costante.
L'angolo di rotazione si pu\`o ottenere con la seguente formula:
\[ \theta_t = \theta_0 (1 - t) + \theta_1 t \]
mentre l'asse di rotazione pu\`o essere ricavato in questo modo:
\[ r_t = \frac{r_0 (1 - t) + r_1 t}{\| r_0 (1 - t) + r_1 t \|} \]
Purtroppo utilizzando questo metodo otterremo che la rotazione sar\`a pi\`u lenta vicino ai due limiti e pi\`u veloce
al centro. A meno che non sia un effetto voluto \`e un qualcosa che vogliamo evitare.

\subsection{Interpolazione Lineare Sferica}
Questo metodo di interpolazione funziona in tutte le dimensioni e garantisce una rotazione uniforme.

Sia
\[ R = \{ r_0, r_1, 0 \} \]
un frame e sia $r(t)$ un vettore espresso tramite coordinate relative a $R$ come
\[ r(t) = \frac{sin{(t \psi)}}{\sin{(\psi)}} r_0 + \frac{\sin{((1 - t) \psi)}}{\sin{(\psi)}} r_1 \]
dove
\[\psi = \arccos{(q_0 q_1)} \]

Con questo metodo posso anche interpolare direttamente tra quaternioni.
\[ q(t) = \frac{sin{(t \psi)}}{\sin{(\psi)}} q_0 + \frac{\sin{((1 - t) \psi)}}{\sin{(\psi)}} q_1 \]