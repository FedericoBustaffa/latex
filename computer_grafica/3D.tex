\chapter{La terza dimensione}\label{3D}
Tutto ci\`o che \`e stato detto fin ad ora per le trasformazioni nello spazio bidimensionale si pu\`o estendere allo
spazio tridimensionale. Per farlo ci baster\`a aggiungere una riga e una colonna alla matrice di trasformazione.

\section{Rotazione}
Nello spazio tridimensionale \`e necessario determinare intorno a quale dei tre assi ruotare. Quel che abbiamo fatto
fino ad ora nello spazio bidimensionale era ruotare la nostra geometria intorno all'asse $z$.

D'ora in poi le matrici di rotazione cambieranno in questo modo:
\begin{itemize}
	\item Rotazione intorno all'asse $x$
	      \[
		      R_x(\theta) = \begin{bmatrix}
			      1 & 0            & 0             & 0 \\
			      0 & \cos(\theta) & -\sin(\theta) & 0 \\
			      0 & \sin(\theta) & \cos(\theta)  & 0 \\
			      0 & 0            & 0             & 1
		      \end{bmatrix}
	      \]
	\item Rotazione intorno all'asse $y$
	      \[
		      R_x(\theta) = \begin{bmatrix}
			      \cos(\theta) & 0 & -\sin(\theta) & 0 \\
			      0            & 1 & 0             & 0 \\
			      \sin(\theta) & 0 & \cos(\theta)  & 0 \\
			      0            & 0 & 0             & 1
		      \end{bmatrix}
	      \]
	\item Rotazione intorno all'asse $z$
	      \[
		      R_x(\theta) = \begin{bmatrix}
			      \cos(\theta) & -\sin(\theta) & 0 & 0 \\
			      \sin(\theta) & \cos(\theta)  & 0 & 0 \\
			      0            & 0             & 1 & 0 \\
			      0            & 0             & 0 & 1
		      \end{bmatrix}
	      \]
\end{itemize}
In questo caso si fa riferimento ad una generica matrice di trasformazione alla quale viene applicata solo la rotazione.
Infatti si ha una matrice che comprende anche un ipotetico vettore di traslazione.

\subsection{Rotazione intorno ad un asse generico}
Per ottenere una rotazione attorno ad un asse generico abbiamo vari metodi a disposizione. Di seguito due dei pi\`u
elementari:
\begin{itemize}
	\item Il primo procedimento \`e analogo a quello nello spazio bidimensionale.
	      \begin{enumerate}
		      \item Considero un frame tale per cui l'asse $z$ \`e l'asse attorno al quel voglio far ruotare la mia
		            figura.
		      \item Applico una trasformazione che porti tale frame a combaciare col frame canonico.
		      \item Effettuo la rotazione intorno all'asse $z$ del frame canonico.
		      \item Riporto il frame nella sua posizione iniziale.
	      \end{enumerate}
	\item Il secondo procedimento usa la formula di Rodriguez per ottenere il punto $p'$
	      \[ p' = \cos(\theta) p + (1 - \cos(\theta))(p \cdot r) r + \sin(\theta)(r \times p) \]
\end{itemize}

\subsection{Quaternioni}
Un altro strumento che ci permette ci effettuare rotazioni in 3D \`e il \textbf{quaternione}. Il quaternione \`e
un'estensione del numero complesso che abbiamo in 2D.

Se ci pensiamo, un numero complesso \`e esprimibile in coordinate polari e come abbiamo gi\`a visto le coordinate polari
identificano un punto sul piano bidimensionale.

Ci\`o che abbiamo visto per le rotazioni in 2D si pu\`o perfettamente ricondurre ad operazioni con numeri complessi.

Per estendere il discorso allo spazio tridimensionale prima dobbiamo chiarire cosa sia un quaternione. Si tratta di un
numero complesso composto da una parte reale e tre parti immaginarie. La rappresentazione \`e analoga ai normali numeri
complessi.
\[ q = w + ix + jy + kz \]
dove $i, j, k$ sono unit\`a immaginarie come lo era $i$ per i numeri complessi in 2D. Godono dunque della stessa
propriet\`a:
\[ i^2 = j^2 = k^2 = -1 \]
ma vale anche
\[ i^2 = j^2 = k^2 = ijk \]
I quaternioni possono anche essere scritti in questo modo:
\[ q = (w, x, y, z) = (s, v) \]
dove $s$ \`e un numero reale e $v$ \`e un vettore in $\mathbb{R}^3$.

Le parti immaginarie dei quaternioni inoltre, se moltiplicate fra loro, danno questi risultati:
\[
	\begin{array}{lr}
		ij = k, & ji = -k \\
		jk = i, & kj = -i \\
		ki = j, & ik = -j
	\end{array}
\]
\subsubsection{Operazioni con i quaternioni}
Definiamo ora le principali operazioni applicabili ai quaternioni. Useremo la notazione
\[ q = (s, v) \]
per indicare un quaternione.
\begin{itemize}
	\item \textbf{Somma}: $q_1 + q_2 = (s_1 + s_2, v_1 + v_2)$
	\item \textbf{Prodotto}: $q_1 q_2 = (s_1 s_2 - v_1 v_2, s_1 v_2 + s_2 v_1 + v_1 \times v_2)$
	\item \textbf{Coniugio}: $\overline{q} = (s, -v)$
	\item \textbf{Norma}: $\| q \| = \sqrt{\overline{q} q} = \sqrt{s^2 + v^2}$
	\item \textbf{Iverso}: $q^{-1} = \displaystyle\frac{\overline{q}}{|q|^2}$
\end{itemize}

\subsubsection{Rotazione con i quaternioni}
Ora abbiamo tutto ci\`o che serve per definire l'operazione di rotazione usando i quaternioni.

Dato un angolo $\theta$ e un vettore $v$, il quaternione
\[
	q = \begin{pmatrix} \cos{ \left( \displaystyle\frac{\theta}{2} \right) },
		 & \sin{ \left( \displaystyle\frac{\theta}{2} \right) } v\end{pmatrix}
\]
rappresenta una rotazione intorno al vettore $v$ di un angolo $\theta$.

Ma come si applica questa rotazione ad un punto nello spazio tridimensionale ? Per capirlo dobbiamo usare i cosiddetti
\textbf{quaternioni puri}. Un quaternione \`e puro se ha la parte reale uguale a 0.
\[ p = (0, v) \]
Se ora interpretiamo $p$ come un punto (che \`e l'origine degli assi traslata di $v$) e sia $q$ il quaternione
\[
	q = \begin{pmatrix}
		\cos{\left( \displaystyle\frac{\theta}{2} \right) },
		 & \sin{ \left( \displaystyle\frac{\theta}{2} \right) v}
	\end{pmatrix}
\]