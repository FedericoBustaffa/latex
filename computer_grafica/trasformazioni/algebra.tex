\chapter{Trasformazioni}
\section{Algebra vettoriale}
Per padroneggiare al meglio le trasformazioni \`e necessaria una conoscenza di base
dell'algebra lineare, in particolare dobbiamo saper maneggiare \textbf{vettori} e
\textbf{matrici} agilmente.

\subsection{Operazioni su punti e vettori}
Definiamo qualche operazione utile per queste due entit\`a:
\begin{itemize}
	\item punto $+$ vettore ritorna un punto
	\item punto $-$ punto ritorna un vettore
	\item vettore $+$ vettore ritorna un vettore
	\item scalare $\cdot$ vettore ritorna un vettore
\end{itemize}

\subsection{Prodotto scalare tra due vettori}
Il prodotto scalare tra due vettori $a$ e $b$ scritto come $a \cdot b$ equivale a:
\[ a \cdot b = \sum_{i=1}^n a_1 b_1 + \dots + a_n b_n  \]
Questo significa che
\[ a \cdot b = \| a \| \| b \| \cos{(\theta)} \]
da qui posso ricavarmi l'angolo in questo modo:
\[ \theta = \arccos{\left( \frac{a \cdot b}{\| a \| \| b \|} \right)} \]
Altre propriet\`a:
\begin{itemize}
	\item Il prodotto scalare tra due vettori \`e 0 quando i due vettori sono \emph{ortogonali}
	      fra loro.
	\item Il prodotto scalare \`e massimo per vettori paralleli (stesso verso) ed \`e minimo
	      per vettori anti-paralleli (verso opposto).
\end{itemize}

\subsection{Prodotto vettoriale tra due vettori}
Il prodotto tra due vettori $a$ e $b$ scritto come $a \times b$ equivale a:
\[
	a \times b = \begin{bmatrix}
		a_y b_z - b_y a_z \\ b_x a_z - a_x b_z \\ a_x b_y - b_x a_y
	\end{bmatrix}
\]
Per ricordarsi il calcolo si pu\`o usare questo metodo: si calcola il determinante della
matrice
\[
	det \begin{bmatrix}
		i   & j   & k   \\
		a_x & a_y & a_z \\
		b_x & b_y & b_z
	\end{bmatrix} =
	i(a_y b_z - b_y a_z) - j(a_x b_z - b_x a_z) + k(a_x b_y - b_x a_y)
\]
Questo prodotto ci fornisce un vettore ortogonale sia ad $a$ che a $b$.
\[ (a \times b) \cdot a = (a \times b ) \cdot b = 0 \]
Non vale la propriet\`a commutativa ma vale la distributiva per la somma.
\[ a \times (b + c) = a \times b + a \times c \]
La norma equivale a
\[ \| a \times b \| = \| a \| \| b \| \sin(\theta) \]

\subsection{Coordinate polari}
Le \textbf{coordinate polari} servono a esprimere punti e vettori tramite la tupla
$(\theta, \rho)$ dove
\begin{itemize}
	\item $\theta$ \`e l'angolo formato con l'asse $x$.
	\item $\rho$ \`e la distanza dall'origine.
\end{itemize}
Se vogliamo passare da coordinate polari a cartesiane
\begin{gather*}
	x = \rho \cdot \cos(\theta) \\
	y = \rho \cdot \sin(\theta)
\end{gather*}
Viceversa se vogliamo passare da coordinate cartesiano a polari
\begin{gather*}
	\theta = \arctan(2 \cdot (y, x)) \\
	\rho = \sqrt{x^2 + y^2}
\end{gather*}

\subsection{Prodotto riga per colonna tra matrici}
Siano $A_{n \times m}, B_{m \times r}$ due matrici, il prodotto riga per colonna
sar\`a una matrice $C$ di formato $n \times r$. Per calcolare l'elemento $c_{ij}$
\[ c_{ij} = \sum_{k = 1}^m a_{i,k} \cdot b_{k,j} \]

Nel caso in cui si voglia moltiplicare una matrice $M$ di formato $p \times q$ e un
vettore $v$ di lunghezza $q$ si pu\`o vedere il prodotto come una combinazione lineare
delle colonne della matrice per le singole componenti del vettore.
\begin{gather*}
	M \cdot v =
	\begin{bmatrix}
		m_{11} & \dots & m_{1p} \\
		\dots  & \dots & \dots  \\
		m_{q1} & \dots & m_{qp}
	\end{bmatrix} \cdot
	\begin{bmatrix}
		v_1 \\ \dots \\ v_q
	\end{bmatrix} \\
	= v_1 \begin{bmatrix}
		m_{11} \\ \dots \\ m_{q1}
	\end{bmatrix} \cdot
	... \cdot
	v_q \begin{bmatrix}
		m_{1p} \\ \dots \\ m_{qp}
	\end{bmatrix}
\end{gather*}
\textbf{NOTA}: Il prodotto riga per colonna \textbf{non} gode della propriet\`a
commutativa, dunque la propriet\`a appena vista vale solo se la moltiplicazione viene
svolta con quell'ordine e il vettore $v$ \`e un vettore colonna.

Vedremo pi\`u avanti che la non commutativit\`a di questa operazione sar\`a fondamentale
per definire diversi tipi di trasformazione.