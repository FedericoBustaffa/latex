\section{Trasformazioni geometriche}
Una \textbf{trasformazione geometrica} \`e una funzione che mappa punti in punti e vettori
in vettori. Quando parliamo vogliamo \emph{trasformare} un oggetto intendiamo l'applicazione
della stessa trasformazione a tutti i punti dell'oggetto.

Per semplicit\`a faremo riferimento a punti nel piano bidimensionale
\[ p = \begin{bmatrix} p_x \\ p_y \end{bmatrix} \]

\subsection{Traslazione}
Se vogliamo traslare un punto $p$ abbiamo bisogno di un vettore $v$ che ci indichi di quando
traslare le sue coordinate.
\[
	T_v(p) = p + v =
	\begin{bmatrix}
		p_x \\ p_y
	\end{bmatrix} +
	\begin{bmatrix}
		v_x \\ v_y
	\end{bmatrix} =
	\begin{bmatrix}
		p_x + v_x \\ p_y + v_y
	\end{bmatrix}
\]
La traslazione \`e indipendente dalle coordinate del punto.

\subsection{Scalatura}
Utile per cambiare la dimensione di un oggetto. Per ottenere questa trasformazione si
moltiplicano tutti i punti della figura per un fattore di scalatura.
\[
	S_{s_x, s_y}(p) =
	\begin{bmatrix}
		s_x \cdot p_x \\
		s_y \cdot p_y
	\end{bmatrix}
\]
Se $s_x = s_y$ allora la scalatura \`e \textbf{uniforme o isotropica} (le proporzioni
dell'oggetto non cambiano). Altrimenti si dice che la scalatura \`e
\textbf{non uniforme o anisotropica}.

\subsection{Rotazione}
Utile per ruotare di un angolo $\theta$ un punto (solo rispetto all'origine ?). Per
effettuare una rotazione abbiamo bisogno di convertire le coordinate cartesiane del
nostro punto in coordinate polari.
\[
	p = \begin{bmatrix}
		p_x \\ p_y
	\end{bmatrix} =
	\begin{bmatrix}
		\rho \cos(\phi) \\
		\rho \sin(\phi)
	\end{bmatrix}\]
Se volessi quindi, ruotare il punto $p$ di un angolo $\theta$ rispetto a dove si trova
dovrei svolgere un'operazione di questo tipo:
\[
	R_\theta(p) =
	\begin{bmatrix}
		\rho \cos(\phi + \theta) \\
		\rho \sin(\phi + \theta)
	\end{bmatrix}
\]
Per semplificare i calcoli possiamo ricordarci che
\begin{gather*}
	\cos(\phi + \theta) = \cos(\phi) \cos(\theta) - \sin(\phi) \sin(\theta) \\
	\sin(\phi + \theta) = \sin(\theta) \cos(\phi) + \sin(\phi) \cos(\theta)
\end{gather*}
Posso quindi riscrivere la funzione di rotazione come
\begin{gather*}
	R_\theta(p) =
	\begin{bmatrix}
		\rho [ \cos(\phi) \cos(\theta) - \sin(\phi) \sin(\theta) ] \\
		\rho [ \sin(\theta) \cos(\phi) + \sin(\phi) \cos(\theta) ]
	\end{bmatrix} \\
	R_\theta(p) = \begin{bmatrix}
		p_x \cos(\theta) - p_y \sin(\theta) \\
		p_x \sin(\theta) + p_y \cos(\theta)
	\end{bmatrix}
\end{gather*}
In questo modo non riesco a modificare il centro di rotazione in nessun modo e nemmeno
ampliare il raggio della rotazione. Per questo secondo problema si potrebbe optare
per una traslazione nella direzione del raggio di opportuna intensit\`a.

\subsection{Strappo}
\`E una traslazione di ognuno dei punti lungo un asse in modo proporzionale al valore
dell'altro.
\[
	Sh_{k, y}(p) =
	\begin{bmatrix}
		p_x + k \cdot p_y \\
		p_y
	\end{bmatrix}
\]
Si pu\`o comunque ottenere come combinazione di rotazioni e scalature non uniformi.

\section{Matrice di trasformazione}
Di norma \`e utile utilizzare una notazione con matrice per le trasformazioni che rende
il codice pi\`u compatto e pi\`u agevole da maneggiare.

Per esempio consideriamo il generico punto $p$ nello spazio bidimensionale. Una generica
trasformazione mi far\`a ottenere il punto $p'$ in questo modo:
\begin{gather*}
	p'_x = a_{xx} p_x + a_{xy} p_y \\
	p'_y = a_{yx} p_x + a_{yy} p_y
\end{gather*}
Dove i vari $a$ sono i coefficienti che dipendono dal tipo di trasformazione che si
vuole applicare.

Posso usare la notazione con matrice e scrivere:
\begin{gather*}
	p' = \begin{bmatrix}
		a_{xx} & a_{xy} \\
		a_{yx} & a_{yy}
	\end{bmatrix} \cdot p
\end{gather*}
Quella appena scritta viene chiamata \textbf{matrice di trasformazione}