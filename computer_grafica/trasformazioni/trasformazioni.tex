\section{Trasformazioni geometriche}
Una \textbf{trasformazione geometrica} \`e una funzione che mappa punti in punti e vettori
in vettori. Quando parliamo vogliamo \emph{trasformare} un oggetto intendiamo l'applicazione
della stessa trasformazione a tutti i punti dell'oggetto.

\subsection{Traslazione}
Tipicamente usata per muovere gli oggetti disegnati. Se vogliamo traslare un punto $p$ abbiamo
bisgno di un vettore $v$ che ci indica di quando traslare le sue coordinate (in questo caso
solo $x$ e $y$).
\[
	T_v(p) = p + v =
	\begin{pmatrix}
		p_x \\ p_y
	\end{pmatrix} +
	\begin{pmatrix}
		v_x \\ v_y
	\end{pmatrix} =
	\begin{pmatrix}
		p_x + v_x \\ p_y + v_y
	\end{pmatrix}
\]

\subsection{Scalatura}
Tipicamente usata per cambiare la dimensione di un oggetto. Per ottenere questa trasformazione
si moltiplicano tutti i punti della figura per un fattore di scalatura.
\[
	S_{(s_x, s_y)}(p) =
	\begin{pmatrix}
		S_x \cdot p_x \\
		S_y \cdot p_y
	\end{pmatrix}
\]
Se $S_x = S_y$ allora la scalatura \`e \textbf{uniforme o isotropica} (le proporzioni
dell'oggetto non cambiano). Altrimenti si dice che la scalatura \`a
\textbf{non uniforme o anisotropica}.

\subsection{Rotazione}
Ruotare di un angolo $\theta$ un punto
\[
	p = \begin{pmatrix}
		\rho \cdot \cos(\phi) \\
		\rho \cdot \sin(\phi)
	\end{pmatrix}\]
equivale a fare
\[
	R_\theta(p) =
	\begin{pmatrix}
		\rho \cdot \cos(\phi + \theta) \\
		\rho \cdot \sin(\phi + \theta)
	\end{pmatrix}
\]
Per semplificare i calcoli possiamo ricordarci che
\begin{gather*}
	\cos(\phi + \theta) = \cos(\phi) \cos(\theta) - \sin(\phi) \sin(\theta) \\
	\sin(\phi + \theta) = \sin(\theta) \cos(\phi) + \sin(\phi) \cos(\theta)
\end{gather*}
Posso quindi riscrivere la funzione di rotazione come
\[
	R_\theta(p) =
	\begin{pmatrix}
		\cos(\theta)p_x - sin(\theta)p_y \\
		\sin(\theta)p_x + \cos(\theta)p_y
	\end{pmatrix}
\]

\subsection{Taglio}
\`E una traslazione di ognuno dei punti lungo un asse in modo proporzionale al valore
dell'altro.
\[
	Sh_{k, y}(p) =
	\begin{pmatrix}
		p_x + k \cdot p_y \\
		p_y
	\end{pmatrix}
\]

\section{Matrice di trasformazione}


\subsection{Composizione di trasformazioni}


\section{Frame}
