\chapter{Trasformazioni}
\section{Algebra vettoriale}
Le entit\`a utilizzate sono \textbf{punti} e \textbf{vettori}, questi ultimi definiti da
\emph{modulo}, \emph{direzione} e \emph{verso}.

\subsection{Operazioni su punti e vettori}
Definiamo qualche operazione utile per queste due entit\`a:
\begin{itemize}
	\item punto $+$ vettore ritorna un punto
	\item punto $-$ punto ritorna un vettore
	\item vettore $+$ vettore ritorna un vettore
	\item scalare $\cdot$ vettore ritorna un vettore
\end{itemize}

\subsection{Prodotto scalare tra due vettori}
Il prodotto scalare tra due vettori $a$ e $b$ scritto come $a \cdot b$ equivale a:
\[ a \cdot b = \sum_{i=1}^n a_1 b_1 + \dots + a_n b_n  \]
Questo significa che
\[ a \cdot b = \| a \| \| b \| \cos{(\theta)} \]
da qui posso ricavarmi l'angolo in questo modo:
\[ \theta = \arccos{\left( \frac{a \cdot b}{\| a \| \| b \|} \right)} \]
Altre propriet\`a:
\begin{itemize}
	\item Il prodotto scalare tra due vettori \`e 0 quando i due vettori sono \emph{ortogonali}
	      fra loro.
	\item Il prodotto scalare \`e massimo per vettori paralleli (stesso verso) ed \`e minimo
	      per vettori anti-paralleli (verso opposto).
\end{itemize}

\subsection{Prodotto vettoriale tra due vettori}
Il prodotto tra due vettori $a$ e $b$ scritto come $a \times b$ equivale a:
\[
	a \times b = \begin{pmatrix}
		a_y b_z - b_y a_z \\ b_x a_z - a_x b_z \\ a_x b_y - b_x a_y
	\end{pmatrix}
\]
Per ricordarsi il calcolo si pu\`o usare questo metodo: si calcola il determinante della
matrice
\[
	det \begin{pmatrix}
		i   & j   & k   \\
		a_x & a_y & a_z \\
		b_x & b_y & b_z
	\end{pmatrix} =
	i(a_y b_z - b_y a_z) - j(a_x b_z - b_x a_z) + k(a_x b_y - b_x a_y)
\]
Questo prodotto ci fornisce un vettore ortogonale sia ad $a$ che a $b$.
\[ (a \times b) \cdot a = (a \times b ) \cdot b = 0 \]
Non vale la propriet\`a commutativa ma vale la distributiva per la somma.
\[ a \times (b + c) = a \times b + a \times c \]
La norma equivale a
\[ \| a \times b \| = \| a \| \| b \| \sin(\theta) \]

\subsection{Coordinate polari}
Le \textbf{coordinate polari} servono a esprimere punti e vettori tramite la tupla
$(\theta, \rho)$ dove
\begin{itemize}
	\item $\theta$ \`e l'angolo formato con l'asse $x$.
	\item $\rho$ \`e la distanza dall'origine.
\end{itemize}
Se vogliamo passare da coordinate polari a cartesiane
\begin{gather*}
	x = \rho \cdot \cos(\theta) \\
	y = \rho \cdot \sin(\theta)
\end{gather*}
Viceversa se vogliamo passare da coordinate cartesiano a polari
\begin{gather*}
	\theta = \arctan(2 \cdot (y, x)) \\
	\rho = \sqrt{x^2 + y^2}
\end{gather*}

\section{Trasformazioni geometriche}
Una \textbf{trasformazione geometrica} \`e una funzione che mappa punti in punti e vettori
in vettori. Quando parliamo vogliamo \emph{trasformare} un oggetto intendiamo l'applicazione
della stessa trasformazione a tutti i punti dell'oggetto.

\subsection{Traslazione}
Tipicamente usata per muovere gli oggetti disegnati. Se vogliamo traslare un punto $p$ abbiamo
bisgno di un vettore $v$ che ci indica di quando traslare le sue coordinate (in questo caso
solo $x$ e $y$).
\[
	T_v(p) = p + v =
	\begin{pmatrix}
		p_x \\ p_y
	\end{pmatrix} +
	\begin{pmatrix}
		v_x \\ v_y
	\end{pmatrix} =
	\begin{pmatrix}
		p_x + v_x \\ p_y + v_y
	\end{pmatrix}
\]

\subsection{Scalatura}
Tipicamente usata per cambiare la dimensione di un oggetto. Per ottenere questa trasformazione
si moltiplicano tutti i punti della figura per un fattore di scalatura.
\[
	S_{(s_x, s_y)}(p) =
	\begin{pmatrix}
		S_x \cdot p_x \\
		S_y \cdot p_y
	\end{pmatrix}
\]
Se $S_x = S_y$ allora la scalatura \`e \textbf{uniforme o isotropica} (le proporzioni
dell'oggetto non cambiano). Altrimenti si dice che la scalatura \`a
\textbf{non uniforme o anisotropica}.

\subsection{Rotazione}
Ruotare di un angolo $\theta$ un punto
\[
	p = \begin{pmatrix}
		\rho \cdot \cos(\phi) \\
		\rho \cdot \sin(\phi)
	\end{pmatrix}\]
equivale a fare
\[
	R_\theta(p) =
	\begin{pmatrix}
		\rho \cdot \cos(\phi + \theta) \\
		\rho \cdot \sin(\phi + \theta)
	\end{pmatrix}
\]
Per semplificare i calcoli possiamo ricordarci che
\begin{gather*}
	\cos(\phi + \theta) = \cos(\phi) \cos(\theta) - \sin(\phi) \sin(\theta) \\
	\sin(\phi + \theta) = \sin(\theta) \cos(\phi) + \sin(\phi) \cos(\theta)
\end{gather*}
Posso quindi riscrivere la funzione di rotazione come
\[
	R_\theta(p) =
	\begin{pmatrix}
		\cos(\theta)p_x - sin(\theta)p_y \\
		\sin(\theta)p_x + \cos(\theta)p_y
	\end{pmatrix}\]
\subsection{Taglio}

\section{Trasformazioni con matrici}

\subsection{Composizione di trasformazioni}

\section{Frame}
