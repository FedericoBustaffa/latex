\chapter{Modelli lineari per problemi di classificazione}
Fino ad ora abbiamo utilizzato modelli lineari solo per trattare problemi di regressione. In realt\`a i modelli lineari
sono molto utili anche per affrontare problemi di classificazione.

Se prima il modelli lineare restituiva un valore numerico reale adesso si limiter\`a restituire solo due valori numerici:
uno per indicare \verb|false| (in genere $0$ o $-1$) e uno per indicare \verb|true| (in genere $1$).
