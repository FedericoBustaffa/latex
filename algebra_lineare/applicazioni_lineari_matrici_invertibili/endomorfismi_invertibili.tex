
\section{Applicazioni lineari e matrici invertibili}
\subsection{Endomorfismi lineari invertibili}

\begin{defn}
	Consideriamo uno spazio vettoriale $V$ di dimensione $n$ sul campo $\mathbb{K}$ e
	una applicazione lineare $L : V \to V$. Una tale applicazione lineare si dice
	\textbf{endomorfismo lineare di $V$}. Indicheremo con $End(V)$ l'insieme di tutti gli
	endomorifsimi lineari di $V$.
\end{defn}

\begin{proposition}
	Un endomorfismo $L$ di $V$ \`e invertibile se e solo se ha rango $n$. La
	funzione inversa $L^{-1} : V \to V$ \`e anch'essa un'applicazione lineare.
\end{proposition}

\begin{observation}
	Dati due spazi $V$ e $W$ entrambi di dimensione $n$ e una applicazione lineare
	$L : V \to W$, l'applicazione lineare $L$ \`e invertibile se e solo se ha rango
	$n$; l'applicazione inversa $L^{-1}$ \`e anch'essa lineare.
	Abbiamo invece gi\`a osservato che se $V$ e $W$ hanno dimensioni diverse,
	rispettivamente $m$ e $n$, nessuna applicazione lineare $L$ da $V$ a $W$ pu\`o
	essere invertibile. Infatti, avendo in mente la relazione che lega la dimensione
	del nucleo di $L$, dell'immagine di $L$ e di $V$
	($dim(Imm(L)) + dim(Ker(L)) = dim(V)$), si ha che:
	\begin{itemize}
		\item se $m > n$ allora la dimensione di $Imm(L)$ \`e al massimo $n$.
		      Quindi la dimensione di $Ker(L)$ \`e almeno $m - n$, ovvero
		      maggiore di 0, e $L$ non \`e iniettiva.
		\item se $m < n$ allora la dimensione di $Imm(L)$ \`e al massimo $m$.
		      Quindi la dimensione di $Imm(L)$ \`e minore della dimensione di
		      $W$, e $L$ non \`e surgettiva.
	\end{itemize}
\end{observation}

Se fissiamo una base di $V$, ad ogni endomorfismo $L \in End(V)$ viene associata
una matrice $[L] \in Mat_{n \times n}(\mathbb{K})$. Se $L$ \`e invertibile,
consideriamo l'inversa $L^{-1}$ e la matrice ad essa associata $[L^{-1}]$.
Visto che $L \circ L^{-1} = L^{-1} \circ L = I$, vale
\begin{equation*}
	[L^{-1}][L] = [L][L^{-1}] = [I] = I
\end{equation*}

Dunque la matrice $[L]$ \`e invertibile e ha per inversa $[L^{-1}]$.
Possiamo affermare anche il viceversa: se la matrice $[L]$ associata ad un
endomorfismo lineare \`e invertibile allora anche $L$ \`e invertibile e la sua
inversa \`e l'applicazione associata alla matrice $[L^{-1}]$.

\begin{corollary}
	Una matrice $A \in Mat_{n \times n}(\mathbb{K})$ \`e invertibile se e solo se
	il suo rango \`e $n$.
\end{corollary}