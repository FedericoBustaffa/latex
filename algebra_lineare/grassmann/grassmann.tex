\chapter{Grassmann}
\section{La formula di Grassmann}
Dati due sottospazi vettoriali $A$ e $B$ in $\R^3$ di dimensione 2,
di che dimensione può essere la loro intersezione ?

Possono intersecarsi lungo una retta: in tal caso si nota che il sottospazio
generato dai vettori di $A \cup B$, ossia $A + B$, è tutto $\R^3$.

Oppure vale $A = B$: allora la loro intersezione è uguale ad $A$ (e a $B$) e
ha dimensione 2, e anche il sottospazio $A + B$ coincide con $A$.

In entrambi i casi, la somma delle dimensioni di $A \cap B$ e di $A + B$ è
sempre uguale a 4.

E se in $\R^4$ consideriamo un piano $C$ e un sottospazio $D$ di
dimensione 3?
Possono darsi tre casi per l'intersezione: $C \cap D = \{O\}$,
$\dim(C \cap D) = 1$, $C \cap D = C$.

Qualunque sia il caso si verifica sempre che
\[
	\dim(C \cap D) + \dim(C + D) = 5 = \dim(C) + \dim(D)
\]

In generale vale la formula
\[
	\dim(A \cap B) + \dim(A + B) = \dim(A) + \dim(B)
\]

Dati due spazi vettoriali $V$ e $W$ sul campo $\K$, sul loro prodotto
cartesiano $V \times W$, c'è una struttura naturale di spazio vettoriale, dove
la somma è definita da:
\[
	(v, w) + (v_1, w_1) = (v + v_1, w + w_1)
\]
e il prodotto per scalare da:
\[
	\lambda(v, w) = (\lambda v, \lambda w)
\]
Si verifica che, se $\{v_1, \dots, v_n\}$ è una base di $V$
e $\{w_1, \dots, w_m\}$ è una base di $W$, allora
$\{(v_1, O), \dots (v_n, O), (O, w_1), \dots, (O, w_m)\}$ è una base di
$V \times W$, che dunque ha dimensione $n + m = (\dim(V)) + (\dim(W))$.

\begin{theorem}[Grassmann]
	Dati due sottospazi $A, B$ di uno spazio vettoriale $V$ sul campo
	$\K$, vale
	\[
		\dim(A) + \dim(B) = \dim(A \cap B) + \dim(A + B)
	\]
	\begin{proof}
		Consideriamo l'applicazione
		\[
			\Phi : A \times B \to V
		\]
		definita da
		\[ \Phi((a, b)) = a - b \]
		Cosa sappiamo dire del nucleo di $\Phi$ ? Per definizione
		\[
			\Ker(\Phi) = \{(a, b) \in A \times B \mid a - b = O\}
		\]
		dunque
		\[
			\Ker(\Phi) = \{(a, b) \in A \times B \mid a = b\}
		\]
		che equivale a scrivere:
		\[
			\Ker(\Phi) = \{(z, z) \in A \times B \mid z \in A \cap B\}
		\]
		Si nota subito che la applicazione lineare
		\[ \theta : A \cap B \to \Ker(\Phi) \]
		è iniettiva e surgettiva, dunque è un isomorfismo. Allora il suo dominio e
		il suo codominio hanno la stessa dimensione, ovvero
		\[
			\dim(\Ker(\Phi)) = \dim(A \cap B)
		\]
		Cosa sappiamo dire invece dell'immagine di $\Phi$ ? Per definizione
		\[
			\Imm(\Phi) = \{a - b \mid a \in A, b \in B\}
		\]
		Visto che $B$, come ogni spazio vettoriale, se contiene un elemento
		$b$ contiene anche il suo opposto $-b$, possiamo scrivere la seguente
		uguaglianza fra insiemi:
		\[
			\{ a - b \mid a \in A, b \in B \} =
			\{ a + b \in V \mid a \in A, b \in B \} =
			A + B
		\]
		Dunque
		\[
			\Imm(\Phi) = A + B
		\]
		Sappiamo che:
		\[
			\dim(A \times B) = \dim(\Ker(\Phi)) + \dim(\Imm(\Phi))
		\]
		Questa formula, viste le osservazioni fatte fin qui, si traduce come:
		\[
			\dim(A) + \dim(B) = \dim(A \cap B) + \dim(A + B)
		\]
	\end{proof}
\end{theorem}
