\chapter{Riduzione a scalini}
\section{Operazioni elementari sulle colonne}
Consideriamo una generica matrice in $\Mat_{m \times n}(\K)$:
\[
	A = \begin{pmatrix}
		a_{11} & a_{12} & \dots & a_{1n} \\
		a_{21} & a_{22} & \dots & \dots  \\
		\dots  & \dots  & \dots & \dots  \\
		a_{m1} & \dots  & \dots & a_{mn}
	\end{pmatrix}
\]
e i tre seguenti tipi di mossa sulle colonne, detti anche
\textbf{operazioni elementari sulle colonne}:
\begin{enumerate}
	\item si somma alla colonna $i$ la colonna $j$ moltiplicata per uno scalare
	      $\lambda$.
	\item si moltiplica la colonna $i$ per uno scalare $\lambda$
	\item si scambiano fra di loro due colonne $i$ e $j$.
\end{enumerate}

\begin{definition}
	La \textbf{profondità} di una colonna è definita come la posizione
	occupata (contata dal basso) dal suo più alto coefficiente diverso da 0.
	Alla colonna nulla si assegna per convenzione profondità uguale a 0.
\end{definition}

\begin{example}
	Consideriamo la seguente matrice:
	\[
		A = \begin{pmatrix}
			0            & 4 - \sqrt{3} & 0  \\
			\sqrt{3} + 1 & 0            & 0  \\
			-2           & -2           & -2
		\end{pmatrix}
	\]
	In questo caso la prima colonna ha profodità uguale a 2, la
	seconda colonna uguale a 3 mentre la terza ha profondità uguale a 1.
\end{example}

\begin{definition}
	Una matrice $A \in \Mat_{m \times n}(\K)$, si dice
	\textbf{in forma a scalini per colonne} se rispetta le seguenti proprietà:
	\begin{itemize}
		\item leggendo la matrice da sinistra a destra, le colonne non nulle si
		      incontrano tutte prima delle colonne nulle.
		\item leggendo la matrice da sinistra a destra, le profondità
		      delle sue colonne non nulle risultano strettamente
		      decrescenti.
	\end{itemize}
\end{definition}

\begin{example}
	Questi sono esempi di matrici in forma \emph{a scalini}:
	\[
		A = \begin{pmatrix}
			1            & 0           & 0 & 0 \\
			\sqrt{3} + 1 & 1           & 0 & 0 \\
			-2           & \frac{5}{2} & 1 & 0
		\end{pmatrix}
	\]
	\[
		B = \begin{pmatrix}
			1  & 0           & 0 & 0 \\
			0  & 1           & 0 & 0 \\
			-2 & \frac{5}{2} & 0 & 0
		\end{pmatrix}
	\]
\end{example}

\begin{definition}
	In una matrice in forma a scalini per colonna, i coefficienti diversi
	da zero più alti di posizione di ogni colonna non nulla si
	chiamano \textbf{pivot}.
\end{definition}

\begin{theorem}
	Data una matrice $A \in \Mat_{m \times n}(\K)$ è sempre
	possibile, usando operazioni elementari sulle colonne, ridurre la
	matrice in forma a scalini per colonne.
\end{theorem}

\begin{observation}
	Quando si riduce una matrice in forma a scalini, la forma a scalini
	ottenuta non è unica.
\end{observation}

\begin{definition}
	Una matrice $A \in \Mat_{m \times n}(\K)$, si dice \textbf{in forma a
		scalini per colonne ridotta} se:
	\begin{itemize}
		\item $A$ è a scalini per colonne.
		\item Tutte le entrate nella stessa riga di un pivot, precedenti al pivot,
		      sono nulle
	\end{itemize}
\end{definition}

\begin{example}
	Esempio di matrice in forma a scalini ridotta:
	\[
		\begin{pmatrix}
			1            & 0           & 0 & 0 \\
			\sqrt{3} + 1 & 1           & 0 & 0 \\
			-2           & \frac{5}{2} & 1 & 0
		\end{pmatrix} \rightarrow
		\begin{pmatrix}
			1 & 0 & 0 & 0 \\
			0 & 1 & 0 & 0 \\
			0 & 0 & 1 & 0
		\end{pmatrix}
	\]
\end{example}

\begin{proposition}
	Data una matrice in forma a scalini per colonne, è sempre possibile, usando
	solo la prima delle operazioni elementari sulle colonne, portare $A$ in forma
	a scalini ridotta.
\end{proposition}

\begin{corollary}
	Ogni matrice $A$ può essere trasformata, tramite le operazioni elementari
	sulle colonne, in una matrice in forma a scalini per colonne ridotta.
\end{corollary}

\begin{proposition}
	Se operiamo attraverso le operazioni elementari sulle colonne, lo
	$Span$ dei vettori colonna rimane invariato. Ovvero se indichiamo con
	$v_1, \dots, v_n$ i vettori colonna di una matrice
	$A \in \Mat_{m \times n}(\K)$, per ogni matrice $A'$ ottenuta da $A$
	attraverso le operazioni elementari sulle colonne, si ha, indicando con
	$w_1, \dots, w_n$ i vettori colonna di $A'$, che:
	\[
		Span(v_1, \dots, v_n) = Span(w_1, \dots, w_n)
	\]
\end{proposition}
