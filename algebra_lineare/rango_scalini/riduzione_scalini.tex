\section{Riduzione a scalini e studio delle basi}

\begin{theorem}
	Sia $V$ uno spazio vettoriale su $\K$ che ammette una base
	finita, allora tutte le basi di $V$ hanno la stessa cardinalità.
\end{theorem}

\begin{corollary}
	In uno spazio vettoriale $V$ di dimensione $n$, dati $n$ vettori
	linearmente indipendenti questi sono anche una base di $V$. Allo
	stesso modo, dati $n$ vettori che generano $V$, questi sono anche una
	base di $V$.
\end{corollary}

Considerzioni simili a quelle fatte fino ad ora ci consentono di descrivere
un criterio concreto per decidere se, dato uno spazio vettoriale $V$ di
dimensione $n$ ed una base $e_1, \dots, e_n$ di $V$, e dati $n$ vettori
$v_1, \dots, v_n$ di $V$, tali vettori costituiscono una base di $V$ o no.

Il criterio è espresso dai seguenti punti:
\begin{enumerate}
	\item Esprimiamo i vettori $v_1, \dots, v_n$ trovandone i coefficienti rispetto
	      alla base $e_1, \dots, e_n$.
	\item Poniamoli in colonna uno accanto all'altro. Così facendo otteniamo una
	      matrice $M$ che è $n \times n$.
	\item Riduciamo $M$ in forma a scalini ridotta ottenendo $M'$.
	\item A questo punto, se $M'$ è l'identità, allora $\{v_1, \dots, v_n\}$
	      è una base di $V$, altrimenti no.
\end{enumerate}

\begin{example}
	Verificare che i seguenti vettori siano una base di $\R^4$.
	\[
		v_1 = \begin{pmatrix}
			1 \\ 1 \\ 0 \\ 0
		\end{pmatrix} \quad
		v_2 = \begin{pmatrix}
			0 \\ 1 \\ 1 \\ 0
		\end{pmatrix} \quad
		v_3 = \begin{pmatrix}
			0 \\ 0 \\ 1 \\ 1
		\end{pmatrix} \quad
		v_4 = \begin{pmatrix}
			0 \\ 0 \\ 0 \\ 1
		\end{pmatrix}
	\]
	Verifichiamo col nuovo metodo che si tratta di una base.
	Scriviamo dunque la matrice
	\[
		\begin{pmatrix}
			1 & 0 & 0 & 0 \\
			1 & 1 & 0 & 0 \\
			0 & 1 & 1 & 0 \\
			0 & 0 & 1 & 1 \\
		\end{pmatrix}
	\]
	Portandola in forma a scalini ridotta diventa
	\[
		\begin{pmatrix}
			1 & 0 & 0 & 0 \\
			0 & 1 & 0 & 0 \\
			0 & 0 & 1 & 0 \\
			0 & 0 & 0 & 1
		\end{pmatrix}
	\]
	Dunque $\{v_1, v_2, v_3, v_4\}$ è una base di $\R^4$.
\end{example}

\begin{theorem}[Completamento]
	Dato uno spazio vettoriale $V$ di dimensione $n$, ogni
	sottoinsieme $B = \{v_1, \dots, v_k\} \subset V$ di vettori
	linearmente indipendenti di cardinalità $k$, con
	$1 \leq k \leq n$, può essere completato ad una base di $V$
	aggiungendo a $B$ $n - k$ vettori di $V \backslash Span(B)$.
	\begin{proof}
		La dimostrazione che segue ci fornisce un algoritmo per trovare
		vettori da aggiungere al sottoinsieme $B$ per riuscire a trovare
		una base di $V$.
		\begin{enumerate}
			\item Si scrivono i vettori $v_1, \dots, v_k$ come
			      vettori colonna rispetto a una base data, e si considera
			      la matrice $M$ che ha questi vettori come colonne.
			\item Si riduce $M$ in forma a scalini.
			\item Tutte le volte le volte che troviamo uno scalino lungo
			      (altezza $\geq 2$, dove l'altezza è la differenza di profondità
			      tra due colonne adiacenti che formano lo scalino lungo) possiamo
			      facilmente trovare $i - 1$ vettori $w_1, \dots, w_{i - 1}$ tali che
			      $\{v_1, \dots, v_k, w_1, \dots, w_{i - 1}\}$ è ancora un
			      insieme di vettori linearmente indipendenti.
			\item Supponiamo infatti che $M$ abbia uno scalino di lunghezza $i$,
			      ovvero in una certa colonna abbia il pivot alla riga $t$ e nella
			      colonna successiva alla riga $t+i$, e sia $i > 1$. Allora
			      per $j$ che varia tra 1 ed $i - 1$.
			\item Scegliere un vettore $w_j$ tale che abbia tutti
			      0 tranne un 1 in corrispondeza della riga $(t + j)$-esima.
			      È facile osservare che ogni $w_j$ non appartiene
			      allo $Span(v_1, \dots, v_k, w_1, \dots, w_{j-1})$. Dunque ad
			      ogni aggiunta di $w_j$, l'insieme
			      $\{v_1, \dots, v_k, w_1, \dots, w_j\}$ rimane un insieme di
			      vettori linearmente indipendenti di $V$.
			\item Ripetiamo questa aggiunta di vettori $w_k$ per ogni scalino
			      lungo che troviamo in $M$, trovando così alla fine $n$ vettori
			      linearmente indipendenti, dunque una base di $V$ come richiesto.
		\end{enumerate}
	\end{proof}
\end{theorem}

\begin{example}
	Illustriamo il metodo descritto con un esempio. Supponiamo che
	$V = \R^7$ e siano dati i 4 vettori linearmente
	indipendenti che, scritti rispetto alla base standard di
	$\R^7$, sono rappresentati come segue:
	\[
		v_1 = \begin{pmatrix}
			1 \\ 1 \\ 0 \\ 2 \\ 1 \\ 0 \\ 1
		\end{pmatrix} \quad
		v_2 = \begin{pmatrix}
			0 \\ 1 \\ 0 \\ 2 \\ 1 \\ 0 \\ 3
		\end{pmatrix} \quad
		v_3 = \begin{pmatrix}
			0 \\ 0 \\ 0 \\ 0 \\ 1 \\ 0 \\ 1
		\end{pmatrix} \quad
		v_4 = \begin{pmatrix}
			0 \\ 0 \\ 0 \\ 0 \\ 1 \\ 0 \\ 2
		\end{pmatrix}
	\]
	La matrice $M$ in questo caso è:
	\[
		M = \begin{pmatrix}
			1 & 0 & 0 & 0 \\
			1 & 1 & 0 & 0 \\
			0 & 0 & 0 & 0 \\
			2 & 2 & 0 & 0 \\
			1 & 1 & 1 & 1 \\
			0 & 0 & 0 & 0 \\
			1 & 3 & 1 & 2
		\end{pmatrix}
	\]
	Riduciamola a scalini per colonne.
	\[
		M' = \begin{pmatrix}
			1 & 0 & 0 & 0 \\
			0 & 1 & 0 & 0 \\
			0 & 0 & 0 & 0 \\
			0 & 2 & 0 & 0 \\
			1 & 0 & 1 & 0 \\
			0 & 0 & 0 & 0 \\
			0 & 1 & 2 & 1
		\end{pmatrix}
	\]
	Il primo scalino lungo ha altezza 3. Come osservato nella dimostrazione
	del teorema del completamento, i vettori
	\[
		w_1 = \begin{pmatrix}
			0 \\ 0 \\ 1 \\ 0 \\ 0 \\ 0 \\ 0
		\end{pmatrix} \quad
		w_2 = \begin{pmatrix}
			0 \\ 0 \\ 0 \\ 1 \\ 0 \\ 0 \\ 0
		\end{pmatrix}
	\]
	non appartengono al sottospazio generato dalle colonne di $M'$, e si
	osserva immediatamente che $\{v_1, v_2, v_3, v_4, w_1, w_2\}$ è un
	insieme di vettori linearmente indipendenti di $\R^7$.

	Similmente prendendo in considerazione il secondo scalino lungo, notiamo
	che il vettore
	\[
		w_3 = \begin{pmatrix}
			0 \\ 0 \\ 0 \\ 0 \\ 0 \\ 1 \\ 0
		\end{pmatrix}
	\]
	non appartiene al sottospazio generato da $v_1, v_2, v_3, v_4, w_1, w_2$.

	A questo punto abbiamo ottenuto l'insieme di vettori linearmente indipendenti
	$\{v_1, v_2, v_3, v_4, w_1, w_2, w_3\}$ di $\R^7$ e dato che
	$\R^7$ ha dimensione 7 abbiamo ottenuto una base di $\R^7$.
\end{example}


