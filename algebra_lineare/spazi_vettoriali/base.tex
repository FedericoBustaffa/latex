
% BASI

\subsection{Base di uno spazio vettoriale}
Sia $V$ uno spazio vettoriale su $\mathbb{K}$.
Per definizione di $V$, se $v_1, v_2, ..., v_n$ sono $n$ vettori
di $V$, allora per qualsiasi scelta di $n$ elementi
$k_1, k_2, ..., k_n$ di $\mathbb{K}$ il vettore:
\begin{equation*}
	v = k_1 v_1 + ... + k_n v_n = \sum_{i=1}^n k_i v_i
\end{equation*}
appartiene a $V$, in quanto $V$ \`e chiuso per somma vettoriale
e prodotto per scalare.

\begin{defn}
	Dato un insieme di vettori $\{v_1, v_2, ..., v_n\}$ di $V$,
	spazio vettoriale sul campo $\mathbb{K}$, il vettore:
	\begin{equation*}
		v = k_1 \cdot v_1 + ... + k_n \cdot v_n
	\end{equation*}
	con $\{k_1, k_2, ..., k_n\}$ scalari di $\mathbb{K}$,
	si dice una \textbf{combinazione lineare} dei vettori
	$\{v_1, v_2, ..., v_n\}$. I $k_i$ sono detti \textbf{coefficienti}
	della combinazione lineare.
\end{defn}

\begin{example}
	Consideriamo lo spazio vettoriale $\mathbb{R}^3$ su $\mathbb{R}$ e i seguenti
	due vettori:
	\begin{equation*}
		v_1 = \begin{pmatrix}
			3 \\ -1 \\ 3
		\end{pmatrix}
		\quad
		v_2 = \begin{pmatrix}
			1 \\ 0 \\ 2
		\end{pmatrix}
	\end{equation*}
	Allora il vettore $v_3$ seguente:
	\begin{equation*}
		v_3 = \begin{pmatrix}
			5 \\ -1 \\ 7
		\end{pmatrix}
		= 1 \cdot \begin{pmatrix}
			3 \\ -1 \\ 3
		\end{pmatrix}
		+ 2 \cdot \begin{pmatrix}
			1 \\ 0 \\ 2
		\end{pmatrix}
	\end{equation*}
	\`E una combinazione lineare dell'insieme dei vettori $\{v_1, v_2\}$ di
	coefficienti 1 e 2.
\end{example}

\begin{defn}
	Dati $\{v_1, v_2, ..., v_t\}$ vettori di $V$ su $\mathbb{K}$,
	si definisce \textbf{span} dei vettori $v_1, v_2, ..., v_t$
	(e si indica con $Span(v_1, v_2, ..., v_t)$) l'insieme di tutte
	le possibili combinazioni lineari dell'insieme di vettori
	$\{v_1, v_2, ..., v_t\}$.
\end{defn}

\begin{defn}
	Un insieme di vettori $\{v_1, v_2, ..., v_t\}$ di $V$ per cui
	$V = Span(v_1, v_2, ..., v_t)$, ovvero $\forall v \in V$, esistono
	degli scalari $a_1, a_2, ..., a_t$ tali che
	\begin{equation*}
		a_1 v_1 + a_2 v_2 + ... + a_t v_t = v
	\end{equation*}
	si dice un \textbf{insieme di generatori} di $V$. In tal caso si dice
	anche che i vettori $v_1, v_2, ..., v_t$ \textbf{generano} $V$.
\end{defn}

L'esistenza di un sistema finito di generatori per uno spazio vettoriale
$V$ su un campo $\mathbb{K}$ \`e un fatto molto importante, dato che
si riduce la descrizione di uno spazio vettoriale con cardinalit\`a
infinita, ad una lista di numero finito di vettori di $V$.


Dato un sistema di generatori $\{v_1, ..., v_t\}$ di $V$ sappiamo dunque
che ogni $v \in V$ si pu\`o scrivere, con opportuni coefficienti
$\{k_1, ..., k_t\}$, come:
\begin{equation*}
	v = \sum_{i=1}^t k_i v_i
\end{equation*}
In generale, tale scrittura non \`e unica, ovvero non ci permette di
identificare univocamente ogni vettore di $v \in V$.

\begin{example}
	Si verifica che i vettori
	\begin{equation*}
		\begin{pmatrix}
			1 \\ 2 \\ 3
		\end{pmatrix} \quad
		\begin{pmatrix}
			1 \\ 0 \\ 1
		\end{pmatrix} \quad
		\begin{pmatrix}
			0 \\ 0 \\ 1
		\end{pmatrix} \quad
		\begin{pmatrix}
			2 \\ 2 \\ 4
		\end{pmatrix}
	\end{equation*}

	generano $\mathbb{R}^3$. Si possono facilmente trovare due distinte
	combinazioni lineari di tali vettori che esprimono il vettore
	\begin{equation*}
		\begin{pmatrix}
			2 \\ 2 \\ 5
		\end{pmatrix}
	\end{equation*}

	Per esempio:
	\begin{equation*}
		\begin{pmatrix}
			1 \\ 2 \\ 3
		\end{pmatrix} +
		\begin{pmatrix}
			1 \\ 0 \\ 1
		\end{pmatrix} +
		\begin{pmatrix}
			0 \\ 0 \\ 1
		\end{pmatrix} =
		\begin{pmatrix}
			2 \\ 2 \\ 5
		\end{pmatrix} =
		\begin{pmatrix}
			2 \\ 2 \\ 4
		\end{pmatrix} +
		\begin{pmatrix}
			0 \\ 0 \\ 1
		\end{pmatrix}
	\end{equation*}
\end{example}

\begin{defn}
	Si dice che un insieme finito di vettori $\{v_1, v_2, ..., v_r\}$ \`e
	un \textbf{insieme di vettori linearmente indipendenti} se l'unico
	modo di scrivere il vettore $O$ come combinazione lineare di questi
	vettori \`e con tutti i coefficienti nulli, ossia se
	\begin{equation*}
		a_1 v_1 + a_2 v_2 + ... + a_r v_r = O \quad \Leftrightarrow \quad
		a_1 = a_2 = ... = a_r = 0
	\end{equation*}
	Si pu\`o dire anche che i vettori sono \textbf{linearmente indipendenti}.
	Se invece i vettori $v_1, v_2, ..., v_r$ non sono linearmente indipendenti
	si dice che sono \textbf{linearmente dipendenti}.
\end{defn}

\begin{proposition}
	Un insieme $A = \{v_1, ..., v_n\}$ di vettori di uno spazio
	vettoriale $V$ su $\mathbb{K}$ \`e un insieme di vettori linearmente
	indipendenti se e solo se nessun $v_i$, appartenente ad $A$, si pu\`o
	scrivere come combinazione lineare dell'insieme
	$B = A \backslash \{v_i\}$ (ovvero $v_i$ non appartiene a $Span(B)$).
\end{proposition}

\begin{defn}
	Sia $V$ uno spazio vettoriale su $\mathbb{K}$, un insieme di vettori
	$\{v_1, v_2, ..., v_n\} \in V$, che generano lo spazio $V$ e che sono
	linearmente indipendenti, si dice una \textbf{base} (finita) di $V$.
\end{defn}

\begin{observation}
	Nella definizione \`e specificato \emph{finita}. Non sempre uno
	spazio vettoriale ammette un numero finito di generatori, e di
	conseguenza nemmeno una base finita.
\end{observation}

Fissata la defizione di base siamo interessati a capire:
\begin{enumerate}
	\item
	      Se la scelta di una base garantisce l'unicit\`a di scrittura di un vettore
	      in termini di combinazione lineare degli elementi della base.
	\item
	      Quando uno spazio vettoriale ammette una base finita, ed in
	      particolare se il fatto che uno spazio vettoriale $V$ abbia un
	      insieme finito di generatori, garantisca che $V$ abbia una
	      base finita o meno.
\end{enumerate}

\begin{proposition}
	Ogni vettore $v \in V$ si scrive \emph{in modo unico} come
	combinazione lineare degli elementi della base.
\end{proposition}

\begin{theorem}
	Sia $V$ uno spazio vettoriale su $\mathbb{K}$ diverso da $\{O\}$
	e generato dall'insieme finito di vettori \emph{non nulli}
	$\{w_1, w_2, ..., w_s\}$. Allora \`e possibile estrarre da
	$\{w_1, w_2, ..., w_s\}$ un sottoinsieme
	$\{w_{i_1}, w_{i_2}, ..., w_{i_n}\}$ (con $n \leq s$) che \`e
	una base di $V$.
\end{theorem}

\begin{defn}
	Sia $V$  uno spazio vettoriale con basi di cardinalit\`a $n$.
	Tale cardinalit\`a $n$ \`e detta \textbf{dimensione} di $V$.
\end{defn}
