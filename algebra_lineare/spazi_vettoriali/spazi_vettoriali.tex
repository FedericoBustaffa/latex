
\section{Spazi Vettoriali}

\subsection{Definizione di spazio vettoriale}
Per fornire la definizione di spazio vettoriale si ha
bisogno di un insieme non vuoto $V$ e di un campo
$\mathbb{K}$, dove sia possibile definire le operazioni
di \textbf{somma vettoriale} e
\textbf{prodotto per scalare}.

\begin{defn}
	Uno \textbf{spazio vettoriale su un campo} $\mathbb{K}$
	\`e un insieme $V$ su cui sono definite la somma fra due
	elementi di $V$ (il cui risultato \`e ancora un elemento
	di V, si dice quindi che $V$ \`e chiuso per la somma),
	e il prodotto di un elemento di $\mathbb{K}$ per un
	elemento di $V$ (il cui risultato \`e sempre un elemento
	di $V$, si dice quindi che V \`e chiuso per il prodotto con
	elementi di $\mathbb{K}$) che verificano le seguenti
	\textbf{propriet\`a}:

	\begin{enumerate}
		\item
		      $\forall u, v, w \in V$ vale
		      \[ (u + v) + w = u + (v + w) \]
		      (propriet\`a associativa dell'addizione)
		\item
		      $\forall v, w \in V$ vale
		      \[ v + w = w + v \] (propriet\`a commutativa della
		      somma)
		\item
		      $\exists O \in V$ tale che $\forall v \in V$
		      vale \[ v + O = v \] (elemento neutro somma)
		\item
		      $\forall v \in V$, $\exists w \in V$
		      tale che \[v + w = O\] (opposto per la somma)
		\item
		      $\forall \lambda, \mu \in \mathbb{K}$ e $\forall v, w \in V$ vale
		      \[ \lambda(v + w) = \lambda v + \lambda w \]
		      e anche
		      \[ (\lambda + \mu)v = \lambda v + \mu v \]
		      (propriet\`a distributiva prodotto per scalare)
		\item
		      $\forall \lambda, \mu \in \mathbb{K}$ e $\forall v \in V$
		      vale \[ (\lambda \mu)v = \lambda(\mu v) \]
		      (propriet\`a associativa prodotto per scalare)
		\item
		      $\forall v \in V$ vale \[ 1v = v \]
		      (invariante moltiplicativo)
	\end{enumerate}
\end{defn}

\begin{observation}
	L'elemento neutro della somma $O$ e lo $0$, elemento neutro di $\mathbb{K}$
	sono due cose ben distinte, il primo è un vettore, il secondo è uno scalare.
\end{observation}

\begin{example}
	Ogni campo $\mathbb{K}$ \`e uno spazio vettoriale su $\mathbb{K}$
	stesso con le operazioni di somma vettoriale e prodotto per scalare
	che sono definite identiche alle operazioni di somma e prodotto sul
	campo. In particolare $\mathbb{R}$ \`e uno spazio vettoriale
	su $\mathbb{R}$, cos\`i come $\mathbb{Q}$ \`e uno spazio vettoriale
	su $\mathbb{Q}$.
\end{example}

\begin{example}
	$\mathbb{R}^2 = \{(a, b) \mid a,b \in \mathbb{R}\}$ \`e uno spazio
	vettoriale su $\mathbb{R}$ con le operazioni di somma vettoriale
	e prodotto scalare definite come segue:
	\begin{gather*}
		(a,b) + (c,d) = (a + c, b + d) \\
		\lambda (a, b) = (\lambda a, \lambda b)
	\end{gather*}
\end{example}

\begin{example}
	Anche l'insieme dei polinomi $\mathbb{K}[x]$, con la somma tra
	polinomi e il prodotto tra polinomi e costanti di $\mathbb{K}$
	definiti come segue:
	\begin{itemize}
		\item
		      Il polinomio somma di $p(x)$ e $q(x)$ \`e quello il cui
		      coefficiente di grado $n$ \`e la somma dei coefficienti di grado
		      $n$ dei polinomi $p(x)$ e $q(x)$.
		\item
		      Il polinomio prodotto di $k \in \mathbb{K}$ e $p(x)$ \`e il
		      polinomio che ha come coefficiente di grado $n$ $k$ volte il
		      coefficiente di grado $n$ di $p(x)$.
	\end{itemize}
	\`e uno spazio vettoriale su $\mathbb{K}$.
\end{example}