
\subsection{Sottospazi vettoriali}

\begin{defn}
	Un \textbf{sottospazio vettoriale} $W$ di $V$ \`e un sottoinsieme di $V$
	che (rispetto alle operazioni $+$ e $\cdot$ che rendono $V$
	uno spazio vettoriale su $\mathbb{K}$) \`e uno spazio
	vettoriale su $\mathbb{K}$.
\end{defn}

\begin{example}
	Dato uno spazio vettoriale $V$ su un campo $\mathbb{K}$, $V$ e l'insieme
	${O}$ sono sempre sottospazi di $V$.
\end{example}

\begin{defn}
	Chiamiamo \textbf{sottospazio proprio} di $V$ un qualsiasi sottospazio
	vettoriale di $V$ che sia diverso da $V$ e dal sottospazio ${O}$.
\end{defn}

\begin{proposition}
	Dato uno spazio vettoriale $V$ su $\mathbb{K}$ e $W \subseteq V$, $W$ \`e
	sottospazio vettoriale di $V$ (rispetto alle operazioni $+$ e $\cdot$ che
	rendono $V$ uno spazio vettoriale su $\mathbb{K}$) se e solo se:
	\begin{enumerate}
		\item Il vettore $O$ appartiene a $W$
		\item $\forall u, v \in W$ vale $u + v \in W$
		\item $\forall k \in \mathbb{K} \and \forall u \in W$ vale $ku \in W$
	\end{enumerate}
\end{proposition}

\begin{example}
	Consideriamo lo spazio vettoriale $\mathbb{R}^2$ su $\mathbb{R}$
	e proviamo vedere se l'insieme
	$X = \{\forall x,y \in \mathbb{R} \mid x^2 + y^2 = 1\}$
	\`e un sottospazio vettoriale di $\mathbb{R}^2$.
	L'insieme in questione \`e l'insieme di punti di una circonferenza.
	Subito notiamo che il vettore $(0, 0)$ non appartiene all'insieme
	dunque possiamo subito concludere che $X$ non \`e un sottospazio di
	$\mathbb{R}^2$.
\end{example}

\begin{observation}
	Tutte le rette passanti per l'origine sono gli unici sottospazi
	vettoriali di $\mathbb{R}^2$. Tutti gli altri sottoinsiemi
	non sono chiusi per somma e prodotto.
\end{observation}

\begin{example}
	Consideriamo il sottoinsieme $L$ di $\mathbb{K}[x]$ che contiene tutti
	e soli i polinomi che hanno radice $1$, ovvero:
	\begin{equation*}
		L = \{p(x) \in \mathbb{K}[x] \mid p(1) = 0\}
	\end{equation*}
	Verifichiamo che $L$ \`e sottospazio vettoriale di $\mathbb{K}[x]$.
	\begin{itemize}
		\item
		      Il polinomio $0$, che \`e il vettore $O$ di $\mathbb{K}[x]$,
		      appartiene a $L$, infatti ha $1$ come radice (addirittura ogni
		      elemento di $\mathbb{K}$ \`e una radice di $0$).
		\item
		      Se $p(x), q(x) \in L$ allora $(p + q)(x)$ appartiene a $L$,
		      infatti:
		      \begin{equation*}
			      (p + q)(1) = p(1) + q(1) = 0 + 0 = 0
		      \end{equation*}
		\item
		      Se $p(x) \in L$ e $k \in \mathbb{K}$ allora $k \cdot p(x) \in L$,
		      infatti:
		      \begin{equation*}
			      (k \cdot p)(1) = k \cdot p(1) = k \cdot 0 = 0
		      \end{equation*}
	\end{itemize}
\end{example}