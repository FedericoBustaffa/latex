\section{Intersezione e somma di sottospazi vettoriali}
Dati due sottospazi vettoriali $U$ e $W$ di uno spazio vettoriale $V$, la somma è il più piccolo sottospazio
vettoriale di $V$ che contenga sia $U$ che $W$ mentre l'intersezione è il più grande sottospazio vettoriale
di $V$ contenuto sia in $U$ che in $W$.

\begin{proposition}
	Sia $V$ uno spazio vettoriale su un campo $\K$, $U$ e $W$ due sottospazi di $V$, allora $U \cap W$
	è un sottospazio vettoriale di $V$.
	\begin{proof}
		Ci interessa verificare che $U \cap W$ verifichi le proprietà della definizione:
		\begin{enumerate}
			\item $O \in U \cap W$, infatti essendo $U$ e $W$ due sottospazi, certamente $O \in U$ e $O \in W$.
			\item Siano $v_1, v_2 \in U \cap W$ allora:
			      \[
				      \begin{cases}
					      v_1 + v_2 \in U \\
					      v_1 + v_2 \in W
				      \end{cases}
				      \Rightarrow{v_1 + v_2 \in U \cap W}
			      \]
			\item Sia $v \in U \cap W$ allora $\forall \lambda \in \K$ si ha:
			      \[
				      \begin{cases}
					      \lambda v \in U \\
					      \lambda v \in W
				      \end{cases}
				      \Rightarrow{\lambda v \in U \cap W}
			      \]
		\end{enumerate}
	\end{proof}
\end{proposition}

Per cercare il più piccolo sottospazio contenente sia $U$ che $W$ verrebbe da pensare all'unione insiemistica,
tuttavia, in generale non è vero che $U \cup W$ è un sottospazio vettoriale di $V$.

Dunque il più piccolo sottospazio vettoriale di $V$ che contiene sia $U$ che $W$ deve necessariamente (per
essere chiuso per la somma) contenere tutti gli elementi della forma $u + v$ dove $u \in U$ e $w \in W$.

\begin{example}
	Provare che se $V = \R^2$ e $U$ e $W$ sono due rette distinte passanti per $O$, allora $U \cup W$
	non è un sottospazio di $V$.

	Basta mostrare che, presi $u \in U$ e $w \in W$, entrambi diversi dall'origine, $v + w$ non appartiene
	all'unione $U \cup W$.
\end{example}

\begin{definition}
	Dati due sottospazi vettoriali $U$ e $W$ di uno spazio vettoriale $V$ su $\K$, chiamo \textbf{somma}
	di $U$ e $W$ l'insieme
	\[ U + W = \{u + w \mid u \in U, w \in W\} \]
\end{definition}

\begin{proposition}
	Dati due sottospazi vettoriali $U$ e $W$ di uno spazio vettoriale $V$ su $\K$, $U + W$ è un
	sottospazio vettoriale di $V$ (ed è il più piccolo contenente $U$ e $W$).
	\begin{proof}
		Partiamo col dire che $O \in U + W$, infatti $O$ appartiene sia ad $U$ che a $W$. Ora, dati
		$a \in \K$ e $x, y \in U + W$, per definizione di $U + W$ esistono $u_1, u_2 \in U$ e
		$w_1, w_2 \in W$ tali che
		\begin{align*}
			x = & u_1 + w_1 \\ y = & u_2 + w_2
		\end{align*}
		Dunque
		\[ x + y = (u_1 + w_1) + (u_2 + w_2) = (u_1 + u_2) + (w_1 + w_2) \in U + W \]
		e
		\[ ax = a(u_1 + w_1) = au_1 + aw_1 \in U + W \]
	\end{proof}
\end{proposition}
