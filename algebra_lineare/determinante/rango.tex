
\subsection{Determinante e calcolo del rango}
Data una matrice $A = (a_{ij}) \in Mat_{n \times m}(\mathbb{K})$ e dato un
numero intero positivo $k$ minore o uguale al minimo fra $m$ e $n$, possiamo
scegliere $k$ righe fra le $n$ righe (diciamo $i_1, i_2, \dots i_k$ e
$k$ colonne fra le $m$ colonne (diciamo $j_1, j_2, \dots, j_k$).

\begin{defn}
	Data la scelta di $k$ righe e $k$ colonne come sopra, chiamiamo
	\textbf{minore} di $A$ di formato $k \times k$, la matrice ottenuta da $A$
	cancellando tutti i coefficienti eccetto quelli che giacciono
	contemporaneamente su una delle $k$ righe e su una delle $k$ colonne scelte,
	in altre parole cancellando tutti i coefficienti eccetto gli $a_{ij}$ per
	cui $i \in \{i_1, i_2, \dots, i_k\}$ e $j \in \{j_1, j_2, \dots j_k\}$.
\end{defn}

\begin{observation}
	Da una matrice $n \times m$ \`e possibile ricavare $\begin{psmallmatrix}
			n \\ k \end{psmallmatrix} \begin{psmallmatrix}
			m \\ k \end{psmallmatrix}$ minori di formato $k \times k$.
\end{observation}

I determinanti dei minori possono essere usati per calcolare il rango di una
matrice $n \times m$ come risulta dal seguente teorema

\begin{theorem}
	Data una matrice $A = (a_{ij}) \in Mat_{n \times m}(\mathbb{K})$, supponiamo
	che esista un minore di formato $k \times k$ il cui determinante \`e diverso
	da 0. Allora il rango di $A$ \`e maggiore o uguale a $k$. Se $k = n$ oppure
	$k = m$ allora il rango di $A$ \`e uguale a $k$. Se $k < n$ e $k < m$ e tutti
	i determinanti dei minori di formato $(k + 1) \times (k + 1)$ sono uguali a 0
	allora il rango di $A$ \`e uguale a $k$.
\end{theorem}

\begin{observation}
	Se una matrice quadrata $n \times n$ ha determinante diverso da zero, allora
	per il teorema precedente ha rango $n$ e dunque \`e invertibile. Sempre per il
	teorema vale anche il viceversa: se una matrice $n \times n$ ha determinante
	uguale a 0, allora il suo rango \`e strettamente minore di $n$ e dunque la
	matrice non \`e invertibile.

	Nel caso $2 \times 2$, data
	\begin{equation*}
		A = \begin{pmatrix}
			a & b \\
			c & d
		\end{pmatrix}
	\end{equation*}
	con determinante diverso da 0, ossia $ad - bc \neq 0$, l'inversa si scrive
	esplicitamente come:
	\begin{equation*}
		A^{-1} = \frac{1}{ad - bc} \begin{pmatrix}
			d  & -b \\
			-c & a
		\end{pmatrix}
	\end{equation*}
\end{observation}

\begin{example}
	La matrice
	\begin{equation*}
		\begin{pmatrix}
			3 & 9          & 4          & 7 & 12         \\
			1 & \textbf{3} & \textbf{2} & 0 & \textbf{5} \\
			1 & \textbf{2} & \textbf{0} & 0 & \textbf{1} \\
			1 & \textbf{4} & \textbf{2} & 7 & \textbf{6}
		\end{pmatrix}
	\end{equation*}
	ha rango maggiore o uguale a 3 in quanto contiene un minore $3 \times 3$
	(quello individuato dalle righe seconda, terza e quarta e dalle colonne
	seconda terza e quinta) che ha determinante diverso da 0. Inoltre, poich\'e
	tutti i minori $4 \times 4$ hanno determinante uguale a zero, la matrice ha
	rango esattamente 3. \`E pi\`u rapido osservare che la prima riga \`e uguale
	alla somma delle altre delle altre tre righe, dunque il rango \`e minore o
	uguale a 3: poich\'e dal calcolo del determinante del minore $3 \times 3$
	sappiamo che il rango \`e maggiore o uguale a 3, allora \`e esattamente 3.
\end{example}

Il calcolo del rango attraverso il teorema precedente pu\`o richiedere molte
verifiche, ed \`e in generale meno conveniente della riduzione di Gauss. Il
seguente teorema pu\`o comunque ridurre i determinanti da calcolare:

\begin{theorem}[Teorema degli orlati]
	Data $A = (a_{ij}) \in Mat_{n \times m}(\mathbb{K})$, supponiamo
	che esista un minore $K$ di formato $k \times k$ il cui determinante \`e
	diverso da 0. Se sono uguali a 0 tutti i determinanti dei minori di formato
	$(k + 1) \times (k + 1)$ che si ottengono aggiungendo una riga e una colonna
	a quelle scelte per formare il minore $K$, allora il rango di $A$ \`e uguale
	a $k$, altrimenti \`e strettamente maggiore di $k$.
\end{theorem}

Dunque se abbiamo un minore di formato $k \times k$ con determinante diverso da 0
e vogliamo decidere se la matrice ha rango $k$ oppure ha rango strettamente
maggiore di $k$ basta controllare $(n - k)(m - k)$ minori di formato
$(k + 1) \times (k + 1)$, non tutti i
$\begin{psmallmatrix} n \\ k + 1 \end{psmallmatrix}$
$\begin{psmallmatrix} m \\ k + 1 \end{psmallmatrix}$ minori di formato
$(k + 1) \times (k + 1)$.