\section{Proprietà del determinante rispetto alle mosse di riga e colonna}
Ricordiamo le operazioni elementari di colonna:
\begin{itemize}
	\item si somma alla colonna $i$ la colonna $j$ moltiplicata per uno
	      scalare $\lambda$.
	\item si moltiplica la colonna $s$ per uno scalare $k \neq 0$.
	\item si permutano tra loro due colonne, diciamo $i$ e $j$.
\end{itemize}

Vediamo ora come cambia il determinante se facciamo un'operazione elementare di colonna
su una matrice $A \in \Mat_{n \times n}(\K)$.
\begin{itemize}
	\item Un'operazione del primo tipo equivale a moltiplicare $A$ a destra per la
	      matrice $n \times n$ che ha tutti 1 sulla diagonale, e 0 in tutte le altre
	      caselle eccetto che nella casella identificata da "riga $j$, colonna $i$",
	      dove troviamo $\lambda$. La matrice $M_{ij}$ è triangolare e il suo
	      determinante è uguale a 1, dunque $\det(AM_{ij})$ è uguale a $\det(A)$ per
	      il teorema di Binet.
	\item Un'operazione di colonna del terzo tipo ha come
	      effetto quello di cambiare il segno del determinante.
	\item Quanto alle operazioni del secondo tipo, dalla definizione stessa di
	      determinante si ricava che, se si moltiplica una colonna per uno scalare
	      $k \neq 0$, anche il determinante della matrice risulterà moltiplicato
	      per $k$.
\end{itemize}
Considerazioni analoghe valgono, ovviamente, per le operazioni elementari di riga.

Una conseguenza di queste osservazioni è che se vogliamo sapere se il
determinante di una certa matrice è uguale a 0 oppure no possiamo,
prima di calcolarlo, fare alcune operazioni di riga e/o di colonna.
Di solito questo risulta utile se con tali operazioni otteniamo una riga
(o una colonna) con molti coefficienti uguali a 0, facilitando il calcolo.