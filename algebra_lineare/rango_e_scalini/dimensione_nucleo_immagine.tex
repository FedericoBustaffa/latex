
\subsection{Il teorema della dimensione del nucleo e dell'immagine di una
	applicazione lineare}
Il teorema del completamento, ha come importante corollario un risultato che stabilisce
una relazione tra la dimensione del nucleo e quella dell'immagine di una applicazione
lineare.

\begin{theorem}
	Considerata una applicazione lineare $L : V \to W$, dove $V$ e $W$ sono spazi
	vettoriali su $\mathbb{K}$, vale
	\begin{equation*}
		dim(Ker(L)) + dim(Imm(L)) = dim(V)
	\end{equation*}
\end{theorem}

\begin{defn}
	Una applicazione lineare bigettiva $L : V \to W$, tra due spazi vettoriali $V$ e $W$
	sul campo $\mathbb{K}$, si dice un \textbf{isomorfismo lineare}.
	
	Dal teorema precedente segue che:
	\begin{itemize}
		\item Se $L : V \to W$ \`e una applicazione lineare iniettiva allora
		      \[ dim(Imm(L)) = dim(V) \]
		      Infatti sappiamo che $Ker(L) = \{O\}$ dunque $dim(Ker(L)) = 0$.
		\item Se $L : V \to W$ \`e un isomorfismo lineare allora \[dim(V) = dim(W)\]
		      Infatti se $L$ \`e bigettiva, in particolare \`e iniettiva e surgettiva.
		\item Se $L : V \to W$ \`e una applicazione lineare iniettiva, allora $L$,
		      pensata come applicazione da $V$ ad $Imm(L)$, \`e un isomorfismo lineare.
	\end{itemize}
\end{defn}