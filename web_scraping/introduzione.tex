\chapter{Introduzione}

Il corso tratterà le modalità di reperimento e analisi di dati da blockchain tramite Python. Si compirà un'analisi
statistica sui dati raccolti tramite \emph{scraping} o API di vario genere per riuscire ad estrapolare informazioni
sulle blockchain prese in esame.

Saranno necessarie nozioni di statistica descrittiva ed inferenziale e si farà uso di strumenti per l'analisi di
grafi.

Le librerie Python necessarie per lavorare al meglio in questo ambito sono:
\begin{itemize}
	\item \verb|BeautifulSoup|, \verb|Scrapy|: web scraping
	\item \verb|Pandas|: gestione di tabelle
	\item \verb|Numpy|: algebra vettoriale e matriciale
	\item \verb|SciPy|: statistica
	\item \verb|Matplotlib|: visualizzazione di grafici
	\item \verb|NetworkX|: analisi di grafi
\end{itemize}
Per installare le librerie scaricare il package manager \verb|pip| di Python:
\begin{verbatim}
	> sudo apt install python3-pip
\end{verbatim}
Installazione delle librerie
\begin{verbatim}
	> pip install beautifulsoup4
	> pip install scrapy
	> pip install pandas
	> pip install numpy
	> pip install scipy
	> pip install matplotlib
	> pip install networkx
\end{verbatim}

\section{Blockchain}
Una \textbf{blockchain} utilizza un modello computazionale differente dal classico modello \textbf{client-server},
si basa infatti sul sistema \textbf{peer-to-peer}.

In questo modello ogni nodo possiede una frazione del potere necessario a compiere determinate azioni all'interno
della blockchain. Il sistema nasce infatti per eliminare l'entità centrale che controlla e gestisce l'intero servizio.

Possiamo vedere una blockchain come un \textbf{registro} \emph{distribuito} e \emph{replicato} tra i nodi di una
rete \emph{peer-to-peer}. In altre parole ogni nodo della rete possiede una copia della blockchain uguale a quella
di tutti gli altri nodi.

Una proprietà fondamentale al fine di mantenere la blockchain consistente e \textbf{immutabile} è la
\textbf{tamper freeness} garantita tramite meccanismi crittografici e algoritmi basati sul consenso.

