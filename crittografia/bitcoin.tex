\chapter{Bitcoin}
Le normali valute come euro, dollari e cos\`i via sono gestite da una banca centrale, come la BCE per l'euro, la quale
gestisce il conio e altri parametri andando a regolare l'inflazione ecc.

Senza entrare nei dettagli finanziari, possiamo dunque dire che le valute comuni sono in mano a sistemi
\textbf{centralizzati}, i quali gestiscono tutto ci\`o che abbiamo detto prima.

Il sistema dei \textbf{bitcoin} nasce dall'esigenza di creare un sistema di valute digitali \textbf{decentralizzato},
ossia un sistema che \`e gestito solo dai suoi utenti.

\section{Introduzione}
Prima di addentrarci nell'argomento chiariamo cosa sia una \emph{valuta digitale} e una \emph{criptovaluta}:
\begin{itemize}
	\item \textbf{Valuta digitale}: valuta che esiste soltanto in forma digitale e che dunque, \`e utilizzabile
	      soltanto da un calcolatore.
	\item \textbf{Criptovaluta}: valuta digitale resa sicura grazie a tecniche di crittografia che rendono quasi
	      impossibile spendere due volte la stessa moneta.
\end{itemize}
Per capire il funzionamento di bitcoin dobbiamo prima introdurre due concetti: la transazione e il registro.
\begin{itemize}
	\item \textbf{Transazione}: passaggio di denaro fra due utenti.
	\item \textbf{Registro}: anche detto \textbf{libro contabile}, \`e un documento per il monitoraggio delle
	      transazioni.
\end{itemize}
Bitcoin \`e una rete \emph{Peer-to-Peer} di nodi su cui gira il software \emph{bitcoin core}, tramite il quale
i vari nodi riescono a comunicare tra di loro e a decidere come gestire le transazioni tra le monete.

Ogni nodo memorizza il registro di tutti i membri della rete. Quest'ultimo dev'essere quindi aggiornato periodicamente
in modo che sia uguale per tutti gli utenti. Affinch\'e questo avvenga il procedimento \`e il seguente
\begin{enumerate}
	\item Si sceglie un nodo \textbf{leader} tramite consenso: i nodi competono tra di loro cercando di risolvere un
	      problema complesso. In media viene scelto un nuovo \emph{leader} ogni 15 minuti.
	\item Il \emph{leader} propone una nuova pagina del libro contabile sulla base delle varie richieste di transazione
	      che gli utenti hanno fatto durante il periodo di selezione del \emph{leader}.
	\item Tutti i computer ricevono la pagina e, tramite la loro copia del registro, controllano che le transazioni
	      inserite siano corrette. Se lo sono accettano la pagina, altrimenti la rifiutano e si cerca un altro
	      \emph{leader}.
	\item Se la pagina viene accettata tutti i computer aggiornano il loro registro e il \emph{leader} viene ricompensato
	      con un premio in bitcoin.
\end{enumerate}

\section{Blockchain}
Ogni pagina del registro \`e collegata alla precedente come in una lista linkata. Ogni pagina, detta \textbf{blocco}
(da qui il nome \textbf{blockchain}), ha un \textbf{header}, in cui sono presenti informazioni necessarie al
mantenimento della struttura dati, e un \textbf{body}, dove sono memorizzate le transazioni.

Nell'\emph{header} sono presenti quattro parametri principali
\begin{itemize}
	\item \textbf{Timestamp}: chi crea la pagina mette un \emph{timestamp} nell'\emph{header} relativo al momento
	      della sua creazione.
	\item \textbf{Nonce}: \`e il frutto della competizione di cui abbiamo parlato prima, ossia il valore ricavato
	      dalla risoluzione del problema posto. Come vedremo pi\`u avanti si tratta del risultato dell'operazione
	      detta \textbf{mining}.
	\item \textbf{Merkle Root}: \`e la radice di un albero di Merkle costruito calcolando l'hash delle transazioni
	      e serve a controllare che una certa transazione, all'interno della pagina, sia integra.
	\item \textbf{Hash Previous Block}: \`e il valore \emph{hash} dell'\emph{header} del blocco precedente.
\end{itemize}

\subsection{Timestamp}
Il \textbf{timestamp} \`e molto importante per riuscire a dare un ordine temporale alle varie pagine il che aiuta nella
verifica della correttezza delle stesse.

\subsection{Nonce}
Come gi\`a anticipato, all'interno dell'\emph{header} di un blocco della \emph{blockchain}, \`e presente un valore
\textbf{nonce}. Tale valore \`e il frutto della competizione a cui gli utenti partecipano per diventare \emph{leader}.

La competizione consiste nel trovare il valore \emph{nonce}, tale che la funzione hash crittografica SHA256 applicata
sull'header del blocco e sul \emph{nonce} sia inferiore di un certo valore detto \textbf{difficulty target}.
\begin{center}
	\verb|SHA256(header, nonce) < difficulty_target|
\end{center}
Pi\`u il \emph{difficulty target} \`e piccolo, pi\`u \`e difficile trovare un \emph{nonce} che soddisfi la disequazione.

Quello che si fa in pratica \`e un forza bruta sui valori del \emph{nonce} fich\'e non si trova quello giusto. Ecco
perch\'e questo procedimento, in gergo, si chiama \textbf{mining}.

Il primo utente che trova un \emph{nonce} valido lo mostra agli altri utenti che ne verificano la correttezza. Se
il valore \`e corretto allora l'utente diventa il nuovo \emph{leader} e propone una nuova pagina del registro contabile.

\subsection{Merkle Root}
Un altro valore presente nell'\emph{header} di un blocco \`e la \textbf{Merkle root}, ossia la radice di un
\textbf{albero di Merkle} calcolato sulle transazioni di un blocco:
\begin{enumerate}
	\item Si calcola l'hash delle singole transazioni.
	\item I valori hash calcolati vengono concatenati a due a due.
	\item Si calcola l'hash di tutte le concatenazioni.
	\item Si continua finch\'e non rimane un solo valore, ossia la \emph{Merkle root}.
\end{enumerate}
Il risultato \`e un albero binario bilanciato che ha come foglie il valore hash delle transazioni e dove il nodo padre
di un generico nodo \`e il valore hash della concatenazione dei due nodi figli. La radice dell'albero \`e ovviamente la
\emph{Merkle root}.

La \emph{Merkle root} serve a controllare l'integrit\`a di una transazione: all'utente viene fornita la
\emph{Merkle root} e vengono forniti i valori hash necessari a calcolarla. Se l'utente, calcolando la \emph{Merkle root}
ottiene la \emph{Merkle root} fornita dal software allora sa che la transazione \`e corretta.

Quando il \emph{leader} deve creare il blocco con le nuove transazioni calcola il nuovo \emph{albero di Merkle} e quindi
la nuova \emph{Merkle root} che va ad inserire nell'\emph{header} del blocco.

Gli altri utenti ricalcolano a loro volta l'albero e controllano che la radice sia corretta.

\subsection{Hash Previous Block}
Questo valore serve a rendere la \emph{blockchain} immutabile. Se provassimo a modificare una o pi\`u transazioni di
un blocco della \emph{blockchain}, la \emph{Merkle root} nell'\emph{header} del blocco non sarebbe pi\`u consistente
con quest'ultimo.

Ricalcoliamo allora l'albero e di conseguenza la nuova radice. A questo punto per\`o \`e praticamente certo che il
valore del \emph{nonce} violi la disequazione e dobbiamo quindi ricalcolarlo.

Supponiamo di riuscire a calcolare un nuovo \emph{nonce}: il valore hash dell'\emph{header} della pagina corrente
(presente nel campo \emph{hash previous block} dell'\emph{header} della pagina successiva) non sar\`a pi\`u consistente
con la pagina corrente.

Si innesca cos\`i un processo di modifiche a catena fino ad arrivare alla pagina pi\`u recente nella blockchain. Anche
riuscendo a modificare tutti i blocchi necessari e a diventare \emph{leader} nell'ultima competizione, una volta
proposta la nuova pagina, questa verrebbe respinta dagli altri utenti in quanto non sarebbe consistente col registro
in loro possesso.

\section{Transazioni}
Nel \emph{body} di un blocco sono salvate tutte le \textbf{transazioni} effettuate fino a quel momento. Una transazione
\`e un invio di bitcoin da un utente $A$ ad un utente $B$ ma in ambito bitcoin sono pi\`u complicate e fanno uso di
firme digitali che hanno la propriet\`a di
\begin{itemize}
	\item \textbf{Autenticazione}: ogni volta che si effettua una transazione si deve essere in possesso di una coppia
	      di chiavi, privata e pubblica, che servono a produrre una firma digitale.
	\item \textbf{Integrit\`a}: una transazione non deve essere modificata da qualche attacco attivo o da problemi di
	      rete.
	\item \textbf{Non ripudio}: non si pu\`o negare di aver effettuato una transazione.
\end{itemize}
Supponiamo di avere due utenti $A$ e $B$ i quali possiedono rispettivamente le chiavi private $k_{A\text{ [priv]}}$ e
$k_{B\text{ [priv]}}$ e le chiavi pubbliche $k_{A\text{ [pub]}}$ e $k_{B\text{ [pub]}}$ e siano $\text{Addr}_A$ e
$\text{Addr}_B$ rispettivamente i loro indirizzi, una transazione da $A$ a $B$ avviene in questo modo:
\begin{enumerate}
	\item $A$ firma la transazione con la sua chiave privata.
	\item $A$ invia il crittogramma a $B$.
	\item $B$ verifica che il messaggio sia inviato da $A$ tramite la chiave $k_{A\text{ [pub]}}$.
\end{enumerate}
Le transazioni vengono effettuate tramite programmi non Turing completi e la loro verifica coinvolge una terza parte
fidata che fa da \emph{garante}.

La transazione non avviene direttamente da $A$ a $B$ ma si invia il messaggio relativo alla transazione ad un
\textbf{indirizzo multisignature} che coinvolge questa terza parte fidata, la quale controlla che la transazione
sia correttamente effettuata e la firma.

Una transazione relativa ad un certo utente \`e strutturata come un blocco in cui sono presenti tre parametri
fondamentali:
\begin{itemize}
	\item \textbf{Input}: somma dei bitcoin ricevuti.
	\item \textbf{Output}: somma dei bitcoin inviati.
	\item \textbf{Unspent Transaction Output}: bitcoin non spesi che vengono comunque mandati in output verso se stessi
	      cos\`i da recuperarli.
\end{itemize}
Il risultato \`e la differenza tra input e output.

In definitiva, nessuno possiede dei bitcoin ma si possiede una chiave privata che consente di spendere bitcoin inviati
ad un certo indirizzo relativo ad una chiave pubblica. Se si perde la chiave privata si perdono tutti i bitcoin legati
ad essa.
