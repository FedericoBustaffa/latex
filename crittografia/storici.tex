\chapter{Cifrari storici}
In questo capitolo andremo a trattare i \textbf{cifrari storici}, chiamati con questo nome perch\'e ad oggi sono stati
tutti forzati e dunque non sono pi\`u cifrari sicuri.

I primi cifrari nascono in un periodo in cui cifratura e decifrazione si facevano "con carta e penna" o quasi e servivano
per cifrare frasi di senso compiuto in linguaggio naturale.

Da un certo punto in poi tutti i cifrari hanno cercato di seguire i cosiddetti \textbf{principi di Bacone}
\begin{itemize}
	\item Le funzioni $C$ e $D$ devonon essere \textbf{facili da calcolare}.
	\item \`E \textbf{impossibile} ricavare la $D$ se la $C$ non \`e nota.
	\item Il crittogramma $c = C(m)$ deve apparire "\textbf{innocente}", deve sembrare cio\`e un testo in chiaro e non
	      una sequenza di caratteri insensata.
\end{itemize}

\section{Cifrario di Cesare}
L'idea di base \`e che il crittogramma $c$ \`e ottenuto dal messaggio in chiaro $m$ sostituendo ogni lettera di $m$ con
quella tre posizioni pi\`u avanti nell'alfabeto.

\begin{center}
	A B C D \dots W X Y Z
	\[ \downarrow \]
	D E F G \dots Z A B C
\end{center}
La decifrazione \`e immediata: basta sostituire ogni lettera del crittogramma con la lettera tre posizioni pi\`u indietro
nell'alfabeto.

\`E un cifrario molto semplice e non utilizza una chiave di cifratura. Una volta scoperto il metodo di cifratura e
decifrazione diventa del tutto inutile.

\subsection{Cifrario di Cesare generalizzato}
Per rendere il cifrario un po' pi\`u robusto basterebbe inserire un chiave $1 \leq k \leq 25$ (26 lascia inalterato il
messaggio) e invece di traslare sempre di tre posizioni le lettere del messaggio in chiaro, le trasliamo di $k$
posizioni.

Ovviamente i due utenti devono possedere la solita chiave per cifrare e decifrare i messaggi.

Con un calcolatore moderno \`e immediato effettuare un attacco a forza bruta: si provano tutte le 25 chiavi.

\subsubsection{Cifratura e decifrazione}
Sia $x$ una lettera dell'alfabeto, $pos(x)$ la sua posizione nell'alfabeto e $k$ la chiave tale che $1 \leq k \leq 25$.

La funzione di cifratura ritorna la lettera $y$ tale che
\[ pos(y) = (pos(x) + k) \mod{26} \]

La funzione di decifrazione ritorna la lettera $x$ tale che
\[ pos(x) = (pos(y) - k) \mod{26} \]