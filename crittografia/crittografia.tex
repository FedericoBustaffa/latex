\documentclass[11pt]{report}

\usepackage[italian]{babel}
\usepackage[utf8]{inputenc}
\usepackage[T1]{fontenc}
\usepackage[hidelinks]{hyperref}
\usepackage{amsmath, amssymb, amsthm, mathtools}
\usepackage{enumitem}

\usepackage{../packages/codestyle}

% --------------- STYLE ---------------
\usepackage[margin=1.25in]{geometry}
\usepackage{xcolor}

\usepackage{fancyhdr}
\usepackage[Sonny]{fncychap}
\usepackage[most]{tcolorbox}

% Font
\renewcommand{\familydefault}{\sfdefault}
\usepackage{sansmath}
\sansmath

\pagestyle{fancy}
\setlength{\headheight}{15pt}
\rhead{\thepage}


% --------------- MATH ---------------
\usepackage{amsmath, amssymb, amsthm, amsfonts, mathtools}

\usepackage{mdframed}
\newtheoremstyle{th_style}
{0pt}{0pt}
{\normalfont}
{}
{\color{green!40!black}}
{\;}{0.25em}
{\thmname{\textbf{#1}}\thmnumber{ \textbf{#2}}{\color{black}\thmnote{\textbf{ -- #3}}}}

\newmdenv[
	rightline=false,
	leftline=true,
	topline=false,
	bottomline=false,
	linecolor=green!40!black,
	innerleftmargin=5pt,
	innerrightmargin=5pt,
	innertopmargin=0pt,
	innerbottommargin=0pt,
	leftmargin=0cm,
	rightmargin=0cm,
	linewidth=4pt
]{dBox}

\newmdenv[
	rightline=false,
	leftline=true,
	topline=false,
	bottomline=false,
	linecolor=green!40!black,
	backgroundcolor=black!5,
	innerleftmargin=5pt,
	innerrightmargin=5pt,
	innertopmargin=5pt,
	innerbottommargin=5pt,
	leftmargin=0cm,
	rightmargin=0cm,
	linewidth=4pt
]{pBox}

\theoremstyle{th_style}
\newtheorem{theoremeT}{Teorema}[chapter]
\newtheorem{definitionT}{Definizione}[chapter]
\newtheorem{propositionT}{Proposizione}[chapter]
\newtheorem{corollary}{Corollario}[chapter]
\newtheorem{lemma}{Lemma}[chapter]
\newtheorem{observation}{Osservazione}[chapter]
\newtheorem{exampleT}{Esempio}[section]

\newenvironment{theorem}{\begin{pBox}\begin{theoremeT}}{\end{theoremeT}\end{pBox}}
\newenvironment{definition}{\begin{dBox}\begin{definitionT}}{\end{definitionT}\end{dBox}}
\newenvironment{proposition}{\begin{pBox}\begin{propositionT}}{\end{propositionT}\end{pBox}}
\newenvironment{example}{\begin{dBox}\begin{exampleT}}{\end{exampleT}\end{dBox}}


\title{Crittografia}
\author{Federico Bustaffa}
\date{14/02/2022}
	
\begin{document}
\maketitle
\tableofcontents

\chapter{Introduzione}
Tutti i calcolatori moderni possono essere schematizzati semplicemente tramite il cosiddetto
\textbf{modello Von Neumann} in cui abbiamo tre componenti principali: \textbf{memoria},
\textbf{processore} e canali di \textbf{I/O}.

\begin{center}
	\begin{tikzpicture}[scale=1.5]
		\node[draw] (mem) at (0, 0) {Memoria};
		\node[draw] (cpu) at (0, -1.5) {CPU};
		\node[draw] (io) at (2, -1.5) {I/O};

		\draw[<->] (mem) -- (cpu);
		\draw[<->] (cpu) -- (io);
	\end{tikzpicture}
\end{center}

Come possiamo vedere dalla figura i collegamenti tra le varie entità sono bidirezionali. Il
processore (ma anche la memoria) è collegato ai canali di I/O e il collegamento che c'è tra memoria
e processore è chiamato \textbf{Von Neumann bottleneck}. Quello che accade tra memoria e processore
è, grosso modo, quello che viene descritto dal seguente pseudocodice ed è denominato ciclo di
\textbf{fetch-decode-execute}.

\begin{minted}{c}
while (true) {
	istr = M[PC]
	decode(istr)
	res = exec(istr)
	update(PC)
	writeback(res)
	interrupt_handling()
}
\end{minted}

In pratica viene estratta dalla memoria l'istruzione puntata da un \textbf{Program Counter}, la
si decodifica e la si esegue. In seguito il Program Counter viene aggiornato e i risultati vengono
consolidati nei registri della CPU oppure in memoria.

Per ognuna delle istruzioni eseguite sul canale presente tra processore e memoria vengono svolte
le seguenti operazioni
\begin{enumerate}
	\item Il processore manda un indirizzo (PC) dove andare a prendere l'istruzione.
	\item La memoria risponde con un'istruzione.
	\item Il processore esegue e durante l'esecuzione può richiedere la lettura o scrittura dalla
	      memoria.
\end{enumerate}
Questo processo avviene molto spesso e rende il traffico sul canale molto intenso creando per
l'appunto un \emph{collo di bottiglia}. Questo è dovuto alla rapida evoluzione dei processori e
alla non altrettanto rapida evoluzione della memoria, la quale ha una velocità nel fornire le
istruzioni minore della velocità che impiega il processore ad eseguirle e richiederne altre.

Per misurare le performance di un processore siamo abituati a ragionare in termini di
\textbf{cicli di clock}. Dato che, approssimativamente, possiamo dire che un ciclo di clock
corrisponde ad un'istruzione eseguita, se ad esempio abbiamo un processore a 1 GHz, questo riuscirà
ad eseguire un'istruzione in 1 nanosecondo.

Durante il corso verrano trattate varie architetture, dalle prime progettate alle più recenti,
cercando di capire come funzionano e i loro pregi e difetti. Tra quelle che tratteremo elenchiamo
\begin{itemize}
	\item \textbf{Single cycle}: modello in cui un ciclo fetch-execute viene effettuato in un ciclo
	      di clock.
	\item \textbf{Pipeline}: modello in cui si hanno componenti distinte per le operazioni di fetch,
	      decode ed execute. In questo modo ogni componente continua a svolgere il proprio lavoro
	      parallelamente alle altre. Una volta arrivato a regime questo processore riesce a
	      eseguire un'istruzione all'inizio di ogni ciclo di clock e non alla fine (come nel
	      modello single cycle).
\end{itemize} 			% INTRODUZIONE
\chapter{Rappresentazione matematica di oggetti}
\section{Metodi di rappresentazione}
\subsection{Alfabeti e sequenze}
Posso definire una alfabeto $\Gamma$ di cardinalit\`a $s$. Nel caso dovessi
rappresentare $N$ oggetti
\begin{itemize}
	\item $d(s, N)$ sarebbe la lunghezza della sequenza pi\`u lunga che rappresenta un
	      oggetto dell'insieme.
	\item $d_{\min}(s, N)$ sarebbe il valore minimo di $d(s, N)$ tra tutte le
	      rappresentazioni possibili.
	\item Un metodo di rappresentazione \`e tanto migliore, quanto pi\`u
	      $d(s, N)$ si avvicina a $d_{\min}(s, N)$.
\end{itemize}

\subsection{Rappresentazione binaria}
In questo caso abbiamo $s = 2$ e l'alfabeto sar\`a $\Gamma = \{ 0, 1 \}$. Questo
alfabeto ha due caratteristiche fondamentali:
\begin{itemize}
	\item Per ogni $k \geq 1$, abbiamo $2^k$ sequenze di lunghezza $k$.
	\item Il numero totale di sequenze lunghe da 1 a $k$ \`e dato da
	      \[
		      \sum_{i = 1}^k 2^i = 2^{k + 1} - 2
	      \]
\end{itemize}
Nel caso in cui avessi $N$ sequenze da rappresentare risulterebbe che
\begin{itemize}
	\item $2^{k + 1} - 2 \geq N$
	\item $k \geq \log_2 (N + 2) - 1$
\end{itemize}
Il numero $\lceil \log_2 N \rceil$ sono i caratteri binari sufficienti per costruire $N$
sequenze differenti. In sostanza posso costruire $N$ sequenze differenti tutte di
$\lceil log_2 N \rceil$ caratteri.

Per esempio se considero $N = 7$ ho che \[ \lceil \log_2 7 \rceil = 3 \] Dunque mi bastano
stringhe binarie di lunghezza 3 per costruire 7 sequenze diverse.
\[ 0 \quad 1 \quad 00 \quad 01 \quad 10 \quad 11 \quad 000 \]

\subsection{Rappresentazione di interi}
La notazione posizionale per rappresentare numeri interi \`e una rappresentazione
efficiente, indipedentemente dalla base $s \geq 2$ scelta.

Un numero $N$ \`e rappresentato con un numero $d$ di cifre tale che
\[ \lceil \log_2 N \rceil \leq d \leq \lceil \log_2 N \rceil + 1 \]
C'\`e quindi una \textbf{riduzione logaritmica} tra il valore $N$ di un numero e la
lunghezza $d$ della sua rappresentazione.

\subsection{Calcolabilit\`a e complessit\`a}
\begin{itemize}
	\item La \textbf{calcolabilit\`a} si occupa delle questioni fondamentali circa la
	      potenza e le limitazioni dei sistemi di calcolo. Si occuopa di definire le
	      nozioni di \textbf{algoritmo} e \textbf{problema non decidibile}. In altre
	      parole si occupa di classificare i problemi in \emph{risolvibili} e
	      \emph{non risolvibili}.
	\item La \textbf{complessit\`a} si dare la nozione di \text{algoritmo efficiente}
	      e di \textbf{problema intrattabile}. In altre parole divide i problemi in
	      \emph{"facili"} e \emph{"difficili"}.
\end{itemize}

\subsection{Insiemi numerabili}
\begin{itemize}
	\item Due insiemi $A$ e $B$ hanno lo stesso numero di elementi se e solo se si pu\`o
	      stabilire una \textbf{corrispondenza biunivoca} tra i loro elementi.
	\item Un insieme \`e \textbf{numerabile} se e solo se i suoi elementi possono essere
	      messi in \textbf{corrispondenza biunivoca con i numeri naturali}.
\end{itemize}

\subsection{Enumerazione delle sequenze}
Se si vogliono elencare in un ordine ragionevole le sequenze di lunghezza finita
costruite su un alfabeto finito ci scontreremmo con un problema: le sequenze non sono
in numero finito, quindi non si potr\`a completare l'elenco.

Lo scopo in questo caso \`e raggiungere qualsiasi sequenza $\sigma$ arbitrariamente
scelta in un numero finito di passi. Per fare ci\`o $\sigma$ deve trovarsi a distanza
finita dall'inizio dell'elenco.

Osserviamo che la numerazione delle sequenze \`e possibile perch\'e esse sono di
lunghezza finita anche se illimitata: scelto $d$ arbitrariamente esistono sequenze
di lunghezza maggiore di $d$. Per sequenze di lunghezza infinita la numerazione non
sarebbe possibile.

\subsection{Il problema dell'arresto}
Si tratta di un algoritmo che indaga sulle proprit\`a di un altro algoritmo, trattato
come dato di input.

Consiste nel verificare se un generico programma termini la sua esecuzione oppure
se entra in un ciclo infinito.

Turing ha tuttavia dimostrato che, riuscire a dimostrare se un programma arbitrario
si arresta e termina la sua esecuzione \`e \textbf{impossibile}.

\begin{theorem}
	Il problema dell'arresto \`e indecidibile.
	\begin{proof}
		Per dimostrarlo proviamo a dobbiamo considerare un generico algoritmo $A$ con
		un generico input $I$. L'algoritmo arresto prende in input $A$ e $I$ e
		ritorna vero se $A$ termina oppure falso se non termina.
	\end{proof}
\end{theorem}
 	% RAPPRESENTAZIONE OGGETTI
\chapter{Teoria della calcolabilit\`a}\label{calcolabilita}
La \textbf{calcolabilit\`a} si occupa delle questioni fondamentali circa la potenza e le limitazioni dei sistemi di
calcolo e cerca di definire le nozioni di \textbf{algoritmo} e di \textbf{problema non decidibile}.

In altre parole si occupa di classificare i problemi in \emph{risolvibili} e \emph{non risolvibili} per via algoritmica.

A differenza invece della \textbf{complessit\`a} che si occupa di definire la nozione di \text{algoritmo efficiente}
e di \textbf{problema intrattabile}.  In altre parole divide i problemi in \emph{facili} e \emph{difficili}.

\section{Insiemi numerabili}\label{insiemi_numerabili}
\begin{itemize}
	\item Due insiemi $A$ e $B$ hanno lo stesso numero di elementi se e solo se si pu\`o stabilire una
	      \textbf{corrispondenza biunivoca} tra i loro elementi.
	\item Un insieme \`e \textbf{numerabile} se e solo se i suoi elementi possono essere messi in
	      \textbf{corrispondenza biunivoca con i numeri naturali}.
\end{itemize}

\subsection{Enumerazione delle sequenze}
Se si volesse elencare, in un ordine ragionevole, le sequenze di lunghezza finita costruite su un alfabeto finito ci
scontreremmo con un problema: le sequenze non sono in numero finito, quindi non si potr\`a completare l'elenco.

Lo scopo in questo caso \`e raggiungere qualsiasi sequenza $\sigma$, arbitrariamente scelta, in un numero finito di passi.
Per fare ci\`o, $\sigma$ deve trovarsi a distanza \emph{finita} dall'inizio dell'elenco.

\subsubsection{Ordinamento canonico}
Per riuscire ad enumerare queste sequenze dobbiamo ricorrere al cosiddetto \textbf{ordinamento canonico}.
\begin{enumerate}
	\item Si ordinano le sequenze in ordine lunghezza crescente.
	\item A parit\`a di lunghezza si ordinano le sequenze secondo l'ordinamento tra i caratteri dell'alfabeto.
\end{enumerate}

A questo punto una sequenza $s$ arbitraria si trover\`a tra quelle di $| s |$ caratteri, in una posizione corrispondente
all'ordine alfabetico relativo all'alfabeto $\Gamma$ che sto utilizzando.

\begin{example}
	Se prendessi un alfabeto composto solo da lettere
	\[ \Gamma = \{ a, b, c, \dots, z \} \]
	e volessi scrivere le sequenze nell'ordine canonico otterrei
	\begin{gather*}
		a, \quad b, \quad c, \quad \dots, \quad z, \quad aa, \quad ab, \quad \dots, \quad az, \\
		ba, \quad \dots, \quad bz, \quad za, \quad \dots, \quad zz, \quad aaa, \quad \dots, \quad zzz, \quad \dots
	\end{gather*}
	come possiamo vedere, si pu\`o costruire un numero infinito di sequenze ma tutte di lunghezza finita e, soprattutto,
	che si possono mettere in corrispondenza biunivoca con l'insieme dei numeri naturali.
\end{example}

Osserviamo che l'enumerazione delle sequenze \`e possibile perch\'e esse sono di lunghezza finita anche se illimitata.
Ovvero, scelto un intero $d$, esistono sempre sequenze di lunghezza maggiore di $d$.

Se le sequenze fossero di lunghezza infinita l'insieme non sarebbe numerabile.

\subsection{Problemi computazionali}
\begin{theorem}
	L'insieme dei \textbf{problemi computazionali} non \`e numerabile.
\end{theorem}

Un problema computazionale pu\`o essere visto come una \emph{funzione matematica} che associa ad ogni insieme di dati,
espressi da $k$ numeri interi, il corrispondente risultato, espresso da $j$ numeri interi.
\[ f : N^k \rightarrow N^j \]
L'insieme delle funzioni $f : N^k \rightarrow N^j$ non \`e numerabile.

\subsection{Diagonalizzazione}
Per dimostrare quanto detto in precedenza, utilizziamo la \textbf{diagonalizzazione}.
\begin{theorem}
	L'insieme di funzioni
	\[ F = \{ f \mid f : N \rightarrow \{0, 1\} \} \]
	nel quale ogni $f \in F$ pu\`o essere rappresentata da una sequenza infinita del tipo
	\[
		\begin{matrix}
			x    &  & 0 & 1 & 2 & \dots & n & \dots \\
			f(x) &  & 0 & 1 & 0 & \dots & 0 & \dots
		\end{matrix}
	\]
	o, se possibile, da una regola finita di costruzione
	\[
		f(x) = \begin{cases}
			0 & x \text{ pari}    \\
			1 & x \text{ dispari}
		\end{cases}
	\]
	\textbf{non \`e numerabile}.
	\begin{proof}
		Vogliamo dimostrare il teorema per assurdo, ammettiamo quindi che $F$ sia numerabile.

		Dato che $F$ \`e enumerabile allora posso assegnare ad ogni funzione $f \in F$ un numero progressivo nella
		numerazione, e costruire una tabella (infinita) di tutte le funzioni.
		\begin{center}
			\begin{tabular}{c | c c c c c}
				$x$      & 0     & 1     & 2     & 3     & \dots \\
				\hline
				$f_0(x)$ & 1     & 0     & 1     & 0     & \dots \\
				$f_1(x)$ & 0     & 0     & 1     & 1     & \dots \\
				$f_2(x)$ & 1     & 1     & 0     & 1     & \dots \\
				$f_3(x)$ & 0     & 1     & 1     & 0     & \dots \\
				\dots    & \dots & \dots & \dots & \dots & \dots
			\end{tabular}
		\end{center}
		Consideriamo adesso la funzione $g \in F$
		\[
			g(x) = \begin{cases}
				0 & f_x (x) = 1 \\
				1 & f_x (x) = 0
			\end{cases}
		\]
		La funzione $g$, cos\`i definita, \textbf{non} corrisponde a nessuna delle $f_i$ in tabella. Questo perch\'e
		differisce sicuramente nei valori posti sulla diagonale principale.
		\begin{center}
			\begin{tabular}{c | c c c c c}
				$x$      & 0     & 1     & 2     & 3     & \dots \\
				\hline
				$f_0(x)$ & 1     & 0     & 1     & 0     & \dots \\
				$f_1(x)$ & 0     & 0     & 1     & 1     & \dots \\
				$f_2(x)$ & 1     & 1     & 0     & 1     & \dots \\
				$f_3(x)$ & 0     & 1     & 1     & 0     & \dots \\
				\dots    & \dots & \dots & \dots & \dots & \dots \\
				\hline
				$g(x)$   & 0     & 1     & 1     & 1     & \dots
			\end{tabular}
		\end{center}
		Ecco che si giunge ad una contraddizione.
	\end{proof}
\end{theorem}

\subsection{Il problema della rappresentazione}
L'informatica rappresenta tutte le sue entit\`a (quindi anche gli algoritmi) in forma digitale, come
\textbf{sequenze finite di simboli di alfabeti finiti}.

Lo stesso vale per gli \textbf{algoritmi}, i quali sono composti da una sequenza finita di operazioni, completamente
e univocamente determinate. Gli algoritmi sono potenzialmente infiniti ma comunque numerabili.

Come gi\`a detto, i problemi computazionali non sono numerabili e dunque abbiamo molti pi\`u problemi che algoritmi.
Questo implica necessariamente che esistano problemi \textbf{privi} di un algoritmo di calcolo per la loro risoluzione.

\section{Il problema dell'arresto}\label{arresto}
Formulato da Turing, consiste in un algoritmo che indaga le proprit\`a di un altro algoritmo, trattato come dato in
input.

Nello specifico il problema \`e cos\`i formulato:
\begin{center}
	Presi ad arbitrio un algoritmo $A$ e i suoi dati di input $D$, decidere in \textbf{tempo finito} se la computazione
	di $A$ su $D$ termina o no.
\end{center}

Sebbene il problema sia lecito, dato che un algoritmo e i relativi dati in ingresso sono codificati con lo stesso
alfabeto, Turing stesso ha dimostrato che \`e \textbf{impossibile} riuscire a stabilire (in tempo finito) se un
programma arbitrario si arresta e termina la sua esecuzione.

\begin{theorem}
	Il problema dell'arresto \`e \textbf{indecidibile}.
	\begin{proof}
		Per dimostrarlo proviamo a dobbiamo considerare un generico algoritmo \verb|A| con un generico input \verb|D|.
		L'algoritmo \verb|ARRESTO| prende in input \verb|A| e \verb|D| e ritorna \verb|true| se \verb|A(D)| termina
		oppure \verb|false| se non termina.

		Se \verb|A(D)| termina \verb|ARRESTO| ritorna \verb|true| e non ci sono problemi. Se invece non termina
		\verb|ARRESTO| non \`e in grado di rispondere \verb|false| in tempo finito.

		Questo perch\'e \verb|ARRESTO| non pu\`o non passare dalla simulazione di \verb|A| su \verb|D| e quindi, nel caso
		\verb|A| non termini, \verb|ARRESTO| non terminerebbe a sua volta.

		Se esistesse l'algoritmo \verb|ARRESTO|, esisterebbe anche il seguente algoritmo
		\begin{lstlisting}[style=pseudo-style]
PARADOSSO(A)
	while (ARRESTO(A, A));
		\end{lstlisting}
		\begin{center}
			\verb|PARADOSSO| termina
			\[ \Leftrightarrow \]
			\verb|ARRESTO(A, A) = 0|
			\[ \Leftrightarrow \]
			\verb|ARRESTO| non termina.
		\end{center}

		Ma cosa succede se provassimo a calcolare \verb|PARADOSSO(PARADOSSO)| ?
		\begin{center}
			\verb|PARADOSSO(PARADOSSO)| termina
			\[ \Leftrightarrow \]
			\verb|ARRESTO(PARADOSSO, PARADOSSO) = 0|
			\[ \Leftrightarrow \]
			\verb|PARADOSSO(PARADOSSO)| non termina
		\end{center}
		Ma ecco che si giunge ad una contraddizione.
	\end{proof}
\end{theorem}
 	% CALCOLABILITA'
\chapter{Teoria della complessit\`a}\label{complessita}
La \textbf{complessit\`a} \`e quella branca dell'informatica teorica che classifica i problemi in base alla
difficolt\`a che si ha nel risolverli, ossia al numero di operazioni necessarie per risolverli.

Nel capitolo \ref{calcolabilita} abbiamo trattato i problemi decidibili e non. In questo capitolo restringiamo il
campo solo ai problemi decidibili ma li dividiamo in due sottocategorie: \textbf{trattabili} e \textbf{intrattabili}.

Si tratta in entrambi i casi di problemi risolvibili, ma nel caso degli intrattabili, il costo computazionale \`e
talmente alto (generalmente esponenziale o pi\`u) da renderli molto simili ai problemi indecidibili, dato che
la risoluzione richiederebbe decine se non centinaia o migliaia (in base alla dimensione dell'input) di anni di
calcolo.

Al contrario, i problemi trattabili, hanno algoritmi di costo polinomiale e dunque sono risolvibili in tempi brevi.

Ci sono infine problemi il cui stato non \`e noto: abbiamo solo algoritmi di costo esponenziale per la loro
risoluzione ma non abbiamo dimostrazione del fatto che siano intrattabili.

\section{Algoritmi polinomiali ed esponenziali}\label{alg_poly_exp}
Studiamo ora la dimensione dei dati trattabili in funzione dell'incremento della velocit\`a dei calcolatori.

\begin{example}
	Siano $C_1$ e $C_2$ due calcolatori, con $C_2$ che \`e $k$ volte pi\`u veloce di $C_1$ e testiamo quanti dati
	riescono a processare i due calcolatori in un tempo $t$.

	Il calcolatore $C_1$ processa $n_1$ dati nel tempo $t$ mentre $C_2$ processa $n_2$ dati sempre in tempo $t$.
	Possiamo osservare che, usare $C_2$, \`e come usare $C_1$ per un tempo $kt$.

	Se l'algoritmo che stiamo usando \`e \textbf{polinomiale} e risolve il problema in un tempo $cn^s$, dove
	$c$ e $s$ sono costanti, abbiamo che
	\[	\begin{array}{llcl}
			C_1 : & c n_1^s = t  & \rightarrow &
			n_1 = \left( \displaystyle\frac{t}{c} \right)^{1/s} \\

			C_2 : & c n_2^s = kt & \rightarrow &
			n_2 = k^{1/s}\left( \displaystyle\frac{t}{c} \right)^{1/s}
		\end{array}
	\]
	Otteniamo quindi
	\[ n_2 = k^{1/s} \cdot n_1 \]
	abbiamo dunque un miglioramento di un \textbf{fattore moltiplicativo} $k^{1/s}$ tanto pi\`u grande quanto pi\`u
	piccolo \`e il grado ($s$) del polinomio.

	Se invece facciamo girare sugli stessi calcolatori un algoritmo di costo esponenziale, che risolve il problema in
	un tempo $c \cdot 2^n$ con $c$ costante, otteniamo
	\[
		\begin{array}{llcl}
			C_1 : & c \cdot 2^{n_1} = t   & \rightarrow & 2^{n_1} = \displaystyle\frac{t}{c}         \\
			C_2 : & c \cdot 2^{n_2} = k t & \rightarrow & 2^{n_2} = k \cdot \displaystyle\frac{t}{c}
		\end{array}
	\]
	Otteniamo quindi
	\[ 2^{n_2} = k \cdot 2^{n_1} \]
	da cui ricaviamo
	\[ n_2 = n_1 + \log k \]
	In questo caso il miglioramento \`e solo di un \textbf{fattore additivo} $\log k$. Questo significa che per
	quanto si incrementi la velocit\`a di elaborazione $k$ della macchina $C_2$ rispetto a $C_1$, i dati elaborati in
	pi\`u sarebbero solo $\log k$.
\end{example}

\section{Problemi}\label{problemi}
Trattiamo ora le varie \textbf{classi di complessit\`a} e come si definiscono. Indichiamo con $\Pi$, un generico
\textbf{problema}, con $I$, l'insieme delle \textbf{istanze} in ingresso e con $S$ l'insieme delle \textbf{soluzioni}.

\subsection{Tipologie di problemi}
Abbiamo tre \textbf{classi} principali di problemi: di \textbf{ricerca}, di \textbf{ottimizzazione} e
\textbf{decisionali}.

\subsubsection{Problemi di ricerca}
Sono problemi in cui, data un'istanza $x$ in input, si richiede di restituire una soluzione $s$, presa da un insieme
$S$ di possibili soluzioni.

Un esempio di problema di ricerca \`e quello in cui si deve calcolare un cammino che unisce due nodi di un grafo. In
questo caso $S$ sar\`a uguale a tutti i possibili cammini che vanno dal nodo di partenza a quello di arrivo.

Le \textbf{istanze positive} saranno le coppie di nodi per le quali esiste almeno un cammino che li unisce, viceversa
le \textbf{istanze negative} saranno le coppie di nodi per le quali non esiste nemmeno un cammino che li unisce.

\subsubsection{Problemi di ottimizzazione}
Nei problemi di ottimizzazione, data un'istanza $x$ in input, si vuole trovare la \textbf{migliore soluzione} $s$ tra
tutte le soluzioni possibili.

In questo caso $S$ sar\`a l'insieme di tutte le soluzioni ottime al problema (diverse fra loro ma con stesso valore).

\subsubsection{Problemi decisionali}
I problemi decisionali richiedono una risposta binaria, in genere \verb|true| o \verb|false|, ossia l'insieme delle
soluzioni \`e
\[ S = \{ 0, 1 \} \]
le \textbf{istanze positive} sono
\[ x \in I \mid \Pi(x) = 1 \]
mentre le \textbf{istanze negative} sono
\[ x \in I \mid \Pi(x) = 0 \]
Tipicamente sono problemi che indagano delle propriet\`a.

In complessit\`a si prende in considerazione solo quest'ultima classe di problemi per due motivi principalmente:
\begin{itemize}
	\item Tutto il tempo di calcolo \`e speso per trovare la soluzione e non per scriverla (la risposta \`e solo 0 o 1).
	\item La complessit\`a di un problema, riformulato in forma decisionale, non cambia rispetto alla sua forma non
	      decisionale.
\end{itemize}
In realt\`a tanti problemi interessanti sono problemi di ottimizzazione e per trattarli dobbiamo riformularli in
forma decisionale. Per farlo, basta verificare l'esistenza di una soluzione (al problema di ottimizzazione) che
soddisfa una certa propriet\`a.

Questo ci dice che il problema di ottimizzazione \`e almeno tanto difficile quanto il corrispondente problema
decisionale.

Caratterizzare il problema decisionale ci da un \textbf{limite inferiore} alla complessit\`a del relativo problema di
ottimizzazione.

\section{Classi di complessit\`a}\label{classi}
Un problema, in base alle risorse spese per la sua risoluzione, viene classificato in diverse
\textbf{classi di complessit\`a}.

\begin{theorem}
	Dato un problema decisionale $\Pi$ ed un algoritmo $A$, diciamo che $A$ \textbf{risolve} $\Pi$ se, data un'istanza di
	input $x$, vale
	\[ A(x) = 1 \quad \Leftrightarrow \quad \Pi(x) = 1 \]
\end{theorem}

\begin{theorem}
	Dato un problema decisionale $\Pi$, un algoritmo $A$ e la dimensione $n$ dell'input, diciamo che $A$ risolve $\Pi$
	in tempo $t(n)$ e spazio $s(n)$ se il tempo di esecuzione e l'occupazione di memoria di $A$ sono rispettivamente
	$t(n)$ e $s(n)$.
\end{theorem}

Data una funzione $f$ qualsiasi e la dimensione $n$ dell'input, possiamo definire due classi di complessit\`a:
\begin{itemize}
	\item \textbf{Time}: l'insieme dei problemi decisionali che possono essere risolti in tempo $O(f(n))$.
	\item \textbf{Space}: l'insieme dei problemi decisionali che possono essere risolti in spazio $O(f(n))$.
\end{itemize}
Da queste due classi di problemi possiamo derivare altre sottoclassi pi\`u specifiche
\begin{itemize}
	\item \textbf{P}: l'insieme dei problemi risolvibili in tempo polinomiale nella dimensione $n$ dell'istanza di input.

	      In questo tipo di problemi abbiamo due costanti: $c$ e $n_0 > 0$ tali che il numero di passi elementari \`e al
	      pi\`u $n^c$ per ogni input di dimensione $n$ e per ogni $n > n_0$.

	\item \textbf{P-Space}: l'insieme dei problemi risolvibili in spazio polinomiale nella dimensione $n$ dell'istanza
	      di input.

	      In questo tipo di problemi abbiamo due costanti: $c$ e $n_0 > 0$ tali che il numero di celle di memoria
	      utilizzate \`e al pi\`u $n^c$ per ogni input di dimensione $n$ e per ogni $n > n_0$.
	\item \textbf{Exp Time}: l'insieme dei problemi risolvibili in tempo esponenziale nella dimensione $n$ dell'istanza
	      di ingresso.
\end{itemize}

\subsection{Relazioni tra classi}
Si congettura che
\[ \text{P} \subseteq \text{P-Space} \]
questo perch\'e un algoritmo di costo polinomiale pu\`o avere accesso al pi\`u ad un numero polinomiale di locazioni di
memoria diverse (in ordine di grandezza).

Si congettura anche che
\[ \text{P-Space} \subseteq \text{Exp Time} \]
questo perch\'e non \`e detto che un algoritmo di costo esponenziale richieda un numero esponenziale di celle di memoria.

\subsection{Classe NP e certificati}
Introduciamo ora una nuova classe di problemi, la classe \textbf{NP}, dove NP sta per \textbf{P}olinomiale su modelli
\textbf{N}on deterministici.

Per le \textbf{istanze accettabili} (\textbf{positive}) $x$ di alcuni problemi, \`e possibile fornire un
\textbf{certificato polinomiale} $y$ che possa convincerci del fatto che l'istanza soddisfa la propriet\`a e dunque
\`e un'istanza \emph{accettabile}.

I certificati servono per dare una risposta in tempi pi\`u brevi a certi problemi decisionali. Un certificato \`e di
fatto una descrizione breve di una soluzione che possiede la propriet\`a cercata.

\begin{example}
	Prendiamo il problema della Clique per esempio. Una clique, dato un grafo non orientato, \`e il sottografo
	connesso pi\`u grande, rispetto al grafo che sto considerando.

	Esiste solo un algoritmo di costo esponenziale per trovare una clique in un grafo ma non sappiamo se possa essere
	risolto in tempo polinomiale.

	Vogliamo verificare che all'interno di questo grafo ci sia una clique di $k$ vertici.

	Se abbiamo un certificato polinomiale contenente un'insieme di $k$ vertici che formano una clique, non ci rimane
	che verificare se quei vertici formano effettivamente una clique (tempo polinomiale).
\end{example}

Un certificato \`e un \emph{attestato breve di esistenza} di una soluzione con determinate propriet\`a e si definisce solo
per istanze accettabili. Infatti, in generale, non \`e facile costruire attestati di non esistenza di una certa soluzione.

\subsubsection{Verifica}
L'\textbf{idea} \`e quella di utilizzare il costo della \textbf{verifica} del certificato per caratterizzare la
complessit\`a del problema.

Un problema $\Pi$ \`e \emph{verificabile in tempo polinomiale} se
\begin{itemize}
	\item Ogni istanza $x$ accettabile di $\Pi$ di dimensione $n$, ammette un certificato $y$ di dimensione
	      polinomiale in $n$.
	\item Esiste un algoritmo di verifica polinomiale in $n$, applicabile a ogni coppia $\langle x, y \rangle$, che
	      permette di attestare che $x$ \`e accettabile.
\end{itemize}
Un altro modo di definire i problemi della classe NP \`e quello di problemi \emph{verificabili in tempo polinomiale}.

\begin{observation}
	Un certificato viene trovato in tempo esponenziale e ci serve \textbf{solo} a caratterizzare un certo problema in
	base al costo della sua verifica.
\end{observation}

\begin{theorem}
	La classe P \`e un sottoinsieme di NP. Un problema risolvibile in tempo polinomiale \`e sicuramente verificabile in
	tempo polinomiale.
\end{theorem}

\subsubsection{Problemi NP-completi}
I problemi \textbf{NP-completi} sono i problemi \emph{pi\`u difficili} all'interno della classe NP. Se esistesse un
algoritmo polinomiale per risolvere uno solo di questi problemi, allora tutti i problemi in NP potrebbero essere risolti
in tempo polinomiale e dunque giungeremmo alla conclusione che $\text{P} = \text{NP}$.

\begin{theorem}\label{th: NP-completi_poly}
	Tutti i problemi NP-completi sono risolvibili in tempo polinomiale oppure nessuno lo \`e.
\end{theorem}

\subsection{Riduzioni polinomiali}
Introduciamo ora le \textbf{riduzioni polinomiali}, necessarie per comprendere meglio il teorema
\ref{th: NP-completi_poly}.

\begin{theorem}
	Siano $\Pi_1$ e $\Pi_2$ due problemi decisionali e siano $I_1$ e $I_2$ gli insiemi delle istanze di input di
	$\Pi_1$ e $\Pi_2$, allora $\Pi_1$ si \textbf{riduce} in tempo polinomiale a $\Pi_2$
	\[ \Pi_1 \leq_p \Pi_2 \]
	se esiste una funzione $f : I_1 \rightarrow I_2$, calcolabile in tempo polinomiale, tale che, per ogni istanza $x$
	di $\Pi_1$, $x$ \`e un'istanza accettabile di $\Pi_1$ se e solo se $f(x)$ \`e un'istanza accettabile di $\Pi_2$.
\end{theorem}

In sintesi, se $\Pi_1$ si riduce a $\Pi_2$, significa che in tempo polinomiale posso
\begin{enumerate}
	\item \emph{Tradurre} $\Pi_1$ in $\Pi_2$.
	\item Risolvere $\Pi_2$.
	\item La soluzione trovata \`e valida anche per $\Pi_1$.
\end{enumerate}

Supponiamo di avere un algoritmo di costo polinomiale per risolvere $\Pi_2$ e vogliamo usare lo stesso algoritmo per
risolvere $\Pi_1$.
\begin{enumerate}
	\item Prendiamo un'istanza $x$ di $\Pi_1$ e traduciamola in un'istanza accettabile per $\Pi_2$ tramite la funzione
	      $f$.
	\item A questo punto facciamo girare l'algoritmo per la risoluzione di $\Pi_2$ sull'istanza $f(x)$.
	\item La soluzione data dall'algoritmo vale anche per $\Pi_1$.
\end{enumerate}

\subsubsection{Problemi NP}

\begin{definition}
	Un problema decisionale $\Pi$ si dice \textbf{NP-arduo} se
	\[	\forall \; \Pi' \in \text{NP}, \quad \Pi' \leq_p \Pi \]
	Ovvero se ogni problema in NP \`e riducibile a $\Pi$.
\end{definition}

\begin{definition}
	Un problema decisionale $\Pi$ si dice \textbf{NP-completo} se
	\[ \Pi \in \text{NP} \quad \wedge \quad \forall \; \Pi' \in \text{NP}, \quad \Pi' \leq_p \Pi \]
	Ovvero se ogni problema in NP \`e riducibile a $\Pi$, anch'esso in NP.
\end{definition}

Dimostrare che un problema \`e in NP pu\`o essere facile: basta esibire un certificato polinomiale. Non \`e altrettanto
facile dimostrare che un problema \`e NP-arduo o NP-completo: bisognerebbe dimostrare che tutti i problemi in NP si
riducono polinomialmente a $\Pi$.

In realt\`a la prima dimostrazione di NP-completezza aggira il problema.

\begin{theorem}[Cook]
	Il problema SAT \`e NP-completo.
\end{theorem}

Cook ha mostrato che, dato un qualunque problema $\Pi \in \text{P}$ ed una qualunque istanza $x$ per $\Pi$, si pu\`o
costruire un'espressione booleana in forma normale congiuntiva, che descrive il calcolo di un algoritmo per risolvere
$\Pi$ su $x$. L'espressione \`e vera se e solo se l'algoritmo restituisce 1.

\begin{theorem}
	Un problema decisionale $\Pi$ \`e NP-completo se
	\[ \Pi \in \text{NP} \quad \wedge \quad \text{SAT} \leq_p \Pi \]
	o se $\Pi$ \`e riducibile ad qualsiasi altro problema NP-completo.
\end{theorem}

\subsubsection{Problemi NP equivalenti}
Il fatto che un problema $\Pi$ si possa ridurre a SAT lo rende NP-completo. Il fatto che $\Pi$ sia NP-completo rende
possibile ridurre SAT a $\Pi$. Questo rende SAT e $\Pi$ due problemi \textbf{NP-equivalenti}.

\begin{theorem}
	Tutti i problemi NP-completi sono tra loro NP-equivalenti.
\end{theorem}

\subsubsection{Classi co-P e co-NP}
Le ultime due classi di problemi che trattiamo sono le classe co-P e co-NP. Fanno parte di queste due classi i problemi
\textbf{complementari} ai problemi in P ed NP.

Dal punto di vista della complessit\`a, passare da un problema al suo complementare, cambia molto a seconda se il problema
sia in P oppure no.

Se ho un problema in P allora anche il suo complementare \`e in P: basta rispondere il contrario del problema di partenza.
Possiamo quindi affermare che P = co-P.

Discorso diverso per i problemi NP. Se un problema NP \`e tale in presenza di un certificato che lo verifichi in tempo
polinomiale, la verifica del suo complementare \`e, in generale, \emph{difficile} e richiede tempo esponenziale. Si
congettura quindi che NP $\neq$ co-NP. 		% COMPLESSITA'
\include{random} 			% SEQUENZE CASUALI
\chapter{Numeri primi}\label{primi}
Iniziamo ora a parlare di \textbf{primalit\`a} e di come costruire algoritmi efficienti per effettuare il
\textbf{test di primalit\`a}, ossia algoritmi in grado di dirci se un numero \`e primo o no e come fare per generare
numeri primi.

Iniziamo con un algoritmo semplice ma che come vedremo risulter\`a molto inefficiente.
\begin{lstlisting}[style=pseudo-style]
Primo(n)
	for i = 2 to sqrt(n)
		if n % i == 0 then
			return false;
	
	return true;
\end{lstlisting}
Come possiamo facilmente constatare, l'algoritmo
\begin{enumerate}
	\item Controlla se uno dei numeri da 2 a $\sqrt{n}$ divide $n$. Si parte da 2 dato che la divisione per 0 \`e in
	      generale indefinita o comunque tende all'infinito e tutti i numeri sono divisibili per 1.

	      Ci si ferma a $\sqrt{n}$ dato che un numero, se composto, possiede sicuramente un divisore minore della sua
	      radice.
	\item Se ne trova uno che divide $n$ allora ritorna \verb|false|.
	\item Se non ne trova nessuno ritorna \verb|true|.
\end{enumerate}
Un'analisi poco attenta potrebbe indurci a pensare che il costo di questo algoritmo sia $O(\sqrt{n})$ dato che faccio al
pi\`u $\sqrt{n}$ iterazioni. Questo \`e in parte vero ma dobbiamo considerare la dimensione dell'istanza di input, la
sua rappresentazione e il costo della divisione.

L'istanza di input, ossia $n$, richiede $\Theta(\log_2 n)$ bit per essere rappresentata mentre la divisione \`e, in
generale, un'operazione quadratica nel numero di cifre. Tutto questo fa salire la complessit\`a a
\[ O(\sqrt{n} \cdot \log^2 n) \]
Ma non \`e finita qui: come abbiamo detto, $n$, necessita di $\Theta(\log_2 n)$ bit per essere rappresentato, dunque $n$
si pu\`o scrivere come $2^{\log n}$ e questo fa diventare la complessit\`a
\[ O(2^\frac{\log n}{2} \cdot \log^2 n) \]
Come possiamo vedere, un algoritmo all'apparenza polinomiale \`e in realt\`a un algoritmo di costo esponenziale nella
dimensione dell'input. Si tratta di un algoritmo \textbf{pseudopolinomiale}.

\section{Algoritmo di esponenziazione veloce}\label{esponenziazione}
Introduciamo l'\textbf{algoritmo di esponenziazione veloce} o \textbf{algoritmo delle quadrature successive} che sar\`a
molto utile per svolgere elevamenti a potenza in modo efficiente. Essere in grado di svolgere elevamenti a potenza
in modo efficiente \`e necessario per avere algoritmi efficienti per il test di primalit\`a.

Sia $b$ la base e $n$ l'esponente e $m$ un intero qualsiasi, vogliamo calcolare
\[ x = b^n \mod{m} \]
con un numero di operazioni dell'ordine di $O(\log_2 n)$.
\begin{enumerate}
	\item Si scompone l'esponente $n$ come somma di potenze di 2.
	      \[ n = \sum_{i=0}^{\lfloor \log_2 n \rfloor} k_i \cdot 2^i \quad \quad k_i \in \{0,1\} \]
	\item Si calcolano tutte le potenze $b^{2^i} \mod{m}$ calcolando ogni potenza come quadrato delle precedente
	      \[ b^{2^i} \mod{m} = \left( b^{2^{i-1}} \right)^2 \mod{m} \]
	      con $1 \leq i \leq \lfloor \log_2 n \rfloor$
	\item Prendiamo, fra le potenze ricavate al punto 2, quelle che compaiono nella scomposizione fatta al
	      punto 1 e moltiplichiamole
	      \[ x = \prod_{i \mid k_i \neq 0} b^{2^i} \mod{m} \]
\end{enumerate}

Al punto 2 vengono compiute esattamente $\log_2 n$ operazioni per il calcolo delle varie potenze e al punto 3 svolgiamo
un numero di moltiplicazioni dell'ordine di $O(\log_2 n)$.

Se consideriamo che la moltiplicazione \`e un'operazione di costo polinomiale nel numero delle cifre \`e facile
stabilire che l'algoritmo sia complessivamente di costo polinomiale.

\begin{example}
	Vogliamo calcolare
	\[ 9^{45} \mod{11} \]
	Scriviamo l'esponente come somma di potenze di 2
	\[ 9^{32 + 8 + 4 + 1} \mod{11} \]
	Calcoliamo ora le potenze $9^{2^i}$ fino ad arrivare a $9^{32}$, ognuna calcolata come quadrato della precedente.
	\[
		\begin{matrix}
			9^2 \mod{11} =    &                & 4 \\
			9^4 \mod{11} =    & 4^2 \mod{11} = & 5 \\
			9^8 \mod{11} =    & 5^2 \mod{11} = & 3 \\
			9^{16} \mod{11} = & 3^2 \mod{11} = & 9 \\
			9^{32} \mod{11} = & 9^2 \mod{11} = & 4
		\end{matrix}
	\]
	Ora non ci rimane che prendere le potenze di cui abbiamo bisogno
	\[ 9^{45} \mod{11} = (9^{32} \mod{11}) \cdot (9^8 \mod{11}) \cdot (9^4 \mod{11}) \cdot (9^1 \mod{11}) \]
	che possiamo comodamente riscrivere come
	\[ 9^{45} \mod{11} = 4 \cdot 3 \cdot 5 \cdot 9 = 1 \]
\end{example}

\section{Algoritmi randomizzati}\label{algoritmi_random}
Gli algoritmi randomizzati sono fondamentali per effettuare test di primalit\`a efficienti dato che un algoritmo
deterministico e polinomiale nella dimensione dell'input esiste ma \`e comunque molto lento.

Questi algoritmi si dividono in due gruppi principali
\begin{itemize}
	\item \textbf{Las Vegas}: generano un risultato \emph{sicuramente corretto} in un tempo \emph{probabilmente breve}.
	\item \textbf{Monte Carlo}: generano un risultato \emph{probabilmente corretto} in un tempo \emph{sicuramente breve}.
\end{itemize}
L'algoritmo che vedremo per il test di primalit\`a \`e della tipologia Monte Carlo ma la probabilit\`a di errore
dev'essere \textbf{misurabile} e \textbf{arbitrariamente piccola}.

\subsection{Classe RP}
La \textbf{classe RP} comprende tutti quei problemi \emph{verificabili} in tempo polinomiale tramite algoritmi
randomizzati.

Sia $\Pi$ un problema decisionale, $x$ un istanza di input di $\Pi$ allora $y$ \`e un \textbf{certificato probabilistico}
di $x$ se
\begin{itemize}
	\item \`E di lunghezza al pi\`u polinomiale in $|x|$ (devo leggerlo in tempo polinomiale).
	\item \`E estratto perfettamente a caso da un insieme associato a $x$.
\end{itemize}
Possiamo anche dire che $A$ \`e un \textbf{algoritmo di verifica} che prende in input $x$ e $y$ se, in tempo polinomiale,
riesce ad attestare che
\begin{itemize}
	\item $x$ non possiede la propriet\`a.
	\item $x$ possiede la propriet\`a con probabilit\`a $> 1/2$.
\end{itemize}
Si congettura che
\[ \text{P} \subset \text{RP} \subset \text{NP} \]

\section{Test di Miller-Rabin}\label{Miller_Rabin}
La prima parte dell'algoritmo \`e composta dei seguenti passi.
\begin{enumerate}
	\item Prendiamo $n$ intero e dispari di cui vogliamo testare la primalit\`a.
	\item Prendiamo $n-1$ (sicuramente pari) e cerchiamo la massima potenza di 2 che lo divide, cos\`i da rappresentare
	      $n-1$ in questo modo
	      \[ n-1 = 2^w \cdot z \quad \text{con $z$ dispari} \]
	      questo \`e sempre possibile perch\'e un numero pari \`e sempre rappresentabile come potenza di 2 che moltiplica
	      un numero dispari.

	      Per determinare $w$ e $z$ impieghiamo in media $O(\log n)$ passi.
	\item Scegliamo un intero $y$ compreso tra 2 e $n-1$.
\end{enumerate}
Se $n$ \`e primo allora valgono i due predicati
\begin{itemize}
	\item $(n, y) = 1$
	\item $y^z \mod{n} \equiv 1$

	      oppure

	      $\exists i, \quad 0 \leq i \leq w-1 \mid y^{2^i} \cdot z \mod{n} \equiv -1$
\end{itemize}
Chiariamo che questi due predicati sono condizioni necessarie alla primalit\`a ma non sufficienti.

\begin{lemma}[Miller-Rabin]
	Se $n$ \`e un numero composto, il numero di interi $y$ compresi tra 2 ed $n-1$, che soddisfano entrambi i predicati
	\`e minore di $n / 4$.
	\[ \# \{ 2 \leq y \leq n-1 \mid P_1(y) = \text{ true} \quad \wedge \quad P_2(y) = \text{ true} \} < \frac{n}{4} \]
\end{lemma}
Questo lemma ci dice che la probabilit\`a di scegliere un $y$ che soddisfa entrambi i predicati \`e minore di $1 / 4$.
Questo \`e banale dato che ho $n - 2$ possibili scelte e $n / 4$ di queste rendono veri entrambi i predicati.
\[ \frac{n/4}{n-2} < \frac{1}{4} \]
Quando scelgo un $y$ vado a testare i due predicati: se anche solo uno \`e falso allora possiamo affermare con certezza
che $y$ \`e composto, se invece sono entrambi soddisfatti \emph{molto probabilmente} \`e primo con probabilit\`a di
errore al pi\`u del $25\%$.

A questo punto possiamo iterare $k$ volte con $k$ scelte casuali e indipendenti di $y$, la probabilit\`a di errore scende
a $(1/4)^k$.

Possiamo quindi concludere l'algoritmo in questo modo
\begin{enumerate}
	\setcounter{enumi}{3}
	\item Verifico che i due predicati siano soddisfatti.
	\item Itero $k$ volte su $k$ scelte diverse e casuali di $y$.
	      \begin{itemize}
		      \item Se anche solo una volta un predicato non \`e soddisfatto allora il numero non \`e primo.
		      \item Altrimenti possiamo affermare che lo sia con probabilit\`a di errore inferiore a $(1/4)^k$
	      \end{itemize}
\end{enumerate}

\begin{lstlisting}[style=pseudo-style]
Verifica(n, y) // true se n e' composto
	if not P1(n, y) or not P2(n, y) then
		return false;
	else
		return true;
\end{lstlisting}

\begin{lstlisting}[style=pseudo-style]
TestMR(n, k)
	for i = 1 to k
		y = random(2, n - 1);
		if Verifica(y, n) then 
			return false;

	return true;
\end{lstlisting}

\subsection{Verifica dei predicati}
Possiamo valutare il costo del ciclo come costante ($k$ cicli) mentre \`e fondamentale capire il costo della verifica
dei predicati.
\begin{itemize}
	\item Per la verifica del primo predicato dobbiamo calcolare il massimo comun divisore con l'algoritmo di Euclide.
	      Un'operazione che richiede costo cubico nel numero di cifre
	      \[ O(\log^3 n) \]
	\item La verifica del secondo predicato \`e pi\`u complessa per via degli elevamenti a potenza.

	      Prima di tutto dobbiamo calcolare $y^z$ e per capire quanto sia costosa questa operazione dobbiamo capire quanto
	      \`e grande $z$. Se andiamo a vedere come si ottiene $z$ possiamo ricavare che $z$ abbia al massimo valore
	      $\frac{n-1}{2}$ dunque $z$ \`e dell'ordine di $n$.

	      Per fare quindi $y^z$ non posso moltiplicare $y$ per se stesso $\frac{n-1}{2}$ volte perch\'e farei un numero
	      di operazioni proporzionale al valore $n$ e non possiamo permettercelo in termini computazionali.

	      Il nostro obbiettivo \`e quello di compiere un numero di operazioni che \`e proporzionale al numero delle cifre
	      ossia $\log_2 n$ e come sappiamo, questo si pu\`o fare con l'\textbf{algoritmo di esponenziazione veloce},
	      esposto al paragrafo \ref{esponenziazione}.

	      Per la verifica della seconda parte del predicato basta semplicemente elevare al quadrato il risultato ottenuto
	      per la verifica della prima parte del predicato. Anche questa seconda verifica ha quindi costo polinomiale.
\end{itemize}
Come possiamo vedere, la verifica dei predicati, ha complessivamente costo polinomiale.

\section{Generazione di numeri primi}\label{generazione_primi}
La generazione di numeri primi casuali si effettua semplicemente generando un numero casuale e in seguito si effettua
il test di primalit\`a di Miller-Rabin. Ripeto la generazione del numero casuale finch\'e non ne trovo uno che posso
dichiarare primo con un possibilit\`a di errore pi\`u bassa possibile.

\begin{theorem}
	Il numero di numeri primi minori di un certo $n$ tende a
	\[ \frac{n}{\ln n} \]
	per $n$ che tende all'infinito.
\end{theorem}

Questo teorema ci dice che, per $n$ sufficientemente grande, nel suo intorno (di ampiezza $\ln n$) cade mediamente un
numero primo.

Mediamente dovremo quindi fare un numero $\ln n$ di tentativi, il che ci va bene dato che \`e polinomiale nel numero
delle cifre.

\subsection{Algoritmo}
Costruiamo ora un algoritmo per generare un numero primo di $n$ bit.

\begin{lstlisting}[style=pseudo-style]
Primo(n)
	S = randomSeq(n-2); // sequenza casuale di n-2 bit
	N = 1 + S + 1; // numero dispari con n bit significativi
	while TestMR(N, k) == 0 
		N = N + 2;

	return N;
\end{lstlisting}
Il costo complessivo \`e $O(n^4)$ dato che facciamo $O(n)$ volte il test di Miller-Rabin, il quale aveva un costo
complessivamente di $O(n^3)$.
 			% NUMERI PRIMI
\chapter{Cifrari storici}
In questo capitolo andremo a trattare i \textbf{cifrari storici}, chiamati con questo nome perch\'e ad oggi sono stati
tutti forzati e dunque non sono pi\`u cifrari sicuri.

I primi cifrari nascono in un periodo in cui cifratura e decifrazione si facevano "con carta e penna" o quasi e servivano
per cifrare frasi di senso compiuto in linguaggio naturale.

Da un certo punto in poi tutti i cifrari hanno cercato di seguire i cosiddetti \textbf{principi di Bacone}
\begin{itemize}
	\item Le funzioni $C$ e $D$ devonon essere \textbf{facili da calcolare}.
	\item \`E \textbf{impossibile} ricavare la $D$ se la $C$ non \`e nota.
	\item Il crittogramma $c = C(m)$ deve apparire "\textbf{innocente}", deve sembrare cio\`e un testo in chiaro e non
	      una sequenza di caratteri insensata.
\end{itemize}

\section{Cifrario di Cesare}
L'idea di base \`e che il crittogramma $c$ \`e ottenuto dal messaggio in chiaro $m$ sostituendo ogni lettera di $m$ con
quella tre posizioni pi\`u avanti nell'alfabeto.

\begin{center}
	A B C D \dots W X Y Z
	\[ \downarrow \]
	D E F G \dots Z A B C
\end{center}
La decifrazione \`e immediata: basta sostituire ogni lettera del crittogramma con la lettera tre posizioni pi\`u indietro
nell'alfabeto.

\`E un cifrario molto semplice e non utilizza una chiave di cifratura. Una volta scoperto il metodo di cifratura e
decifrazione diventa del tutto inutile.

\subsection{Cifrario di Cesare generalizzato}
Per rendere il cifrario un po' pi\`u robusto basterebbe inserire un chiave $1 \leq k \leq 25$ (26 lascia inalterato il
messaggio) e invece di traslare sempre di tre posizioni le lettere del messaggio in chiaro, le trasliamo di $k$
posizioni.

Ovviamente i due utenti devono possedere la solita chiave per cifrare e decifrare i messaggi.

\subsubsection{Cifratura e decifrazione}
Sia $x$ una lettera dell'alfabeto, $pos(x)$ la sua posizione nell'alfabeto e $k$ la chiave tale che $1 \leq k \leq 25$.

La funzione di cifratura ritorna la lettera $y$ tale che
\[ pos(y) = (pos(x) + k) \mod{26} \]

La funzione di decifrazione ritorna la lettera $x$ tale che
\[ pos(x) = (pos(y) - k) \mod{26} \]

\`E immediato effettuare un attacco a forza bruta per forzarlo: si provano tutte le 25 chiavi.

\subsubsection{Osservazioni}
Il cifrario gode della propriet\`a \textbf{commutativa}: data una sequenza di chiavi e di operazioni di cifratura e
decifrazione, l'ordine delle operazioni pu\`o essere permutato arbitrariamente senza modificare il crittogramma finale.

Date due chiavi $k_1$ e $k_2$, e una sequenza $s$ vale
\[
	\begin{matrix}
		C(C(s, k_2), k_1) & = & C(s, k_1 + k_2) \\
		D(D(s, k_2), k_1) & = & D(s, k_1 + k_2)
	\end{matrix}
\]
Quindi una sequenza di operazioni di cifratura e decifrazione pu\`o essere ridotta ad una sola operazione di cifratura
o decifrazione.

In generale, comporre pi\`u cifrari \textbf{non} aumenta la sicurezza del sistema.

\section{Cifrari a sostituzione}
I \textbf{cifrari a sostituzione} sostituiscono ogni lettera del messaggio in chiaro con una o pi\`u lettere
dell'alfabeto secondo una regola prefissata.

Questi cifrari si suddividono a loro volta in due sottocategorie
\begin{itemize}
	\item Cifrari a \textbf{sostituzione monoalfabetica}: alla stessa lettera del messaggio corrisponde sempre una
	      stessa lettera nel crittogramma.
	\item Cifrari a \textbf{sostituzione polialfabetica}: alla stessa lettera del messaggio corrisponde una lettera
	      scelta in un insieme di lettere possibili, secondo una regola opportuna.
\end{itemize}

\subsection{Cifrari a sostituzione monoalfabetica}
Per questo tipo di cifrari si possono usare funzioni di cifratura e decifrazione pi\`u complesse dell'addizione e
sottrazione in modulo.

Andiamo cos\`i ad ampliare molto lo spazio delle chiavi non ottenendo per\`o grandi  miglioramenti per quanto
riguarda la sicurezza.

\subsubsection{Cifrario affine}
Questo cifrario usa una chiave composta da due valori
\[ k = \langle a, b \rangle \]
e cifra una lettera $x$ del messaggio in chiaro con la lettera $y$ che occupa la posizione
\[ pos(y) = (a * pos(x) + b) \mod{26} \]
nell'alfabeto.

La funzione di decifrazione invece mappa la lettera cifrata $y$ nella lettera $x$ del messaggio in chiara che ha posizione
\[ pos(x) = a^{-1} * (pos(y) - b) \mod{26} \]
nell'alfabeto. Si deve quindi trovare l'\textbf{inverso} moltiplicativo di $a$, ossia, quel numero che moltiplicato per $a$
mi da 1 (ovviamente in modulo).

Se vi ricordate qualcosa del corso di matematica discreta noterete che sorgono due problemi.
\begin{itemize}
	\item Il primo \`e che se $a$ \`e nullo allora la lettera $x$, in fase di cifratura viene sempre mappata nella lettera
	      di posizione $b$.
	\item Il secondo problema \`e dovuto al fatto che, in fase di decifrazione, dobbiamo usare l'inverso modulare di $a$ ma,
	      l'esistenza di quest'ultimo, non \`e sempre e comunque garantita.

	      Affinch\'e $a$ abbia un inverso modulare esso deve essere coprimo col numero di simboli del mio alfabeto (26 nel
	      nostro caso).
\end{itemize}
Il parametro $a$ deve quindi soddisfare queste due propriet\`a affinch\'e il cifrario sia utilizzabile
\[ a \neq 0 \quad \wedge \quad (a, 26) = 1 \]
Se $a$ non \`e coprimo col numero di simboli dell'alfabeto la funzione di cifratura non \`e pi\`u iniettiva e la decifrazione
diventa impossibile.

Se consideriamo un alfabeto di 26 caratteri, dobbiamo contare gli interi minori di 26 coprimi con esso.
\begin{itemize}
	\item I fattori primi di 26 sono 2 e 13.
	\item 26 \`e pari, dunque $a$ non pu\`o essere pari.
\end{itemize}
Ne deduciamo che il valore di $a$ deve essere un numero dispari compreso tra 1 e 25 ad eccezione di 13. Abbiamo quindi 12
possibili valori.

Se siete tra quelli bravi a matematica discreta non vi sar\`a sfuggito che il numero di possibili valori di $a$, ossia il
numero di valori coprimi con un certo numero $n$, \`e equivalente al valore della \textbf{funzione di Eulero} calcolata
su $n$. Nel nostro caso abbiamo
\[ \phi(26) = 12 \]
Per coloro che invece hanno bisogno di un ripasso elenchiamo qualche propriet\`a comoda per il calcolo di $\phi$
\begin{itemize}
	\item Se $n$ \`e primo allora
	      \[ \phi(n) = n - 1 \]
	\item Se $n = p \cdot q$ con $p$ e $q$ primi allora
	      \[ \phi(n) = (p - 1)(q - 1) \]
\end{itemize}
Il parametro $b$ invece pu\`o assumere un qualsiasi valore tra 1 e 25. Abbiamo in totale un numero di chiavi pari a
\[ 12 * 25 = 311 \]
Abbiamo aumentato molto il numero delle chiavi ma il cifrario rimane comunque molto debole.

\subsubsection{Cifrario completo}
Con questo cifrario andiamo a rendere lo spazio delle chiavi molto pi\`u ampio rispetto ad un cifrario affine andando a
prendere una permutazione arbitraria dell'alfabeto come chiave.

Ora abbiamo quindi che la chiave non \`e pi\`u una coppia di valori ma una permutazione lunga quanto l'intero alfabeto che
sto utilizzando.

Con questo tipo di cifratura la lettera in posizione $i$ viene cifrata nella lettera di posizione $i$ nella permutazione.

Con un alfabeto di $n$ caratteri ottengo uno spazio delle chiavi di dimensione
\[ n! - 1 \]
La permutazione in cui l'alfabeto \`e nell'ordine "canonico" non viene contata.

Il cifrario \`e tuttavia molto semplice da forzare sfruttando la struttura logica del messaggio in chiaro e l'occorrenza
statistica delle lettere.

\subsection{Cifrari a sostituzione polialfabetica}
Questi cifrari, come gi\`a anticipato, cifrano una lettera $x$ in una lettera $y$ scegliendo quest'ultima in un insieme di
possibili lettere secondo una regola opportuna.

In generale, ad una lettera $x$ non corrisponde sempre la stessa lettera $y$ che magari abbiamo ottenuto in una cifratura
precedente. Ogni lettera pu\`o essere quindi cifrata in pi\`u lettere e questo rende il cifrario molto pi\`u robusto di un
cifrario a sostituzione monoalfabetica.

\subsubsection{Cifrario di Augusto}
Questo cifrario, ideato dall'imperatore Augusto, veniva usato da quest'ultimo per cifrare informazioni utili che egli voleva
custodire e mantenere segrete.

Il funzionamento si basava sul primo libro dell'Illiade
\begin{enumerate}
	\item Si prende la lettera in posizione $i$ nel documento che vogliamo cifrare $a$.
	\item Si prende la lettera in posizione $i$ nel primo libro dell'Illiade $b$.
	\item Si calcola la distanza fra $a$ e $b$ nell'alfabeto che stiamo considerando.
	\item Si scrive in posizione $i$ la lettera dell'alfabeto nella posizione identificata dalla distanza trovata al punto
	      precedente.
\end{enumerate}

\subsubsection{Disco cifrante di Leon Battista Alberti}
Il cifrario si compone di due dischi: uno esterno e uno interno.
\begin{itemize}
	\item \textbf{Disco esterno}: composto dai caratteri dell'alfabeto (in ordine) e in pi\`u aggiungo qualche carattere
	      speciale (in genere dei numeri) alla fine.
	      \begin{center}
		      A B C D E F G H I L M N O P Q R S T U V Z 1 2 3 4 5
	      \end{center}
	\item \textbf{Disco interno}: composto dallo stesso numero di caratteri del primo disco ma si usa un alfabeto
	      pi\`u ricco e i caratteri sono permutati.
	      \begin{center}
		      E Q H C W L M V P D N X A O G Y I B Z R J T S K U F
	      \end{center}
\end{itemize}
Per riuscire a comunicare, i due utenti, devono possedere ognuno una copia identica del cifrario.

Il cifrario fa uso di una chiave composta da una coppia di caratteri: il primo deve essere presente sul disco esterno, il
secondo su quello interno. Una volta decisa la chiave iniziale si impostano i due dischi.

Cifratura e decifrazione avvengono mettendo in corrispondenza disco esterno ed interno. Se per esempio sto cifrando il
messaggio mi baster\`a far corrispondere alla lettera del messaggio la lettera corrispondete sul disco permutato.

Ci\`o che rende il cifrario interessante e anche molto robusto \`e il \textbf{cambio di chiave} che si pu\`o effettuare
in qualsiasi momento. Anzi, pi\`u cambi di chiave effettuiamo, pi\`u il nostro messaggio sar\`a difficile da decifrare.

Per effettuare il cambio di chiave (in fase di cifratura) dobbiamo inserire, all'interno del messaggio, i caratteri speciali.

Di seguito due possibili tecniche per l'utilizzo efficace del cambio di chiave.
\begin{itemize}
	\item Nel primo metodo inseriamo un carattere speciale nel messaggio in chiaro e lo cifriamo secondo la chiave che stiamo
	      usando. A questo punto abbiamo cambiato chiave di cifratura: la prima lettera della chiave, ossia quella del disco
	      esterno rimane la stessa ma la seconda lettera diventa la lettera nella quale il carattere speciale viene cifrato.

	      Questo procedimento equivale a ruotare il disco interno e iniziare quindi a cifrare il messaggio in un modo diverso.
	\item Il secondo metodo, a "\textbf{indice mobile}", \`e pi\`u complesso e rende il cifrario un po' pi\`u robusto.

	      Come prima cosa inserisco un carattere speciale (stavolta dev'essere necessariamente un numero) all'interno del
	      messaggio e lo cifro con la chiave corrente. Quel numero mi dice quanti caratteri sono ancora cifrati con la chiave
	      corrente.

	      Dopo quel numero di caratteri avr\`o un carattere che non fa parte del messaggio ma che mi indicher\`a la nuova
	      chiave da usare.
\end{itemize}
Cambiando molto spesso la chiave, attacchi di tipo statistico, sono praticamente inutili.

\subsubsection{Cifrario di Vigen\`ere}
\`E un cifrario in cui si fa uso di una chiave \textbf{corta} e \textbf{ripetuta ciclicamente}.

Il cifrario funziona cos\`i
\begin{enumerate}
	\item Si prende una sequenza corta di caratteri dell'alfabeto, diciamo lunga $n$ caratteri.
	\item Si prende la posizione che ciascun carattere occupa nell'alfabeto. La posizione di ciascun carattere
	      nell'alfabeto indica una traslazione che si dovr\`a effettuare in fase di cifratura.
	\item Consideriamo il messaggio a gruppi di $n$ caratteri.
	\item Cifriamo la lettera in posizione $i$ della parte di messaggio in chiaro che sto considerando andando a traslarla
	      di un valore pari al valore di traslazione corrispondente alla lettera in posizione $i$ nella chiave.
	\item Ripeto il procedimento per tutto il messaggio cifrando sempre gruppi di $n$ caratteri.
\end{enumerate}
In questo modo, alla stessa lettera del messaggio, potrebbe corrispondere una traslazione diversa a seconda del valore di
traslazione indicato dalla chiave.

La debolezza del cifrario consiste nell'utilizzo di una chiave corta e ripetuta. Se si riesce infatti a stimare
correttamente la lunghezza della chiave si pu\`o capire si possono decifrare parti di messaggio andando anche a fare
analisi sulla frequenza dei caratteri nel linguaggio naturale.

Per ovviare al problema basta usare una chiave casuale e non riutilizzabile lunga quanto il messaggio in chiaro. In questo
modo il cifrario diventa \textbf{inattacabile} e si ottiene un \textbf{one-time pad}, che andremo a trattare pi\`u avanti.

\section{Cifrari a trasposizione}
I \textbf{cifrari a trasposizione} permutano le lettere del messaggio in chiaro secondo una regola prefissata ed
eventualmente aggiungono altre lettere che verranno poi ignorate in fase di decifrazione.

\subsection{Cifrario a permutazione semplice}
In questo cifrario utilizziamo un intero $h$ e una permutazione $\pi$ dei primi $h$ interi come chiave.

Per effettuare la cifratura si opera in questo modo
\begin{enumerate}
	\item Si suddivide il messaggio in blocchi di $h$ lettere.
	\item Si permutano le lettere di ciascun blocco secondo $\pi$, ovvero, si va a vedere il numero in posizione $i$
	      nella permutazione e si va a spostare il carattere in posizione $\pi[i]$ nel blocco alla posizione $i$.
\end{enumerate}
Le chiavi di cifratura sono $h! - 1$ dato che consideriamo tutte possibili permutazioni di $h$ elementi e togliamo quella
identica.

\subsection{Cifrario a permutazione di colonne}
In questo cifrario usiamo una chiave composta da due interi $c$ ed $r$ e da una permutazione $\pi$ dei primi $c$ interi.

I valori $c$ ed $r$ denotano il numero di righe e di colonne di una \textbf{tabella} di lavoro $T$.

\begin{enumerate}
	\item Il messaggio viene decomposto in blocchi di $c \times r$ caratteri.
	\item Scrivo ogni blocco per righe dall'alto verso il basso dentro $T$.
	\item Permuto le colonne di $T$ secondo $\pi$ come nel cifrario a permutazione semplice.
	\item Formo il crittogramma leggendo la tabella permutata per colonne.
\end{enumerate}
Il numero delle chiavi \`e teoricamente esponenziale dato che non ci sono limiti sulla scelta di $r$ e $c$.

\subsection{Cifrario a griglia}
La chiave segreta \`e una griglia quadrata $q \times q$ con $q$ pari.
\begin{enumerate}
	\item Si considera un numero
	      \[ s = q^2 / 4 \]
	      di celle (un quarto del totale).
	\item Si scrivono i primi $s$ caratteri del messaggio nelle posizioni corrispondenti alle $s$ celle scelte.
	\item Si ruota la griglia di 90 gradi e si scrivono altri $s$ caratteri del messaggio.
	\item Ripeto la rotazione per 3 volte.
\end{enumerate}
Le posizioni delle $s$ celle devono essere scelte in modo che non si sovrappongano mai nelle quattro rotazioni. Il numero
delle chiavi \`e dunque $4^s$ dato che ho 4 possibili scelte per ognuna delle $s$ celle.

\section{Crittoanalisi statistica}
Come abbiamo solo accennato in qualche occasione, la sicurezza di un cifrario \`e legata alla dimensione dello spazio
delle chiavi, tuttavia, questa \`e condizione necessaria ma non sufficiente a rendere il cifrario sicuro.

Uno spazio di chiavi molto grande protegge solamente da attacchi di tipo \emph{forza bruta}. Un crittoanalista ha tuttavia
molti metodi a disposizione per riuscire a forzare un cifrario: le chiavi potrebbero essere generate male, troppo corte,
prevedibili o riutilizzate oppure si potrebbe conoscere il formato del messaggio.

Un crittoanalista pu\`o aggrapparsi a tutte queste cose per riuscire a forzare il cifrario anche se dal punto di vista
algoritmico esso non presenta vulnerabilit\`a.

Un altro metodo con il quale si \`e riusciti a forzare molti dei cifrari storici \`e la \textbf{crittoanalisi statistica}.

Quello che si fa non \`e cercare una vulnerabilit\`a nell'algoritmo che ha prodotto il crittogramma ma si va a fare
analisi statistica sul crittogramma in nostro possesso.

Avendo a disposizione solo il crittogramma possiamo compiere attacchi di tipo \textbf{cipher text} andando per esempio
a fare \textbf{analisi delle frequenze}.

Si va cio\`e a vedere quali sono le lettere pi\`u frequenti nel crittogramma e le confrontiamo con le lettere pi\`u
frequenti del linguaggio naturale (ovviamente nella lingua in cui stiamo comunicando).

\subsection{Attacchi}
Facciamo prima qualche ipotesi affinch\'e abbia senso fare della crittoanalisi statistica.
\begin{itemize}
	\item Si deve conoscere il metodo impiegato per la cifratura e decifrazione.
	\item Si deve conoscere il linguaggio in cui \`e scritto il messaggio.
	\item Si ammette che il messaggio sia sufficientemente lungo da poter rilevare alcuni dati statistici sui caratteri
	      che compongono il crittogramma.
\end{itemize}
Premettiamo che la frequenza con cui appaiono in media le lettere dell'alfabeto \`e ben studiata in ogni lingua e sono
anche presenti dati simili per \emph{digrammi}, \emph{trigrammi} ecc.

\subsubsection{Sostituzione monoalfabetica}
Se il metodo di cifratura \`e di tipo \emph{monoalfabetico} significa che ad una lettera del crittogramma corrisponde
sempre la stessa lettera del messaggio e viceversa.

Questo significa anche che la frequenza della lettera cifrata nel crittogramma \`e uguale alla frequenza della lettera in
chiaro nel linguaggio naturale.

Basta dunque prendere un istogramma delle frequenze delle varie lettere e dopo pochi tentativi la procedura converge in
qualche corrispondenza significativa.

\subsubsection{Sostituzione polialfabetica}
In questo caso, l'istogramma delle frequenze del crittogramma risulta piatto dato che con questo metodo di sostituzione
una lettera del crittogramma potrebbe corrispondere a lettere diverse del messaggio.

Il cifrario di Vigen\`ere \`e stato tuttavia forzato sfruttando la debolezza relativa ad avere una chiave corta e ripetuta.

Se la chiave contiene $h$ caratteri, le apparizioni della stessa lettera, distanti un multiplo di $h$ nel messaggio, si
sovrappongono alla stessa lettera della chiave. Supponiamo quindi di conoscere la lunghezza $h$ della chiave
\begin{enumerate}
	\item Si costruisce un sottomessaggio formato dal carattere in posizione $i$ ($i \leq h$) e da tutti i caratteri
	      in posizione $i + k \cdot h$.
	\item Ognuno di questi sottomessaggi \`e cifrato in modo monoalfabetico (stesso valore di traslazione).
	\item Si effettua un'analisi delle frequenze per ciascun sottomessaggio.
\end{enumerate}

Se per\`o non conosciamo la lunghezza $h$ della chiave dobbiamo andare a sfruttare qualche propriet\`a del linguaggio
naturale per riuscire a stimarla pi\`u correttamente possibile.

Il messaggio contiene quasi sicuramente gruppi di lettere ripetuti pi\`u volte come ad esempio trigrammi pi\`u frequenti
nella lingua o parole ricorrenti nel testo.

Pu\`o capitare che questi $q$-grammi vengano cifrati con la stessa porzione di chiave e quindi in modo identico l'uno
all'altro.

Andiamo quindi a cercare nel crittogramma coppie di sequenze identiche. Se trovo due sequenze, in posizione $p_1$ e $p_2$,
con le caratteristiche citate in precedenza, \`e probabile che la distanza $p_2 - p_1$ sia la lunghezza della
chiave o ad un suo multiplo.

\subsubsection{Cifrari a trasposizione}
Come detto in precedenza questi cifrari permutano i caratteri del messaggio. Non ha dunque senso condurre un attacco di
tipo statistico basato sulle frequenze.

Sono tuttavia utili le propriet\`a sintattiche e lessicali del linguaggio, per esempio lo studio dei $q$-grammi.

Se si vuole forzare un cifrario a permutazione semplice e si conosce la lunghezza $h$ della chiave si opera in questo
modo:
\begin{enumerate}
	\item Si divide il crittogramma in blocchi di $h$ caratteri.
	\item Si cercano gruppi di $q$ lettere che formano i $q$-grammi pi\`u frequenti nel linguaggio.
	\item Se un blocco della permutazione deriva da un $q$-gramma si scopre parte della permutazione.
\end{enumerate}

\subsection{Conclusione}
In crittoanalisi statistica lo studio della frequenza delle varie lettere nel linguaggio \`e quindi molto utile per
capire il metodo di cifratura utilizzato
\begin{itemize}
	\item Nei cifrari a trasposizione l'istogramma delle frequenze coincide approssimativamente con quello proprio del
	      linguaggio.
	\item Nei cifrari a sostituzione monoalfabetica l'istogramma relativo al crittogramma e quello delle frequenze delle
	      varie lettere nel linguaggio naturale si somigliano a meno di una permutazione delle lettere.
	\item Nei cifrari a sostituzione polialfabetica l'istogramma delle frequenze \`e piatto.
\end{itemize}

\section{ENIGMA}
L'ultimo cifrario storico di cui andiamo a parlare \`e la macchina \textbf{ENIGMA}. Di grande rilievo storico \`e questo
cifrario perch\'e segna un punto di svolta verso sistemi crittografici automatizzati.

Storicamente fu impiegato dai tedeschi nella seconda guerra mondiale per l'invio di informazioni criptate.

Come vedremo si tratta di un'\emph{estensione} del cifrario di Alberti, visto precedentemente.

\subsection{Utilizzo}
La macchina aveva le sembianze di una macchina da scrivere. Quando si voleva formare il crittogramma si digitava sulla
tastiera il messaggio in chiaro ma ci\`o che veniva effettivamente era del testo cifrato in un modo che vedremo pi\`u
avanti.

Al contrario, quando si voleva decifrare un crittogramma, lo si digitava sulla macchina e questa forniva il relativo testo
in chiaro.

\subsection{Funzionamento}
Come gi\`a anticipato, quando si premeva un tasto sulla tastiera, il carattere corrispondente non era quello premuto ma
un altro che variava a seconda di un po' di cose.

Come prima cosa si scambiava lo spinotto che collegava il tasto di una lettera con una lettera diversa.

Per complicare ulteriormente le cose, ENIGMA, era dotata di alcuni \textbf{rotori}, i quali servivano a mappare una
lettera in un'altra lettera. In genere ogni macchina ENIGMA possedeva tre rotori.

Ogni volta che un rotore effettuava uno scambio, per contatto, andava a attivare la lettera di un altro rotore che a sua
volta effettuava un ulteriore scambio.

Una volta finiti i rotori si arrivava ad un \textbf{riflettore}, il quale compiva lo stesso scambio effettuato internamente
dai riflettori e innescava un ulteriore sequenza di scambi ripassando da tutti i rotori in senso inverso.

Detto cos\`i sembrerebbe un cifrario a permutazione semplice ma quello ci\`o che non abbiamo detto \`e che i rotori, ogni
volta che si batteva una lettera, avanzavano di un passo.

In realt\`a era solo un rotore ad avanzare, almeno finch\'e non aveva compiuto 26 passi. Una volta digitate 26 lettere
il rotore tornava in posizione iniziale e il rotore successivo iniziava la rotazione.

In questo modo la chiave cambiava ogni qual volta si digitava un carattere.

\subsection{Chiavi}
Analizziamo quante sono le possibili permutazioni generabili da ENIGMA.

Ogni rotore genera 26 permutazioni con tutte le sue rotazioni giungendo cos\`i ad un totale di $26^3$ permutazioni
possibili (circa $17.000$).

Un numero troppo basso per la comunicazione di messaggi in ambito militare. Vedremo infatti che la macchina verr\`a
complicata.

\subsection{Debolezze}
\begin{itemize}
	\item Numero di chiavi basso.
	\item I rotori erano immutabili e uguali su ogni macchina ENIGMA.
	\item Le $26^3$ permutazioni erano sempre le stesse ed erano applicate sempre nello stesso ordine.
\end{itemize}

\subsection{Modifiche}
Una prima modifica per aumentare il numero di possibili permutazioni \`e stata quella di permutare tra loro i rotori. In
questo modo giungiamo cos\`i ad un numero di possibili permutazioni pari a $3! \cdot 26^3$ (pi\`u di $10^5$).

Un'altra modifica \`e stata l'aggiunta del \textbf{plugboard} tra la tastiera e il primo rotore che consentiva di scambiare
i caratteri di 6 coppie scelte arbitrariamente.

Ogni cablaggio \`e descritto da una sequenza di 12 caratteri scelti da un insieme totale di 26. Abbiamo quindi un numero
di possibili combinazioni pari a $\begin{psmallmatrix} 26 \\ 12 \end{psmallmatrix}$ che \`e circa $10^7$.

Ogni gruppo di 12 lettere si pu\`o presentare in $12!$ permutazioni diverse, ma non tutte producono lo stesso effetto.
Dobbiamo quindi dividere $10^7$ per $12!$ e per $2^6$.

Il numero di chiavi totale \`e quindi dato da
\[
	3! \cdot 26^3 \cdot
	\begin{pmatrix}
		26 \\ 12
	\end{pmatrix}
	\frac{12!}{6! \cdot 2^6}
\]
Per un totale $10^{16}$ chiavi.

Durante la seconda guerra mondiale si davano in dotazione 8 rotori da cui sceglierne 3 e il numero di coppie scambiabili
con il plugboard \`e passato da 6 a 10.

\subsection{Decifrazione}
La decifrazione di ENIGMA \`e merito degli inglesi, tra cui anche Alan Turing, i quali erano venuti in possesso di un
modello di ENIGMA e grazie alla costruzione di un simulatore di quest'ultima, il famoso COLOSSUS, il quale provava a
capire quale fosse una possibile configurazione della macchina a partire dal crittogramma.

Ogni possessore di una macchina ENIGMA aveva un registro di \textbf{chiavi giornaliere}. Le chiavi nel registro erano usate
per impostare l'assetto iniziale della macchina. Con quell'assetto si comunicava un nuovo assetto da usare per quella
specifica trasmissione. Il nuovo assetto veniva usato per effettuare l'effetiva cifratura del messaggio. 			% CIFRARI STORICI
\chapter{Cifrari perfetti}\label{perfetti}
I \textbf{cifrari perfetti}, detti anche \textbf{cifrari a sicurezza incondizionata}: sono cifrari per uso ristretto e
nascondono l'informazione con certezza assoluta (anche per macchine quantistiche).

Un \textbf{cifrario perfetto} \`e tale se non si riesce ad estrapolare alcuna informazione dall'analisi del crittogramma.

Proviamo a formalizzare matematicamente quanto appena detto. Per farlo dobbiamo considerare
\begin{itemize}
	\item \textbf{MSG}: spazio dei messaggi.
	\item \textbf{CRITTO}: spazio dei crittogrammi.
	\item \textbf{M}: variabile aleatoria che descrive il comportamento del	mittente.
	\item \textbf{C}: variabile aleatoria che descrive la comunicazione sul canale.
\end{itemize}
Indichiamo ora con
\[ P(M = m) \]
la probabilit\`a che il mittente voglia inviare il messaggio $m \in$ MSG. Indichiamo invece con
\[ P(M = m \mid C = c) \]
la probabilit\`a condizionata che il messaggio inviato sia proprio $m$, posto che sul canale stia transitando il
crittogramma $c \in$ CRITTO. In altre parole, quest'ultima espressione, indica la probabilit\`a che $c$ sia $m$ cifrato.

\begin{theorem}\label{th: cifrario_perfetto}
	Un cifrario \`e \textbf{perfetto} se $\forall m \in \text{MSG}$ e $\forall c \in \text{CRITTO}$ vale che
	\[ P(M = m \mid C = c) = P(M = m) \]
\end{theorem}

Mettiamoci per un attimo in uno scenario di massimo pessimismo in cui il crittoanalista sa:
\begin{itemize}
	\item La distribuzione di probabilit\`a con cui il mittente invia messaggi.
	\item Il cifrario utilizzato.
	\item Lo spazio delle chiavi.
\end{itemize}
Supponiamo inoltre che di voler inviare un messaggio $m$ con probabilit\`a
\[ P(M = m) = p > 0 \quad \quad \text{con } 0 < p < 1 \]
e analizziamo i due casi, estremi e opposti l'uno all'altro. Nel primo caso, diciamo che esiste un crittogramma $c$
tale che
\[ P(M = m \mid C = c) = 1 \]
e nel secondo caso diciamo che esiste un crittogramma $c$ tale che
\[ P(M = m \mid C = c) = 0 \]
In entrambi i casi, vedere il crittogramma, raffina la conoscenza del crittoanalista. L'unico caso in cui il
crittoanalista non ricava nulla dal crittogramma \`e il caso descritto dal teorema \ref{th: cifrario_perfetto}.


\section{Svantaggi}\label{svantaggi_perfetti}
L'estrema solidit\`a di un cifrario perfetto ha per\`o un costo in termini di numero di chiavi.

\begin{theorem}[Shannon]
	In un cifrario perfetto l'insieme delle chiavi deve essere grande almeno quanto l'insieme dei messaggi possibili.
	Dove per \textbf{messaggio possibile} indichiamo un messaggio $m \in$ MSG tale che
	\[ P(M = m) > 0 \]
	Questa \`e condizione necessaria ma non sufficiente affinch\'e il cifrario sia perfetto.
	\begin{proof}
		Dimostriamo il teorema per assurdo e andiamo ad indicare con $N_k$ il numero delle chiavi e con $N_m$ il numero
		dei messaggi possibili. Supponiamo per assurdo che
		\[ N_m > N_k \]
		e consideriamo ora un crittogramma $c$ che pu\`o transitare sul canale con probabilit\`a
		\[ P(C = c) > 0 \]
		Se provassimo a decifrare $c$ con una generica chiave $k_i$, otterremo un messaggio $m_i$. Facciamo per\`o
		attenzione al fatto che, cifrando $c$ con una chiave $k_j$, potremmo ottenere il messaggio $m_i$, ottenibile
		anche con la chiave $k_i$. Indichiamo quindi con $s \leq N_k$ il numero dei messaggi che potrebbero corrispondere ù
		al crittogramma $c$. Ma per ipotesi
		\begin{gather*}
			N_k < N_m \\
			\Downarrow \\
			s \leq N_k < N_m
		\end{gather*}
		Dunque il numero dei messaggi che possono corrispondere al crittogramma $c$ \`e strettamente minore	del numero
		dei messaggi possibili.

		Questo significa che esiste un messaggio $m'$, appartenente allo spazio dei messaggi possibili, che non pu\`o
		corrispondere a quel crittogramma.
		\[ P(M = m' \mid C = c) = 0 \]
		Giungiamo quindi all'assurdo dato che un cifrario \`e perfetto se un crittogramma pu\`o corrispondere ad uno
		qualsiasi dei messaggi possibili.
	\end{proof}
\end{theorem}

\section{One-Time Pad}\label{one_time_pad}
Come abbiamo in parte anticipato, il cifrario \textbf{One-Time Pad}, altro non \`e che un cifrario di Vigen\`ere che
cifra e decifra sequenze binarie e dove la chiave, invece di essere corta e ripetuta, \`e lunga quanto il messaggio.

La prima parte del nome (One-Time) \`e relativa alla chiave: ogni chiave dev'essere utilizzata una sola volta e poi
buttata via.

\subsection{Funzionamento}\label{funzionamento_otp}
Consideriamo
\begin{itemize}
	\item \textbf{MSG}: lo spazio dei messaggi.
	\item \textbf{CRITTO}: lo spazio dei crittogrammi.
	\item \textbf{KEY}: lo spazio delle chiavi.
\end{itemize}
Messaggio, chiave e crittogramma sono una sequenza di $n$ bit. Il crittogramma si compone facendo lo XOR bit a bit di
messaggio e chiave
\[ c = m \oplus k \]
Lo XOR ritorna 0 se i bit che stiamo confrontando sono uguali, ritorna 1 altrimenti, dunque il crittoanalista
\begin{itemize}
	\item quando vede uno 0 in posizione $i$, sa che i bit di messaggio e chiave in posizione $i$ sono uguali
	      ma non sa se siano 0 o 1.
	\item quando vede un 1 in posizione $i$, sa che i bit di messaggio e chiave in posizione $i$ sono diversi
	      ma non sa quale sia 1 e quale sia 0.
\end{itemize}
Per effettuare la decifrazione basta rifare lo XOR bit a bit di crittogramma e chiave
\[ c_i \oplus k_i = m_i \]
si vede facilmente che il procedimento funziona
\begin{gather*}
	c_i = m_i \oplus k_i \\
	\Downarrow \\
	m_i \oplus k_i \oplus k_i = m_i
\end{gather*}
ma $k_i \oplus k_i$ \`e un sequenza di 0 e quindi
\[ m_i \oplus 0 = m_i \]

\subsection{Debolezza}\label{debolezza_otp}
La debolezza si ha dal punto di vista della generazione delle chiavi. Come abbiamo detto, la chiave dev'essere monouso.

Prendiamo come esempio il caso in cui due messaggi, $m_1$ ed $m_2$ siano cifrati, con la stessa chiave $k$, in due
crittogrammi
\begin{gather*}
	c_1 = m_1 \oplus k \\
	c_2 = m_2 \oplus k
\end{gather*}
A questo punto il crittoanalista potrebbe applicare lo XOR bit a bit fra i due crittogrammi per ottenere
\[ c_1 \oplus c_2 = (m_1 \oplus k) \oplus (m_2 \oplus k) \]
dato che vale la propriet\`a associativa pu\`o scrivere
\[ c_1 \oplus c_2 = (m_1 \oplus m_2) \oplus (k \oplus k) \]
Come prima $k \oplus k$ \`e una sequenza di 0 e quindi otteniamo
\[ c_1 \oplus c_2 = m_1 \oplus m_2 \]
Dalla sequenza di bit ottenuta, si pu\`o raffinare la propria conoscenza del messaggio andando a cercare lunghe sequenze
di 0, le quali, indicano che quella parte di messaggio \`e stata inviata due volte.

\subsection{Sicurezza}\label{sicurezza_otp}
Vogliamo ora dimostrare che il cifrario \`e perfetto. Per farlo lavoriamo sotto alcune ipotesi
\begin{itemize}
	\item Tutti i messaggi sono lunghi $n$. Se il messaggio \`e pi\`u corto di $n$ si fa del \emph{padding} (si
	      aggiungono caratteri casuali in fondo).
	      Se invece il messaggio \`e pi\`u lungo di $n$ faccio una cifratura a blocchi.
	\item Tutte le sequenze di $n$ bit sono messaggi possibili.
	\item I messaggi privi di significato vengono utilizzati per confondere la crittoanalisi e ognuno di essi ha una
	      probabilit\`a molto bassa, ma comunque maggiore di 0, di essere inviato.
	\item La chiave deve essere casuale e unica per ogni messaggio.
\end{itemize}

\begin{theorem}
	Sotto le ipotesi appena elencate, One-Time Pad \`e un cifrario perfetto e impiega un numero minimo di chiavi.
	\begin{proof}
		Dimostriamo per prima cosa la \textbf{minimalit\`a}, ossia
		\[ N_n = N_k = 2^n \]
		ma questo \`e immediato dato che le chiavi sono sequenze di bit lunghe quanto i messaggi.
	\end{proof}

	\begin{proof}
		Dimostriamo ora che il cifrario \`e perfetto. Come sappiamo, un cifrario \`e perfetto se, per ogni $m \in$ MSG
		e per ogni $c \in$ CRITTO, vale
		\[ P(M = m \mid C = c) = P(M = m) \]
		Partiamo dicendo che
		\[ P(M = m \mid C = c) = \frac{P(M = m \wedge C = c)}{P(C = c)} \]
		dove il valore al numeratore \`e la probabilit\`a che il messaggio inviato sia $m$ e che sia stato cifrato in
		$c$.

		Per come \`e fatto lo XOR, fissato $m$, chiavi diverse producono crittogrammi diversi. Esiste dunque
		un'\textbf{unica} chiave $k$ che cifra $m$ in $c$. Pi\`u formalmente, possiamo affermare che la probabilit\`a
		che il crittogramma sia $c$ \`e uguale alla probabilit\`a di scegliere a caso l'unica chiave $k$ che cifra $m$
		in $c$
		\[ P(C = c) = \frac{1}{2^n} \]
		Se la probabilit\`a di ottenere il crittogramma $c$ dipende solo dalla chiave, allora i due eventi al
		numeratore sono indipendenti e possiamo riscrivere la formula iniziale in questo modo:
		\[ P(M = m \mid C = c) = \frac{P(M = m) \cdot P(C = c)}{P(C = c)} \]
		Semplificando \`e immediato ottenere
		\[ P(M = m \mid C = c) = P(M = m) \]
	\end{proof}
\end{theorem}

\subsection{Scambio delle chiavi}\label{chiavi_otp}
Un metodo ragionevole per lo scambio di chiavi \`e quello che prevede lo scambio tra i due utenti del generatore casuale
e del suo assetto iniziale (seme).
\begin{enumerate}
	\item I due generatori vengono impostati allo stesso modo con lo stesso seme.
	\item Si scrive un messaggio $m$ e si genera un chiave $k$ lunga $|m|$ con il generatore.
	\item Si cifra il messaggio con $k$.
	\item Si genera la chiave $k$ di $|c|$ bit con il secondo generatore, che ricordiamo essere uguale al primo e
	      inizializzato con lo stesso seme.
	\item Si decifra il crittogramma $c$ con la chiave $k$ generata dal secondo generatore.
\end{enumerate}
Alla fine di questo processo si ha che i due generatori sono impostati di nuovo alla stessa maniera e si pu\`o quindi
continuare la comunicazione.

Il generatore deve essere crittograficamente sicuro e il seme deve essere molto lungo in modo da essere al riparo da
attacchi a forza bruta sul seme.

\subsection{Conclusioni}\label{conclusioni_otp}
Proviamo a rimuovere l'ipotesi secondo cui ogni messaggio sia possibile, anche quelli non significativi.

Dato che le chiavi devono essere tante quante i messaggi possibili, se restringiamo questo insieme anche lo spazio
delle chiavi diventerebbe pi\`u piccolo e con esso anche la lunghezza delle chiavi diminuirebbe.

In lingua inglese i messaggi significativi lunghi $n$ bit sono circa $\alpha^n$ con
\[ \alpha = 1.1 \]
Se consideriamo quindi solo l'insieme dei messaggi significativi in inglese, la cardinalit\`a dell'insieme di chiavi
passa da $2^n$ a $1.1^n$.

Come gi\`a detto, il numero delle chiavi dev'essere almeno quanto il numero dei messaggi possibili, quindi
\[ N_k \geq N_m = \alpha^n < 2^n \]
e dato che $\alpha^n < 2^n$ \`e possibile descrivere le chiavi con $t$ bit, con $t$ tale che
\begin{gather*}
	2^t \geq \alpha^n \\
	\Downarrow \\
	t \geq n \log \alpha \quad \approx \quad 0.12 \cdot n
\end{gather*}
Abbiamo cos\`i ridotto il numero di chiavi possibili a circa un decimo del numero di chiavi che avevamo prima.

Il problema \`e che avendo ridotto cos\`i tanto l'insieme delle chiavi, un attacco di tipo forza bruta torna ad avere
senso.

Quello che si fa in genere, per riuscire a mitigare il problema riuscendo comunque a diminuire un po' il numero di
chiavi e far s\`i che decifrando un crittogramma con chiavi diverse si riesca a risalire a diversi messaggi
significativi.

In altre parole, cifrando messaggi diversi con chiavi diverse si ottiene lo stesso crittogramma.

Per fare questo il numero di coppie $\langle m, k \rangle$ deve essere molto maggiore del numero di crittogrammi.
Supponiamo di usare chiavi casuali di $t$ bit. Se il numero di messaggi significativi \`e $\alpha^n$ abbiamo
\[ \alpha^n \cdot 2^t \]
possibili coppie $\langle m, k \rangle$, mentre il numero dei crittogrammi rimane $2^n$. Otteniamo dunque che
\begin{gather*}
	\alpha^n \cdot 2^t >> 2^n \\
	\Downarrow \\
	n \log \alpha + t >> n
\end{gather*}
Sviluppando ancora i calcoli otteniamo che le chiavi devono essere lunghe
\[ t >> n - n \log \alpha \quad \rightarrow \quad t >> 0.88 \cdot n \]
affinch\'e si verifichi il fenomeno descritto in precedenza.

La rimozione dell'ipotesi non ci permette quindi di risparmiare sui bit della chiave se si vuole mantenere un buon grado
di sicurezza. Siamo comunque riusciti a diminuire il numero delle chiavi. 			% CIFRARI PERFETTI
\chapter{Cifrari simmetrici}\label{critto_sim_massa}
Questi cifrari basano la loro sicurezza sulla difficolt\`a nel risolvere problemi complessi. Si dice quindi che hanno
una sicurezza di tipo computazionale e nascondono l'informazione a patto che il crittoanalista abbia risorse
computazionali limitate e sotto l'ipotesi che P $\neq$ NP.

La loro sicurezza si basa sui due \textbf{principi di Shannon}, i quali, rendono questi cifrari robusti alla
crittoanalisi statistica:
\begin{itemize}
	\item \textbf{Diffusione}: Il testo in chiaro si deve distribuire su tutto il crittogramma.

	      Ogni carattere del crittogramma deve cio\`e dipendere da tutti i caratteri del blocco del messaggio ottenendo
	      cos\`i un istogramma delle frequenze piatto.
	\item \textbf{Confusione}: Messaggio e crittogramma sono combinati fra loro in modo molto complesso per non
	      permettere al crittoanalista di separare le due sequenze tramite l'analisi statistica del crittogramma.

	      Per far s\`i che questo avvenga devono essere vere due condizioni:
	      \begin{itemize}
		      \item La chiave deve essere ben distribuita sul testo cifrato.
		      \item Ogni bit del crittogramma deve dipendere da tutti i bit della chiave.
	      \end{itemize}
\end{itemize}

\section{DES}\label{DES}
\`E stato il primo cifrario \textbf{certificato} proposto da IBM e che proponeva una struttura di questo tipo:
\begin{itemize}
	\item Il messaggio \`e diviso in blocchi, ciascuno cifrato e decifrato indipendetemente dall'altro.
	\item Ogni blocco \`e di 64 bit.
	\item Cifratura e decifrazione procedono in $r$ fasi o \textbf{round} in cui si ripetono le stesse operazioni. Noi
	      considereremo la versione del cifrario con 16 round.
	\item La chiave \`e composta da 8 byte. I primi 7 bit di ciascun byte sono scelti arbitrariamente e l'ottavo \`e
	      aggiunto per il controllo di parit\`a.
	      \begin{itemize}
		      \item La chiave contiene dunque 64 bit: 56 arbitrari e 8 di parit\`a.
		      \item Dalla chiave vengono create $r$ \textbf{sottochiavi di fase}.
	      \end{itemize}
\end{itemize}

\subsection{Funzionamento}\label{funzionamento_DES}
Sia $m$ il messaggio da inviare, $c$ il rispettivo crittogramma e $k$ la chiave. Il processo di cifratura \`e il seguente
\begin{enumerate}
	\item I bit del messaggio vengono permutati (blocco PI).
	\item La chiave viene privata dei bit di controllo parit\`a e i rimanenti vengono permutati (blocco T).
	\item Si dividono i bit del messaggio in due parti (S e D), ciascuna di 32 bit.
	\item Si entra in un ciclo di 16 fasi e per ogni fase $i$ abbiamo in input, l'output della fase precedente.

	      Alla chiave $k$ si applicano queste operazioni:
	      \begin{itemize}
		      \item I 56 bit della chiave vengono divisi in due parti da 28 bit ciascuna e si applica, a ciascuna delle
		            due parti, uno shift ciclico di 1 o 2 bit a seconda della fase in cui ci si trova.

		            Procedimento necessario affinch\'e vengano usati tutti i bit della chiave (\emph{confusione}).
		      \item Si estraggono 48 bit dai due blocchi di 28 bit del punto precedente, i quali andranno a formare la
		            sottochiave di fase.
		      \item Riconcateniamo le due sequenze shiftate che andranno poi a comporre la chiave per la fase
		            successiva.
	      \end{itemize}
	      I due blocchi del messaggio vengono trattati in questo modo:
	      \begin{itemize}
		      \item Si mandano i 32 bit di destra (input) nella parte di sinistra (output)
		            \[ S[i] = D[i-1] \]
		      \item Vengono copiati 16 bit della parte di destra, andando cos\`i a produrre un blocco da 48 bit.
		      \item Si fa lo XOR bit a bit tra il blocco appena prodotto e la sottochiave di fase.
		      \item I blocchi di 48 bit vengono riportati a 32 bit grazie alla \textbf{S-box} (approfondimento pi\`u
		            avanti).
		      \item Si permutano i bit prodotti al passo precedente.
		      \item Si fa lo XOR bit a bit tra il blocco appena prodotto e la parte sinistra in input, andando cos\`i
		            a comporre il nuovo blocco di destra.
	      \end{itemize}
	\item Parte destra e parte sinistra vengono unite di nuovo.
	\item Si permutano i bit del blocco ottenuto (blocco PF).
\end{enumerate}

\subsubsection{S-Box}
La \textbf{S-Box} \`e una funzione composta da 8 sotto-funzioni, ciascuna che prende in input 6 bit e ne restituisce
4.

Per farlo si prendono il primo e l'ultimo bit in input e se ne ricava un indice di riga, mentre con i rimanenti bit
si ricava un indice di colonna.

Tramite questi due indici si ottiene un valore presente in una tabella, le cui righe contengono ognuna una permutazione
dei primi 16 interi. Il valore identificato dai due indici \`e resituito in output di 4 bit.

\subsection{Sicurezza}\label{sicurezza_DES}
Un cifrario ha una sicurezza di $b$ bit se il costo del miglior attacco \`e di ordine $O(2^b)$ operazioni e richiede di
esplorare uno spazio delle chiavi di cardinalit\`a $2^b$.

Nel caso del DES abbiamo chiavi da 56 bit ma lo spazio delle chiavi ha cardinalit\`a $2^{55}$ dato che, se cifriamo
il complemento del messaggio col complemento della chiave, otteniamo il complemento del crittogramma. I bit di
sicurezza non sono quindi 56 ma 55.

In sostanza escludere una chiave ci permette di escludere anche il suo complemento.

\subsection{Attacchi}\label{attacchi_DES}
Il DES, per quanto complesso, si \`e rivelato vulnerabile a diversi attacchi di diversa natura.

\subsubsection{Attachi distribuiti}
Uno degli attacchi di cui il DES \`e stato vittima \`e quello di tipo \textbf{distribuito}, ossia, un attacco a
forza bruta, distribuito su pi\`u macchine. Con questo tipo di attacco si \`e riusciti a forzare il cifrario in tempi
sempre pi\`u brevi.

\subsubsection{Chosen plain text}
\begin{enumerate}
	\item Si prende un messaggio $m$ e lo si cifra in $c_1$.
	\item Si cifra $\overline{m}$, ossia il complemento di $m$, in $c_2$.
	\item Per ogni chiave $k$ si prova a cifrare $m$ con $k$.
	      \begin{itemize}
		      \item Se si ottiene $c_1$ molto probabilmente $k$ \`e la chiave (non sicuramente dato che potrebbero
		            esserci altre chiavi che mappano $m$ in $c_1$).
		      \item Se la cifratura ha invece prodotto $\overline{c_2}$ allora \`e probabile che $\overline{k}$ sia
		            la chiave.

		            Questo perch\'e provando a cifrare il complemento del messaggio col complemento della chiave si
		            ottiene il complemento del crittogramma. Nel nostro caso
		            \[ C(\overline{m}, \overline{k}) = \overline{\overline{c_2}} = c_2 \]
		      \item Altrimenti n\'e $k$ n\'e $\overline{k}$ sono le chiavi ma con una sola cifratura vengono scartate
		            due chiavi.
	      \end{itemize}
\end{enumerate}

\subsubsection{Crittoanalisi differenziale}
Un altro attacco di tipo \emph{chosen plain text} si basa sulla \textbf{crittoanalisi differenziale}, la quale necessita
di almeno $2^{47}$ coppie $\langle m, c \rangle$ per funzionare e sfrutta l'analisi probabilistica per stimare quale
chiave \`e stata usata andando a cercare variazioni nei vari crittogrammi.

Il costo di questo attacco \`e tuttavia dell'ordine di $O(2^{55})$ operazioni per via delle 16 fasi del cifrario, le
quali, rendono l'attacco leggermente pi\`u dispendioso del forza bruta.

\subsubsection{Crittoanalisi lineare}
L'ultima tecnica di attacco che vediamo \`e basata sulla \textbf{crittoanalisi lineare}. \`E un attacco di tipo
\emph{know plain text} e serve a stimare alcuni bit della chiave.

Per effettuare l'attacco si necessita di $2^{43}$ coppie $\langle m, c \rangle$ ed \`e meno costosa del forza bruta.

\subsection{Miglioramenti}\label{miglioramenti_DES}
Visti i problemi del DES e le sue, ormai note vulnerabilit\`a, si \`e provato a migliorarlo apportando qualche modifica.

\subsubsection{Chiavi}
Si \`e provato a cambiare sempre le chiavi di fase, arrivando ad avere 768 bit di chiave complessivi. In realt\`a,
per attacchi basati su crittoanalisi differenziale, i bit di sicurezza sono 61, aggiungendo cos\`i solo 6 bit di
sicurezza al fronte di una chiave molto pi\`u lunga.

\subsubsection{Cifratura doppia}
L'approccio che invece \`e stato adottato \`e stata la \textbf{cifratura multipla}, ossia, la composizione del DES con
se stesso. Scelte due chiavi $k_1$ e $k_2$ qualsiasi, vale che
\[ C(C(m, k_1), k_2) \neq C(m, k_3) \]
per qualunque chiave $k_3$ nello spazio delle chiavi. In questo modo otteniamo chiavi di 112 bit ma con 57 bit di
sicurezza.

Ci\`o che riduce molto i bit di sicurezza sono gli attacchi di tipo \textbf{meet in the middle}: data una coppia
$\langle m, c \rangle$
\begin{enumerate}
	\item Per ogni $k_1$ si calcola e si salva in una tabella
	      \[ C(m, k_1) \]
	\item Per ogni $k_2$ si calcola e si cerca nella tabella
	      \[ D(c, k_2) \]
	\item Se troviamo una corrispondenza $k_1$ e $k_2$ probabilmente sono le chiavi.
\end{enumerate}
L'attacco si basa sul fatto che se il crittogramma $c$ \`e generato da
\[ C(C(m, k_1), k_2) \]
allora vale
\[ D(c, k_2) = C(m, k_1) \]
Quello che di fatto andiamo a fare \`e enumerare tutte le chiavi due volte (non tutte le coppie di chiavi) e poi
cerchiamo una corrispondenza.

Se le chiavi sono $2^{56}$ basta moltiplicare per 2 e otteniamo cos\`i $2^{57}$ operazioni per forzare il cifrario al
fronte di una chiave lunga 112 bit.

\subsubsection{Cifratura tripla}
Per giungere ad una sicurezza significativa si \`e arrivati a comporre il DES con se stesso tre volte. Si parla di
3TDEA, nel caso si utilizzino tre chiavi
\[ c = C(D(C(m, k_1), k_2), k_3) \]
o di 2TDEA, nel caso si utilizzino due chiavi
\[ c = C(D(C(m, k_1), k_2), k_1) \]
\textbf{NOTA}: usare la funzione di decifrazione tra le due cifrature non aumenta n\'e diminuisce la sicurezza, \`e
solo per rendere il sistema retrocompatibile con l'applicazione singola del DES.

Con questo nuovo metodo andiamo ad ottenere, in entrambi i casi, 112 bit di sicurezza ma, come vedremo fra poco, la
versione a due chiavi si rivela pi\`u conveniente.

Un attacco di tipo \emph{meet in the middle} sul 3TDEA sfrutta la relazione
\[ C(D(C(m, k_1), k_2), k_3) \]
la quale pu\`o essere riscritta come
\[ D(c, k_3) = D(C(m, k_1), k_2) \]
e se cifriamo con chiave $k_2$ entrambi i membri otteniamo
\[ C(D(c, k_3), k_2) = C(m, k_2) \]
Sapendo questo e data una coppia $\langle m, c \rangle$, l'attacco si compone delle seguenti fasi
\begin{enumerate}
	\item Si enumerano tutte le $2^{56}$ possibili chiavi $k_1$ e si calcola
	      \[ C(m, k_1) \]
	      salvando i risultati in una tabella.
	\item Si enumerano tutte le $2^{112}$ possibili coppie di chiavi $\langle k_2, k_3 \rangle$ e si calcola
	      \[ C(D(c, k_3), k_2) \]
	\item Se si trova una corrispondeza, allora le chiavi $k_1$, $k_2$ e $k_3$, molto probabilmente, sono le chiavi
	      usate.
\end{enumerate}
L'attacco ha quindi un costo complessivo di $O(2^{112})$ operazioni mentre un attacco di tipo forza bruta ne richiede
$O(2^{168})$.

La versione 2TDEA ha comunque 112 bit di sicurezza ma utilizza solo due chiavi da 56 bit ciascuna e dunque, una chiave
complessiva di 112 bit cos\`i da avere tutti i bit della chiave come bit di sicurezza.

\section{AES}\label{AES}
Si tratta di un cifrario che fa uso di chiavi brevi (128, 192 o 256 bit) e ripetute, le quali devono essere cambiate
per ogni nuova sessione di comunicazione.

Il messaggio \`e diviso in blocchi lunghi sempre 128 bit (a prescindere dalla lunghezza della chiave) ma cambia il
numero di fasi di cui si compone il processo di cifratura: 10 per chiavi da 128 bit, 12 per chiavi da 192 bit e 14
per chiavi da 256 bit.

\subsection{Gestore delle chiavi}
Per ora vediamo solamente la versione con 128 bit di chiave, la quale viene caricata per colonne in una matrice $W$ da
16 byte. Tale matrice viene ampliata aggiungendo ricorsivamente 40 colonne a partire dalle 4 iniziali per generare le
10 sottochiavi di fase secondo questa regola
\[
	W[i] = \begin{cases}
		W[i - 4] \oplus W[i - 1]    & \text{se } 4 \nmid i \\
		W[i - 4] \oplus T(W[i - 1]) & \text{se } 4 \mid i
	\end{cases}
\]
dove $T$ \`e una trasformazione \emph{non lineare} (S-Box) che rende il cifrario robusto ad attacchi basati su
\emph{crittoanalisi lineare}.

\subsection{Operazioni}
Dato che ogni blocco \`e di 128 bit, il cifrario lo organizza in una matrice $B$ di dimensione $4 \times 4$ da 16 byte.
Dopo di che iniziano le varie fasi, le quali si compongono di 4 operazioni principali.

\subsubsection{S-Box}
La \textbf{S-Box}, in questo caso, \`e una matrice di $16 \times 16$ byte, che contiene una permutazione di tutti i 256
interi rappresentabili con 8 bit.

Sia $B$ la matrice che contiene il blocco del messaggio e sia $b_{i, j}$ un byte in posizione $(i, j)$ di tale matrice,
la S-Box mappa $b_{i, j}$ in $a_{i, j}$ usando i primi 4 bit di $b_{i, j}$ per ricavare un indice $r$ di riga e gli
ultimi 4 per ricavare un indice $c$ di colonna.

Il valore $a_{i, j}$ restituito, \`e il valore presente nella S-box in posizione $(r, c)$ scritto con 8 bit.

\paragraph{Formulazione matematica}
La S-box calcola l'\textbf{inverso moltiplicativo} di ogni byte $b_{i, j}$, considerando il byte come un elemento del
campo \emph{finito} GF($2^8$). Questa operazione \`e ci\`o che rende la funzione $T$, che abbiamo citato in precedenza,
\emph{non lineare}.

Ogni byte $b_{i, j}$ viene prima sostituito con il suo inverso moltiplicativo in GF($2^8$) e poi moltiplicato per una
matrice di $8 \times 8$ bit sommato con un vettore colonna.

\subsubsection{Shift delle righe}
I byte di ogni riga vengono shiftati ciclicamente verso sinistra di 0, 1, 2, e 3 posizioni, rispettivamente. In questo
modo i 4 byte di ogni colonna si disperdono su 4 colonne diverse.

\subsubsection{Rimescolamento delle colonne}
Ogni colonna del blocco, trattata come un vettore di 4 elementi, viene moltiplicata per una matrice $M$ prefissata di
$4 \times 4$ byte.

La moltiplicazione \`e eseguita modulo $2^8$ e la somma modulo 2 (operazioni del campo GF($2^8$)).

Questo fa s\`i che ogni byte della colonna rimescolata dipenda da tutti i byte della colonna di partenza.

\subsubsection{Somma della chiave di fase}
L'ultima operazione di ogni fase \`e la somma della chiave di fase con lo XOR bit a bit.

Si utilizza la chiave di fase fornita dal gestore delle chiavi e si fa lo XOR bit a bit con il blocco in output
dall'operazione di rimescolamento delle colonne e ottengo cos\`i un nuovo blocco da dare in input alla fase successiva.

\subsection{Sicurezza}
Tutti i bit sono di sicurezza e, ad oggi, nessun attacco \`e stato in grado di compromettere AES anche nella sua
versione pi\`u semplice con chiave a 128 bit.

Esistono attacchi pi\`u efficienti di un attacco esauriente sulle chiavi per le versioni con 6 fasi, ma nessun attacco
\`e pi\`u efficiente se le fasi sono almeno 7.

Si conoscono attacchi \textbf{side-channel} che sfruttano debolezze della piattaforma sui cui esso \`e implementato.

\section{Cifrario a blocchi}
In un \textbf{cifrario a blocchi}, per trattare messaggi con lunghezza diversa da un multiplo della lunghezza del
messaggio, si fa del \textbf{padding}, ovvero si inseriscono bit casuali per riempire il blocco finale.

Cifrare a blocchi espone tuttavia la comunicazione ad attacchi dato che
\begin{itemize}
	\item Blocchi uguali nel messaggio producono blocchi uguali nel crittogramma (se cifrati con la stessa
	      chiave).
	\item C'\`e \textbf{poca diffusione} tra un blocco e l'altro.
	\item C'\`e \textbf{periodicit\`a} nel crittogramma.
\end{itemize}
Per risolvere il problema \`e necessario che a blocchi uguali del messaggio corrispondano blocchi diversi del
crittogramma. Per riuscire ad ottenere questo risultato, si fa ricorso ad una tecnica chiamata
\textbf{composizione di	blocchi}, in cui, la cifratura di un blocco, deve dipendere dalla cifratura dei precedenti.

\subsection{Cipher Block Chaining - CBC}
Per la cifratura del blocco $m_i$ si calcola
\[ c_i = C(m_i \oplus c_{i-1}, k) \]
dove $c_{i-1}$ \`e il blocco cifrato al passo precedente.

Al primo passo si deve usare una sequenza $c_0$ per effettuare la prima cifratura. Questa sequenza pu\`o essere
scambiata in chiaro e pu\`o essere una qualsiasi sequenza di 128 bit, a patto che sia cambiata per ogni blocco che
viene cifrato.

Per la decifrazione del blocco $c_i$ si calcola
\[ m_i = c_{i-1} \oplus D(c_i, k) \]
La fase di decifrazione pu\`o essere fatta in parallelo, a differenza della cifratura che deve essere sequenziale. 			% CIFRARI SIMMETRICI PER CRITTOGRAFIA DI MASSA
\chapter{Cifrari a chiave pubblica}
Nei cifrari simmetrici si ha un grosso problema, ovvero, lo \textbf{scambio della chiave}, che, come sappiamo, deve
essere la stessa per entrambi gli utenti.

Fino ad ora abbiamo sempre assunto che i due utenti fossero gi\`a in possesso della chiave e abbiamo solo parlato del
metodo di cifratura e decifrazione. Come abbiamo visto i messaggi scambiati sono cifrati e dunque la comunicazione \`e
sicura ma come avviene lo scambio della chiave ?
\begin{itemize}
	\item Se avviene di persona allora tanto vale scambiarsi direttamente il messaggio.
	\item Se avviene in chiaro non ha pi\`u senso cifrare il messaggio dato che chiunque potrebbe intercettare la
	      chiave e decifrarlo senza sforzo.
	\item Se inviassimo la chiave cifrata si innescherebbe lo stesso problema all'infinito: il destinatario avrebbe
	      bisogno della chiave di cifratura per decifrare e dunque si dovrebbe inviare un'altra chiave e cos\`i via.
\end{itemize}
I \textbf{cifrari a chiave pubblica} risolvono il problema.

\section{Soluzione ingenua}
Se abbiamo un sistema con $N$ utenti, ogni utente pu\`o memorizzare $N-1$ chiavi diverse e condivise con ciascun altro
utente.

In questo modo abbiamo un numero quadratico di chiavi nel numero di utenti del sistema.

Quando un utente $i$ vuole comunicare con l'utente $j$ manda un messaggio a $j$ cifrandolo con la chiave $k_j$. L'utente
$j$ a questo punto decifra il messaggio con la sua chiave $k_j$ e invia un messaggio a $i$ cifrandolo con la chiave $k_i$.

\section{TTP}
Una soluzione migliore \`e rappresentata dal \textbf{TTP} o \textbf{trusted 3rd party}, ossia una terza parte
\emph{fidata}, a cui gli utenti si appoggiano per comunicare.

Ogni utente deve ricordarsi una sola chiave mentre TTP gestisce la creazione e lo scambio delle chiavi condivise tra i
due utenti.

Siano $A$ e $B$ i due utenti che vogliono comunicare, il processo di scambio funziona in questo modo:
\begin{enumerate}
	\item $A$ e $B$ possiedono rispettivamente $k_A$ e $k_B$, due chiavi generate da loro stessi.
	\item $A$ comunica a TTP di voler comunicare con $B$.
	\item TTP genera casualmente una chiave $k_{AB}$ che potranno usare i due utenti per quella comunicazione.
	\item TTP cifra genera due crittogrammi $c_A$ e $c_B$ in questo modo
	      \[
		      \begin{matrix}
			      c_A & = & C(k_{AB}, k_A) \\
			      c_B & = & C(k_{AB}, k_B)
		      \end{matrix}
	      \]
	\item TTP invia $c_A$ e $c_B$ ad $A$.
	\item $A$ decifra $c_A$ con $k_A$ e invia a $c_B$ a $B$.
	\item $B$ decifra $c_B$ con $k_B$.
\end{enumerate}
Alla fine di questo processo i due utenti sono in possesso di una chiave $k_{AB}$ che potranno usare per quella
comunicazione in un cifrario simmetrico.

Le problematiche di questo sistema sono due:
\begin{itemize}
	\item TTP deve essere sempre online.
	\item TTP conosce tutte le chiavi.
\end{itemize}
\`E un approccio utilizzabile solo in un sistema ristretto come un'universit\`a o un'azienda.

\section{Chiave pubblica}
In questo tipo di cifratura si vuole implementare un meccanismo che permette a chiunque di inviare messaggi cifrati a un
certo utente ma permettere solo a quell'utente di decifrarli.

Le operazioni di cifratura e decifrazione sono pubbliche e utilizzano due chiavi diverse:
\begin{itemize}
	\item \textbf{Chiave pubblica}: \`e nota a tutti.
	\item \textbf{Chiave private}: nota solo al destinatario.
\end{itemize}
Questa coppia di chiavi \`e generata dall'utente in veste di destinatario, il rende nota la sua chiave pubblica e mantiene
segreta la sua chiave privata.

Se volessimo inviare un messaggio all'utente $i$ si dovrebbe cifrare il messaggio con la sua chiave pubblica. L'utente $i$
decifra il crittogramma con la sua chiave privata.

In questo sistema l'unica cosa non nota a tutti \`e la chiave privata del destinatario. Le funzioni di cifratura e
decifrazione e la chiave pubblica sono note a qualsiasi utente.

\subsection{Requisiti}
Perch\'e la cifratura a chiave pubblica devono essere soddisfatti alcuni requisiti.
\begin{itemize}
	\item Il procedimento di cifratura e decifrazione devono essere implementati correttamente. Il destinatario deve essere
	      in grado di decifrare qualsiasi messaggio con la propria chiave privata.
	      \[ D(C(m, k_{\text{pub}}), k_{\text{priv}}) = m \]
	\item Efficienza e sicurezza del sistema:
	      \begin{itemize}
		      \item La coppia di chiavi \`e \emph{facile} da generare e deve risultare praticamente impossibile che due
		            utenti scelgano la stessa chiave.
		      \item Dati $m$ e $k_{\text{pub}}$ \`e \emph{facile} per il mittente produrre il crittogramma.
		      \item Dati $c$ e $k_{\text{priv}}$ \`e \emph{facile} per il destinatario produrre il messaggio originale.
		      \item Pur conoscendo la chiave pubblica e le funzioni di cifratura e decifrazione deve essere
		            \emph{difficile} per il crittoanalista risalire al messaggio in chiaro.
	      \end{itemize}
\end{itemize}
La soluzione risiede nel trovare una funzione di tipo \textbf{one-way trap-door}, ovvero, una funzione facile da calcolare
e difficile da invertire a meno che non si conosca la chiave privata.

\section{RSA}
Questo cifrario usa l'algebra modulare e si basa sulla moltiplicazione di due numeri primi $p$ e $q$ poich\'e calcolare
\[ n = p \cdot q \]
richiede tempo quadratico nella lunghezza della loro rappresentazione ma, ricostruire $p$ e $q$ a partire da $n$ richiede
tempo esponenziale se $p$ e $q$ sono primi.

Se si conosce tuttavia uno dei due fattori, risalire all'altro \`e facile (basta fare una divisione). 	% CIFRARI A CHIAVE PUBBLICA
\chapter{Crittografia su curve ellittiche}
La \textbf{crittografia su curve ellittiche} nasce per alleggerire il carico computazionale che si porta dietro
la crittografia a chiave pubblica basata sull'algebra modulare.

I problemi su cui si basa la crittografia a chiave pubblica, come la fattorizzazione e il calcolo del logaritmo
discreto hanno due problemi:
\begin{itemize}
	\item Sebbene si risolvano in tempo polinomiale sono comunque \textbf{calcoli molto pesanti}.
	\item Gli algoritmi odierni per il calcolo di questi due problemi non hanno pi\`u costo esponenziale ma,
	      come abbiamo gi\`a visto, \textbf{subesponenziale}. Ne \`e una diretta conseguenza l'aumento della
	      lunghezza delle chiavi.
\end{itemize}
Come vedremo, la crittografia su curve ellittiche, propone i cifrari visti in precedenza (protocollo DH e cifrario
di ElGamal) in un contesto matematico diverso e basati sul calcolo del logaritmo discreto e non pi\`u sulla
fattorizzazione.

Il risultato \`e stato quello di ottenere un costo del miglior attacco, come vedremo, \textbf{puramente esponenziale}.

\section{Curve ellittiche}
Prima di vedere i dettagli implementativi definiamo cosa sono le curve ellittiche.

Le curve ellittiche sono curve algebriche descritte da equazioni cubiche (simili a quelle utilizzate nel calcolo
della lunghezza degli archi delle ellissi).
\[ E(a, b) = \{ (x, y) \in \mathbb{R}^2 \mid y^2 = x^3 + ax + b \} \]
qui sono definite sui reali, nella cosiddetta \emph{forma normale di Weierstrass}, ma possiamo scrivere l'equazione
su un campo $\mathbb{K}$ qualisiasi.

L'insieme appena descritto contiene anche il cosiddetto \textbf{punto all'infinito} $O$ in direzione dell'asse $y$
(la curva ha un asintoto verticale), il quale rappresenta l'\textbf{elemento neutro per l'addizione}.

La curva si pu\`o presentare in due forme
\begin{itemize}
	\item A \textbf{due componenti} nel caso in cui la cubica abbia tre radici reali.
	\item Ad \textbf{una componente} nel caso in cui la cubica abbia una sola radice reale e due complesse coniugate.
\end{itemize}
A prescindere dalla forma, in ogni punto della curva \`e possibile mandare una \textbf{tangente}.

Per le applicazioni crittografiche si assume che il discriminante della cubica sia
\[ 4 a^3 + 27 b^2 \neq 0 \]
Questo ci assicura che
\begin{itemize}
	\item La cubica non abbia radici multiple.
	\item La curva sia priva di punti singolari come \emph{cuspidi} o \emph{nodi}, dove non sarebbe definita in modo
	      univoco la tangente.
\end{itemize} 	% CIFRARI SU CURVE ELLITTICHE
\chapter{Protocolli}
In questo capitolo andremo a trattare i protocolli di sicurezza tra i quali distingueremo
\begin{itemize}
	\item \textbf{Identificazione}: un sistema di elaborazione ha bisogno di accertarsi dell'\textbf{identit\`a}
	      di un utente che vuole accedere ai suoi servizi.
	\item \textbf{Autenticazione}: il destinatario di un messaggio deve essere in grado di accertare
	      l'\textbf{identit\`a del mittente} e l'\textbf{integrit\`a del crittogramma} ricevuto.
	\item \textbf{Firma digitale}: la firma digitale si pone tre obbiettivi:
	      \begin{itemize}
		      \item Il mittente non deve poter \textbf{negare l'invio} di un messaggio.
		      \item Il destinatario deve essere in grado di \textbf{autenticare} il messaggio.
		      \item Il destinatario non deve poter sostenere di aver ricevuto un messaggio diverso da quello inviato
		            dal mittente.
	      \end{itemize}
\end{itemize}
Come si pu\`o notare, ognuna di queste funzionalit\`a, estende l'altra
\begin{itemize}
	\item L'autenticazione garantisce l'identificazione.
	\item La firma digitale garantisce l'autenticazione.
\end{itemize}
Ognuna di queste funzionalit\`a ha il compito di proteggere le comunicazioni da attacchi attivi, come per esempio
gli attacchi \emph{man in the middle}.

\section{Funzioni hash}
Per l'implementazione di queste funzionalit\`a faremo ricorso alla \textbf{funzioni hash}. Una funzione hash
\[ f : X \rightarrow Y \]
\`e una funzione tale che
\[ n = |X| >> m = |Y| \]
ossia, \`e tale se, definito il dominio $X$, il codominio $Y$ della funzione \`e molto pi\`u piccolo.

Inoltre, una funzione hash, ha molti elementi del dominio che vengono mappati nella stessa immagine del codominio.
Questo ci permette di \emph{partizionare} il dominio in sottoinsiemi
\[ X = X_1 \cup X_2 \cup \dots \cup X_m \]
tali che ogni elemento del sottoinsieme \`e mappato, dalla funzione hash, nella stessa immagine del codominio
\[ \forall i, \quad \forall x \in X_i \quad \quad f(x) = y \]
Una funzione hash \`e buona se la cardinalit\`a di ogni sottoinsieme sia circa la stessa.

\subsection{Funzioni hash one-way}
Le funzioni hash usate in crittografia devono soddisfare tre requisiti principali
\begin{itemize}
	\item Per ogni $x \in X$ deve essere \emph{facile} calcolare
	      \[ y = f(x) \]
	\item \textbf{One-way}: per la maggior parte degli $y \in Y$ deve essere \emph{difficile} determinare $x \in X$
	      tale che
	      \[ f(x) = y \]
	\item \textbf{Claw-free}: deve essere \emph{difficile} determinare una coppia $x_1, x_2 \in X$ tali che
	      \[ f(x_1, x_2) \]
\end{itemize}

\subsection{Funzione hash SHA-1}
Una delle funzioni, storicamente usata in crittografia, \`e \textbf{SHA-1}, la quale opera su sequenze lunghe fino
a $2^{64} - 1$ bit e produce immagini di 160 bit.

In particolare opera su blocchi di 160 bit, contenuti in un buffer di 5 registri da 32 bit ciascuno, in cui sono
caricati inizialmente dei valori pubblici.

Il messaggio viene concatenato con una sequenza di \emph{padding} che rende la sua lunghezza un multiplo di 512 bit.

Il contenuto dei registri varia nel corso dei cicli successivi in cui questi valori si combinano tra loro e con
blocchi di 32 bit provenienti dal messaggio.

La funzione sfrutta shift ciclici e una componente non lineare che varia per riuscire ad ottenere il valore hash per
ciascun messaggio in input.

\section{Identificazione}
Vediamo ora come viene applicato il protocollo di \textbf{identificazione} sia nel caso in cui ci trovassimo su un
canale sicuro sia nel caso in cui ci trovassimo su un canale insicuro.

\subsection{Canale sicuro}
Prendiamo per esempio la situazione in cui un utente voglia accedere ai propri file personali memorizzati su un
calcolatore ad accesso riservato ai membri della sua organizzazione.
\begin{enumerate}
	\item L'utente invia in chiaro nome utente e password.
	\item Dato che il canale \`e sicuro, un attacco pu\`o essere sferrato solo da un utente locale al sistema o da
	      un hacker.
\end{enumerate}
Il meccanismo di identifiaczione dovrebbe per\`o prevedere una \textbf{cifratura} della password tramite funzioni hash
one-way anche su canali sicuri.

Dobbiamo distinguere due casi: quando l'utente si registra e quando l'utente effettua tutti gli accessi successivi.

\subsubsection{Registrazione}
Possiamo associare la fase di registrazione alla fase di cifratura con una funzione hash one-way $h$:
\begin{enumerate}
	\item L'utente $u$ si registra fornendo per la prima volta la password $p$.
	\item Il sistema associa a $p$ due sequenze binarie che memorizza nel file delle password al posto di $p$.
	      \begin{itemize}
		      \item Un \textbf{seme} casuale $s$ prodotto da un generatore.
		      \item Il \textbf{valore hash} della concatenazione tra $p$ ed $s$.
		            \[ q = h(p s) \]
	      \end{itemize}
\end{enumerate}
Quello che quindi viene salvato dal sistema \`e l'\textbf{immagine hash} della password concatenata ad un seme casuale
e non la password in chiaro.

\subsubsection{Accesso}
In fase di accesso il sistema compie i seguenti passaggi
\begin{enumerate}
	\item Recupera $s$ dal file delle password.
	\item Concatena $s$ con la password $p$ fornita da $u$.
	\item Calcola il valore hash della sequenza $p s$.
	\item Se $q = h(p s)$ l'identificazione ha successo.
\end{enumerate}
Avere accesso al file delle password non fornisce informazioni interessanti perch\'e \`e computazionalmente difficile
ricavare la password dalla sua immagine hash.

\subsection{Canale insicuro}
Nel caso di canale insicuro non si pu\`o inviare la password in chiaro e dunque andremo a lavorare con sistemi a
chiave pubblica per l'invio della password.

In realt\`a un sistema non dovrebbe mai poter maneggiare password in chiaro ma solo una loro immagine hash.

Supponiamo che l'utente $u$ voglia accedere ai servizi forniti da un certo sistema $s$ e per farlo \`e necessario
che si identifichi.

Supponiamo che il sistema adotti un cifrario a chiave pubblica (per esempio l'RSA) per lo scambio sicuro dei dati
dell'utente. L'utente dispone quindi di una chiave pubblica $\langle e, n \rangle$ e di una chiave privata $d$.

Quello che avviene all'atto pratico \`e questo:
\begin{enumerate}
	\item $s$ genera un numero casuale $r < n$ e lo invia in chiaro a $u$.
	\item $u$ calcola
	      \[ f = r^d \mod{n} \]
	      che rappresenta la \textbf{firma} di $u$ su $s$ e lo invia a $s$.
	\item $s$ verifica che
	      \[ f^e \mod{n} = r \]
	      se l'uguaglianza \`e soddisfatta, l'identificazione ha successo.
\end{enumerate}
Come possiamo notare, le operazioni di cifratura e decifrazione sono invertite rispetto all'impiego standard dell'RSA
ed \`e possibile farlo dato che sono commutative
\[ (x^e \mod{n})^d \mod{n} = (x^d \mod{n})^e \mod{n} \]
Chiariamo inoltre che solo $u$ pu\`o produrre $f$ dato che \`e l'unico che possiede il valore $d$.

Se il passo 3 va a buon fine, il sistema ha la garanzia che l'utente che ha richiesto l'identificazione sia
effettivamente $u$, anche se il canale \`e insicuro.

Il protocollo funziona bene a patto che il sistema sia onesto. Dato che \`e il sistema che fornisce $r$, se $r$ \`e
effettivamente un valore casuale allora tutto va a buon fine, se invece $r$ \`e un dato con particolari propriet\`a
utili a ricavare la chiave privata di $u$.

\subsection{Protocollo a conoscenza zero}
Questo protocollo permette ad un utente di dimostrare la sua identit\`a o la sua conoscenza di un certo segreto,
convincendo un altro utente di essere in possesso di una certa capacit\`a, senza rivelare niente oltre alla
veridicit\`a di questa sua capacit\`a.

In questo nuovo paradigma abbiamo due utenti:
\begin{itemize}
	\item \textbf{Prover}: indicato con $P$, ha il compito di dimostrare la sua capacit\`a o conoscenza.
	\item \textbf{Verifier}: indicato con $V$, ha il compito di verificare la veridicit\`a di ci\`o che afferma $P$.
\end{itemize}
Supponiamo quindi che $P$ voglia dimostrare a $V$ di possedere una certa capacit\`a, un protocollo \`e il seguente
\begin{enumerate}
	\item $V$ mette alla prova $P$ chiedendogli di dimostrare la sua capacit\`a su un problema che lui gli pone.
	\item $P$ risponde.
	\item Se la risposta \`e corretta, $V$ genera un valore casuale e modifica il problema in base al valore generato.
	\item $V$ chiede a $P$ di risolvere il problema modificato.
	\item Se $P$ risponde bene, allora $V$ continua a modificare il problema e a chiedere a $P$ di risolverlo, fino ad
	      essere sicuro che $P$ possegga effettivamente la capacit\`a da lui dichiarata.
	\item Se $P$ sbaglia anche solo una risposta, allora $V$ pu\`o immediatamente dedurre che $P$ dica il falso.
\end{enumerate}
Se le sfide proposte da $V$ sono $k$, la probabilit\`a che $P$ stia dicendo il falso, \`e
\[ \left( \frac{1}{2} \right)^k \]
Possiamo quindi concludere che, pi\`u $k$ \`e alto, meno sono le probabilit\`a che $P$ stia dicendo il falso.

\subsubsection{Propriet\`a fondamentali}
Di seguito le propriet\`a fondamentali del protocollo:
\begin{itemize}
	\item \textbf{Completezza}: se $P$ \`e onesto, $V$ accetta sempre la dimostrazione.
	\item \textbf{Correttezza}: se $P$ \`e disonesto, la probabilit\`a che $P$ riesca ad ingannare $V$ \`e al pi\`u
	      $(1/2)^k$ con $k$ scelto da $V$.
	\item \textbf{Conoscenza zero}: se $P$ \`e onesto, un verificatore (anche disonesto), non pu\`o acquisire nessuna
	      informazione se non la veridicit\`a dell'affermazione.
\end{itemize}

\subsection{Protocollo di Fiat-Shamir}
Questa non \`e altro che un'applicazione del protocollo a conoscenza zero con chiavi pubbliche e private.

Il \emph{prover} \`e impersonato dall'utente che vuole dimostrare di essere il legittimo proprietario di una chiave
privata associata ad una certa chiave pubblica senza usarla su dati scelti dall'utente \emph{verifier}.

Il protocollo si basa sulla difficolt\`a del calcolo di una radice in modulo un numero composto.

L'utente $P$, in fase di preparazione
\begin{enumerate}
	\item Sceglie due numeri primi $p$ e $q$ molto grandi.
	\item Calcola $n = pq$.
	\item Sceglie una sorta di chiave privata $s < n$.
	\item Calcola
	      \[ t = s^2 \mod{n} \]
	\item Rende nota la coppia $\langle t, n \rangle$ e mantiene privata la tripla $\langle p, q, s \rangle$.
\end{enumerate}
A questo punto $P$ vuole dimostrare a $V$ di conoscere una radice di $t$, ovvero $s$ senza per\`o inviargliela.

Per $k$ volte, con $k$ scelto da $V$, si ripetono i seguenti passi
\begin{enumerate}
	\item $V$ chiede a $P$ di iniziare una iterazione.
	\item $P$ genera un numero casuale $r < n$, calcola
	      \[ u = r^2 \mod{n} \]
	      e lo invia a $V$.
	\item $V$ genera un bit casuale $e$ e lo invia a $P$.
	\item $P$ calcola
	      \[ z = r \cdot s^e \mod{n} \]
	      e lo invia a $V$.
	      \begin{itemize}
		      \item Se $e = 0$ allora $z = r \mod{n}$.
		      \item Se $e = 1$ allora $z = r \cdot s \mod{n}$
	      \end{itemize}
	\item $V$ calcola
	      \[ x = z^2 \mod{n} \]
	      e controlla se
	      \[ x = u \cdot t^e \mod{n} \]
	      Se l'uguaglianza \`e vera si torna al passo 1, altrimenti $P$ non \`e identificato.
\end{enumerate}

\subsubsection{Completezza}
In questo caso, se $P$ \`e davvero in possesso di una radice di $t$, il verificatore lo identifica.
\begin{itemize}
	\item Se $e = 0$ allora
	      \[ x \quad = \quad u \cdot t^e \mod{n} \quad = \quad u \mod{n} \]
	\item Se $e = 1$ allora
	      \[ x \quad = \quad z^2 \mod{n} \quad = \quad (r \cdot s^e)^2 \mod{n} \quad = \quad u \cdot t \mod{n} \]
\end{itemize}
Quindi $P$ supera tutte le iterazioni se conosce $s$.

\subsubsection{Correttezza}
Supponiamo che $P$ sia disonesto e che quindi non conosca $s$. Per ingannare $V$ deve riuscire a prevedere il bit
casuale generato da $V$ ad ogni iterazione.

Distinguiamo due casi:
\begin{itemize}
	\item Se $P$ prevede di ricevere $e = 0$ non modifica il protocollo e se la previsione \`e corretta supera il
	      test.
	\item Se $P$ prevede di ricevere $e = 1$
	      \begin{enumerate}
		      \item Al passo 2 del protocollo e invia
		            \[ r^2 \cdot t^{-1} \mod{n} \]
		      \item Al passo 4 del protocollo invia
		            \[ z = r \mod{n} \]
	      \end{enumerate}
	      Se al passo 5 la previsione \`e corretta, $P$ supera il test perch\'e $V$ controlla se
	      \[ x = z^2 = u \cdot t^e \]
	      e se $e = 1$ allora
	      \[ u \cdot t^e = u \cdot t \]
	      Dato che $z = r$ e che
	      \[ u = r^2 \cdot t^{-1} \]
	      otteniamo
	      \[ r^2 = r^2 \cdot t^{-1} \cdot t \]
	      e quindi
	      \[ r^2 = r^2 \]
\end{itemize}
Come possiamo vedere, il metodo funziona a patto che la previsione sul bit sia corretta e la previsione deve essere
fatta prima di ricevere $e$.

La probabilit\`a di prevedere il bit ad ogni passo \`e quindi di un $1/2$. Per $k$ passi abbiamo quindi una
probabilit\`a di $(1/2)^k$ di prevedere tutti i bit.

\section{Autenticazione}
L'autenticazione riguarda il messaggio e si occupa di accertare l'identit\`a del mittente e l'integrit\`a del
messaggio.

\subsection{Canale insicuro}
Per questo protocollo, su canale insicuro, useremo un sistema basato su cifrari simmetrici in cui mittente e
destinatario devono quindi accordarsi su una chiave segreta $k$.

Nella pratica il mittente
\begin{enumerate}
	\item Allega al messaggio un \textbf{MAC} (Message Authentication Code) $A(m, k)$, allo scopo di garantire la
	      provenienza e l'integrit\`a del messaggio.
	\item A questo punto ha due opzioni
	      \begin{itemize}
		      \item Invia la coppia $\langle m, A(m, k) \rangle$ in chiaro.
		      \item Cifra $m$ e spedisce $\langle C(m, k'), A(m, k) \rangle$ dove $C$ \`e una funzione di cifratura
		            e $k'$ la chiave pubblica o la chiave simmetrica segreta del cifrario scelto.
	      \end{itemize}
\end{enumerate}

Il destinatario invece
\begin{enumerate}
	\item Riceve il messaggio (se cifrato lo decifra).
	\item Dato che conosce $A$ e $k$ calcola a sua volta il MAC.
	\item Confronta il MAC calcolato con il MAC ricevuto.
\end{enumerate}
Se la verifica ha successo il messaggio \`e autenticato altrimenti lo si scarta.

\subsubsection{MAC}
Il MAC \`e un'immagine breve del messaggio che pu\`o essere generata solo da un mittente conosciuto dal destinatario
e pu\`o essere calcolato con cifrari asimettrici, simmetrici o funzioni hash one-way.

Quest'ultima opzione implementativa \`e la pi\`u frequente
\[ A(m, k) = h(m k) \]
dato che il calcolo di una funzione hash \`e molto veloce per chi sta cifrando ma computazionalmente molto dispendioso
per un crittoanalista, che, anche disponendo di $h$ e $m$ non \`e comunque in grado di risalire a $k$ in tempo
polinomiale dato che $k$ viaggia all'interno di un MAC e quindi si dovrebbe invertire la funzione hash (costo
esponenziale).

Il crittoanalista non pu\`o nemmeno sostituire (facilmente) il messaggio $m$ con un altro messaggio $m'$ perch\'e
dovrebbe allegare a $m'$ il MAC $A(m', k)$ che pu\`o produrre solo conoscendo $k$.

\section{Firma digitale}
Questo protocollo cerca di portare tutte le propriet\`a di una \textbf{firma manuale} (con carta e penna) in ambito
tecnologico. Una firma manuale infatti
\begin{itemize}
	\item \`E autentica e non falsificabile.
	\item Non \`e riutilizzabile.
	\item Non pu\`o essere ripudiata.
\end{itemize}
Anche il documento firmato deve essere \textbf{inalterabile}.

Come vedremo, una \textbf{firma digitale}
\begin{itemize}
	\item Non pu\`o consistere nella digitalizzazione di un documento scritto firmato manualmente perch\'e si
	      potrebbe facilmente contraffare.
	\item Deve avere una forma che dipenda dal documento su cui viene apposta, per essere inscindibile da
	      quest'ultimo.
	\item Pu\`o essere generata sia tramite cifrari simmetrici che asimmetrici.
\end{itemize}

\subsection{Protocollo 1: Diffie Hellman}
In questo protocollo il messaggio $m$ \`e in chiaro e firmato.

Supponiamo che un utente $u$, in possesso di una coppia di chiavi $\langle k_\text{pub}, k_\text{priv} \rangle$ e
che ha a disposizione una funzione $C$ di cifratura e una funzione $D$ di decifrazione (commutative), voglia firmare
$m$ e inviarlo a $v$.

Per firmare $m$, l'utente $u$
\begin{enumerate}
	\item Genera la firma $f$ per $m$ calcolando
	      \[ f = D(m, k_\text{priv}) \]
	\item $u$ invia all'utente $v$ la tripla $\langle u, m, f \rangle$.
\end{enumerate}
L'utente $v$
\begin{enumerate}
	\item Riceve $\langle u, m, f \rangle$.
	\item Verifica l'autenticit\`a della firma calcolando e controllando che
	      \[ m = C(f, k_\text{pub}) \]
\end{enumerate}
Se la verifica va a buon fine allora $v$ accetta la firma.

\subsubsection{Limiti}
Questo protocollo ha il grosso limite di non riuscire a proteggere il messaggio in lettura, infatti anche se
inviassimo un crittogramma $c$ di $m$, sarebbe il risultato della cifratura di $m$ con la chiave pubblica.

Possiede tuttavia tutti i requisiti di una firma manuale
\begin{itemize}
	\item La chiave $k_\text{priv}$ \`e nota solo a $u$ e per ottenerla si fa un numero esponenziale di operazioni.
	\item Se $m$ venisse alterato dal crittoanalista non ci sarebbe pi\`u consistenza tra $m$ e $f$ e la verifica
	      fallirebbe.
	\item Poich\'e solo $u$ pu\`o aver prodotto $f$ non pu\`o ripudiarla.
	\item Dato che la firma \`e un'immagine di $m$ non \`e riutilizzabile su un altro messaggio $m'$.
\end{itemize}

\subsection{Protocollo 2}
Questo protocollo si propone di risolvere il problema del precedente relativo all'impossibilit\`a di proteggere il
messaggio.

L'utente $u$
\begin{enumerate}
	\item Genera la firma $f$ per $m$ calcolando
	      \[ f = D(m, k_\text{priv}) \]
	\item Cifra la firma calcolando
	      \[ c = C(f, k) \]
	      dove $k$ pu\`o essere la chiave pubblica del destinatario oppure una chiave simmetrica segreta.
	\item Invia la coppia $\langle u, c \rangle$ a $v$.
\end{enumerate}
L'utente $v$
\begin{enumerate}
	\item Riceve $\langle u, c \rangle$.
	\item Ricava $f$ calcolando
	      \[ f = D(c, k) \]
	      con $k$ che pu\`o essere la sua chiave privata o una chiave simmetrica.
	\item Cifra $f$ con la chiave pubblica del mittente ottenendo il messaggio $m$
	      \[ m = C(f, k_\text{pub}) \]
\end{enumerate}
Se il messaggio \`e significativo l'identit\`a di $u$ \`e attestata altrimenti si butta via il messaggio.

\subsubsection{Algoritmo con RSA}
In questo caso abbiamo due coppie di chiavi
\begin{gather*}
	d_u, \quad \langle e_u, n_u \rangle \\
	d_v, \quad \langle e_v, n_v \rangle
\end{gather*}
La coppia del mittente \`e usata per produrre la firma e verificarla mentre la coppia del destinatario per decifrare
il crittogramma.

Supponiamo che $u$ sia il mittente e $v$ il destinatario, l'utente $u$
\begin{enumerate}
	\item Genera la firma del messaggio $m$ calcolando
	      \[ f = m^{d_u} \mod{n_u} \]
	\item Cifra $f$ con la chiave pubblica di $v$ calcolando
	      \[ c = f^{e_v} \mod{n_v} \]
	\item Invia la coppia $\langle u, c \rangle$ a $v$.
\end{enumerate}
L'utente $v$
\begin{enumerate}
	\item Riceve la coppia $\langle u, c \rangle$.
	\item Decifra $c$ calcolando
	      \[ f = c^{d_v} \mod{n_v} \]
	\item Decifra $f$ con la chiave pubblica di $u$
	      \[ m = f^{e_u} \mod{n_u} \]
\end{enumerate}
Se $m$ \`e significativo allora l'identit\`a del mittente \`e attestata.

Affinch\'e il procedimento venga effettuato correttamente \`e necessario che
\[ f < n_v \]
e perch\'e questo accada c'\`e bisogno che
\[ n_u \leq n_v \]
Questo impedisce a $v$ di inviare messaggi firmati e cifrati da $u$.

Di solito ogni utente ha due coppie di chiavi diverse, una per la firma e una per la cifratura, tali che le chiavi
per la firma siano per esempio minori di un certo valore $H$ e quelle di cifratura siano invece maggiori. Il valore
$H$ \`e un valore molto grande su cui i due utenti si devono accordare.

\subsubsection{Attacco}
Un crittoanalista potrebbe procurarsi la firma di un utente su messaggi apparentemente privi di senso.

Prendiamo uno scenario in cui un crittoanalista $x$ si procura la firma dell'utente da messaggi privi di senso
per l'utente.

Supponiamo che
\begin{itemize}
	\item Il destinatario $v$ di un messaggio invii sempre una risposta $ack$ al mittente $u$ ogni volta che
	      riceve un messaggio (prima della verifica della firma).
	\item Il segnale di $ack$ sia il crittogramma della firma di $v$ su $m$.
\end{itemize}
In queste condizioni, un attacco attivo, pu\`o avere successo se
\begin{enumerate}
	\item $x$ intercetta il crittogramma $c$ firmato inviato da $u$ a $v$, lo rimuove dal canale e lo rispedisce a
	      $v$, facendogli credere che $c$ sia stato inviato da lui.
	\item $v$ spedisce automaticamente a $x$ un ack.
	\item $x$ usa l'ack ricevuto per risalire al messaggio originale applicando le funzioni del cifrario con le
	      chiavi pubbliche di $u$ e $v$.
\end{enumerate}
Per risalire a $m$, il crittoanalista compie dei passaggi algebrici che avranno complessivamente costo polinomiale.

Prima di tutto, il fatto che $u$ abbia inviato il crittogramma $c$ a $v$, significa che
\begin{gather*}
	c = C(f, k_{v [\text{pub}]}) \\
	f = D(m, k_{u [\text{priv}]})
\end{gather*}
A questo punto, dopo che $x$ ha intercettato $c$ e l'ha rispedito a $v$, l'utente $v$ decifra $c$ ottenendo
\[ f = D(c, k_{v [\text{priv}]}) \]
e verifica la firma con la chiave pubblica di $x$ ottenendo un messaggio
\[ m' = C(f, k_{x [\text{pub}]}) \]
Il messaggio $m'$, molto probabilmente, non \`e significativo ma l'ack $c'$ \`e gi\`a stato inviato in automatico
calcolando
\begin{gather*}
	f' = D(m', k_{v [\text{priv}]}) \\
	c' = C(f', k_{x [\text{pub}]})
\end{gather*}
A questo punto $x$ ha tutto ci\`o che serve:
\begin{enumerate}
	\item Decifra $c'$ e trova $f'$
	      \[ f' = D(c', k_{x [\text{priv}]}) = D(C(f', k_{x [\text{pub}]}), k_{x [\text{priv}]}) \]
	\item Verifica la firma $f'$ e trova $m'$ usando la chiave pubblica di $v$
	      \[ m' = C(f', k_{v [\text{pub}]}) = C(D(m', k_{v [\text{priv}]}), k_{v [\text{pub}]}) \]
	\item Da $m'$ ricava la firma $f$
	      \[ f = D(m', k_{x [\text{priv}]}) = D(C(f, k_{x [\text{pub}]}), k_{x [\text{priv}]}) \]
	\item Verifica $f$ con la chiave pubblica di $u$ e trova $m$
	      \[ m = C(f, k_{u [\text{pub}]}) = C(D(m, k_{u [\text{priv}]}), k_{u [\text{pub}]}) \]
\end{enumerate}
Come possiamo vedere si usano sempre le funzioni di cifratura e decifrazione che, come sappiamo, hanno sempre costo
polinomiale e quindi anche l'attacco ha costo complessivamente polinomiale.

Per proteggersi da questo attacco non si devono inviare ack automatici, almeno finch\'e non si \`e concluso la fase
di verifica e si deve sempre firmare un'immagine hash del messaggio.

\subsection{Protocollo 3}
Questo protocollo si propone di risolvere anche i problemi presentati del secondo. In questo caso il messaggio
$m$ \`e cifrato e firmato con una funzione hash.

Il mittente $u$
\begin{enumerate}
	\item Calcola l'hash del messaggio $h(m)$ e genera la firma calcolando
	      \[ f = D(h(m), k_{u [\text{priv}]}) \]
	\item Cifra il messaggio calcolando
	      \[ c = C(m, k_{v [\text{pub}]}) \]
	\item Invia la tripla $\langle u, c, f \rangle$ a $v$.
\end{enumerate}
Il destinatario $v$
\begin{enumerate}
	\item Riceve la tripla $\langle u, c, f \rangle$.
	\item Decifra $c$ calcolando
	      \[ m = D(c, k_{v [\text{priv}]}) \]
	\item Calcola $h(m)$.
	\item Verifica la firma calcolando e verificando che
	      \[ h(m) = C(f, k_{u [\text{pub}]}) \]
\end{enumerate}
Come per tutti i protocolli a chiave pubblica, anche questo \`e vulnerabile ad attacchi di tipo
\emph{man in the middle}.

\subsection{Certification Authority}
Un algoritmo \`e tanto robusto quanto la sicurezza delle sue chiavi ma lo scambio o la generazione della chiave \`e
un passo cruciale.

\`E proprio in questo frangente che i crittoanalisti sfruttano attacchi di tipo \emph{man in the middle} per
riuscire a forzare facilmente i sistemi crittografici.

Sono dunque nate delle infrastrutture, chiamate \textbf{certification authority}, adibite a garantire la validit\`a
delle chiavi pubbliche e a regolare il loro uso, gestendone la distribuzione.

Le CA rilasciano un \textbf{certificato digitale} che autentica l'associazione tra un utente e la sua chiave pubblica.

Un certificato digitale consiste della chiave pubblica di una lista di informazioni relative al suo proprietario,
opportunamente firmate dalla CA. Per falsificare un certificato si deve falsificare la firma delle CA.

Una CA mantiene un archivio di chiavi pubbliche sicuro, accessibile a tutti e protetto da attacchi in scrittura non
autorizzati.

La chiave pubblica della CA \`e nota a tutti gli utenti che la mantengono protetta sui propri dispositivi e la
utilizzano per verificare la firma della CA stessa sui certificati.

Le CA hanno in genere una struttura gerarchica ad albero e dunque si avvia una sorta di verifica a catena risalendo
le varie CA.

\subsubsection{Certificazione}
Supponiamo che $u$ voglia comunicare con $v$
\begin{enumerate}
	\item $u$ richiede la chiave pubblica di $v$ alla CA.
	\item La CA invia a $u$ il certificato digitale $c_v$ di $v$.
	\item Dato che $u$ conosce la chiave pubblica della CA, controlla l'autenticit\`a del certificato verificandone il
	      periodo di validit\`a e la firma della CA.
	\item $u$ estrae dal certificato la chiave pubblica di $v$ e inizia il protocollo di comunicazione.
\end{enumerate}
Gli attacchi \emph{man in the middle} sono sempre possibili ma devono essere effettuati falsificando la certificazione
ma si assume che la CA sia fidata e il suo archivio di chiavi inattaccabile.

\subsection{Protocollo 4}
Ultimo protocollo che vediamo consiste nel cifrare, firmare e certificare un messaggio $m$.

Il mittente $u$
\begin{enumerate}
	\item Si procura il certificato $\text{cert}_v$ di $v$ e verifica che sia autentico.
	\item Calcola $h(m)$ e genera la firma calcolando
	      \[ f = D(h(m), k_{u [\text{priv}]}) \]
	\item Cifra $m$ calcolando
	      \[ c = C(m, k_{v [\text{pub}]}) \]
	\item Invia la tripla $\langle \text{cert}_u, c, f \rangle$ a $v$.
\end{enumerate}
Il destinatario $v$
\begin{enumerate}
	\item Riceve la tripla $\langle \text{cert}_u, c, f \rangle$ a $v$.
	\item Verifica l'autenticit\`a di $\text{cert}_u$ con la chiave pubblica della CA che tiene in locale.
	\item Decifra $c$ con la sua chiave privata calcolando
	      \[ m = D(c, k_{v [\text{priv}]}) \]
	\item Verifica che la firma sia autentica controllando che
	      \[ h(m) = C(f, k_{u [\text{pub}]}) \]
\end{enumerate}
L'unico punto debole di questo metodo \`e rappresentato dall'uso di certificati revocati.

\section{SSL}
Il \textbf{protocollo SSL} \`e molto usato per costruire comunicazioni sicure e garantisce \textbf{confidenzialit\`a}
e \textbf{affidabilit\`a} ed \`e progettato per permettere a due computer che non conoscono le reciproche
funzionalit\`a di comunicare.

Supponiamo che un utente $u$ voglia accedere ad un servizio fornito da un sistema $s$. Il protocollo SSL garantisce
\begin{itemize}
	\item \textbf{Confidenzialit\`a}: la trasmissione \`e cifrata mediante un sistema ibrido in cui si usa un
	      cifrario asimmetrico per costruire le chiavi e uno simmetrico per la comunicazione.
	\item \textbf{Autenticazione}: il protocollo accerta l'identit\`a dei due utenti tramite un cifrario asimmetrico
	      o facendo uso di certificati digitali e inserendo un MAC nei messaggi.
\end{itemize}
L'SSL sta tra il protocollo di trasporto (TCP/IP) e il protocollo applicativo (HTTP) ed \`e completamente
indipendente da quest'ultimo.

\subsection{Livelli}
Il protocollo \`e organizzato su due livelli: \textbf{record} e \textbf{handshake}.

\subsubsection{SSL Record}
Il livello \emph{record} \`e a livello pi\`u basso ed \`e connesso direttamente al protocollo di trasporto.

Ha come obbiettivo incapsulare i dati spediti dai protocolli dei livelli superiori, assicurando confidenzialit\`a
e integrit\`a della comunicazione.

Implementa fisicamente il canale su cui viaggiano i messaggi.

\subsubsection{SSL Handshake}
Il livello \emph{handshake} \`e responsabile dell'instaurazione e del mantenimento dei parametri usati dal livello
\emph{record} e permette al sistema di
\begin{itemize}
	\item Autenticarsi
	\item Negoziare gli algoritmi di cifratura e firma
	\item Stabilire le chiavi per i singoli algoritmi crittografici e per il MAC
\end{itemize}
In definitiva il livello \emph{handshake} crea un canale \textbf{sicuro}, \textbf{affidabile} e \textbf{autenticato}
tra utente e sistema, entro il quale il livello \emph{record} fa viaggiare i messaggi.

Affinch\'e avvenga l'\emph{handshake} deve esserci uno scambio preliminare di messaggi indispensabile alla creazione
di un canale sicuro. Attraverso questi messaggi, client e server si identificano a vicenda e cooperano per la
costruzione delle chiavi simmetriche usate per le comunicazioni successive.

\subsection{Creazione del canale}
Vediamo ora tutti i passi che compie il protocollo per la costruzione del canale sicuro.

\begin{enumerate}
	\item \textbf{Client hello}: $u$ invia a $s$ un messaggio di \emph{client hello} con cui
	      \begin{itemize}
		      \item Richiede la creazione di una connessione SSL
		      \item Specifica le prestazioni di sicurezza che desidera siano garantite durante la comunicazione
		            (\textbf{cipher suite}).
		      \item Invia una sequenza di byte casuali.
	      \end{itemize}
	\item \textbf{Server hello}: $s$ riceve il messaggio di \emph{client hello} e manda un messaggio di
	      \emph{server hello} con cui
	      \begin{itemize}
		      \item Specifica una \emph{cipher suite} che anch'esso \`e in grado di supportare.
		      \item Invia una sequenza di byte casuali.
	      \end{itemize}
	      Se $u$ non riceve il \emph{server hello} interrompe la comunicazione.
	\item \textbf{Invio del messaggio}: $s$ si autentica con $u$ inviandogli il proprio certificato digitale e se
	      i servizi offerti da $s$ devono essere protetti negli accessi, $s$ pu\`o richiedere a $u$ di autenticarsi
	      inviando il suo certificato digitale.
	\item \textbf{Server hello done}: $s$ invia il messaggio \emph{server hello done} con cui sancisce la fine degli
	      accordi sulla \emph{cipher suite} e sui parametri crittografici a essa associati.
	\item \textbf{Autenticazione del sistema}: per accerta l'autenticit\`a del certificato ricevuto da $s$, $u$ deve
	      controllare che
	      \begin{itemize}
		      \item Il certificato sia ancora valido.
		      \item La CA che ha firmato il certificato sia tra quelle \emph{fidate}.
		      \item La firma apposta sul certificato sia autentica.
	      \end{itemize}
	\item \textbf{User master secret}: $u$ a questo punto
	      \begin{itemize}
		      \item Costruisce un \textbf{pre-master secret} costituito da una nuova sequenza di byte casuali.
		      \item Lo cifra con il cifrario a chiave pubblica su cui si \`e accordato con $s$.
		      \item Invia il crittogramma a $s$.
	      \end{itemize}
	      Il pre-master secret viene poi usato per calcolare un \textbf{master secret}.
	\item \textbf{System master secret}: $s$ riceve il crittogramma contenente il \emph{pre-master secret} inviato
	      da $u$ e calcola il \emph{master secret} compiendo le stesse operazioni compiute da $u$.
	\item \textbf{Invio del certificato}: questo passo \`e opzionale e si fa solo nel caso in cui $s$ richieda un
	      certificato al client.
	\item \textbf{Finished}: \`E il primo messaggio protetto mediante il \emph{master secret} e la \emph{cipher suite}
	      su cui le due parti si sono accordate.

	      Il messaggio viene prima costruito da $u$ e spedito a $s$ e dopo avviene il contrario ma il messaggio \`e
	      diverso.
\end{enumerate}
Se tutto questo processo \`e andato a buon fine si possono costruire le chiavi crittografiche simmetriche in modo
sicuro.

\subsection{Sicurezza}
Nei passi di \emph{hello} i due utenti si inviano due sequenze casuali per la costruzione del \emph{master secret},
che risulta cos\`i, diverso in ogni sessione di SSL.

Un crittoanalista non pu\`o riutilizzare i messaggi di \emph{handshake} catturati sul canale per sostituirsi a $s$
in una successiva comunicazione con $u$ perch\'e questi sono riferiti a valori \emph{usa e getta}.

I blocchi vengono cifrati con un MAC (anch'esso cifrato) e quindi un crittoanalista dovrebbe riuscire ad invertire
una funzione hash.

Un attacco volto a modificare la comunicazione tra i due utenti \`e difficile quanto un attacco volto alla decriptazione
dei messaggi.

Il sistema \`e autenticato con un certificato ed \`e dunque robusto ad attacchi \emph{man in the middle}.

La sicurezza del protocollo SSL \`e sicuro almeno quanto il pi\`u debole \emph{cipher suite} supportato. 		% PROTOCOLLI DI SICUREZZA
\chapter{Bitcoin}
Le normali valute come euro, dollari e cos\`i via sono gestite da una banca centrale, come la BCE per l'euro, la quale
gestisce il conio e altri parametri andando a regolare l'inflazione ecc.

Senza entrare nei dettagli finanziari, possiamo dunque dire che le valute comuni sono in mano a sistemi
\textbf{centralizzati}, i quali gestiscono tutto ci\`o che abbiamo detto prima.

Il sistema dei \textbf{bitcoin} nasce dall'esigenza di creare un sistema di valute digitali \textbf{decentralizzato},
ossia un sistema che \`e gestito solo dai suoi utenti.

\section{Introduzione}
Prima di addentrarci nell'argomento chiariamo cosa sia una valuta digitale e una criptovaluta:
\begin{itemize}
	\item \textbf{Valuta digitale}: valuta che esiste soltanto in forma digitale e che dunque, \`e utilizzabile
	      soltanto da un calcolatore.
	\item \textbf{Criptovaluta}: valuta digitale resa sicura grazie a tecniche di crittografia che rendono quasi
	      impossibile spendere due volte la stessa moneta.
\end{itemize}
Per capire il funzionamento di bitcoin dobbiamo prima introdurre due concetti: la transazione e il registro.
\begin{itemize}
	\item \textbf{Transazione}: passaggio di denaro fra due utenti.
	\item \textbf{Registro}: anche detto \textbf{libro contabile}, \`e un documento per il monitoraggio delle
	      transazioni.
\end{itemize}
Il bitcoin \`e una rete \emph{Peer-to-Peer} di nodi su cui gira il software \emph{bitcoin core}, con il quale
riescono a comunicare tra di loro e decidono come gestire le transazioni tra le monete.

Ogni nodo memorizza il registro di tutti i membri della rete. Quest'ultimo dev'essere quindi aggiornato periodicamente
in modo che sia uguale per tutti gli utenti. Affinch\'e questo avvenga il procedimento \`e il seguente
\begin{enumerate}
	\item Si sceglie un nodo \textbf{leader} tramite consenso: i nodi competono tra di loro cercando di risolvere un
	      problema complesso.
	\item Il \emph{leader} propone una nuova pagina del libro contabile sulla base delle varie richieste di transazione
	      tra gli utenti.
	\item Tutti i computer ricevono la pagina e, tramite la loro copia del registro, controllano che le transazioni
	      inserite siano corrette. Se lo sono accettano la pagina, altrimenti la rifiutano e si cerca un altro
	      \emph{leader}.
	\item Tutti i computer aggiornano il loro registro e il \emph{leader} viene ricompensato con un premio in bitcoin.
\end{enumerate}

\section{Blockchain}
Ogni pagina del registro \`e collegata alla precedente come in una lista linkata. Ogni pagina, detta \textbf{blocco}
(da qui il nome \textbf{blockchain}), ha un \textbf{header}, in cui sono presenti informazioni necessarie al
mantenimento della struttura dati, e un \textbf{body}, dove sono memorizzate le transazioni.

\subsection{Header}
Nell'\emph{header} sono presenti quattro parametri principali
\begin{itemize}
	\item \textbf{Timestamp}: chi crea la pagina mette un \emph{timestamp} nell'\emph{header} relativo al momento
	      della creazione di essa.
	\item \textbf{Nonce}: \`e il frutto della competizione di cui abbiamo parlato prima.

	      In pratica si fa una sorta di forza bruta cercando un valore \emph{nonce} tale che il valore della funzione
	      hash SHA256 calcolata sull'\emph{header} dell'ultima pagina della \emph{blockchain} e il \emph{nonce} sia
	      minore di un certo valore.

	      Trovare un \emph{nonce} di questo tipo equivale a vincere la competizione.
	\item \textbf{Merkle Root}: \`e la radice di un albero di Merkle costruito calcolando l'hash delle transazioni.
	      e serve a controllare che una certa transazione, all'interno della pagina, sia integra.
	\item \textbf{Hash previous block}: \`e il \emph{nonce} della pagina precedente.
\end{itemize}

\subsection{Transazioni}
Nel \emph{body} di un blocco sono salvate tutte le \textbf{transazioni} effettuate fino a quel momento.

Una transazione
\`e un invio di bitcoin da un utente $A$ ad un utente $B$
\[ Tx(A, B, BTC) \]
In ambito bitcoin sono pi\`u complicate e fanno uso di firme digitali che hanno la propriet\`a di
\begin{itemize}
	\item \textbf{Autenticazione}: ogni volta che si effettua una transazione si deve essere in possesso di una coppia
	      di chiavi, privata e pubblica, che servono a produrre una firma digitale.
	\item \textbf{Integrit\`a}: una transazione non deve essere modificata da qualche attacco attivo o da problemi di
	      rete.
	\item \textbf{Non ripudio}: non si pu\`o negare di aver effettuato una transazione.
\end{itemize}
Supponiamo di avere due utenti $A$ e $B$ i quali possiedono rispettivamente le chiavi private $sk_A$ e $sk_B$ e le
chiavi pubbliche $pk_A$ e $pk_B$ e siano $Addr_A$ e $Addr_B$ rispettivamente i loro indirizzi, una transazione da $A$
a $B$ avviene in questo modo:
\begin{enumerate}
	\item $A$ firma la transazione con la sua chiave privata.
	\item $A$ invia il crittogramma a $B$.
	\item $B$ verifica che il messaggio sia inviato da $A$ tramite la chiave $pk_A$.
\end{enumerate}
Le transazioni vengono effettuate tramite programmi non Turing completi e la loro verifica coinvolge una terza parte
fidata che fa da \emph{garante}.

La transazione non avviene direttamente da $A$ a $B$ ma si invia il messaggio relativo alla transazione ad un
\textbf{indirizzo multisignature} che coinvolge questa terza parte fidata, la quale controlla che la transazione
sia correttamente effettuata e la firma.

Una transazione relativa ad un certo utente \`e strutturata come un blocco in cui sono presenti tre parametri
fondamentali:
\begin{itemize}
	\item \textbf{Input}: somma dei bitcoin ricevuti.
	\item \textbf{Output}: somma dei bitcoin inviati.
	\item \textbf{UTXO}: sta per \emph{Unspent Transaction Output} e sono i bitcoin non spesi che vengono comunque
	      mandati in output verso se stessi cos\`i da recuperarli.
\end{itemize}
Il risultato \`e la differenza tra input e output.

In definitiva, nessuno possiede dei bitcoin ma si possiede una chiave privata che consente di spendere bitcoin inviati
ad un certo indirizzo relativo ad una chiave pubblica. Se si perde la chiave privata si perdono tutti i bitcoin
legati ad essa.

\section{Conclusioni}
\subsection{Come ottenere bitcoin}
\subsection{Come utilizzare bitcoin}
\subsection{Valore del bitcoin} 			% BITCOIN
\chapter{Crittografia quantistica}
In questo capitolo non andremo a parlare di macchine quantistiche ma andremo a vedere gli effetti della
\textbf{meccanica quantistica} sulla crittografia.

Esistono infatti dei protocolli crittografici che sfruttano alcuni dei suoi principi per lo scambio delle chiavi in
contesti in cui \`e richiesta estrema sicurezza e ai quali si affianca un One-Time Pad come cifrario simmetrico per
le comunicazioni.

\section{Meccanica quantistica}
Introduciamo alcuni principi di meccanica quantistica necessari a comprendere il protocollo
\begin{itemize}
	\item \textbf{Sovrapposizione}: propriet\`a di un sistema quantistico di trovarsi in diversi stati
	      contemporaneamente.
	\item \textbf{Decoerenza}: la misurazione di un sistema quantistico disturba il sistema. Il sistema disturbato
	      perde la sovrapposizione degli stati e collassa in uno stato singolo.
	\item \textbf{No Cloning}: impossibilit\`a di copiare uno stato quantistico non noto.
	\item \textbf{Entanglement}: possibilit\`a che due o pi\`u elementi si trovino in uno stato quantistico correlato
	      in modo che, anche se portati a grande distanza, mantengono la correlazione.
\end{itemize}

\subsection{Fotoni polarizzati}
Per comprendere al meglio il protocollo che vedremo in seguito dobbiamo anche parlare di \textbf{fotoni polarizzati}.

Un fotone ha una propriet\`a, detta \textbf{polarizzazione}, che pu\`o assumere due valori e che pu\`o essere misurata
facendo riferimento ad una \textbf{base}, anch'essa di due tipi:
\begin{itemize}
	\item \textbf{Ortogonale}: si indica con $+$ e pu\`o assumere valore \textbf{verticale} (a $90^\circ$), indicato con
	      $v$ oppure \textbf{orizzontale} (a $0^\circ$), indicato con $h$.
	\item \textbf{Diagonale}: si indica con $\times$ e pu\`o valere $+45^\circ$ o $-45^\circ$.
\end{itemize}

\subsubsection{Misurazione}
Non \`e possibile distinguere, con una misura, uno dei quattro casi: si deve scegliere una delle due basi e, dopo la
misura, \`e possibile distinguere uno dei due casi relativi alla base scelta.

Per la misurazione viene usato uno strumento (\textbf{Polarizing Beam Splitter}) il quale, una volta scelta la base di
riferimento, misura il valore di polarizzazione relativo alla base.

Tuttavia, la misurazione \`e corretta solo nel caso in cui il fotone sia preparato con la stessa base del PBS. Nel caso
in cui la base non sia quella corretta si hanno pari probabilit\`a di misurare uno dei due valori.

Il PBS ha due uscite, A ed R, che stanno rispettivamente per \emph{assorbimento} e \emph{riflessione}, e un asse di
polarizzazione S. Anche il fotone ha un suo asse di polarizzazione F e indichiamo con $\theta$ l'angolo che questo
forma con S.

In fase di misurazione si hanno due possibili scenari:
\begin{itemize}
	\item Il fotone esce dall'uscita A con probabilit\`a $\cos^2 \theta$ e assume polarizzazione S.
	\item Il fotone esce dall'uscita R con probabilit\`a $\sin^2 \theta$ e assume polarizzazione perpendicolare a S
	      ($\text{S}^\perp$).
\end{itemize}
Procediamo per casi supponendo di usare un PBS impostato su base ortogonale e quindi con S = 0:
\begin{itemize}
	\item Se $\theta = 0$ (F = S), allora il fotone esce da A con probabilit\`a 1 con polarizzazione S = F.
	\item Se $\theta = 90^\circ$, allora il fotone esce da R con probabilit\`a 1 con polarizzazione
	      $\text{S}^\perp = \text{F}$.
	\item Se $\theta = \pm 45^\circ$, allora il fotone esce con pari probabilit\`a da A o da R con polarizzazione
	      a $0^\circ$ o $90^\circ$.
\end{itemize}
Dunque la lettura ha distrutto lo stato quantistico precedente, il quale non \`e pi\`u recuperabile.

Per fare riferimento ai principi elencati all'inizio, possiamo dire che si \`e perso lo stato di \emph{sovrapposizione}
e si \`e dunque verificata una situazione di \emph{decoerenza}.

Dato che il protocollo prevede l'invio di fotoni polarizzati tra i due utenti, l'azione di crittoanalista che tenta di
intercettare la comunicazione lascia tracce, proprio perch\'e rompe lo stato di \emph{sovrapposizione}.

\section{BB84}
Ideato da Bennet e Brassard, \`e un protocollo per lo scambio di chiavi che fa uso di fotoni polarizzati.

Supponiamo che un utente $A$ voglia comunicare con l'utente $B$ tramite il protocollo BB84. Una prima fase viene
effettuata sul \textbf{canale quantistico}, al quale \`e associato un valore QBER (Quantum Bit Error Rate), che indica
la percentuale di errori dovuti al rumore.
\begin{enumerate}
	\item L'utente $A$ genera un sequenza iniziale di $n$ bit $S_A$, da cui sar\`a estratta la chiave (rappresentata
	      con un codice a correzione di errori).
	\item Per $n$ volte
	      \begin{enumerate}[label=(\Roman*)]
		      \item $A$ sceglie una base a caso, codifica $S_A[i]$ e invia il fotone a $B$.
		      \item $B$ sceglie una base a caso, interpreta il fotone ricevuto e costruisce $S_B[i]$.
	      \end{enumerate}
\end{enumerate}
Queste due sequenze coincidono con certezza dove le basi scelte coincidono. Al contrario, dove le basi non coincidono
i bit avranno pari probabilit\`a di coincidere o di non coincidere.

A questo punto i due utenti passano ad una comunicazione sul canale standard e si accordano su una funzione hash
crittografica $h$.
\begin{enumerate}
	\item $B$ comunica ad $A$ le basi scelte per ogni bit.
	\item $A$ risponde a $B$ comunicando le basi comuni.
	\item $A$ e $B$ estraggono due sottosequenze $S_A'$ e $S_B'$ di $S_A$ ed $S_B$, corrispondenti alle basi comuni.
	\item $A$ e $B$ estraggono due sottosequenze di $S_A''$ ed $S_B''$ da $S_A'$ ed $S_B'$ in posizioni concordate per
	      accertarsi se c'\`e stato l'intervento di un crittoanalista.
	\item $A$ e $B$ si scambiano $S_A''$ e $S_B''$ e se la percentuale di bit diversi \`e maggiore di QBER allora
	      possiamo dedurre che ci sia stato l'intervento di un crittoanalista e dunque si ripete tutto il procedimento
	      da capo.
	\item $A$ e $B$ calcolano rispettivamente $S_A' \backslash S_A''$ e $S_B' \backslash S_B''$, le decifrano con un
	      codice di correzione degli errori e ottengono una sequenza comune $S_C$.
	\item $A$ e $B$ calcolano $h(S_C)$ e la usano come chiave.
\end{enumerate}

\subsection{Sicurezza}
Il crittoanalista vuole scoprire i bit della chiave e per farlo deve misurare la polarizzazione dei fotoni in transito
sul canale. Dato che nessuno apparte $A$ sa quale sia la base usata, il crittoanalista
\begin{enumerate}
	\item Intercetta il fotone.
	\item Lo misura con una base scelta a caso.
	\item Lo invia a $B$.
	\item Costruisce la sua sequenza $S$ (non necessariamente per ogni fotone inviato).
\end{enumerate}
Cos\`i facendo, nel caso in cui il crittoanalista abbia usato la stessa base di $A$, il fotone non viene perturbato
nel suo stato di polarizzazione, ma nel caso in cui la base sia diversa, il fotone cambia la sua polarizzazione.

Se fosse possibile copiare lo stato quantistico di un fotone sarebbe possibile, per il crittoanalista, copiare il
lo stato del fotone prima di misurarlo, inviare la copia a $B$ e poi misurare il fotone originale. Come detto
all'inizio del capitolo, il principio di \emph{no cloning} non permette che ci\`o avvenga.

Il crittoanalista, per provare a confondere le perturbazioni dovute alle misure che fa sui fotoni, potrebbe decidere
di intercettarne solo alcuni e nel caso riuscisse passare inosservato, conoscerebbe alcuni bit della chiave.

Questo \`e potenzialmente molto pericoloso dato che, per ogni bit conosciuto, lo spazio delle chiavi si dimezza e quindi
un attacco a \emph{forza bruta} molto meno costoso.

Ecco perch\'e $A$ e $B$ usano l'immagine hash di $S_C$: usando l'immagine hash, la sequenza cambia radicalmente e i bit
intercettati diventano dunque inutili per il crittoanalista. 			% CRITTOGRAFIA QUANTISTICA

\end{document}