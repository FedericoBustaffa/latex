\chapter{Teoria della complessit\`a}
La \textbf{complessit\`a} \`e quella branca dell'informatica teorica che classifica i problemi in base alla
difficolt\`a che si ha nel risolverli, ossia al numero di operazioni necessarie per risolverli.

Nel capitolo precedente abbiamo trattato problemi decidibili e non, ossia problemi per i quali esiste un algoritmo
di risoluzione o meno. In questo capitolo restringiamo il campo solo ai problemi decidibili ma li dividiamo in due
sottocategorie: \textbf{trattabili} e \textbf{intrattabili}.

Si tratta in entrambi i casi di problemi risolvibili, ma nel caso degli intrattabili, il costo computazionale \`e
talmente alto (generalmente esponenziale o pi\`u) da renderli molto simili ai problemi indecidibili. Questo perch\'e
la risouzione richiede decine se non centinaia o migliaia (in base alla dimensione dell'input) di anni di calcolo.

Al contrario i problemi trattabili sono quelli che hanno algoritmi di costo polinomiale e dunque sono risolvibili in
tempi ragionevoli.

Ci sono infine problemi il cui stato non \`e noto: abbiamo solo algoritmi di costo esponenziale per la risoluzione ma
non abbiamo dimostrazione del fatto che siano intrattabili.

\section{Algoritmi polinomiali ed esponenziali}
Studiamo ora la dimensione dei dati trattabili in funzione dell'incremento della velocit\`a dei calcolatori.

\begin{example}
	Siano $C_1$ e $C_2$ due calcolatori con $C_2$ che \`e $k$ volte pi\`u veloce di $C_1$ e testiamo quanti dati
	riescono a processare i due calcolatori in un tempo $t$.

	Il calcolatore $C_1$ tratta $n_1$ dati nel tempo $t$ mentre $C_2$ tratta $n_2$ dati sempre in tempo $t$.

	Osserviamo che usare $C_2$ \`e come usare $C_1$ per un tempo $k \cdot t$.

	Se l'algoritmo che stiamo usando \`e \textbf{polinomiale} e risolve il problema in un tempo $c \cdot n^s$, dove
	$c$ e $s$ sono costanti, abbiamo che
	\begin{itemize}
		\item $C_1 : \quad c \cdot n_1^s = t \quad \rightarrow \quad n_1 = (t / c)^{1/s}$
		\item $C_2 : \quad c \cdot n_2^s = k \cdot t \quad \rightarrow \quad n_2 = k^{1/s}(t / c)^{1/s}$
	\end{itemize}
	Da qui otteniamo che
	\[ n_2 = k^{1/s} \cdot n_1 \]
	dunque abbiamo un miglioramento di un \textbf{fattore moltiplicativo} $k^{1/s}$, che sar\`a tanto pi\`u grande
	quanto pi\`u piccolo sar\`a il grado ($s$) del polinomio.

	Se invece facciamo girare sugli stessi calcolatori un algoritmo di costo esponenziale, che risolve il problema in
	un tempo $c 2^n$ con $c$ costante, otteniamo
	\begin{itemize}
		\item $C_1 : \quad c 2^{n_1} = t \quad \rightarrow \quad 2^{n_1} = t / c$
		\item $C_2 : \quad c 2^{n_2} = k \cdot t \quad \rightarrow \quad 2^{n_2} = k \cdot t/c$
	\end{itemize}
	Da qui si ricava che
	\[ 2^{n_2} = k \cdot 2^{n_1} \]
	quindi
	\[ n_2 = n_1 + \log_2 k \]
	In questo caso il miglioramento \`e solo di un \textbf{fattore additivo} $\log_2 k$. Il che significa che per quanto
	io aumenti la velocit\`a di elaborazione $k$ della macchina $C_2$ rispetto a $C_1$, riuscirei a trattare solo
	$\log_2 k$ dati in pi\`u.
\end{example}

\section{Problemi}
Trattiamo ora le varie \textbf{classi di complessit\`a} e come si definiscono. Indicheremo con $\Pi$ un generico
\textbf{problema}, con $I$ l'insieme delle \textbf{istanze} in ingresso e con $S$ l'insieme delle \textbf{soluzioni}.

\subsection{Tipologie di problemi}
Abbiamo tre \textbf{classi} principali di problemi: di \textbf{ricerca}, di \textbf{ottimizzazione} e
\textbf{decisionali}.

\subsubsection{Problemi di ricerca}
Sono problemi in cui, data un'istanza $x$ in input, richiedono di restituire una soluzione $s$.

Un esempio di problema di ottimizzazione \`e quello di calcolare un cammino che unisce due nodi di un grafo. In questo
caso $S$ sar\`a uguale a tutti i possibili cammini che vanno dal nodo di partenza a quello di arrivo.

Le \textbf{istanze positive} saranno le coppie di nodi per le quali esiste almeno un cammino che li unisce, viceversa
le \textbf{istanze negative} saranno le coppie di nodi per le quali non esiste nemmeno un cammino che li unisce e dunque
non \`e possibile dare una risposta.

\subsubsection{Problemi di ottimizzazione}
Nei problemi di ottimizzazione si da un'istanza $x$ in input e si vuole trovare la \textbf{migliore soluzione} $s$ tra
tutte le soluzioni possibili.

In questo caso $S$ sar\`a l'insieme di tutte le soluzioni ottime al problema (diverse fra loro ma con stesso valore).

\subsubsection{Problemi decisionali}
I problemi decisionali richiedono una risposta binaria, in genere \verb|true| o \verb|false|, ossia l'insieme delle
soluzioni \`e
\[ S = \{ 0, 1 \} \]
le \textbf{istanze positive} sono
\[ x \in I \mid \Pi(x) = 1 \]
mentre le \textbf{istanze negative} sono
\[ x \in I \mid \Pi(x) = 0 \]
Tipicamente sono problemi che indagano delle propriet\`a.

In complessit\`a si prende in considerazione solo quest'ultima classe di problemi per due motivi principalmente
\begin{itemize}
	\item Tutto il tempo di calcolo \`e speso per trovare la soluzione e non per scriverla (la risposta \`e solo 0 o 1).
	\item La complessit\`a di un problema riformulato in forma decisionale non cambia rispetto alla sua forma non
	      decisionale.
\end{itemize}
In realt\`a tanti problemi interessanti sono problemi di ottimizzazione. Per trattarli dobbiamo quindi riformularli in
forma decisionale. Per farlo basta verificare l'esistenza di una soluzione (al problema di ottimizzazione) che soddisfa
una certa propriet\`a.

Questo ci dice che il problema di ottimizzazione \`e almeno tanto difficile quanto il corrispondete problema decisionale.

Caratterizzare il problema decisionale ci da un \textbf{limite inferiore} alla complessit\`a del primo.

\section{Classi di complessit\`a}

\begin{theorem}
	Dato un problema decisionale $\Pi$ ed un algoritmo $A$, diciamo che $A$ \textbf{risolve} $\Pi$ se, data un'istanza di
	input $x$ vale
	\[ A(x) = 1 \Leftrightarrow \Pi(x) = 1 \]
\end{theorem}

\begin{theorem}
	Dato un problema decisionale $\Pi$ ed un algoritmo $A$, diciamo che $A$ risolve $\Pi$ in tempo $t(n)$ e spazio
	$s(n)$ se il tempo di esecuzione e l'occupazione di memoria di $A$ sono rispettivamente $t(n)$ e $s(n)$.
\end{theorem}

Data una funzione qualsiasi $f(n)$ possiamo ora definire diverse classi di complessit\`a:
\begin{itemize}
	\item \textbf{Time}: l'insieme dei problemi decisionali che possono essere risolti in tempo $O(f(n))$.
	\item \textbf{Space}: l'insieme dei problemi decisionali che possono essere risolti in spazio $O(f(n))$.
\end{itemize}
Da queste due classi di problemi possiamo derivare altre sottoclassi pi\`u specifiche
\begin{itemize}
	\item \textbf{P}: l'insieme dei problemi risolvibili in tempo polinomiale nella dimensione $n$ dell'istanza di input.

	      In questo tipo di problemi abbiamo due costanti: $c$ e $n_0 > 0$ tali che il numero di passi elementari \`e al
	      pi\`u $n^c$ per ogni input di dimensione $n$ e per ogni $n > n_0$.

	\item \textbf{P-Space}: l'insieme dei problemi risolvibili in spazio polinomiale nella dimensione $n$ dell'istanza
	      di input.

	      In questo tipo di problemi abbiamo due costanti: $c$ e $n_0 > 0$ tali che il numero di celle di memoria
	      utilizzate \`e al pi\`u $n^c$ per ogni input di dimensione $n$ e per ogni $n > n_0$.
	\item \textbf{Exp Time}: l'insieme dei problemi risolvibili in tempo esponenziale nella dimensione $n$ dell'istanza
	      di ingresso.
\end{itemize}

\subsection{Relazioni tra classi}
Si congettura che
\[ \text{P} \subseteq \text{P-Space} \]
questo perch\'e un algoritmo di costo polinomiale pu\`o avere accesso al pi\`u ad un numero polinomiale di locazioni di
memoria diverse (in ordine di grandezza).

Si congettura invece che
\[ \text{P-Space} \subseteq \text{Exp Time} \]
questo perch\'e non \`e detto che un algoritmo di costo esponenziale richieda un numero esponenziale di celle di memoria.

\subsection{Classe NP e certificati}
Introduciamo ora una nuova classe di problemi, la classe \textbf{NP}, dove NP sta per \textbf{P}olinomiale su modelli
\textbf{N}on deterministici.

Per alcuni problemi, per le \textbf{istanze accettabili} (\textbf{positive}) $x$ \`e possibile fornire un
\textbf{certificato} $y$ che possa convincerci del fatto che l'istanza soddisfa la propriet\`a e dunque \`e un'istanza
\emph{accettabile}.

\begin{example}
	Prendiamo il problema della Clique per esempio.
\end{example}