\chapter{Crittografia quantistica}
In questo capitolo non andremo a parlare di macchine quantistiche ma andremo a vedere gli effetti della
\textbf{meccanica quantistica} sulla crittografia.

Esistono infatti dei protocolli crittografici che sfruttano alcuni dei suoi principi per lo scambio delle chiavi in
contesti in cui \`e richiesta estrema sicurezza e in cui si affianca un One-Time Pad come cifrario simmetrico per
le comunicazioni.

\section{Principi di meccanica quantistica}
Introduciamo alcuni principi di meccanica quantistica necessari a comprendere il protocollo
\begin{itemize}
	\item \textbf{Sovrapposizione}: propriet\`a di un sistema quantistico di trovarsi in diversi stati
	      contemporaneamente.
	\item \textbf{Decoerenza}: la misurazione di un sistema quantistico disturba il sistema. Il sistema disturbato
	      perde la sovrapposizione degli stati e collassa in uno stato singolo.
	\item \textbf{No-Cloning}: impossibilit\`a di copiare uno stato quantistico non noto.
	\item \textbf{Entanglement}: possibilit\`a che due o pi\`u elementi si trovino in uno stato quantistico correlato
	      in modo che, anche se portati a grande distanza, mantengono la correlazione.
\end{itemize}

\subsection{Fotoni polarizzati}
Per comprendere al meglio il protocollo che vedremo in seguito dobbiamo anche parlare di \textbf{fotoni polarizzati}.

Un fotone ha una propriet\`a, detta \textbf{polarizzazione}, che pu\`o assumere due valori e che pu\`o essere misurata
facendo riferimento ad una \textbf{base}, anch'essa di due tipi:
\begin{itemize}
	\item \textbf{Ortogonale}: si indica con $+$ e pu\`o assumere valore \textbf{verticale} (a $90^\circ$), indicato con
	      $v$ oppure \textbf{orizzontale} (a $0^\circ$), indicato con $h$.
	\item \textbf{Diagonale}: si indica con $\times$ e pu\`o valere $+45^\circ$ o $-45^\circ$.
\end{itemize}
Non \`e possibile distinguere, con una misura, uno dei quattro casi: si deve scegliere una delle due basi e, dopo la
misura, \`e possibile distinguere uno dei due casi relativi alla base scelta.

Per la misurazione viene usato uno strumento (\textbf{Polarizing Beam Splitter}) il quale, una volta scelta la base di
riferimento, misura il valore di polarizzazione relativo alla base.

Tuttavia, la misurazione \`e corretta solo nel caso in cui il fotone sia preparato con la stessa base del PBS. Nel caso
in cui la base non sia quella corretta si hanno pari probabilit\`a di misurare uno dei due valori.

\section{BB84}
Ideato da Bennet e Brassard, \`e un protocollo per lo scambio di chiavi che fa uso di \textbf{fotoni polarizzati}.

Supponiamo che un utente $A$ voglia comunicare con l'utente $B$ tramite il protocollo BB84. In una fase preliminare si
associa, per ognuna delle due basi, il bit 0 ad un valore della vase e il bit 1 all'altro. Per esempio
\[
	\begin{array}{lrr}
		\text{Ortogonale:} & 90^\circ  \rightarrow 0 & 0^\circ   \rightarrow 1 \\
		\text{Diagonale:}  & +45^\circ \rightarrow 0 & -45^\circ \rightarrow 1
	\end{array}
\]
Le varie associazioni vengono fatte a piacere ma devono essere uguali per entrambi gli utenti. A questo punto $A$
\begin{enumerate}
	\item Sceglie casualmente una delle due basi e uno dei due valori di polarizzazione ad essa associati.
	\item Invia a $B$ un fotone sul canale quantistico, con il valore di polarizzazione scelto al punto 1.
\end{enumerate}
Per inviare il fotone, $A$, usa un dispositivo (\textbf{One Photon Gun}) che spara un fotone, preparato sempre con il
solito valore di polarizzazione, al quale si impone il valore scelto dall'utente con un altro dispositivo.

L'utente $B$ dispone di due \emph{PBS} per la misurazione della polarizzazione: uno per la base ortogonale e uno per
la base diagonale.

Una volta che $A$ ha inviato il fotone, l'utente $B$
\begin{enumerate}
	\item Riceve il fotone ma non sa n\'e la base scelta da $A$ n\'e il valore di polarizzazione.
	\item Genera a caso un bit, il quale indica quale dei due strumenti usare per la misurazione.
	\item A seconda del valore del bit generato al passo precedente, $B$ utilizza uno dei due strumenti e misura la
	      polarizzazione del fotone. Si vengono a creare due possibili circostanze:
	      \begin{itemize}
		      \item Se la base del fotone corrisponde alla base scelta da $B$, la misura della polarizzazione avviene
		            correttamente.
		      \item Altrimenti si hanno pari probabilit\`a di misurare uno dei due valori dell'altra base.
	      \end{itemize}
\end{enumerate}