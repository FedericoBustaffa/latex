\chapter{Cifrari moderni}\label{moderni}
Passiamo ora ai cosiddetti \textbf{cifrari moderni} i quali si dividono in due grandi gruppi
\begin{itemize}
	\item \textbf{Cifrari a sicurezza incondizionata}: sono cifrari per uso ristretto e nascondono l'informazione con
	      certezza assoluta (anche per macchine quantistiche).
	\item \textbf{Cifrari a sicurezza computazionale}: sono cifrari adibiti alla crittografia di massa e nascondono
	      l'informazione solo se il crittoanalista ha accesso a risorse computazionali limitate (su macchine quantistiche
	      si forzano in tempo polinomiale). Anche nel caso in cui si riuscisse a dimostrare che P = NP ognuno di questi
	      cifrari crollerebbe e si potrebbe forzare in tempo polinomiale.
\end{itemize}

\section{Cifrari perfetti}
Un \textbf{cifrario perfetto} \`e tale se non si riesce ad estrapolare alcuna informazione dall'analisi del crittogramma.

Proviamo a formalizzare matematicamente quanto appena detto. Per farlo dobbiamo considerare
\begin{itemize}
	\item \textbf{MSG}: spazio dei messaggi.
	\item \textbf{CRITTO}: spazio dei crittogrammi.
	\item \textbf{M}: variabile aleatoria che descrive il comportamento del	mittente e assume i valori in MSG.
	\item \textbf{C}: variabile aleatoria che descrive la comunicazione sul canale.
\end{itemize}
Indichiamo ora con
\[ P(M = m) \]
la probabilit\`a che il mittente voglia inviare il messaggio $m \in$ MSG. Indichiamo invece con
\[ P(M = m \mid C = c) \]
la probabilit\`a condizionata che il messaggio inviato sia proprio $m$, posto che sul canale stia transitando il
crittogramma $c \in$ CRITTO. In altre parole quest'ultima espressione indica la probabilit\`a che $c$ sia $m$ cifrato.

\begin{theorem}\label{th: cifrario_perfetto}
	Un cifrario \`e \textbf{perfetto} se $\forall m \in \text{MSG}$ e $\forall c \in \text{CRITTO}$ vale che
	\[ P(M = m \mid C = c) = P(M = m) \]
\end{theorem}

\begin{example}
	Mettiamoci per un attimo in uno scenario di massimo pessimismo in cui il crittoanalista sa:
	\begin{itemize}
		\item La distribuzione di probabilit\`a con cui il mittente invia messaggi.
		\item Il cifrario utilizzato.
		\item Lo spazio delle chiavi.
	\end{itemize}
	Supponiamo inoltre che di voler inviare un messaggio $m$ con probabilit\`a
	\[ P(M = m) = p > 0 \quad \quad \text{con } 0 < p < 1 \]
	e analizziamo due casi estremi e opposti l'uno all'altro. Nel primo caso diciamo che esiste un crittogramma $c$ tale
	che
	\[ P(M = m \mid C = c) = 1 \]
	e nel secondo caso diciamo che esiste un crittogramma $c$ tale che
	\[ P(M = m \mid C = c) = 0 \]
	In entrambi i casi, vedere il crittogramma, raffina la conoscenza del crittoanalista. L'unico caso in cui il
	crittoanalista non ricava nulla dal crittogramma \`e il caso descritto dal teorema \ref{th: cifrario_perfetto}.
\end{example}

\subsection{Svantaggi}
L'estrema solidit\`a di un cifrario perfetto ha per\`o un costo in termini di numero di chiavi.

\begin{theorem}[Shannon]
	In un cifrario perfetto l'insieme delle chiavi deve essere grande almeno quanto l'insieme dei messaggi possibili.
	Dove per \textbf{messaggio possibile} indichiamo un messaggio $m \in$ MSG tale che
	\[ P(M = m) > 0 \]
	Questa \`e condizione necessaria ma non sufficiente affinch\'e il cifrario sia perfetto.
	\begin{proof}
		Dimostriamo il teorema per assurdo e andiamo ad indicare con $N_k$ il numero delle chiavi e con $N_m$ il numero
		dei messaggi possibili.

		Supponiamo per assurdo che
		\[ N_m > N_k \]
		e consideriamo ora un crittogramma $c$ che pu\`o transitare sul canale con probabilit\`a
		\[ P(C = c) > 0 \]
		Se provassimo a decifrare $c$ con una generica chiave $k_i$ otterremo un messaggio $m_i$. Facciamo per\`o
		attenzione al fatto che cifrando $c$ con una chiave $k_j$ potremmo ottenere il messaggio $m_i$, ottenibile anche
		con la chiave $k_i$.

		Indichiamo quindi con $s$ tale che
		\[ s \leq N_k \]
		il numero dei messaggi che potrebbero corrispondere al crittogramma $c$. Ma per ipotesi abbiamo che
		\[ N_k < N_m \]
		quindi
		\[ s \leq N_k < N_m \]
		Ho ottenuto che il numero dei messaggi che possono corrispondere al crittogramma $c$ \`e strettamente minore
		del numero dei messaggi possibili.

		Questo vuol dire che esiste un messaggio $m'$ appartenente allo spazio dei messaggi possibili che non pu\`o
		corrispondere a quel crittogramma.
		\[ P(M = m' \mid C = c) = 0 \]
		Giungiamo quindi all'assurdo dato che un cifrario \`e perfetto se un crittogramma pu\`o corrispondere ad uno
		qualsiasi dei messaggi possibili.
	\end{proof}
\end{theorem}

\subsection{One-Time Pad}