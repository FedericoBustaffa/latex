\chapter{Cifrari a chiave pubblica}
Nei cifrari simmetrici si ha un grosso problema, ovvero, lo \textbf{scambio della chiave}, che, come sappiamo, deve
essere la stessa per entrambi gli utenti.

Fino ad ora abbiamo sempre assunto che i due utenti fossero gi\`a in possesso della chiave e abbiamo solo parlato del
metodo di cifratura e decifrazione. Come abbiamo visto i messaggi scambiati sono cifrati e dunque la comunicazione \`e
sicura ma come avviene lo scambio della chiave ?
\begin{itemize}
	\item Se avvenisse di persona allora tanto vale scambiarsi direttamente il messaggio.
	\item Se avvenisse in chiaro non ha pi\`u senso cifrare il messaggio dato che chiunque potrebbe intercettare la
	      chiave e decifrare senza sforzo.
	\item Se inviassimo la chiave cifrata si innescherebbe lo stesso problema all'infinito: il destinatario avrebbe
	      bisogno della chiave di cifratura per decifrare e dunque si dovrebbe inviare un'altra chiave e cos\`i via.
\end{itemize}
Il problema \`e risolto grazie ai \textbf{cifrari a chiave pubblica}.

\section{Soluzione ingenua}
Se abbiamo un sistema con $N$ utenti, ogni utente pu\`o memorizzare $N-1$ chiavi diverse e condivise con ciascun altro
utente.

In questo modo abbiamo un numero quadratico di chiavi nel numero di utenti del sistema.

\section{TTP}
Una soluzione migliore \`e rappresentata dal \textbf{TTP} o \textbf{trusted 3rd party}, ossia una terza parte
\emph{fidata}, a cui gli utenti si appoggiano per comunicare.

Ogni utente deve ricordarsi una sola chive mentre TTP gestisce la creazione e lo scambio delle chiavi condivise tra i
due utenti.

Siano $A$ e $B$ i due utenti che vogliono comunicare, il processo di scambio funziona in questo modo:
\begin{enumerate}
	\item $A$ e $B$ possiedono rispettivamente $k_A$ e $k_B$, due chiavi generate da loro stessi.
	\item $A$ comunica a TTP di voler comunicare con $B$.
	\item TTP genera casualmente una chiave $k_{AB}$ che potranno usare i due utenti per quella comunicazione.
	\item TTP cifra genera due crittogrammi $c_A$ e $c_B$ in questo modo
	      \[
		      \begin{matrix}
			      c_A & = & C(k_{AB}, k_A) \\
			      c_B & = & C(k_{AB}, k_B)
		      \end{matrix}
	      \]
	\item TTP invia $c_A$ e $c_B$ ad $A$.
	\item $A$ decifra $c_A$ con $k_A$ e invia a $c_B$ a $B$.
	\item $B$ decifra $c_B$ con $k_B$.
\end{enumerate}
Alla fine di questo processo i due utenti sono in possesso di una chiave $k_{AB}$ che potranno usare per quella
comunicazione in un cifrario simmetrico.

Le problematiche di questo sistema sono due:
\begin{itemize}
	\item TTP deve essere sempre online.
	\item TTP conosce tutte le chiavi.
\end{itemize}

\subsection{Problematiche}
