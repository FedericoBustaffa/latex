\chapter{Introduzione}
\section{Crittografia}
La \textbf{crittografia} si occupa dei metodi di cifratura mentre la \textbf{crittoanalisi} dei metodi di decifrazione
di un messaggio. Entrambe si possono riunire sotto la branca della \textbf{crittologia}.

Un tipico scenario, usato per capire meglio le varie tecniche crittografiche, \`e quello che vede due personaggi (Alice
e Bob) impegnati nello scambio di messaggi su di un canale di trasmissione \emph{insicuro}. \`E quindi possibile
intercettare i messaggi che transitano su di esso.

Per proteggere la comunicazione, Alice e Bob, devono adottare un \textbf{metodo di cifratura} che impedisca, ad un terzo
personaggio (Eve), in veste di crittoanalista, di leggere i messaggi che si inviano.

Ci\`o che viagger\`a sul canale di trasmissione sar\`a dunque un crittogramma con le seguenti caratteristiche:
\begin{itemize}
	\item \textbf{incomprensibile} al crittoanalista.
	\item \textbf{facilmente decifrabile} da Bob.
\end{itemize}

\section{Cifratura}
In crittografia si usa una \textbf{funzione di cifratura} $\mathcal{C} : MSG \rightarrow CRITTO$, dove $MSG$ \`e l'insieme
dei messaggi in chiaro e $CRITTO$ \`e l'insieme dei crittogrammi, di questo tipo
\[ \mathcal{C}(m) = c \]
che cifra un certo \textbf{messaggio} $m$ in un \textbf{crittogramma} $c$. Deve esserci anche una
\textbf{funzione di decifrazione} $\mathcal{D} : CRITTO \rightarrow MSG$, \textbf{inversa} di $\mathcal{C}$, di questo
tipo:
\[ \mathcal{D}(c) = m \]
che trasforma un crittogramma $c$ in un messaggio in chiaro $m$.

La funzione $\mathcal{C}$ \textbf{deve} essere \textbf{iniettiva} poich\'e deve esserci una corrispondenza biunivoca
tra un messaggio e il relativo crittogramma.

Se ad un crittogramma corrispondessero pi\`u messaggi in chiaro, anche il destinatario del messaggio non \`e in grado
di capire quale sia il messaggio inviato dal mittente.

Come gi\`a detto, $\mathcal{C}$ e $\mathcal{D}$ devono essere l'una l'inversa dell'altra, vale quindi la seguente
relazione:
\[ \mathcal{D}(c) = \mathcal{D}(\mathcal{C}(m)) = m \]

\section{Cifrari}
\subsection{Cifrario di Cesare}
Si tratta di un cifrario per uso ristretto la cui idea \`e ottenere un crittogramma sostituendo ogni lettere del messaggio
con la lettera tre posizioni pi\`u avanti nell'alfabeto. Se si supera la Z, si riparte dalla A.

La segretezza dipendeva dalla conoscenza del metodo. Una volta scoperto il metodo di cifratura il cifrario diventa
inutile.

\subsection{Chiavi}
Se un cifrario \`e utilizzato da molti utenti, la \textbf{parte segreta} del metodo non pu\`o essere la funzione di
cifratura e/o quella di decifrazione, poich\'e sarebbe troppo difficile mantenerla segreta.

Si deve quindi introdurre un nuovo elemento, ossia la \textbf{chiave}, nota solo agli utenti che stanno comunicando e
tramite la quale possono cifrare e decifrare il messaggio.

Il numero delle chiavi dev'essere di ordine abbastanza grande da rendere impossibile o molto dispendioso il metodo di
provarle tutte (\textbf{forza bruta}). La chiave deve essere scelta in modo \textbf{casuale}.

Le funzioni di cifratura e decifrazione diventano quindi di questo tipo:
\[ \mathcal{C}(m, k) = c \quad \mathcal{D}(c, k) = m \]
dove $k$ \`e la chiave, inserita come parametro per entrambi i metodi.

Ricapitolando, il crittoanalista, anche conoscendo $\mathcal{C}$, $\mathcal{D}$ e $c$, non deve essere in grado di
risalire a $m$ senza conoscere $k$.

\subsection{Cifrari per uso ristretto e per uso generale}
\begin{itemize}
	\item I \textbf{cifrari per uso ristretto} hanno la caratteristica di tenere la funzione di cifratura e
	      decifrazione, segrete in ogni loro aspetto.
	\item I \textbf{cifrari per uso generale}, al contrario, rendono pubbliche le funzioni di cifratura e decifrazione.
	      Tengono tuttavia una chiave segreta che sar\`a diversa per ogni coppia di utenti e inserita come parametro
	      nelle funzioni di cifratura e decifrazione.

	      Se non si conosce la chiave non si deve poter risalire al messaggio in chiaro neppure conoscendo le due
	      funzioni.
\end{itemize}

\subsection{Cifrari perfetti}
Un \textbf{cifrario perfetto} deve possedere le seguenti caratteristiche:
\begin{itemize}
	\item La chiave deve essere segreta, lunga quanto il messaggio e nuova per ogni messaggio.
	\item Il messaggio in chiaro e il crittogramma risultano del tutto scorrelati tra loro.
	\item Nessuna informazione sul testo in chiaro pu\`o filtrare dal crittogramma.
	\item La conoscenza del crittoanalista non cambia dopo aver osservato un crittogramma.
\end{itemize}
I cifrari perfetti sono tuttavia computazionalmente molto costosi e sono quindi inutilizzabili per la crittografia di
massa.

\subsection{Cifrari sicuri}
Ad oggi i cifrari usati non sono perfetti ma sono dichiarati \textbf{sicuri} perch\'e rimasti inviolati e perch\'e la
loro violazione si basa su problemi matematici molto complessi. Se si riuscisse a trovare delle soluzioni efficienti
per tali problemi (tempo polinomiale) la sicurezza di tali cifrari crollerebbe.

\subsection{Cifrari simmetrici}
In questi cifrari la \textbf{chiave di cifratura} \`e uguale alla \textbf{chiave di decifrazione} ed \`e nota solo ai
due comunicanti. Il messaggio \`e diviso in blocchi lunghi come la chiave. La chiave \`e utilizzata per trasformare un
blocco del messaggio in un blocco del crittogramma.

\subsection{Cifrari a chiave pubblica (asimmetrici)}
I \textbf{cifrari a chiave pubblica} nascono dall'esigenza di risolvere il problema dello scambio delle chiavi tra i due
utenti.

In questo caso, mittente e destinatario hanno due chiavi diverse. La chiave di cifratura e le funzioni di cifratura e
decifrazione sono \textbf{pubbliche} mentre invece la chiave di decifrazione \`e \textbf{privata} e solo il destinatario
del messaggio ne \`e in possesso.

L'obbiettivo \`e quello di permettere a tutti di inviare messaggi cifrati ad un certo destinatario ma solo lui deve
essere in grado di decifrarli.

In questo tipo di cifrari, la funzione di cifratura dev'essere di tipo \textbf{one-way trap-door}, ovvero, cifrare
dev'essere computazionalmente facile (tempo polinomiale) e decifrare dev'essere computazionalmente difficile (tempo
esponenziale) a meno che non si conosca la chiave.

Tra i vantaggi abbiamo che si elimina la necessit\`a di effettuare uno scambio di chiavi e abbiamo inoltre una riduzione
sostanziale sul numero di chiavi da generare: se gli utenti di un sistema sono $n$, il numero complessivo di chiavi
pubbliche e private \`e $2n$ anzich\'e $n (n-1) / 2$.

D'altra parte abbiamo dei cifrari molto lenti, dunque difficilmente utilizzabili in crittografia di massa e sono inoltre
soggetti ad attacchi di tipo \emph{chosen plain text}.

\section{Attacchi crittografici}
Il crittoanalista pu\`o avere due tipi di \textbf{comportamento} nella fase di attacco ad un sistema crittografico:
\begin{itemize}
	\item \textbf{Passivo}: si limita ad ascoltare la comunicazione intercettando i crittogrammi inviati.
	\item \textbf{Arrivo}: modifica il contenuto dei messaggi che i due utenti si inviano, oppure si spaccia per uno di
	      dei due per riuscire a trarre informazioni utili alla decifrazione del messaggio.
\end{itemize}

L'obbiettivo di un attacco \`e ovviamente quello di forzare il sistema. Il metodo e il livello di pericolosit\`a
dipendono dalle informazioni in possesso del crittoanalista.

\subsection{Attacchi passivi}
Negli attacchi passivi il crittoanalista si limita dunque a intercettare i messaggi per provare a estrapolarne
informazioni. Tra gli attacchi passivi pi\`u comuni abbiamo:
\begin{itemize}
	\item \textbf{Cipher Text}: Il crittoanalista rileva solo crittogrammi sul canale e cerca di estrapolarne informazioni.
	\item \textbf{Known Plain Text}: Il crittoanalista conosce delle coppie ($m$, $c$).
	\item \textbf{Chosen Plain Text}: Il crittoanalista si procura delle coppie ($m$, $c$) relative
	      ai messaggi in chiaro da lui scelti.
	\item \textbf{Chosen Cipher Text}: Il crittoanalista si procura delle coppie ($m$, $c$) relative a crittogrammi
	      da lui scelti.
	\item \textbf{Brute force}: Il crittoanalista prova tutte le chiavi.
\end{itemize}

\subsubsection{Attacchi chosen plain-text}
Dato che la funzione di cifratura e anche la sua chiave sono pubbliche, il crittoanalista pu\`o cifrare un numero a
piacimento di messaggi, confrontarli con i messaggi crittografati che sta provando a decifrare e, nel caso in cui
trovi una corrispondenza tra qualcuno dei suoi crittogrammi e qualcuno da decifrare significa che ha trovato il
messaggio in questione.

\subsection{Attacchi attivi}
Come anticipato in un attacco di tipo attivo, il crittoanalista si intromette tra le comunicazioni che arrivano tra
Alice e Bob e cerca di disturbare la comunicazione in vari modi.

\subsubsection{Man in the middle}
Uno degli attacchi di tipo attivo pi\`u comune \`e il \textbf{man in the middle}. In questo tipo di attacco il
crittoanalista interrompe le comunicazioni dirette tra i due utenti e sostituisce i loro messaggi con i propri e convince
ciascun utente che tali messaggi arrivino dall'altro.