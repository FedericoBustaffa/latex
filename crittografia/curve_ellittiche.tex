\chapter{Crittografia su curve ellittiche}
La \textbf{crittografia su curve ellittiche} nasce per alleggerire il carico computazionale che si porta dietro
la crittografia a chiave pubblica basata sull'algebra modulare.

I problemi su cui si basa la crittografia a chiave pubblica, come la fattorizzazione e il calcolo del logaritmo
discreto hanno due problemi:
\begin{itemize}
	\item Sebbene si risolvano in tempo polinomiale sono comunque \textbf{calcoli molto pesanti}.
	\item Gli algoritmi odierni per il calcolo di questi due problemi non hanno pi\`u costo esponenziale ma,
	      come abbiamo gi\`a visto, \textbf{subesponenziale}. Ne \`e una diretta conseguenza l'aumento della
	      lunghezza delle chiavi.
\end{itemize}
Come vedremo, la crittografia su curve ellittiche, propone i cifrari visti in precedenza (protocollo DH e cifrario
di ElGamal) in un contesto matematico diverso e basati sul calcolo del logaritmo discreto e non pi\`u sulla
fattorizzazione.

Il risultato \`e stato quello di ottenere un costo del miglior attacco, come vedremo, \textbf{puramente esponenziale}.

\section{Curve ellittiche}
Prima di vedere i dettagli implementativi definiamo cosa sono le curve ellittiche.

Le curve ellittiche sono curve algebriche descritte da equazioni cubiche (simili a quelle utilizzate nel calcolo
della lunghezza degli archi delle ellissi).
\[ E(a, b) = \{ (x, y) \in \mathbb{R}^2 \mid y^2 = x^3 + ax + b \} \]
qui sono definite sui reali, nella cosiddetta \emph{forma normale di Weierstrass}, ma possiamo scrivere l'equazione
su un campo $\mathbb{K}$ qualisiasi.

L'insieme appena descritto contiene anche il cosiddetto \textbf{punto all'infinito} $O$ in direzione dell'asse $y$
(la curva ha un asintoto verticale), il quale rappresenta l'\textbf{elemento neutro per l'addizione}.

La curva si pu\`o presentare in due forme
\begin{itemize}
	\item A \textbf{due componenti} nel caso in cui la cubica abbia tre radici reali.
	\item Ad \textbf{una componente} nel caso in cui la cubica abbia una sola radice reale e due complesse coniugate.
\end{itemize}
A prescindere dalla forma, in ogni punto della curva \`e possibile mandare una \textbf{tangente}.

Per le applicazioni crittografiche si assume che il discriminante della cubica sia
\[ 4 a^3 + 27 b^2 \neq 0 \]
Questo ci assicura che
\begin{itemize}
	\item La cubica non abbia radici multiple.
	\item La curva sia priva di punti singolari come \emph{cuspidi} o \emph{nodi}, dove non sarebbe definita in modo
	      univoco la tangente.
\end{itemize}