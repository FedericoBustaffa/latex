\chapter{Sicurezza}
Come possiamo notare, il modello stratificato non comprende protocolli
di sicurezza in maniera nativa. La sicurezza è infatti una sorta di 
strato aggiuntivo i cui obbiettivi sono:
\begin{itemize}
	\item \textbf{Riservatezza e segretezza}: solo mittente e 
		destinatario devono essere in grado di comprendere il contenuto
		del messaggio trasmesso.
	\item \textbf{Integrità}: il contenuto del messaggio deve essere
		\textbf{integro} e quindi non deve essere stato manomesso.
	\item \textbf{Disponibilità}: deve essere garantita la 
		disponibilità di informazioni e servizi.
\end{itemize}
A minare questi punti cardine delle comunicazioni di rete ci sono 
diversi attacchi come l'intercettazione, lo spoofing e DDoS.

\section{Crittografia}
Il meccanismo fondamentale per garantire la riservatezza di una
comunicazione è la \textbf{crittografia}. La quale usa sistemi di
cifratura a chiave \textbf{simmetrica} o \textbf{asimmetrica}.

Nel primo caso i due utenti in comunicazione utilizzano la stessa 
chiave per cifrare e decifrare i messaggi. Il problema è quello di 
riuscire a scambiarsi tale chiave tra sistemi remoti.

Per risolvere questo problema è nata la crittografia a chiave 
asimmetrica. Questo modello prevede l'utilizzo di una coppia di chiavi:
\begin{itemize}
	\item \textbf{Pubblica}: conosciuta da entrambi i nodi.
	\item \textbf{Privata}: mittente e destinatario possiedono una
		chiave privata diversa l'una dall'altra.
\end{itemize}
Il messaggio viene cifrato con la chiave pubblica ma viene decifrato
con quella privata. Ogni nodo, attraverso la chiave pubblica e un 
altro dato che varia da cifrario a cifrario è in grado di costruirsi
una sua chiave privata.

Il problema della cifratura asimmetrica è l'efficienza in quanto le 
computazioni coinvolte sono molto pesanti. Ecco che spesso si usa la
cifratura asimmetrica per lo scambio delle chiavi simmetriche (usate
per il resto della comunicazione).

\section{Hash}
Le funzioni \textbf{hash} hanno il compito di produrre un \emph{hash}
del messaggio scambiato. Questo hash equivale ad una sorta di impronta
digitale del messaggio.

Ecco che, quando è richiesta integrità, si calcola l'hash del messaggio
insieme al messaggio stesso. Una volta arrivato a destinazione viene
ricalcolato l'hash e si confronta con l'hash inviato. Se i due hash
corrispondono allora il messaggio è integro.

\subsection{MAC}
Per aggiungere un ulteriore strato al controllo dell'integrità del
messaggio si fa uso di un \textbf{MAC}, che non ha niente a che fare
con l'indirizzo MAC di cui abbiamo parlato in precedenza.

Il MAC (Message Authentication Code) altro non è che il risultato della
funzione hash applicata ad una concatenazione di una chiave (condivisa 
e segreta tra i due utenti) e del messaggio inviato.

In questo modo, quando si ricalcola il MAC in fase di ricezione, nel
caso ci sia una corrispondonza, si ha la conferma che il messaggio 
proviene dall'utente con cui abbiamo scambiato la chiave segreta.

\subsection{Firma digitale}
Il MAC non risolve il problema della \textbf{non ripudiabilità}. Ecco
che viene introdotta la \textbf{firma digitale}, la quale genera un
codice tramite la chiave privata dell'utente mittente.

Tale chiave è conosciuta solo dall'utente mittente e questo fa sì che
questo si identifichi e non possa negare di aver inviato il messaggio.

\section{TLS e SSL}
I protocolli di sicurezza \textbf{TLS} (Transport Layer Security) ed
\textbf{SSL} (Secure Sockets Layer) sono implementati a livello di
trasporto che forniscono riservatezza, integrità e autenticazione.

Il protocollo SSL è visto come un sottostrato che si frappone tra lo
strato di trasporto e lo strato applicativo ed è composto da tre fasi
principali:
\begin{enumerate}
	\item \textbf{Handshake}: fase iniziale in cui mittente e 
		destinatario si autenticano e si scambiano dati sensibili.
	\item \textbf{Derivazione delle chiavi}: mittente e destinatario
		utilizzano i dati condivisi per generare le chiavi.
	\item \textbf{Trasferimento dati}: i dati da trasferire sono 
		scomposti in serie di record.
\end{enumerate}
Se questo protocollo viene utilizzato correttamente. l'unica cosa che
un attaccante è in grado di vedere in chiaro è indirizzo IP e porta e
quanti dati vengono approssimativamente inviati.

Potrebbero essere in grado di terminare la connessione ma sia mittente
che destinatario sarebbero in grado di capire che questo è stato fatto
da una terza parte.

\section{IPSec}
Si tratta di un insieme di protocolli in grado di fornire sicurezza a 
livello rete tramite l'autenticazione la confidenzialità dei pacchetti
IP tramite
\begin{itemize}
	\item \textbf{Modalità trasporto}: vengono aggiunti un header e un
		trailer IPSec, rispettivamente all'inizio e alla fine del frame
		in arrivo dal livello di trasporto. Viene poi aggiunta la 
		solita intestazione IP. Viene dunque protetto il payload ma non
		l'intestazione IP.
	\item \textbf{Modalità tunnel}: si protegge l'intero pacchetto IP
		andando ad aggiungere header e trailer rispettivamente 
		all'inizio e alla fine dell'intero pacchetto IP. In seguito
		si andrà ad aggiungere una ulteriore intestazione IP.
\end{itemize}
Questo è molto utilizzato nel caso in cui si voglia proteggere la 
comunicazione tra due router.

\subsection{ESP}
Il protocollo \textbf{ESP} è una delle versioni più usate di IPSec, il
quale aggiunge un trailer ESP al datagramma che viene cifrato insieme
al payload.

Aggiunge anche un'intestazione ESP che viene autenticato tramite un 
MAC e infine si aggiunge la solita intestazione IP con un flag in 
grado di segnalare che il pacchetto IP trasportato è protetto da IPSec.

\subsection{VPN}
Una \textbf{VPN} (Virtual Private Network) utilizza la modalità di 
tunnel di IPSec per andare a creare una rete \emph{privata} sulla rete
pubblica incapsulando il contenuto dei datagrammi IP.
