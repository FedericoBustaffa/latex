\chapter{Reti peer to peer}
Fino ad ora abbiamo visto un tipo di comunicazione basato sul paradigma
\textbf{client-server}, in cui un processo \emph{client} faceva delle
richieste ad un processo \emph{server} che cercava di fornire dei
servizi.

Il paradigma \textbf{peer to peer} (abbreviato in P2P) invece, prevede 
che tutti gli host fungano da client e da server allo stesso tempo.
Tra le applicazioni P2P più conosciute ed utilizzate al giorno d'oggi,
abbiamo BitTorrent e applicazioni che prevedono l'utilizzo di 
blockchain (Bitcoin).

In una rete P2P tutti i nodi hanno la stessa importanza, si trovano
nella periferia della rete e sono indipendenti l'uno dall'altro.

Il vantaggio delle reti P2P sta nella loro grande dinamicità e 
nell'assenza di un sistema centralizzato. Quest'ultimo punto però è
un'arma a doppio taglio in quanto
\begin{itemize}
	\item Un sistema centralizzato nel caso subisse malfunzionamenti
		non riuscirebbe più a fornire servizi. D'altro canto, in 
		assenza di tali malfunzionamenti, riesce a fornire i servizi
		con più \emph{solidità}.
	\item Una rete P2P ha il vantaggio di essere distribuita e dunque,
		nel caso si richieda un servizio, è probabile che il nodo in
		grado di fornircelo sia vicino al nostro. \`E anche vero che
		quel nodo appartiene ad un utente che potrebbe disconettersi
		dalla rete in ogni momento (anche durante l'erogazione del
		servizio).
\end{itemize}
Ecco che quest'ultima casistica deve essere gestita e può avere un 
costo variabile a seconda dei casi.

\section{Contattare altri peer}
Altro problema della rete peer to peer è quello di riuscire a 
contattare altri peer. Si pone infatti il problema di riuscire a
instaurare una connessione con altri nodi non sapendo però il loro
nome o indirizzo IP (che cambia ogni volta che il nodo accede alla 
rete).

Ogni peer offre poi servizi differenti e quindi, anche se fossimo in
grado di raggiungere un certo numero di nodi, dobbiamo essere in grado
di filtrarli per quali offrono i servizi che ci interessano.

\subsection{Directory centralizzata}
Il metodo più semplice sta nel creare una directory centralizzata in
cui ogni peer, quando si collega alla rete, scrive il proprio indirizzo
IP e i servizi o file di cui è in possesso e che vuole condividere con
gli altri peer.

Questo modello è molto semplice ma ha il problema di avere un unico 
punto di fallimento (directory centralizzata). Punto che può anche 
creare colli di bottiglia per le prestazioni.

\subsection{Reti decentralizzate}
Per risolvere queste problematiche si è passati ad un tipo di rete
decentralizzato in cui i peer si organizzano in un \textbf{overlay 
network}, ossia un rete logica in cui abbiamo delle particolari 
connessioni e in cui si possono evidenziare dei ruoli tra i peer.

\subsubsection{Reti non strutturate}
Uno dei primi approcci prevede l'uso di reti non strutturate in cui
non ci sono particolari modelli di organizzazione tra i nodi. Questo
implica l'assenza di un'organizzazione nella locazione delle risorse.

Sebbene questo approccio risolva il problema della centralizzazione,
rende l'accesso alle risorse molto difficoltoso per via dell'assenza
di struttura. 

Uno dei vantaggi è che la rimozione/aggiunta di nodi alla rete è poco
costosa. Questo permette di gestire reti i cui nodi hanno un 
comportamento fortemente transiente.

Si va ad aumentare anche il traffico in quanto, non avendo una
struttura, la ricerca delle risorse avviene "\emph{alla cieca}"
coinvolgendo molti nodi.

\subsubsection{Copertura gerarchica}
In questo modello vengono \emph{eletti} dei peer come \textbf{group
leader} di altri peer in base a quanto sono "potenti" in banda o in
risorse.

La rete è poi organizzata con una topologia che prevede connessioni TCP
dirette tra i peer e il loro group leader e tra alcune coppie di group
leader.

Ogni group leader mantiene una lista dei contenuti appartenenti ai peer
che sono a lui collegati.

\section{BitTorrent}
Il protocollo BitTorrent viene utilizzato per la distribuzione di file.
Esso si basa sulla suddivisione del file in pezzi da 256 KB che poi 
verranno ridistribuiti dai vari peer ai destinatari in questo modo:
\begin{enumerate}
	\item Si riduce il carico di ogni sorgente.
	\item Si riduce la dipendenza dal distributore originale.
	\item Si fornisce \emph{ridondanza}.
\end{enumerate}
Quando avviene una condivisione, un insieme di peer (\textbf{swarm})
partecipa alla distribuzione scambiandosi pezzi (\textbf{chunk}) di 
tale file. Ognuno dei peer preleva e trasmette allo stesso tempo più
chunk.

Esiste un nodo particolare, chiamato \textbf{tracker}, che ha il
compito di coordinare la distribuzione del file tramite dei file (con
estensione .torrent) contenenti metadati sul file condiviso. Questi
file .torrent sono vengono forniti al tracker dai nodi che vogliono
condividere un file o un chunk.

Quando un nuovo peer si collega e richiede un file, il tracker gli 
fornisce gli indirizzi IP di alcuni nodi dello swarm che possiedono
il file o alcuni chunk di quest'ultimo.

