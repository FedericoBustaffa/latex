\chapter{Reti peer to peer}
Fino ad ora abbiamo visto un tipo di comunicazione basato sul paradigma
\textbf{client-server}, in cui un processo \emph{client} faceva delle
richieste ad un processo \emph{server} che cercava di fornire dei
servizi.

Il paradigma \textbf{peer to peer} (abbreviato in P2P) invece, prevede 
che tutti gli host fungano da client e da server allo stesso tempo.
Tra le applicazioni P2P più conosciute ed utilizzate al giorno d'oggi,
abbiamo BitTorrent e applicazioni che prevedono l'utilizzo di 
blockchain (Bitcoin).

In una rete P2P tutti i nodi hanno la stessa importanza, si trovano
nella periferia della rete e sono indipendenti l'uno dall'altro.


