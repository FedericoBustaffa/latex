\chapter{Livello di trasporto}
Come abbiamo visto, il livello applicativo, genera messsaggi che poi 
vengono consegnati al \textbf{livello di trasporto}. Nel livello di 
trasporto vengono specificati indirizzo IP e porta del processo remoto 
verso il quale devono essere inviati tali messaggi.

Il livello di trasporto realizza una \textbf{comunicazione logica} fra 
processi residenti in host diversi comportandosi come se gli host 
fossero direttamente collegati. Non si preoccupa quindi dei dettagli 
dell'infrastruttura fisica usata per la comunicazione.

Lo strato applicativo interagisce con il livello di trasporto per 
trasmettere o ricevere dati scegliendo la \emph{modalità} di trasporto.
Deve quindi fornire al livello di trasporto i dati nella forma 
richiesta perché questo riesca a consegnarli.

Affinché la comunicazione avvenga, il livello di trasporto si appoggia 
allo strato di rete che si occupa della comunicazione tra host 
consegnando il datagramma all'host destinatario (non al processo).

Il livello di trasporto è implementato solo nei sistemi terminali per 
l'invio e la consegna di dati fra processi. Tutti i sistemi intermedi 
arrivano ad implementare solo il livello di rete.

\section{Servizi}
A livello di trasporto abbiamo due tipi di servizio:
\begin{itemize}
	\item \textbf{Connectionless}: il processo mittente consegna i 
		messaggi al livello di trasporto uno per uno. Il livello di 
		trasporto tratta ogni messaggio come entità singole senza 
		mantenere alcuna relazione fra di essi. Non c'è né garanzia 
		sull'arrivo del segmento né che l'ordine di invio sia 
		preservato anche in fase di consegna.
	\item \textbf{Connection-oriented}: viene stabilita una connessione
		logica fra il processo client e il processo server.
\end{itemize}
In entrambi i casi questi servizi forniscono:
\begin{itemize}
	\item \textbf{Multiplexing/Demultiplexing}: incapsulano il 
		messaggio del livello applicativo in modo che sia possibile 
		inviarlo sulla rete e, in fase di consegna, lo decapsulano per
		consegnarlo al processo destinatario.
	\item \textbf{Controllo degli errori}: nel caso di TCP è 
		obbligatorio mentre UDP lo implementa solamente.
\end{itemize}
I due servizi principalmente usati sono \textbf{TCP} e \textbf{UDP}. Il
servizio TCP, come già anticipato in precedenza offre una serie di 
garanzie tra cui:
\begin{itemize}
	\item Gestione della connessione.
	\item Consegna affidabile.
	\item Controllo del flusso dati.
	\item Controlli sulla congestione del traffico internet.
\end{itemize}
Il servizio UDP invece non garantisce niente di tutto ciò ma ha come
vantaggio principale la velocità di trasmissione in quanto non paga la 
fase iniziale di connessione.

\subsection{Multiplexing/Demultiplexing}
Come abbiamo appena detto, sia TCP che UDP, fanno \textbf{Multiplexing}
e \textbf{Demultiplexing}.

Il Demultiplexing è un'operazione che viene eseguita in seguito 
all'arrivo del pacchetto dalla rete. In questa fase infatti il 
pacchetto è arrivato all'host destinatario e non al processo 
destinatario. Ecco che l'operazione di Demultiplexing si occupa proprio
di recapitare il pacchetto al processo giusto.

L'operazione di Multiplexing si occupa invece di gestire il messaggio 
in arrivo dallo strato applicativo, inserire la giusta 
\emph{intestazione} e infine inviarlo al protocollo di rete.

Queste due operazioni si basano sui \textbf{socket address} dei 
processi, identificati dalla coppia indirizzo IP e porta.

\section{Indirizzo IP e porta}
Abbiamo già parlato di porte ma ora andiamo a chiarire meglio il 
concetto. Una \textbf{porta} è un numero di 16 bit che viene assegnato 
a un processo, o più precisamente a un punto di Demultiplexing dei 
protocolli TCP o UDP.

L'\textbf{indirizzo IP} è invece un numero a 32 bit presente nello 
stack TCP/IP. La coppia formata da indirizzo IP e porta identifica in 
modo univoco la connessione TCP o UDP.

Il range di porte (da 0 a 65535) è organizzato in tre diverse categorie
come segue:
\begin{itemize}
	\item \textbf{Porte di sistema}: da 0 a 1023 e identificano i 
		processi server.
	\item \textbf{Porte utente}: da 1024 a 49151 utilizzabili .
	\item \textbf{Porte dinamiche}: da 49152 a 65535.
\end{itemize}
Il sistema operativo assegna dinamicamente le porte ai processi che ne 
fanno richiesta.

\subsection{Demultiplexing per UDP}
Con UDP, la coppia IP e porta è sono sufficienti per l'operazione di 
Demultiplexing. Quello che viene fatto nello specifico è andare a 
controllare indirizzo IP e porta del destinatario per recapitare il 
messaggio.

Da tenere di conto anche il fatto che datagrammi con IP e/o porta 
differenti ma stessi IP e porta destinatari vengono consegnati alla 
stessa socket. In questa situazione è possibile distinguere processi o 
host mittenti differenti tramite indirizzo IP e porta del mittente, 
presenti
nell'intestazione del pacchetto.

\subsection{Demultiplexing per TCP}
In questo caso i processi comunicano tramite una connessione logica 
instaurata dal livello di trasporto e la socket TCP è identificata da 
quattro parametri:
\begin{itemize}
	\item Indirizzo IP di origine.
	\item Numero di porta di origine.
	\item Indirizzo IP di destinazione.
	\item Numero di porta di destinazione.
\end{itemize}
L'host ricevente usa tutti e quattro i per inviare il segmento alla 
socket appropriata. Un host può inoltre supportare più socket 
contemporaneamente ma ogni socket è specificata per una sola coppia
mittente destinatario.
