\section{Commutazione}
Una internet è data dall'interconnessione di reti composte da link e dispositivi capaci di
scambiarsi informazioni. In particolare distinguiamo i dispositivi in host e dispositivi di
interconnessione.

Nel momento in cui vogliamo inviare un \textbf{pacchetto} si deve definire un percorso nella rete
lungo il quale questo viaggia per arrivare a destinazione. Le tecniche di \textbf{commutazione}
definiscono i metodi con il quale viene determinato tale percorso. In particolare andiamo a
distinguere due principali metodi di commutazione: commutazione di \textbf{pacchetto} e
commutazione di \textbf{circuito}.

In generale non esiste un solo percorso per connettere due dispositivi. Ecco che i due modelli di
commutazione entrano in gioco fornendo due possibili alternative nella scelta di tale percorso.
Vedremo che i due tipi di commutazione non forniscono solo un metodo per determinare un percorso
di rete ma influiscono anche sulla velocità di connessione tra dispositivi e sulla trasmissione
dell'informazione.

\subsection{Commutazione di circuito}
La commutazione di circuito instaura un cammino dedicato tra i due dispositivi che vogliono
comunicare tramite una serie di linee di trasmissione e dispositivi di commutazione che collegano
sorgente e destinazione.
\begin{enumerate}
	\item \textbf{Inizializzazione}: In genere si ha una fase iniziale in cui si richiede di
		aprire una connessione verso una determinata entità e una prima infrastruttura si occupa 
		di determinare una connessione dedicata tra mittente e destinatario. I dispositivi di 
		mezzo sono incaricati di gestire tale collegamento di modo che la connessione non si
		interrompa.
	\item \textbf{Risorse}: una volta determinato il percorso si dedicano le risorse necessarie 
		a tale connessione. Una volta allocate, tali risorse, sono disponibile per tutta la durata 
		della comunicazione.
\end{enumerate}
Gli svantaggi di questo tipo di commutazione sono principalmente due:
\begin{itemize}
	\item \`E necessaria una fase iniziale di inizializzazione prima di iniziare a comunicare.
	\item Le risorse dedicate ad una connessione rimangono inattive se non utilizzate.
\end{itemize}
D'altro canto abbiamo prestazioni molto elevate in quanto, una volta stabilita la connessione,
tutte le risorse sono allocate e non si subiscono interruzioni di alcun tipo dato che la fase 
iniziale ha assicurato che il percorso scelto fosse in grado di sostenere la connessione richiesta.

\subsection{Commutazione di pacchetto}
Questo paradigma per la commutazione cerca di risolvere i due problemi della commutazione di
circuito, minimizzando il tempo speso nella fase iniziale e migliorando la gestione delle risorse
occupate dalle varie connessioni.

Il metodo consiste nel dividere l'informazione in pacchetti che vengono inviati in rete e sono
totalmente indipendenti dagli altri (nella fase di trasmissione).
\begin{enumerate}
	\item L'informazione viene suddivisa in pacchetti, i quali potranno viaggiare in modo
		indipendente sulla rete, anche su percorsi diversi.
	\item Nel caso in cui la rete sia congestionata il router tiene i pacchetti in un
		\textbf{buffer} fin quando non ci sarà di nuovo la possibilità di inviarli.
	\item Una volta che la rete è disponibile all'inoltro del pacchetto, il router determina,
		tramite le informazioni del pacchetto quali sono i possibili nodi a cui inviare il
		pacchetto e valuta quale tra questi è l'opzione migliore basandosi su parametri come la 
		densità del traffico in quel momento ecc.
	\item Alla fine l'informazione dovrà essere ricomposta prima di essere letta dal destinatario
		in modo tale che sia significativa.
\end{enumerate}
Il principio che sta alla base della scelta del percorso di ogni pacchetto è l'impiego di risorse
a seconda della necessità.

Nel caso in cui i pacchetti siano tenuti nel buffer si introduce un \textbf{ritardo} nella
trasmissione. Ritardo che con la commutazione di circuito non avevamo dato che l'unico momento di
attesa era dovuto all'inizializzazione.

Teniamo anche di conto che i buffer hanno dimensione finita e quindi nel caso in cui la rete sia 
congestionata e il buffer sia pieno, si rischia di perdere uteriore pacchetti ricevuti.

Il vantaggio sta nella maggiore flessibilità. Se ad esempio uno dei dispositivi di interconnessione
si guasta, i pacchetti possono comunque continuare il loro percorso passando da altri nodi.

Il commutatore (router) deve ricevere l'intero pacchetto prima di poterlo inoltrare ad un altro
commutatore. Questo perché ha bisogno di informazioni per riuscire a determinare il canale d'uscita
di tale pacchetto (\textbf{store and forward}).

Ovviamente c'è la possibilità di creare \textbf{code di priorità} per ottimizzare e dare la
precedenza a tipologie di pacchetto che necessitano una trasmissione più rapida.

