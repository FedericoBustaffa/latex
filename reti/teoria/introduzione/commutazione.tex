\section{Commutazione}
Una internet è data dall'interconnessione di reti composte da link e dispositivi capaci di
scambiarsi informazioni. In particolare distinguiamo i dispositivi in host e dispositivi di
interconnessione.

Nel momento in cui vogliamo inviare un \textbf{pacchetto} si deve definire un percorso nella rete
lungo il quale questo viaggia per arrivare a destinazione. Le tecniche di \textbf{commutazione}
definiscono i metodi con il quale viene determinato tale percorso. In particolare andiamo a
distinguere due principali metodi di commutazione: commutazione di \textbf{pacchetto} e
commutazione di \textbf{circuito}.

In generale non esiste un solo percorso per connettere due dispositivi. Ecco che i due modelli di
commutazione entrano in gioco fornendo due possibili alternative nella scelta di tale percorso.
Vedremo che i due tipi di commutazione non forniscono solo un metodo per determinare un percorso
di rete ma influiscono anche sulla velocità di connessione tra dispositivi e sulla trasmissione
dell'informazione.

\subsection{Commutazione di circuito}
La commutazione di circuito instaura un cammino dedicato tra i due dispositivi che vogliono
comunicare tramite una serie di linee di trasmissione e dispositivi di commutazione che collegano
sorgente e destinazione.
\begin{enumerate}
	\item \textbf{Inizializzazione}: In genere si ha una fase iniziale in cui si richiede di
		aprire una connessione verso una determinata entità e una prima infrastruttura si occupa 
		di determinare una connessione dedicata tra mittente e destinatario. I dispositivi di 
		mezzo sono incaricati di gestire tale collegamento di modo che la connessione non si
		interrompa.
	\item \textbf{Risorse}: una volta determinato il percorso si dedicano le risorse necessarie 
		a tale connessione. Una volta allocate, tali risorse, sono disponibile per tutta la durata 
		della comunicazione.
\end{enumerate}
Gli svantaggi di questo tipo di commutazione sono principalmente due:
\begin{itemize}
	\item \`E necessaria una fase iniziale di inizializzazione prima di iniziare a comunicare.
	\item Le risorse dedicate ad una connessione rimangono inattive se non utilizzate.
\end{itemize}
D'altro canto abbiamo prestazioni molto elevate in quanto, una volta stabilita la connessione,
tutte le risorse sono allocate e non si subiscono interruzioni di alcun tipo dato che la fase 
iniziale ha assicurato che il percorso scelto fosse in grado di sostenere la connessione richiesta.

\subsection{Commutazione di pacchetto}
Questo paradigma per la commutazione cerca di risolvere i due problemi della commutazione di
circuito, ossia la minimizzazione del tempo speso nella fase iniziale e la minimizzazione della

