\section{Metriche}
Vogliamo ora capire come misurare le prestezioni della rete. Per farlo dobbiamo introdurre concetti
come
\begin{itemize}
	\item Ampiezza di banda e bitrate
	\item Throughput
	\item Latenza
	\item Perdita di pacchetti
\end{itemize}
Una volta chiariti certi concetti sarà anche più facile interpretare i risultati forniti dai famosi
\emph{speed test} che si trovano online.

\subsection{Larghezza di banda e bitrate}
Quando si parla di \textbf{larghezza di banda} si parla della larghezza dell'intervallo delle
frequenze utilizzato dal mezzo trasmissivo e si misura in Hertz (Hz).

Il \textbf{bitrate} è invece la velocità di trasmissione del mezzo trasmissivo e indica quanti bit
possono essere trasmessi nell'unità di tempo e si misura in bits per second (bps).

\subsection{Throughput}
Il \textbf{throughput} indica la quantità di dati che possono essere trasmessi da un nodo sorgente
a un nodo destinazione in un certo intervallo di tempo. Da non confondere con il bitrate in quanto
il throughput indica la velocità con cui trasferiamo i dati tenendo di conto di eventuali
variazioni del mezzo trasmissivo, duplicazioni, protocolli ecc.

Se per esempio fino a un certo nodo abbiamo un bitrate $R_1$ e da quel nodo in poi abbiamo un
bitrate $R_2$, il throughput sarà
\[ \text{throughput} = \min (R_1, R_2) \]
In generale quando ci sono più connessioni i cui host interagiscono l'uno con l'altro si ha spesso
una situazione in cui il throughput finale equivale alla capacità trasmissiva peggiore all'interno
della rete.

\subsection{Latenza e perdite}
Con \textbf{latenza} indichiamo il tempo richiesto affinché un messaggio arrivi a destinazione dal
momento in cui il primo bit parte dalla sorgente. La latenza si può ricavare tramite la seguente 
formula
\[ l = rp + rt + ra + re \]
dove gli elementi a secondo membro indicano rispettivamente i ritardi di propagazione, trasmissione
accodamento ed elaborazione.

\subsubsection{Accodamento}
Come abbiamo già visto, i pacchetti che non possono essere elaborati dal router vengono messi in
attesa in un buffer. Questo ovviamente genera ritardo, ossia il \textbf{ritardo di accodamento},
che equivale al tempo di permanenza del pacchetto nel buffer. Quando il buffer del router è pieno,
tutti i pacchetti in arrivo vengono scartati.

\subsubsection{Elaborazione}
Altro ritardo introdotto dalla commutazione di pacchetto è il \textbf{ritardo di elaborazione}
dovuto a controllo di errori sui bit e all'elaborazione di una scelta del canale d'uscita del
pacchetto.

\subsubsection{Trasmissione}
Il \textbf{ritardo di trasmissione} dipende dalla velocità di trasmissione del collegamento e dalla
lunghezza del pacchetto. Rappresenta il tempo impiegato per trasmettere il pacchetto.
\[ rt = \frac{l}{r} \]
dove $l$ è la lunghezza del pacchetto e $r$ è la velocità di trasmissione (rate) del mezzo
trasmissivo.

\subsubsection{Propagazione}
Tempo impiegato da un bit per essere propagato da un nodo all'altro. Esso dipende dalla lunghezza
del collegamento fisico e dalla velocità di propagazione nel mezzo.
\[ rp = \frac{d}{s} \]
dove $d$ è la lunghezza del mezzo trasmissivo e $s$ è la velocità di propagazione di tale mezzo.
