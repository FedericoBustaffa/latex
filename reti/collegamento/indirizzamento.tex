\section{Indirizzi a livello collegamento}
Un indirizzo a livello collegamento, detto \textbf{MAC}, è associato
alla scheda di rete dal fornitore e non al nodo, tipicamente 
permanente.

All'interno di una stessa rete locale l'indirizzo MAC di ogni 
dispositivo deve essere univoco.

L'indirizzo MAC ha una struttura piatta che identifica in modo univoco
una scheda di rete tramite un codice a 48 bit.

\subsection{Indirizzi LAN}
In una LAN un adattatore inserisce l'indirizzo MAC di destinazione per
inviare un frame. Ogni scheda controlla se l'indirizzo MAC corrisponde
al proprio e in caso negativo deve scartare il frame, in caso 
affermativo lo decapsula e passa al livello superiore.

\subsection{ARP}
All'accensione di una macchina, questa conosce il proprio indirizzo 
MAC, l'indirizzo IP (e la rete locale cui appartiene) e il suo 
indirizzo alfanumerico. Non conosce però le macchine che ha attorno.

Entra quindi in gioco il protocollo \textbf{ARP} (Address Resolution
Protocol), che si pone a livello di rete producendo un payload di
richiesta per la risoluzione di indirizzi MAC di cui non si è a 
conoscenza.

Ogni nodo della LAN mantiene una tabella ARP, gestita dinamicamente,
che contiene delle corrispondenze tra indirizzi IP e MAC delle
varie macchine presenti, e ci associa un TTL. Alla fine del TTL 
la entry nella tabella viene deallocata.

Se un dispositivo che conosce l'indirizzo IP di un altro dispositivo
vuole risalire al suo indirizzo MAC deve fare una richiesta nella LAN
in modo da risolvere l'indirizzo mediante la tabella ARP di qualche
altro dispositivo.

Tale richiesta viene fatta tramite un \textbf{pacchetto ARP} in cui
si specificano vari parametri come il proprio indirizzo IP e MAC,
l'indirizzo IP della macchina di cui si vuole ricavare il MAC ecc.

Questo pacchetto IP viene poi incapsulato in un frame verso
l'indirizzo MAC di broadcast dato che non sappiamo a chi inviarlo.
La risposta dal nodo in grado di effettuare la risoluzione 
dell'indirizzo avviene in unicast.
