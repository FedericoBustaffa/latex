\section{Protocolli MAC}
Nel caso di \textbf{broadcast} il collegamento è \emph{condiviso} e
questo genera \textbf{collisioni} quando avvengono due o più 
trasmissioni simultanee tra i nodi, creando \emph{interferenza}.

Più nello specifico, un nodo che sta in ascolto sul canale condiviso,
non riesce a distinguere messaggi in arrivo da un nodo piuttosto che
da un altro. \`E quindi necessario definire dei protocolli che 
gestiscano l'accesso multiplo alla risorsa.

Un tipo di soluzione \emph{ideale} prevede l'utilizzo di un canale 
broadcast con rate $R$ bps e la gestione di tale canale dovrebbe 
avvenire in questo modo:
\begin{enumerate}
	\item Quando un solo nodo trasmette può inviare dati con un rate 
		di $R$ bps.
	\item Quando $M$ nodi vogliono trasmettere, ciascuno trasmette ad
		un rate medio di $R / M$ bps.
	\item Deve essere \textbf{decentralizzato}, ossia non ci deve
		essere un nodo che coordina gli altri ma ogni nodo trasmette
		senza che ci siano fasi preliminari.
	\item Semplice.
\end{enumerate}
Nella pratica si è cercato di costruire un modello quanto più simile
a quello appena descritto. Si è dunque suddiviso i \textbf{protocolli
MAC} in tre classi principali:
\begin{itemize}
	\item \textbf{Suddivisione del canale}: si suddivide il canale
		in \emph{pezzi} più piccoli e la risorsa (che può essere 
		diversa da caso a caso) è allocata in modo esclusivo tra le
		entità trasmittenti.
	\item \textbf{Accesso random}: il canale è condiviso e possono 
		esserci collissioni. Esistono tuttavia del meccanismi per
		recuperare dati in seguito ad una collisione.
	\item \textbf{Rotazione}: i nodi possono trasmettere a rotazione.
\end{itemize}

\subsection{Suddivisione del canale}
Come vedremo tra poco parleremo di diversi protocolli che permettono
di \textbf{partizionare} il canale in base a diverse tipologia di
risorsa.

\subsubsection{TDMA}
Il protocollo \textbf{TDMA} (Time Division Multiple Access) fornisce
l'accesso al canale ad intervalli di tempo. Ciascuno di questi 
intervalli è suddiviso a sua volta in \textbf{slot}.

Ogni stazione ha a disposizione uno slot di lunghezza fissa in ogni
intervallo e quelli non utilizzati sono sprecati.

In questo caso non abbiamo collisioni dato che ogni stazione trasmette
quando le altre non possono e dunque la trasmissione avviene la massimo
delle capacità del canale.

\subsubsection{FDMA}
Il protocollo \textbf{FDMA} (Frequency Division Multiple Access)
suddivide il canale in base allo spettro delle frequenze utilizzate 
dalle varie stazioni.

Possiamo infatti suddividere il canale in più \textbf{bande} ed 
assegnare ad ogni stazione un banda per la trasmissione.

Se delle stazioni non trasmettono le bande non utilizzate vanno 
sprecate.

\subsection{Accesso random}
Passiamo ora ai protocolli ad accesso \textbf{random} al canale. In
questo caso, quando un nodo deve trasmettere, lo fa al massimo rate
del canale e non c'è un coordinamento a priori tra i nodi.

Se però due o più nodi trasmettono simultaneamente avviene una 
collisione. Un protocollo MAC ad accesso random specifica però come
\textbf{rilevare} le collisioni e come \textbf{recuperare} dopo una
collisione.	

\subsubsection{Slotted ALOHA}
Anche in questo caso abbiamo intervalli di tempo suddivisi in slot
uguali e fissi. I nodi iniziano a trasmettere all'inizio dello slot 
in modo sincronizzato.

Se due o più nodi trasmettono simultaneamente sono in grado di 
rilevare la collisione e gestirla.

Quando il nodo deve trasmettere un frame, comincia a trasmetterlo
all'inizio dello slot successivo:
\begin{itemize}
	\item Se non ci sono collisioni il nodo può inviare un nuovo
		frame nello slot successivo.
	\item Se ci sono collisioni il nodo rileva la connessione e 
		ritrasmette il frame con probabilità $p$ nello slot successivo
		finché non ha successo.
\end{itemize}
Lo slotted ALOHA è un'evoluzione dell'ALOHA, che invece non prevedeva
gli slot andando ad aumentare la probabilità di collisioni, anche 
parziali.

\subsubsection{CSMA}
Il protocollo CSMA (Carrier Sense Multiple Access) prevede una fase di
\emph{ascolto} preliminare del traffico sul canale di modo da 
implementare un politica più \emph{educata} di accesso alla risorsa.

In linea generale, attuando un ascolto sul canale, questo protocollo è
in grado di capire quando trasmettere e quando no. Se il canale risulta
occupato ritarda la trasmissione, se invece è libero trasmette il 
frame.

Questo non evita affatto il problema delle collisioni in quanto due
nodi potrebbero iniziare a parlare nello stesso momento avendo rilevato
il canale come libero.

Il meccanismo di rilevamento delle collisioni sta in ascolto sul
canale e interrompe subito la comunicazione non appena rileva una
collisione.

Il tempo che intercorre tra una ritrasmissione e l'altra dipende dal
numero di collisioni rilevate. Se vengono rilevate tante collisioni 
si sceglie un intervallo in maniera random alla fine del quale si può
ricominciare a trasmettere.

\subsection{Rotazione}
Come detto in precedenza, i protocolli a rotazione permettono ai nodi
di comunicare uno dopo l'altro.

\subsubsection{Polling}
Questo protocollo prevede la presenza di un nodo \textbf{master} che 
va a gestire gli altri nodi \textbf{slave}, \emph{invitandoli} ad uno
ad uno a comunicare.

Se il nodo non deve comunicare niente si passa la successivo altrimenti
comunica il messaggio entro il tempo che il master gli fornisce.

\subsubsection{Token passing}
Questo protocollo non prevede la presenza di un nodo master ma il 
funzionamento è comunque molto simile in quanto un nodo può comunicare
solo quando è in possesso di un messaggio particolare, ossia il 
\textbf{token}.

Una volta che comunica il messaggio invia il token ad un altro nodo
che a quel punto può comunicare.

