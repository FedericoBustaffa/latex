\section{Compilazione ed esecuzione}
Per compilare i sorgenti sarà sufficiente digitare il comando
\begin{center}
	\verb|$ javac -cp <libs> *.java|
\end{center}
dove al posto di \verb|libs| è necessario inserire una stringa (quindi delimitata da doppi apici)
contenente i percorsi dei file \verb|.jar| della libreria \verb|Jackson|. Nel mio caso sono tre:
\begin{itemize}
	\item \verb|jackson-core-2.9.7.jar|
	\item \verb|jackson-databind-2.9.7.jar|
	\item \verb|jackson-annotations-2.9.7.jar|
\end{itemize}
Per permettere al compilatore correttamente tutti i file \verb|jar| presenti nella stringa è
necessario separarli con dei doppi punti (\verb|:|).

Il risultato della compilazione sarà una serie di file \verb|.class| di cui due eseguibili:
\begin{itemize}
	\item \verb|ServerMain.class|
	\item \verb|ClientMain.class|
\end{itemize}
Per eseguire i due file sarà sufficiente invocare l'interprete Java passando come argomento (sia
per client che per server) il relativo file di configurazione. Per esempio se volessimo avviare il
server basterà digitare
\begin{center}
	\verb|$ java ServerMain server_config.json|
\end{center}