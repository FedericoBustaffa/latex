\section{Compilazione ed esecuzione}
Per compilare i sorgenti sarà necessario aggiungere al \verb|ClassPath| di java anche le classi
della libreria \verb|Jackson| utilizzate:
\begin{lstlisting}
$ javac -cp .:jackson-core-2.9.7.jar:jackson-databind-2.9.7.jar:jackson-annotations-2.9.7.jar *.java
\end{lstlisting}
Per permettere al compilatore di distinguere correttamente tutti i file \verb|jar| presenti nella
stringa è necessario separarli con dei doppi punti (\verb|:|).

Similmente alla compilazione, anche per l'esecuzione è necessario notificare all'interprete Java
che stiamo usando dei file \verb|.jar| della libreria \verb|Jackson|:
\begin{lstlisting}
$ java -cp .:jackson-core-2.9.7.jar:jackson-databind-2.9.7.jar:jackson-annotations-2.9.7.jar ServerMain server_config.json
\end{lstlisting}
In questo caso ho fatto l'esempio dell'esecuzione del \verb|Server| ma il discorso è analogo per il
client in quanto anch'esso utilizza componenti della libreria \verb|Jackson|.

Nel caso si vogliano eseguire direttamente i file \verb|.jar| basterà digitare il comando
\begin{lstlisting}
$ java -jar Server.jar server_config.json
\end{lstlisting}
Discorso analogo per il client, basterà solamente cambiare i parametri adeguatamente.