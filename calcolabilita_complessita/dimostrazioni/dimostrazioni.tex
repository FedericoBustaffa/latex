\documentclass[12pt]{article}

\usepackage[T1]{fontenc}
\usepackage[italian]{babel}
\usepackage[hidelinks]{hyperref}
\usepackage{tikz, pgfplots}
\usepackage{graphicx}
\usepackage{tabularx}
\usepackage[margin=1.25in]{geometry}
\usepackage{amsmath, amssymb, amsthm, amsfonts, mathtools}

\pgfplotsset{compat=newest}

\newcommand{\N}{\mathbb{N}}
\newcommand{\Z}{\mathbb{Z}}
\newcommand{\R}{\mathbb{R}}
\newcommand{\RR}{\mathcal{R}}
\newcommand{\C}{\mathcal{C}}
\newcommand{\F}{\mathbb{F}}
\newcommand{\K}{\mathcal{K}}
\newcommand{\B}{\mathcal{B}}

\newcommand{\tx}{\tilde{x}}
\newcommand{\norm}[1]{\left\lVert#1\right\rVert}
\newcommand{\start}{\triangleright}

\DeclareMathOperator{\sign}{sign}
\DeclareMathOperator{\trn}{trn}
\DeclareMathOperator{\arr}{arr}
\DeclareMathOperator{\fl}{fl}
\DeclareMathOperator{\dist}{dist}
\DeclareMathOperator{\Ker}{Ker}
\DeclareMathOperator{\diag}{diag}
\DeclareMathOperator{\nnz}{nnz}
\DeclareMathOperator{\Var}{Var}
\DeclareMathOperator{\dom}{dom}

\title{Dimostrazioni calcolabilità e complessità}
\author{}
\date{}

\begin{document}

\maketitle
\tableofcontents

\section{Teoremi principali}

% \subsection{Teorema di enumerazione}

\subsection{Teorema del parametro}

Esiste una funzione $s$ calcolabile totale e iniettiva tale che
$\forall i, x, y$ vale
\[ \lambda y . \varphi_i(x, y) = \varphi_{s(i,x)} (y) \]

\subsubsection{Dimostrazione}

La dimostrazione si basa sul costruire una funzione $s$
calcolabile, totale e iniettiva, tale per cui la funzione
$\varphi_i$ valutata su $x$ e $y$, nel caso in cui il valore
$x$ sia fissato (sia cioè un \textbf{parametro}), restituisca
lo stesso risultato della funzione indicizzata dal valore di
$s(i, x)$.

In altre parole, nel momento in cui ci troviamo una funzione
a 2 parametri e uno di questi sia fissato vogliamo trovare una
funzione che ci ritorna lo stesso risultato ma in cui il
parametro è parte della funzione stessa. Prendiamo l'esempio
della funzione somma
\[ f(x, y) = x + y \]
e supponiamo che $x$ sia sempre fissato a 2. Possiamo sia
calcolare $f(2, y)$ sia andare a cercare una seconda funzione
$g$ che prende un solo parametro $y$ ed è definita in questo
modo:
\[ g(y) = 2 + y \]
In sostanza stiamo definendo una funzione che prende un numero
ed è solo in grado di sommargli 2.

Per svolgere la dimostrazione replichiamo quindi quanto fatto
con l'esempio. I passi che seguono (da quanto ho capito) hanno
come intuizione la definizione di una nuova macchina
"modificata".

Dato che a primo membro abbiamo $\varphi_i (x,y)$ possiamo
seguire lo stesso schema di sempre e dire che la macchina $M_i$
calcola $\varphi_i$. Ma tale macchina prende ancora 2 argomenti
ignoti a priori. Noi ci stiamo ponendo nell'ipotesi in cui $x$
sia fissato a priori e ne conosciamo quindi il valore.

Quello che stiamo cercando di fare è recuperare una macchina
$M_j$, che prende un solo argomento $y$ e la quale (ad esempio)
non ha l'istruzione che legge la variabile $x$ ma $x$ è
direttamente scritta dentro la definizione della macchina $M_j$.
Per trovare tale indice $j$ possiamo
\begin{enumerate}
	\item Recuperare l'$i$-esima macchina $M_i$.
	\item Scrivere sul nastro di tale macchina il parametro $x$
	      e la variabile $y$.
\end{enumerate}
Da notare che $x$ è un parametro e dunque non abbiamo bisogno di
\emph{leggerlo}. Non abbiamo cioè bisogno di controllare quanto
cose come la sua lunghezza o il suo tipo ecc. Il parametro $x$
è dato e viene scritto direttamente sul nastro.

Questa procedura definisce un algoritmo, in particolare
l'algoritmo di indice $j$, il quale è identificato in funzione
di $i$ e $x$ in quanto ci serve $i$ per recuperare l'$i$-esima
macchina e ci serve $x$ il quale essendo noto viene scritto
immediatamente sul suo nastro. A questo punto è facile
convincersi che la procedura algoritmica che termina sempre e
quindi esiste $s$ calcolabile totale tale che
\[ j = s(i, x) \]
In particolare $s$ è
\begin{itemize}
	\item Calcolabile poiché, come abbiamo appena detto, è
	      possibile definire una procedura algoritmica che la
	      calcola e che termina sempre.
	\item Totale poiché è definita per ogni input. Non esistono
	      infatti un indice $i \in \N$ e un input $x$ tali per
	      cui la funzione $s$ non è definita. Siamo sempre in
	      grado di trovare l'$i$-esima macchina e scrivere $x$
	      sul suo nastro.
\end{itemize}
Ci rimane da dimostrare che $s$ è iniettiva. Per essere
iniettiva abbiamo bisogno che non si verifichi mai una
situazione di questo genere
\[ s(i, x) = s(i', x') \]
Per il padding lemma sappiamo però che esiste un'infinità
numerabile di algoritmi che calcolano la stessa funzione o, se
preferiamo, esiste un'infinità numerabile di indici che
identificano una macchina in gradi di calcolare la stessa
funzione. Premesso ciò è dunque possibile costruire una funzione
$s'$ tale che
\[ \varphi_{s(i, x)} = \varphi_{s'(i, x)} \]
che genera indici in modo strettamente crescente. Deve quindi
valere che
\[ s'(i_0, x_0) < s'(i_1, x_1) \]
dove la codifica della coppia $(i_0, x_0)$ è minore della
codifica della coppia $(i_1, x_1)$. Come codifica possiamo ad
esempio prendere la coda di colomba che mappa una coppia di
naturali in un un'unico naturale.
\subsection{Teorema di ricorsione (Kleene II)}

Per ogni funzione $f$ calcolabile totale esiste $n$ tale che
\[ \varphi_n = \varphi_{f(n)} \]

\subsubsection{Dimostrazione}

Dobbiamo quindi dimostrare che esiste questa $n$ per ogni $f$
calcolabile totale. Per dimostrare che esiste andiamo a definire
la funzione calcolabile
\[
	\psi (x, y) = \begin{cases}
		\varphi_{\varphi_x(x)} (y) & \text{se } \varphi_x (x) \downarrow \\
		\text{indefinita}          & \text{altrimenti}
	\end{cases}
\]
Tale funzione è calcolabile poiché per calcolarla prendiamo
$M_x$ e gli diamo $x$ in input
\begin{itemize}
	\item Se termina siamo nel primo ramo e soprattutto
	      otteniamo un indice $\varphi_x(x)$ che è l'indice di
	      $\varphi_{\varphi_x(x)} (y)$.
	\item Se non termina siamo nel secondo ramo e quindi non
	      otteniamo un indice.
\end{itemize}
La funzione è inoltre totale poiché è ovunque definita (il
calcolo di $\varphi_x (x)$ o termina o non termina).

Se $\psi$ è calcolabile, per Church-Turing ha un indice $i$
e dunque vale che
\[ \psi (x, y) = \varphi_i (x, y) \]
A questo punto possiamo applicare il teorema del parametro e
ottenere
\[ \psi (x, y) = \varphi_i (x, y) = \varphi_{s(i, x)} (y) \]
Ma la $\psi$ è fissata e dunque anche l'indice $i$ lo è. Possiamo
allora scrivere che
\[
	\psi (x, y) = \varphi_i (x, y) = \varphi_{s(i, x)} (y) =
	\varphi_{d(x)} (y)
\]
Abbiamo quindi che se $\varphi_x (x) \downarrow$ vale che
\[
	\psi (x, y) = \varphi_i (x, y) = \varphi_{s(i, x)} (y) =
	\varphi_{d(x)} (y) = \varphi_{\varphi_x(x)} (y)
\]
Per il teorema del parametro sappiamo che $d$ è calcolabile,
totale e iniettiva, e come possiamo vedere, non dipende da $f$
poiché $f$ non compare ancora da nessuna parte. Noi però siamo
ora in possesso di questo oggetto
\[ \varphi_{d(x)} \]
e vogliamo metterlo in correlazione con ciò che enuncia il
teorema. In particolare supponiamo che $d(x) = n$ e quindi
per dimostrare la tesi
\[ \varphi_n = \varphi_{f(n)} \]
vogliamo porre in qualche modo
\[ \varphi_{d(x)} = \varphi_{f(d(x))} \]
Per il momento non l'abbiamo ancora fatto, questo non è un vero
passaggio. \`E solo per mostrare l'intuizione. Notiamo ora che,
dato che $d$ ed $f$ sono entrambe calcolabili e totali, anche la
loro composizione (chiamiamola $g$) lo è e dunque ha un indice.
Vale quindi che
\[ f(d(x)) = g(x) = \varphi_j (x) \]
Tale funzione è totale perché sia $f$ che $d$ lo sono e quindi
\[ \varphi_j (j) \downarrow \]
Ma se questo converge abbiamo che
\[ \varphi_{d(j)} = \varphi_{\varphi_j(j)} \]
Ora calcoliamo $d(j)$ il quale ci darà un risultato $n$,
otteniamo quindi che
\[ \varphi_{d(j)} = \varphi_n \]
ma come appena detto vale che
\[ \varphi_{\varphi_j(j)} = \varphi_{d(j)} = \varphi_n \]
A questo punto usiamo il fatto che
\[ f(d(x)) = \varphi_j (x) \]
per dedurre che
\[
	\varphi_{f(d(j))} = \varphi_{\varphi_j(j)} =
	\varphi_{d(j)} = \varphi_n
\]
Abbiamo però detto che $d(j) = n$ e quindi vale
\[
	\varphi_{f(n)} = \varphi_{f(d(j))} =
	\varphi_{\varphi_j(j)} = \varphi_{d(j)} = \varphi_n
\]

% da cui si deduce che
% \[ d(x) = \varphi_x(x) \]

\section{Riduzioni}

\subsection{K e insiemi non ricorsivi}

Quando ci viene chiesto di dimostrare che un insieme $A$ è non
ricorsivo possiamo ridurre
\[ K = \{ x \mid \varphi_x (x) \downarrow \} \]
a tale insieme secondo
\[ rec = \{ \varphi_x \mid \forall y \in \N . \varphi_x(y) \downarrow \} \]
\subsection{K si riduce secondo rec a CONST}

Per dimostrare che $K \leq_{rec} CONST$ ricordiamoci prima come
sono definiti i vari insiemi

\begin{gather*}
	K = \{ x \mid \varphi_x (x) \downarrow \} \\
	CONST = \{ x \mid \varphi \text{ è totale e costante} \} \\
	rec = \{ \varphi_x \mid \forall y \in \N . \varphi_x (y) \downarrow \}
\end{gather*}

Dire quindi che $K \leq_{rec} CONST$ significa che esiste una funzione
$f \in rec$ tale che
\[ \forall x \in K \implies f(x) \in CONST \]
Dire che $x \in K$ equivale a dire che $\varphi_x (x)$ converge,
dire invece che $f(x) \in CONST$ equivale a dire che la funzione
$\varphi_{f(x)}$ è totale e costante. L'obbiettivo della
dimostrazione è quindi quello di trovare la $f$ tale per cui sia
vera quest'ultima cosa. Iniziamo con il definire la funzione
\[
	\psi (x, y) = \begin{cases}
		1                 & \text{se } \exists z > y \mid
		\varphi_x(x) \downarrow \text{ in meno di $z$ passi} \\
		\text{indefinita} & \text{altrimenti}
	\end{cases}
\]
Ci chiediamo ora se $\psi$ è calcolabile. Intuitivamente possiamo
prendere l'$x$-esima macchina $M_x$ ed effettuare al più $z$
passi nel calcolo di $M_x(x)$.
\begin{itemize}
	\item Se converge in meno di $z$ passi allora cadiamo nel
	      primo ramo e $\psi(x, y) = 1$.
	\item Se invece non converge in meno di $z$ passi allora
	      cadiamo nel secondo ramo e $\psi(x,y)$ è indefinita.
\end{itemize}
Abbiamo quindi trovato una procedura che calcola $\psi$ che
termina sempre e dunque la funzione è calcolabile. Dato che la
funzione è calcolabile allora ha un indice $i$ e possiamo quindi
scrivere
\[ \psi(x, y) = \varphi_i (x, y) \]
A questo punto possiamo applicare il teorema del parametro per
ottenere
\[ \psi(x, y) = \varphi_i (x, y) = \varphi_{s(i,x)} (y) \]
Se notiamo che $i$ è costante (è fissato perché la funzione
$\psi$ è fissata), possiamo scrivere
\[
	\psi(x, y) = \varphi_i (x, y) =
	\varphi_{s(i,x)} (y) = \varphi_{f(x)} (y)
\]
Ecco che abbiamo ritrovato la stessa struttura che avevamo messo
in evidenza all'inizio. Avevamo detto che se $x \in K$ allora
$f(x) \in CONST$. Dire che $f(x) \in CONST$ equivale a dire che
$f(x)$ è l'indice di una funzione ($\varphi_{f(x)}$) totale e
costante.

Vogliamo quindi dimostrare che se $x \in K$, allora $f(x)$ è
l'indice di una funzione totale e costante. Vediamo quindi che
succede se $x \in K$
\[ x \in K \implies \varphi_x (x) \downarrow \]
e quindi
\[ \psi (x, y) = \varphi_{f(x)} (y) = 1 \]
e quindi come possiamo vedere $f(x)$ è l'indice di una funzione
$\varphi_{f(x)}$ che è totale e costante (in quanto uguale a
$1$). Possiamo quindi concludere che $f(x) \in CONST$. Per
terminare dobbiamo dimostrare che
\[ x \notin K \implies f(x) \notin CONST \]
Seguiamo la stessa catena di implicazioni:
\[ x \notin K \implies \varphi_x (x) \uparrow \]
quindi sicuramente supera gli $z$ passi di limite che avevamo
definito e dunque
\[ \psi (x, y) = \varphi_{f(x)} (y) = \text{indefinita} \]
E dunque in questo caso $f(x)$ è l'indice di una funzione
indefinita e dunque non totale (requisito necessario) affinché
$f(x) \in CONST$. Concludiamo quindi che
\[ x \notin K \implies f(x) \notin CONST \]

\subsection{INF si riduce secondo rec a CONST}

Per dimostrare che $INF \leq_{rec} CONST$ ricordiamoci prima
come sono definiti i vari insiemi

\begin{gather*}
	INF = \{ x \mid \dom (\varphi_x) \text{ è infinito} \} \\
	CONST = \{ x \mid \varphi \text{ è totale e costante} \} \\
	rec = \{ \varphi_x \mid \forall y \in \N . \varphi_x (y) \downarrow \}
\end{gather*}

Dire quindi che $INF \leq_{rec} CONST$ significa che esiste una funzione
$f \in rec$ tale che
\[ \forall x \in INF \implies f(x) \in CONST \]
Dire che $x \in INF$ equivale a dire che $\dom (\varphi_x)$ è infinito,
dire invece che $f(x) \in CONST$ equivale a dire che la funzione
$\varphi_{f(x)}$ è totale e costante. L'obbiettivo della
dimostrazione è quindi quello di trovare la $f$ tale per cui sia
vera quest'ultima cosa. Iniziamo con il definire la funzione
\[
	\psi (x, y) = \begin{cases}
		1                 & \text{se } \exists z > y .
		\varphi_x(z) \downarrow                        \\
		\text{indefinita} & \text{altrimenti}
	\end{cases}
\]
In queste dimostrazioni vogliamo sempre creare una situazione
in cui se $x \in A$ (in questo caso se $x \in INF$) allora
prendiamo il primo ramo. Dire che $x \in INF$ equivale a dire
che $\dom (\varphi_x)$ è infinito. Avere il dominio infinito
non significa che la funzione è definita per ogni input ma
significa che riesco sempre a trovare un certo $z$ per cui
la funzione converge. In particolare, preso un qualsiasi $y$
in input, è sempre possibile trovare un valore $z > y$ tale che
$\varphi_x (z) \downarrow$. Ci chiediamo ora se $\psi$ è
calcolabile. Intuitivamente possiamo prendere l'$x$-esima
macchina $M_x$ e calcolare $M_x (z)$.
\begin{itemize}
	\item Se converge allora cadiamo nel primo ramo e
	      $\psi(x, y) = 1$.
	\item Se diverge cadiamo nel secondo ramo e
	      $\psi(x,y)$ è indefinita.
\end{itemize}
Abbiamo quindi trovato una procedura che calcola $\psi$ che
termina sempre e dunque la funzione è calcolabile. Dato che la
funzione è calcolabile allora (per Church-Turing) ha un indice
$i$ e possiamo quindi scrivere
\[ \psi(x, y) = \varphi_i (x, y) \]
A questo punto possiamo applicare il teorema del parametro per
ottenere
\[ \psi(x, y) = \varphi_i (x, y) = \varphi_{s(i,x)} (y) \]
Se notiamo che $i$ è costante (è fissato perché la funzione
$\psi$ è fissata), possiamo scrivere
\[
	\psi(x, y) = \varphi_i (x, y) =
	\varphi_{s(i,x)} (y) = \varphi_{f(x)} (y)
\]
Ecco che abbiamo ritrovato la stessa struttura che avevamo messo
in evidenza all'inizio. Avevamo detto che se $x \in INF$ allora
$f(x) \in CONST$. Dire che $f(x) \in CONST$ equivale a dire che
$f(x)$ è l'indice di una funzione ($\varphi_{f(x)}$) totale e
costante.

Vogliamo quindi dimostrare che se $x \in INF$, allora $f(x)$ è
l'indice di una funzione totale e costante. Vediamo quindi che
succede se $x \in INF$
\[ x \in INF \implies \exists z > y \mid \varphi_x (z) \downarrow \]
e quindi
\[ \psi (x, y) = \varphi_{f(x)} (y) = 1 \]
e quindi come possiamo vedere $f(x)$ è l'indice di una funzione
$\varphi_{f(x)}$ che è totale e costante (in quanto uguale a $1$).
Possiamo quindi concludere che $f(x) \in CONST$. Per terminare
dobbiamo dimostrare che
\[ x \notin INF \implies f(x) \notin CONST \]
Seguiamo la stessa catena di implicazioni:
\[ x \notin INF \implies \forall z > y . \varphi_x (z) \uparrow \]
perché se $x \notin INF$ il dominio di $\varphi_x$ non è infinito
e dunque esiste un $y$ tale per cui, per ogni $z > y$, la funzione
$\varphi_x(z)$ diverge. Ecco che ci troviamo nel secondo caso
\[ \psi (x, y) = \varphi_{f(x)} (y) = \text{indefinita} \]
E dunque in questo caso $f(x)$ è l'indice di una funzione
indefinita e dunque non totale (requisito necessario) affinché
$f(x) \in CONST$. Concludiamo quindi che
\[ x \notin K \implies f(x) \notin CONST \]
\subsection{TOT si riduce secondo rec a INF}

Per dimostrare che $TOT \leq_{rec} INF$ ricordiamoci prima
come sono definiti i vari insiemi

\begin{gather*}
	TOT = \{ x \mid \dom (\varphi_x) = \N \} \\
	INF = \{ x \mid \dom (\varphi_x) \text{ è infinito} \} \\
	rec = \{ \varphi_x \mid \forall y \in \N . \varphi_x (y) \downarrow \}
\end{gather*}

Dire quindi che $TOT \leq_{rec} INF$ significa che esiste una funzione
$f \in rec$ tale che
\[ \forall x \in TOT \implies f(x) \in INF \]
Dire che $x \in TOT$ equivale a dire che $\dom (\varphi_x)$ è l'insieme
di tutti i naturali, dire invece che $f(x) \in INF$ equivale a dire che
il dominio della funzione $\varphi_{f(x)}$ è infinito. L'obbiettivo della
dimostrazione è quindi quello di trovare la $f$ tale per cui sia vera
quest'ultima cosa. Iniziamo con il definire la funzione
\[
	\psi (x, y) = \begin{cases}
		1                 & \text{se } \forall z < y .
		\varphi_x(z) \downarrow                        \\
		\text{indefinita} & \text{altrimenti}
	\end{cases}
\]
Come prima vogliamo trovarci nella situazione in cui se $x \in TOT$ allora
scegliamo il primo ramo. Ma dire che $x \in TOT$ equivale a dire che il
dominio della funzione $\varphi_x$ è tutto $\N$, ossia che $\varphi_x$ è
totale. Se la funzione è totale vale che per un qualsiasi valore di $y$ e
per ogni $z < y$ vale che $\varphi_x(z)$ converge. In sostanza stiamo dicendo
che preso $y$ tutti i numeri tra $0$ e $y$ possono essere dati in pasto
a $\varphi_x$ e otterremo sempre un risultato. Questo può essere fatto per
un qualsiasi $y$ e dunque la funzione è totale.

Ci chiediamo ora se $\psi$ è calcolabile. Intuitivamente possiamo
prendere l'$x$-esima macchina $M_x$ e calcolare $M_x (z)$ per ogni
$z < y$.
\begin{itemize}
	\item Se converge ogni volta allora cadiamo nel primo ramo e
	      $\psi(x, y) = 1$.
	\item Se diverge anche solo una volta cadiamo nel secondo ramo e
	      $\psi(x,y)$ è indefinita.
\end{itemize}
Abbiamo quindi trovato una procedura che calcola $\psi$ che
termina sempre e dunque la funzione è calcolabile. Dato che la
funzione è calcolabile allora (per Church-Turing) ha un indice
$i$ e possiamo quindi scrivere
\[ \psi(x, y) = \varphi_i (x, y) \]
A questo punto possiamo applicare il teorema del parametro per
ottenere
\[ \psi(x, y) = \varphi_i (x, y) = \varphi_{s(i,x)} (y) \]
Se notiamo che $i$ è costante (è fissato perché la funzione
$\psi$ è fissata), possiamo scrivere
\[
	\psi(x, y) = \varphi_i (x, y) =
	\varphi_{s(i,x)} (y) = \varphi_{f(x)} (y)
\]
Ecco che abbiamo ritrovato la stessa struttura che avevamo messo
in evidenza all'inizio. Avevamo detto che se $x \in TOT$ allora
$f(x) \in INF$. Dire che $f(x) \in INF$ equivale a dire che
$f(x)$ è l'indice di una funzione ($\varphi_{f(x)}$) il cui dominio
è infinito.

Vogliamo quindi dimostrare che se $x \in TOT$, allora $f(x)$ è
l'indice di una funzione con dominio infinito. Vediamo quindi che
succede se $x \in TOT$
\[ x \in TOT \implies \forall z < y . \varphi_x (z) \downarrow \]
e quindi
\[ \psi (x, y) = \varphi_{f(x)} (y) = 1 \]
e quindi come possiamo vedere $f(x)$ è l'indice di una funzione
$\varphi_{f(x)}$ il cui dominio è infinito in quanto per ogni $y$
vale che $\varphi_x$ è una funzione totale e dunque cadiamo sempre
nel primo caso. Notiamo come in questo caso il dominio della funzione
$\psi = \varphi_{f(x)}$ non solo è infinito ma $\N$ stesso. Possiamo
quindi concludere che $f(x) \in INF$. Per terminare dobbiamo dimostrare
che
\[ x \notin TOT \implies f(x) \notin INF \]
Seguiamo la stessa catena di implicazioni:
\[ x \notin TOT \implies \exists z < y . \varphi_x (z) \uparrow \]
e dunque
\[ \psi (x, y) = \varphi_{f(x)} (y) = \text{indefinita} \]
E dunque in questo caso $f(x)$ è l'indice di una funzione
indefinita e dunque non totale (requisito necessario) affinché
$f(x) \in INF$. Concludiamo quindi che
\[ x \notin TOT \implies f(x) \notin INF \]

\end{document}