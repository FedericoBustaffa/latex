\subsection{Teorema di ricorsione (Kleene II)}

Per ogni funzione $f$ calcolabile totale esiste $n$ tale che
\[ \varphi_n = \varphi_{f(n)} \]

\subsubsection{Dimostrazione}

Dobbiamo quindi dimostrare che esiste questa $n$ per ogni $f$
calcolabile totale. Per dimostrare che esiste andiamo a definire
la funzione calcolabile
\[
	\psi (x, y) = \begin{cases}
		\varphi_{\varphi_x(x)} (y) & \text{se } \varphi_x (x) \downarrow \\
		\text{indefinita}          & \text{altrimenti}
	\end{cases}
\]
Tale funzione è calcolabile poiché per calcolarla prendiamo
$M_x$ e gli diamo $x$ in input
\begin{itemize}
	\item Se termina siamo nel primo ramo e soprattutto
	      otteniamo un indice $\varphi_x(x)$ che è l'indice di
	      $\varphi_{\varphi_x(x)} (y)$.
	\item Se non termina siamo nel secondo ramo e quindi non
	      otteniamo un indice.
\end{itemize}
La funzione è inoltre totale poiché è ovunque definita (il
calcolo di $\varphi_x (x)$ o termina o non termina).

Se $\psi$ è calcolabile, per Church-Turing ha un indice $i$
e dunque vale che
\[ \psi (x, y) = \varphi_i (x, y) \]
A questo punto possiamo applicare il teorema del parametro e
ottenere
\[ \psi (x, y) = \varphi_i (x, y) = \varphi_{s(i, x)} (y) \]
Ma la $\psi$ è fissata e dunque anche l'indice $i$ lo è. Possiamo
allora scrivere che
\[
	\psi (x, y) = \varphi_i (x, y) = \varphi_{s(i, x)} (y) =
	\varphi_{d(x)} (y)
\]
Abbiamo quindi che se $\varphi_x (x) \downarrow$ vale che
\[
	\psi (x, y) = \varphi_i (x, y) = \varphi_{s(i, x)} (y) =
	\varphi_{d(x)} (y) = \varphi_{\varphi_x(x)} (y)
\]
Per il teorema del parametro sappiamo che $d$ è calcolabile,
totale e iniettiva, e come possiamo vedere, non dipende da $f$
poiché $f$ non compare ancora da nessuna parte. Noi però siamo
ora in possesso di questo oggetto
\[ \varphi_{d(x)} \]
e vogliamo metterlo in correlazione con ciò che enuncia il
teorema. In particolare supponiamo che $d(x) = n$ e quindi
per dimostrare la tesi
\[ \varphi_n = \varphi_{f(n)} \]
vogliamo porre in qualche modo
\[ \varphi_{d(x)} = \varphi_{f(d(x))} \]
Per il momento non l'abbiamo ancora fatto, questo non è un vero
passaggio. \`E solo per mostrare l'intuizione. Notiamo ora che,
dato che $d$ ed $f$ sono entrambe calcolabili e totali, anche la
loro composizione (chiamiamola $g$) lo è e dunque ha un indice.
Vale quindi che
\[ f(d(x)) = g(x) = \varphi_j (x) \]
Tale funzione è totale perché sia $f$ che $d$ lo sono e quindi
\[ \varphi_j (j) \downarrow \]
Ma se questo converge abbiamo che
\[ \varphi_{d(j)} = \varphi_{\varphi_j(j)} \]
Ora calcoliamo $d(j)$ il quale ci darà un risultato $n$,
otteniamo quindi che
\[ \varphi_{d(j)} = \varphi_n \]
ma come appena detto vale che
\[ \varphi_{\varphi_j(j)} = \varphi_{d(j)} = \varphi_n \]
A questo punto usiamo il fatto che
\[ f(d(x)) = \varphi_j (x) \]
per dedurre che
\[
	\varphi_{f(d(j))} = \varphi_{\varphi_j(j)} =
	\varphi_{d(j)} = \varphi_n
\]
Abbiamo però detto che $d(j) = n$ e quindi vale
\[
	\varphi_{f(n)} = \varphi_{f(d(j))} =
	\varphi_{\varphi_j(j)} = \varphi_{d(j)} = \varphi_n
\]

% da cui si deduce che
% \[ d(x) = \varphi_x(x) \]