\subsection{INF si riduce secondo rec a CONST}

Per dimostrare che $INF \leq_{rec} CONST$ ricordiamoci prima
come sono definiti i vari insiemi

\begin{gather*}
	INF = \{ x \mid \dom (\varphi_x) \text{ è infinito} \} \\
	CONST = \{ x \mid \varphi \text{ è totale e costante} \} \\
	rec = \{ \varphi_x \mid \forall y \in \N . \varphi_x (y) \downarrow \}
\end{gather*}

Dire quindi che $INF \leq_{rec} CONST$ significa che esiste una funzione
$f \in rec$ tale che
\[ \forall x \in INF \implies f(x) \in CONST \]
Dire che $x \in INF$ equivale a dire che $\dom (\varphi_x)$ è infinito,
dire invece che $f(x) \in CONST$ equivale a dire che la funzione
$\varphi_{f(x)}$ è totale e costante. L'obbiettivo della
dimostrazione è quindi quello di trovare la $f$ tale per cui sia
vera quest'ultima cosa. Iniziamo con il definire la funzione
\[
	\psi (x, y) = \begin{cases}
		1                 & \text{se } \exists z > y .
		\varphi_x(z) \downarrow                        \\
		\text{indefinita} & \text{altrimenti}
	\end{cases}
\]
In queste dimostrazioni vogliamo sempre creare una situazione
in cui se $x \in A$ (in questo caso se $x \in INF$) allora
prendiamo il primo ramo. Dire che $x \in INF$ equivale a dire
che $\dom (\varphi_x)$ è infinito. Avere il dominio infinito
non significa che la funzione è definita per ogni input ma
significa che riesco sempre a trovare un certo $z$ per cui
la funzione converge. In particolare, preso un qualsiasi $y$
in input, è sempre possibile trovare un valore $z > y$ tale che
$\varphi_x (z) \downarrow$. Ci chiediamo ora se $\psi$ è
calcolabile. Intuitivamente possiamo prendere l'$x$-esima
macchina $M_x$ e calcolare $M_x (z)$.
\begin{itemize}
	\item Se converge allora cadiamo nel primo ramo e
	      $\psi(x, y) = 1$.
	\item Se diverge cadiamo nel secondo ramo e
	      $\psi(x,y)$ è indefinita.
\end{itemize}
Abbiamo quindi trovato una procedura che calcola $\psi$ che
termina sempre e dunque la funzione è calcolabile. Dato che la
funzione è calcolabile allora (per Church-Turing) ha un indice
$i$ e possiamo quindi scrivere
\[ \psi(x, y) = \varphi_i (x, y) \]
A questo punto possiamo applicare il teorema del parametro per
ottenere
\[ \psi(x, y) = \varphi_i (x, y) = \varphi_{s(i,x)} (y) \]
Se notiamo che $i$ è costante (è fissato perché la funzione
$\psi$ è fissata), possiamo scrivere
\[
	\psi(x, y) = \varphi_i (x, y) =
	\varphi_{s(i,x)} (y) = \varphi_{f(x)} (y)
\]
Ecco che abbiamo ritrovato la stessa struttura che avevamo messo
in evidenza all'inizio. Avevamo detto che se $x \in INF$ allora
$f(x) \in CONST$. Dire che $f(x) \in CONST$ equivale a dire che
$f(x)$ è l'indice di una funzione ($\varphi_{f(x)}$) totale e
costante.

Vogliamo quindi dimostrare che se $x \in INF$, allora $f(x)$ è
l'indice di una funzione totale e costante. Vediamo quindi che
succede se $x \in INF$
\[ x \in INF \implies \exists z > y \mid \varphi_x (z) \downarrow \]
e quindi
\[ \psi (x, y) = \varphi_{f(x)} (y) = 1 \]
e quindi come possiamo vedere $f(x)$ è l'indice di una funzione
$\varphi_{f(x)}$ che è totale e costante (in quanto uguale a $1$).
Possiamo quindi concludere che $f(x) \in CONST$. Per terminare
dobbiamo dimostrare che
\[ x \notin INF \implies f(x) \notin CONST \]
Seguiamo la stessa catena di implicazioni:
\[ x \notin INF \implies \forall z > y . \varphi_x (z) \uparrow \]
perché se $x \notin INF$ il dominio di $\varphi_x$ non è infinito
e dunque esiste un $y$ tale per cui, per ogni $z > y$, la funzione
$\varphi_x(z)$ diverge. Ecco che ci troviamo nel secondo caso
\[ \psi (x, y) = \varphi_{f(x)} (y) = \text{indefinita} \]
E dunque in questo caso $f(x)$ è l'indice di una funzione
indefinita e dunque non totale (requisito necessario) affinché
$f(x) \in CONST$. Concludiamo quindi che
\[ x \notin K \implies f(x) \notin CONST \]