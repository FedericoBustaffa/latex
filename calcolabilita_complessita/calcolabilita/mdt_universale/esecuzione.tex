\section{Esecuzione}
Passiamo ora a descrivere come avviene l'esecuzione di $U$ e
cosa ci aspettiamo che succeda. Definiamo la macchina $U$ come
\[ U = (Q_U, \Sigma_U, \delta_U, s_U) \]
e diciamo che per ogni generica $M$ e per ogni dati in ingresso
$w \in \Sigma^*$ ci aspettiamo che
\begin{itemize}
	\item Se $M$ termina su $w$ e da come risultato $uav$
	      \[
		      (s, \underline{\start}, w) \rightarrow^*_0
		      (h, u \underline{a} v)
	      \]
	      allora $U$ termina su $\rho(M) \tau(w)$ e il nastro è
	      la codifica di $uav$
	      \[
		      (s_U, \underline{\start} \rho(M) \tau(w))
		      \rightarrow^*_U (h, \tau(uav) \underline{\#})
	      \]
	\item Viceversa se $U$ termina su $\rho(M) \tau(w)$ e da come
	      risultato $u'a'v'$
	      \[
		      (s_U, \underline{\start} \rho(M) \tau(w))
		      \rightarrow^*_U (h, u' \underline{a'}v')
	      \]
	      allora $a' = \#$ è il bianco, $v' = \epsilon$ è la
	      marca di fine stringa e $u' = \tau(uav)$ è la codifica
	      di $uav$ per quale $u$, $a$ e $v$ tali che $M$ termina
	      su $w$ e da come risultato $uav$
	      \[
		      (s, \underline{\start}, w) \rightarrow^*_0
		      (h, u \underline{a} v)
	      \]
\end{itemize}
Nella nostra MdT universale abbiamo che nel primo nastro c'è la
codifica $\tau(w)$ di $w$, nel secondo nastro la codifica di
$\rho(M)$ di $M$ e nel terzo c'è la codifica dello stato in cui
si trova $M$.

Ovviamente, prima di iniziare il calcolo, la MdT universale deve
attraversare una fase di inizializzazione in cui scrive sui vari
nastri le codifiche di $M$, $w$ ed $s$.

Dopo la fase di inizializzazione passiamo a descrivere coem
avviene la fase di calcolo. Dato che abbiamo tre nastri, per
simplicità chiameremo cursore $i$ (con $i = \{ 1, 2, 3\}$) il
cursore sull'$i$-esimo nastro. Iniziamo con il calcolo:
\begin{enumerate}
	\item Se il terzo nastro (quello con lo stato) contiene la
	      codifica di $q_{h_i}$, allora spostiamo il cursore $2$
	      sul primo carattere della stringa che codifica $M$.
	\item Se il cursore $1$ è all'inizio di
	      $\kappa(\sigma_{k_j})$, allora spostiamo il cursore
	      $2$ sulla "quitupla" $S_{i,j}$.
	\item Se $S_{i,j} = c \; \mid^p \; c \; \mid^q \; c \;
		      \mid^{p'} \; c \; \mid^{q'} \; c \; \mid^r \; c$
	      \begin{itemize}
		      \item Si scrive sul terzo nastro $\mid^{p'}$ al
		            posto di $\kappa(q_{h_i})$.
		      \item Si scrive $\mid^{q'}$ sul primo nastro al
		            posto di $\kappa(\sigma_{k_j})$,
		            eventualmente spostando a destra o a
		            sinistra la porzione di nastro a destra di
		            $\kappa(\sigma_{k_j})$, a seconda che la
		            lunghezza sia minore o maggiore di quella di
		            $\mid^{q'}$.
		      \item Si sposta il cursore $1$ al precedente o
		            al successivo gruppo $c \; \mid^v \; c$,
		            oppure lo si lascia dov'è a seconda del
		            valore di $r$.
	      \end{itemize}
	\item Se $S_{i,j} = c \; \mid^p \; c \; \mid^q \;
		      c \; d \; c \; d \; c \; d \; c$
	      abbiamo una condizione di terminazione con
	      \textbf{errore}.
	\item Se invece $S_{i,j} = c \; \mid^p \; c
		      \; \mid^q \; c \; \mid \; c \; \mid^{p'} \; c
		      \; \mid^j \; c$ abbiamo una condizione di
	      terminazione con \textbf{successo}.
\end{enumerate}
Se non fosse chiaro, un paragone con il comportamento di un
classico interprete potrebbe chiarire alcuni concetti:
\begin{itemize}
	\item I primi due passi corrispondono al \verb|fetch|
	      dell'istruzione. Attraverso il contenuto del terzo
	      nastro e mediante il simbolo corrente si realizza
	      il \verb|program counter|.
	\item L'ultimo passo incorpora la decodifica dell'istruzione
	      corrente e l'esecuzione vera e propria di quest'utima,
	      con relativa memorizzazione del risultato. Questo se
	      l'istruzione decodificata non è un errore o
	      l'istruzione di fine esecuzione.
\end{itemize}
Questa è una delle infinite possibili implementazioni di una
MdT universale, la quale può anche essere riscritta a singolo
nastro o a più di tre nastri. Come vedremo nella parte di
complessità il numero di nastri non aumenta in alcun modo la
capacità espressiva della macchina anche se (almeno in teoria)
è possibile implementare del \emph{"parallelismo"}.

Concludiamo riprendendo quanto detto all'inizio sulla differenza
tra semantica e sintassi. La semantica è, in generale, la
funzione che vogliamo calcolare. La sintassi è invece come la
calcoliamo, o per meglio dire, l'algoritmo che la calcola.
Possiamo quindi avere diverse sintassi che fanno riferimento
alla stessa semantica.

Nel caso di una MdT universale le due cose sembrano mischiarsi
ma non è così, la funzione rimane la stessa, semplicemente la
sintassi della macchina $M$ viene convertita in quella della
macchina $U$.