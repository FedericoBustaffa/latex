\chapter{Costruzione di una MdT universale}
Proviamo ora a costruire una MdT universale a tre nastri per
riuscire ad avere una visione più concreta di ciò di cui stiamo
parlando. Lo facciamo anche per riuscire ad apprezzare meglio la
differenza tra \emph{sintassi} e \emph{semantica}. Tale
costruzione ci tornerà utile più avanti per lo studio della
complessità e delle MdT non deterministiche.

Se ci pensiamo un attimo, un qualsiasi calcolatore fisico è una
macchina di turing universale, in quanto prende i nostri
programmi e li esegue ritornando lo stesso risultato che
darebbero se fossero eseguiti a mano da un operatore umano.

Come abbiamo appena visto, i teoremi di
\hyperref[th: fn]{forma normale} e di
\hyperref[th: enum]{teorema di enumerazione} ci forniscono la
prova che una MdT universale esiste, passiamo ora alla
costruzione di una possibile implementazione di essa.

Chiariamo che avere tre nastri non aumenta in alcun modo né la
capacità espressiva né le prestazioni della macchina (questo
punto verrà trattato meglio nella parte di complessità). Serve
solo a noi per riuscire a definire la macchina più comodamente.
