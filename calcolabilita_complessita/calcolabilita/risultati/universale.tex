\section{Formalismo universale}
Supponiamo ora di avere a disposizione un formalismo
\textbf{universale}, in grado cioè di esprimere \emph{tutte}
le funzione calcolabili. Questo sarebbe così potente da
riuscire ad esprimere l'interprete dei propri programmi.

\begin{theorem}[Enumerazione] \label{th: enum}
	Esiste una funzione numerabile parziale $\varphi_z(i, x)$
	tale che $\forall i,x$ vale
	\[ \varphi_i(x) = \varphi_z (i, x) \]
	\begin{proof}
		Poiché la funzione $U(\mu y . T(i, x, y))$ usata nel
		\hyperref[th: fn]{teorema di forma normale} è definita
		per composizione e $\mu$-ricorsione a partire da
		funzioni ricorsive primitive, essa stessa è una una
		funzione calcolabile in due argomenti $i$ e $x$. Avrà
		quindi un indice che chiamiamo $z$ e per cui vale
		\[ \varphi_z (i, x) = U(\mu y. T(i, x, y)) \]
		Applichiamo quindi il teorema di forma normale per
		ottenere
		\[ U(\mu y . T(i, x, y)) = \varphi_i (x) \]
		da qui otteniamo la tesi per transitività
		dell'uguaglianza
		\[
			\varphi_z (i, x) = U(\mu y . T(i, x, y)) =
			\varphi_i (x)
		\]
		Più informalmente la macchina $M_z$ recupera la
		descrizione della macchina $M_i$ e la applica a $x$.
	\end{proof}
\end{theorem}

Il teorema, scritto in forma molto compatta, in sostanza ci
dice che esiste un algoritmo (o una MdT) $z$ che prende in
input un altro algoritmo (o MdT) $i$, i dati $x$ su cui lavora
$i$ e restituisce lo stesso risultato dell'algoritmo $i$-esimo
applicato a $x$. In genere la macchina di Turing $z$ viene
chiamata \textbf{macchina di Turing universale}. In realtà
esistono un'infinità numerabile di MdT universali.

Passiamo ora ad un teorema che possiamo vedere un po' come il
duale del teorema di enumerazione. Enunceremo una prima versione
semplificata e poi la sua forma generale.

\begin{theorem}[Teorema del parametro s-1-1]
	\label{th: s-1-1}
	Esiste una funzione $s$ calcolabile totale e iniettiva tale
	che per ogni $i$, $x$ e $y$ vale
	\[ \lambda y . \varphi_i (x, y) = \varphi_{s (i, x)} (y) \]
	\begin{proof}
		Iniziamo con il definire due passi:
		\begin{enumerate}
			\item Da $i$ prendiamo l'$i$-esima MdT.
			\item Scriviamo sul nastro di $M_i$ l'input $x$ e
			      $y$.
		\end{enumerate}
		Questi due passi definiscono un algoritmo il quale, per
		la tesi di Church-Turing, ha un indice $j$ tale che
		\[ j = s(i, x) \]
		Tale indice identifica la macchina che calcola
		\[ \varphi_{s(i,x)} \]
		A questo punto dobbiamo evitare di trovarci nella
		situazione in cui
		\[ s(i, x) = s(i, x') \]
		abbiamo quindi bisogno di una funzione per effettuare la
		ricerca dell'indice $j$ che sia iniettiva. Per il
		\hyperref[th: padding lemma]{padding lemma} sappiamo che
		esiste un'infinità numerabile di algoritmi che calcolano
		la stessa funzione e dunque cerchiamo $h$ maggiore di
		tutti i valori ottenuti fino ad ora. Otteniamo così una
		funzione strettamente crescente e quindi iniettiva.
	\end{proof}
\end{theorem}

Il teorema ci dice che esiste una funzione $s$ calcolabile
totale e iniettiva tale che, per ogni indice $i$ e per ogni
\emph{parametro} $x$, questa individua una macchina $j$ in grado
di calcolare $\varphi_i (x, y)$ qualora $x$ sia fissato, sia
cioè un \textbf{parametro}.

\begin{example}
	Prendiamo ad esempio la funzione
	\[ x + y \]
	Una volta fissata la $x$ ad un certo valore, supponiamo
	$2$, la $x$ a questo punto non è più un argomento ma
	diventa un parametro. A questo punto la funzione
	\[ 2 + y \]
	può essere calcolata da un'altra macchina che a regola
	dovrebbe essere un po' più efficiente di quella di partenza.
\end{example}

Questo sta alla base della \textbf{valutazione parziale} che è
molto utile in quei contesti in cui si vuole mantenere un alto
livello di generalizzazione.

\begin{theorem}[Teorema del parametro s-m-n]
	\label{th: s-m-n}
	Per ogni $m, n \geq 0$ esiste una funzione calcolabile
	totale (iniettiva) $s_n^m$ con $m+1$ argomenti, tale che
	per ogni $x, y_1, \dots, y_m$, vale
	\[
		\varphi_{s_n^m (x, y_1, \dots, y_m)}^{(n)} =
		\lambda z_1, \dots, z_n .
		\varphi_x^{(m+n)} (y_1, \dots, y_m, z_1, \dots, z_n)
	\]
\end{theorem}

Si noti come il teorema del parametro e quello di enumerazione
siano in un certo senso l'inverso l'uno dell'altro. Infatti
l'uno "abbassa" un argomento nella posizione di indice, mentre
l'altro "innalza" un indice nella posizione di argomento.

\begin{theorem}[Espressività]
	Un formalismo è \textbf{Turing-equivalente}, ossia calcola
	tutte e sole le funzioni T-calcolabili ed è universale, se
	e solo se
	\begin{itemize}
		\item Ha un algoritmo universale (vale cioè il teorema
		      di enumerazione).
		\item Vale il teorema del parametro.
	\end{itemize}
\end{theorem}

Grazie al teorema del parametro si dimostra un altro teorema
che ha un ruolo fondamentale sia in informatica che in teoria
della calcolabilità.

\begin{theorem}[Punto fisso] \label{th: punto_fisso}
	Per ogni funzione $f$ calcolabile totale $\exists n$ tale
	che
	\[ \varphi_n = \varphi_{f(n)} \]
	\begin{proof}
		Definiamo la seguente funzione calcolabile "diagonale"
		\[
			\psi (u, z) = \varphi_{d(u)} (z) =
			\begin{cases}
				\varphi_{\varphi_u (u)} (z) &
				\text{se } \varphi_u (u) \downarrow \\
				\text{indefinita}           &
				\text{altrimenti}
			\end{cases}
		\]
		Per il teorema del parametro, $d(u)$ è totale e
		iniettiva (e non dipende da $f$). Data $f$, allora
		anche $f \circ d$ è calcolabile e sia $v$ proprio
		un indice tale che
		\[ \varphi_v(x) = f(d(x)) \]
		Tale funzione è totale (perché sia $d$ che $f$ lo sono)
		e quindi $\varphi_v (v)$ converge. Pertanto, in accordo
		con la definizione data in precedenza di $\psi (u,z)$
		abbiamo che
		\[ \varphi_{d(v)} = \varphi_{\varphi_v(v)} \]
		Calcoliamo ora $d(v)$ e supponiamo che il risultato sia
		$n$, cioè poniamo
		\[ n = d(v) \]
		Dimostriamo che è un punto fisso di $f$. Infatti vale
		la seguente catena di eguaglianze
		\[
			\varphi_n  = \varphi_{d(v)}
			= \varphi_{\varphi_v (v)}
			= \varphi_{f(d(v))}
			= \varphi_{f(n)}
		\]
		Nell'eguaglianza più a sinistra si sfrutta l'iniettività
		che viene garantita dal teorema del parametro.
	\end{proof}
\end{theorem}

Un tale indice viene detto \textbf{punto fisso} di $f$. Il
teorema inoltre ci dice che la funzione $f$ trasforma algoritmi
in algoritmi, proprio come fa un compilatore.

In altre parole il punto fisso non cambia la funzione calcolata
ma trasforma l'algoritmo $P_n$ nell'algoritmo $P_{f(n)}$ con la
stessa semantica.

\begin{property}
	Nelle ipotesi del teorema di ricorsione
	\begin{itemize}
		\item Il punto fisso è calcolabile mediante una funzione
		      totale (iniettiva) $g$ a partire dall'indice di
		      $f$.
		\item Ci sono $\# (\N)$ punti fissi di $f$.
	\end{itemize}
\end{property}

\begin{proof}
	Per dimostrare il primo punto prendiamo $h(x)$
	calcolabile totale tale che $\forall n$ vale
	\[ \varphi_{h(x)} (n) = \varphi_x (d(n)) \]
	Allora vale
	\[ g(x) = d(h(x)) \]
	Il secondo punto segue invece dal teorema
	\ref{th: n calc}.
\end{proof}

C'è anche un altro modo per dimostrare il teorema del punto
fisso, o meglio per specificare come deve essere implementata
la ricorsione nei linguaggi di programmazione.

Supponiamo di avere una procedura ricorsiva $P$ il cui corpo
sia $C$, all'interno del quale ovviamente compare la chiamata
a $P$ stessa.