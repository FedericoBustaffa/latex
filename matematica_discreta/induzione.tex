\section{Induzione}
\subsection{Definizioni ricorsive}
L'idea alla base di una definizione ricorsiva \`e quella di ricondursi ai casi
precedenti per definire una funzione o una successione.

\begin{example}
	Prendiamo ad esempio la funzione fattoriale
	\begin{equation*}
		\begin{cases}
			0!       & = 1          \\
			(n + 1)! & = n! (n + 1)
		\end{cases}
	\end{equation*}
	Dunque $4! = 1 \cdot 2 \cdot 3 \cdot 4$.
\end{example}

\begin{example}
	Prendiamo anche in esempio la successione di Fibonacci. Essa \`e definita come
	segue
	\begin{equation*}
		\begin{cases}
			F(0)     & = 0               \\
			F(1)     & = 1               \\
			F(n + 2) & = F(n + 1) + F(n)
		\end{cases}
	\end{equation*}
\end{example}

\subsection{Dimostrazioni per induzione}
L'idea sta nel dimostrare che se una certa propriet\`a \`e vera per un certo $n$,
fissato come valore iniziale, allora lo \`e anche per $n + 1$.

Come prima cosa c'\`e la verifica del \textbf{caso iniziale}. Dobbiamo quindi trovare
(per tentativi) un certo $n$ che soddisfi la propriet\`a.

Una volta trovato si procede
col \textbf{passo induttivo}. Il passo induttivo consiste nel dimostrare che data
ad esempio una certa formula $f(n)$,
\begin{equation*}
	f(n + 1) = f(n) + (n + 1)
\end{equation*}
Chiariamo meglio il concetto con qualche esempio.

\begin{example}
	Prendiamo ad esempio la seguente propriet\`a della somma di numeri naturali.
	\begin{equation*}
		1 + 2 + \cdots + n = \frac{n (n + 1)}{2}
	\end{equation*}

	\begin{enumerate}
		\item \emph{Caso iniziale}: Si vede facilmente che la propriet\`a \`e
		      valida per $n = 1$.
		\item \emph{Passo induttivo}: Per prima cosa scrivo la propriet\`a in termini
		      di $n + 1$ e ottengo
		      \begin{equation*}
			      1 + 2 + \cdots + n + (n + 1) = \frac{(n + 1)(n + 2)}{2}
		      \end{equation*}
		      Notiamo che al primo membro abbiamo $1 + 2 + \cdots + n$ e sappiamo
		      a caso equivale. Sostituiamo e sviluppiamo l'espressione.
		      \begin{align*}
			      1 + 2 + \cdots + n + (n + 1)  & = \\
			      \frac{n (n + 1)}{2} + (n + 1) & = \\
			      \frac{n(n + 1) + 2(n + 1)}{2} & = \\
			      \frac{(n + 1)(n + 2)}{2}
		      \end{align*}
		      Abbiamo quindi ottenuto la formula in funzione di $n + 1$.
	\end{enumerate}
\end{example}

Proviamo adesso a dimostrare la stessa propriet\`a ma senza induzione.

\begin{example}
	Abbiamo sempre la solita formula
	\begin{equation*}
		1 + 2 + \cdots + n = \frac{n(n + 1)}{2}
	\end{equation*}
	Proviamo a ragionare in questo modo
	\begin{equation*}
		1 + 2 + \cdots + n + n + (n - 1) + \cdots + 1
	\end{equation*}
	equivale a sommare $n$ volte $n + 1$ quindi l'espressione equivale a $n(n + 1)$. Notiamo
	per\`o che l'espressione equivale anche a raddoppiare $1 + 2 + \cdots + n$. Dividendo
	per $2$ otteniamo quindi la formula desiderata, ovvero
	\begin{equation*}
		\frac{n(n + 1)}{2}
	\end{equation*}
\end{example}

\subsection{Sommatorie}
Iniziamo a parlare di sommatorie e delle loro propriet\`a. Consideriamo per la generica
successione
\begin{equation*}
	a_0, a_1, \dots, a_n
\end{equation*}
Abbiamo che
\begin{align*}
	\sum_{i = 1}^{10} a_i    & = a_1 + a_2 + \cdots a_{10}                     \\
	\sum_{i = 5}^7 a_i       & = a_5 + a_6 + a_7                               \\
	\sum_{i = 0}^k a_i       & = a_0 + a_1 + \cdots + a_k                      \\
	\sum_{i = 0}^{k + 1} a_i & = \left( \sum_{i = 0}^k a_i \right) + a_{k + 1}
\end{align*}

Facciamo ora qualche esempio
\begin{example}
	Prendiamo la seguente somma come esempio
	\begin{equation*}
		1 + 3 + 5 + \cdots + (2n + 1) = \sum_{i = 0}^n (2i + 1)
	\end{equation*}
	Si tratta della somma dei primi $n + 1$ numeri dispari.
	Se volessi la somma dei primi $n$ numeri dispari la sommatoria sarebbe di questo tipo
	\begin{equation*}
		\sum_{i = 1}^n (2i - 1) = n^2
	\end{equation*}
	E questo lo si pu\`o dimostrare anche graficamente.
\end{example}

Vediamo ora qualche propriet\`a delle sommatorie
\begin{enumerate}
	\item $c \left( \sum_{i = 1}^n a_i \right) = \sum_{i = 1}^n (c a_i)$
	\item $\sum_{i = 1}^n a_i + \sum_{i = 1}^n b_i = \sum_{i = 1}^n (a_i + b_i)$
\end{enumerate}

Con queste propriet\`a possiamo ora dimostrare la formula di prima in questo modo
\begin{align*}
	\sum_{i = 1}^n (2i - 1)               & =     \\
	\sum_{i = 1}^n 2i - \sum_{i = 1}^n 1  & =     \\
	2 \left( \sum_{i = 1}^n i \right) - n & =     \\
	n(n + 1) - n                          & =     \\
	n [n + 1 - 1]                         & = n^2
\end{align*}

\subsection{Induzione forte}
In questo caso ho pi\`u casi iniziali che do per veri e il passo induttivo diventa di questo
tipo
\begin{equation*}
	p(1) \wedge p(2) \wedge \cdots \wedge p(n) \Rightarrow p(n + 1)
\end{equation*}

Fissiamo meglio le idee con un esempio.

\begin{example}
	Un numero $p$ \`e \emph{primo} se \`e divisibile solo per $1$ e per se stesso.
	Proviamo ora a dimostrare con l'induzione forte che ogni $n \in \mathbb{N}$, con $n > 1$
	\`e un prodotto di numeri primi.
	\begin{enumerate}
		\item Nel caso in cui $n$ sia primo ho finito.
		\item Nel caso in cui $n$ non sia primo \`e divisibile per qualche $k \neq 1 \neq n$.
		      quindi
		      \begin{equation*}
			      \begin{array}{lcr}
				      n = qk & k, q > 1 & 1 < k, q < n
			      \end{array}
		      \end{equation*}
		      Per ipotesi induttiva forte $k$ e $q$ si scompongono in prodotto di fattori primi
		      quindi anche $n$.
	\end{enumerate}
\end{example}

\begin{example}[Disuguaglianza di Bernoulli]
	Partiamo con lo scrivere la disuguaglianza
	\begin{equation*}
		(1 + x)^n \geq 1 + nx
	\end{equation*}
	Posso verificare per induzione su $n$.
	\begin{enumerate}
		\item \emph{Caso iniziale}: $n = 0$ verifica la disuguaglianza, infatti
		      \begin{equation*}
			      \begin{array}{rcl}
				      (1 + x)^0 & \geq & 1 + 0x \\
				      1         & \geq & 1
			      \end{array}
		      \end{equation*}
		\item \emph{Passo induttivo}: Scriviamo la formula per $n + 1$ e procediamo con qualche
		      passaggio aritmetico
		      \begin{equation*}
			      \begin{array}{ll}
				      (1 + x)^{n + 1}        & \geq 1 + (n + 1)x \\
				      (1 + x)^n (1 + x)      & \geq 1 + nx + x   \\
				      (1 + x)^n + x(1 + x)^n & \geq 1 + nx + x
			      \end{array}
		      \end{equation*}
		      Arrivati a questo punto notiamo che parte della disuguaglianza \`e vera per ipotesi
		      quindi ci rimane da verificare che
		      \begin{equation*}
			      x(1 + x)^n \geq x
		      \end{equation*}
		      La dimostrazione \`e molto semplice infatti
		      \begin{equation*}
			      \begin{array}{ll}
				      x(1 + x)^n & \geq x \\
				      (1 + x)^n  & \geq 1
			      \end{array}
		      \end{equation*}
		      Ma sappiamo che $(1 + x)^n \geq 1 + nx$ quindi la disuguaglianza \`e sempre vera.
	\end{enumerate}
\end{example}

\subsection{Progressioni aritmetiche e geometriche}
\begin{defn}
	Una \textbf{progressione aritmetica} \`e una successione di numeri in cui, ad ogni passo,
	$a_{n + 1}$ si ottiene sommando ad $a_n$ una quantit\`a finita $b$.
	Per fissare meglio le idee consideriamo la generica successione
	\begin{equation*}
		a_0, a_1, a_2, a_3, \dots
	\end{equation*}
	Ho che
	\begin{equation*}
		\begin{array}{ll}
			a_1 & = a_0 + b  \\
			a_2 & = a_0 + 2b \\
			a_3 & = a_0 + 3b \\
			\dots
		\end{array}
	\end{equation*}
\end{defn}

\begin{defn}
	Una \textbf{progressione geometrica} \`e una di numeri in cui, ad ogni passo, $a_{n + 1}$
	si ottiene moltiplicando ad $a_n$ una quantit\`a finita $b$.
	Anche in questo caso consideriamo la successione generica di prima e otteniamo che
	\begin{equation*}
		\begin{array}{lll}
			a_1 & = a_0 \cdot b   &               \\
			a_2 & = a_0 \cdot b^2 & = a_1 \cdot b \\
			\dots
		\end{array}
	\end{equation*}
\end{defn}

Studiamo ora le sommatorie di progressioni aritmetiche e geometriche con qualche esempio
\begin{example}
	Consideriamo la progressione
	\begin{equation*}
		a_0, a_1, a_2, \dots
	\end{equation*}
	e vogliamo calcolare
	\begin{equation*}
		\sum_{i = 0}^n a_i
	\end{equation*}
	Poniamo ad esempio $a_0 = 3$ e $a_{n + 1} = a_n + 5$ e poniamo $n = 100$.
	\begin{gather*}
		\sum_{i = 0}^{100} a_i                        = \\
		\sum_{i = 0}^{100} (3 + 5i)                   = \\
		\sum_{i = 0}^{100} 3 + \sum_{i = 0}^{100} 5i  = \\
		3 \cdot 101 + 5 \sum_{i = 0}^{100} i          = \\
		3 \cdot 101 + 5 \left( \frac{100 \cdot 101}{2} \right)
	\end{gather*}
	Ne concludiamo che
	\begin{equation*}
		\sum_{i = 0}^n (3 + 5i) = 3(n + 1) + 5 \cdot \frac{n(n + 1)}{2}
	\end{equation*}
\end{example}

Vediamo ora un esempio per le progressioni geometriche
\begin{example}
	In questo caso consideriamo $a_0 = 1$ e moltiplichiamo ogni volta per $x$. Otteniamo quindi
	la progressione
	\begin{equation*}
		1, x, x^2, x^3, \dots
	\end{equation*}
	Mettendola in una sommatoria otteniamo
	\begin{gather*}
		\sum_{i = 0}^5 a_i = \\
		\sum_{i = 0}^5 x^i = \\
		1 + x + x^2 + x^3 + x^4 + x^5 = \\
		\frac{x^6 - 1}{x - 1}
	\end{gather*}
	Si pu\`o dimostrare facilemente moltiplicando la successione per $x - 1$.
	Quindi possiamo affermare che
	\begin{equation*}
		\sum_{i = 0}^n x^i = \frac{x^{n + 1} - 1}{x - 1}
	\end{equation*}
\end{example}

Elenchiamo di seguito qualche regola per le disuguaglianze
\begin{itemize}
	\item $x \geq y \Rightarrow cx \geq cy$ se $c \geq 0$.
	\item $x \geq y \Rightarrow cx \leq cy$ se $c < 0$.
	\item $x \geq y \Rightarrow x + c \geq y + c$ per qualsiasi $c$.
\end{itemize}
