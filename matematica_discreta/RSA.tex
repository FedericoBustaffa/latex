\chapter{Algoritmo RSA}
\section{Introduzione}
\begin{theorem}[RSA]
	Se $p, q$ sono primi e sia $m \in \mathbb{Z}$ tale che
	\begin{equation*}
		0 \leq m < pq
	\end{equation*}
	e siano $e, d \in \mathbb{Z}$ tali che
	\begin{equation*}
		ed \equiv 1 \mod{(p - 1)(q - 1)}
	\end{equation*}
	allora
	\begin{equation*}
		m^{ed} \equiv m \mod{pq}
	\end{equation*}
	\begin{proof}
		Possiamo usare il teorema di Eulero e vedere che
		\begin{equation*}
			\phi (pq) = \phi (p) \cdot \phi(q) = (p - 1)(q - 1)
		\end{equation*}
		Poi da
		\begin{equation*}
			ed \equiv 1 \mod{(p - 1)(q - 1)}
		\end{equation*}
		ricaviamo
		\begin{equation*}
			\begin{array}{c}
				ed = 1 + (p - 1)(q - 1)k                                \\
				\Downarrow                                              \\
				m^{ed} \equiv m^{1 + (p - 1)(q - 1)k} \equiv            \\
				m \cdot [m^{(p - 1)(q - 1)}]^k \equiv                   \\
				m \cdot [\overbrace{m^{\phi (pq)}}^{\equiv 1}]^k \equiv \\
				m \mod{pq}
			\end{array}
		\end{equation*}
		Si possono sviluppare i calcoli anche nel caso in cui $m$ non sia invertibile modulo $pq$
		e il teorema funziona lo stesso.
	\end{proof}
\end{theorem}

\begin{example}
	Prendiamo
	\begin{equation*}
		\begin{array}{ll}
			p  & = 73   \\
			q  & = 101  \\
			pq & = 7373 \\
			m  & = 2091 \\
			e  & = 19   \\
			d  & = 379
		\end{array}
	\end{equation*}
	Il messaggio in questione \`e il nostro $m$. Se voglio criptarlo
	elevandolo alla $e \mod{pq}$. Se voglio decriptarlo devo elevarlo alla $d \mod{pq}$.
	Nel nostro caso
	\begin{equation*}
		2091^{19} \equiv 6542 \mod{7373}
	\end{equation*}
	e
	\begin{equation*}
		6542^{379} \equiv 2091 \mod{7373}
	\end{equation*}
	Abbiamo dunque criptato e decriptato $m$ con pochi passaggi.
\end{example}

\section{Comunicazione sicura con l'algoritmo RSA}
Vediamo ora come funziona l'algortimo RSA (verr\`a approfondito a crittografia).

Vogliamo simulare la comunicazione tra $A$ e $B$ dove $A$ manda un messaggio a $B$.
Questi sono i passaggi che devono seguire i due per inviarsi un messaggio in modo sicuro:
\begin{enumerate}
	\item $B$ prende due numeri $p, q$ primi molto grandi (300 cifre).
	\item $B$ rende pubblico $pq = M$ ma tiene segreti $p$ e $q$.
	\item $B$ prende $e \in \mathbb{Z}$ coprimo con $(p - 1)(q - 1)$.
	\item $B$ calcola l'inverso $d$ di $e \mod{(p - 1)(q - 1)}$
	      \begin{equation*}
		      de \equiv 1 \mod{(p - 1)(q - 1)}
	      \end{equation*}
	\item $B$ rende pubblico $e$.
	\item $A$ conosce dunque $pq$ ed $e$. Cripta $m$ e invia quindi $m^e \mod{pq}$.
	\item $B$ decripta $m^e$ elevandolo alla $d$ poich\'e $m^e \equiv (m^e)^d \equiv m \mod{pq}$.
\end{enumerate}