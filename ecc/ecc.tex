\documentclass[12pt]{report}

\usepackage[T1]{fontenc}
\usepackage[italian]{babel}
\usepackage[hidelinks]{hyperref}
\usepackage{tikz, pgfplots}
\usepackage{graphicx}
\usepackage{tabularx}

\hypersetup {
    colorlinks=true,
    urlcolor=blue,
    linkcolor=black
}

% --------------- STYLE ---------------
\usepackage[margin=1.25in]{geometry}
\usepackage{xcolor}

\usepackage{fancyhdr}
\usepackage[Sonny]{fncychap}
\usepackage[most]{tcolorbox}

% Font
\renewcommand{\familydefault}{\sfdefault}
\usepackage{sansmath}
\sansmath

\pagestyle{fancy}
\setlength{\headheight}{15pt}
\rhead{\thepage}


% --------------- MATH ---------------
\usepackage{amsmath, amssymb, amsthm, amsfonts, mathtools}

\usepackage{mdframed}
\newtheoremstyle{th_style}
{0pt}{0pt}
{\normalfont}
{}
{\color{green!40!black}}
{\;}{0.25em}
{\thmname{\textbf{#1}}\thmnumber{ \textbf{#2}}{\color{black}\thmnote{\textbf{ -- #3}}}}

\newmdenv[
	rightline=false,
	leftline=true,
	topline=false,
	bottomline=false,
	linecolor=green!40!black,
	innerleftmargin=5pt,
	innerrightmargin=5pt,
	innertopmargin=0pt,
	innerbottommargin=0pt,
	leftmargin=0cm,
	rightmargin=0cm,
	linewidth=4pt
]{dBox}

\newmdenv[
	rightline=false,
	leftline=true,
	topline=false,
	bottomline=false,
	linecolor=green!40!black,
	backgroundcolor=black!5,
	innerleftmargin=5pt,
	innerrightmargin=5pt,
	innertopmargin=5pt,
	innerbottommargin=5pt,
	leftmargin=0cm,
	rightmargin=0cm,
	linewidth=4pt
]{pBox}

\theoremstyle{th_style}
\newtheorem{theoremeT}{Teorema}[chapter]
\newtheorem{definitionT}{Definizione}[chapter]
\newtheorem{propositionT}{Proposizione}[chapter]
\newtheorem{corollary}{Corollario}[chapter]
\newtheorem{lemma}{Lemma}[chapter]
\newtheorem{observation}{Osservazione}[chapter]
\newtheorem{exampleT}{Esempio}[section]

\newenvironment{theorem}{\begin{pBox}\begin{theoremeT}}{\end{theoremeT}\end{pBox}}
\newenvironment{definition}{\begin{dBox}\begin{definitionT}}{\end{definitionT}\end{dBox}}
\newenvironment{proposition}{\begin{pBox}\begin{propositionT}}{\end{propositionT}\end{pBox}}
\newenvironment{example}{\begin{dBox}\begin{exampleT}}{\end{exampleT}\end{dBox}}


\pgfplotsset{compat=newest}

\newcommand{\N}{\mathbb{N}}
\newcommand{\Z}{\mathbb{Z}}
\newcommand{\R}{\mathbb{R}}
\newcommand{\C}{\mathcal{C}}
\newcommand{\F}{\mathbb{F}}
\newcommand{\K}{\mathcal{K}}
\newcommand{\B}{\mathcal{B}}

\newcommand{\tx}{\tilde{x}}
\newcommand{\norm}[1]{\left\lVert#1\right\rVert}
\newcommand{\start}{\triangleright}

\DeclareMathOperator{\sign}{sign}
\DeclareMathOperator{\trn}{trn}
\DeclareMathOperator{\arr}{arr}
\DeclareMathOperator{\fl}{fl}
\DeclareMathOperator{\dist}{dist}
\DeclareMathOperator{\Ker}{Ker}
\DeclareMathOperator{\diag}{diag}
\DeclareMathOperator{\nnz}{nnz}
\DeclareMathOperator{\Var}{Var}

\title{Elementi di calcolabilità e complessità}
\author{Federico Bustaffa}
\date{15/04/2024}

\begin{document}

\maketitle
\tableofcontents

\chapter{Introduzione}

Il corso tratterà le modalità di reperimento e analisi di dati da blockchain tramite Python. Si compirà un'analisi
statistica sui dati raccolti tramite \emph{scraping} o API di vario genere per riuscire ad estrapolare informazioni
sulle blockchain prese in esame.

Saranno necessarie nozioni di statistica descrittiva ed inferenziale e si farà uso di strumenti per l'analisi di
grafi.

Le librerie Python necessarie per lavorare al meglio in questo ambito sono:
\begin{itemize}
	\item \verb|BeautifulSoup|, \verb|Scrapy|: web scraping
	\item \verb|Pandas|: gestione di tabelle
	\item \verb|Numpy|: algebra vettoriale e matriciale
	\item \verb|SciPy|: statistica
	\item \verb|Matplotlib|: visualizzazione di grafici
	\item \verb|NetworkX|: analisi di grafi
\end{itemize}
Per installare le librerie scaricare il package manager \verb|pip| di Python:
\begin{verbatim}
	> sudo apt install python3-pip
\end{verbatim}
Installazione delle librerie
\begin{verbatim}
	> pip install beautifulsoup4
	> pip install scrapy
	> pip install pandas
	> pip install numpy
	> pip install scipy
	> pip install matplotlib
	> pip install networkx
\end{verbatim}

\section{Blockchain}
Una \textbf{blockchain} utilizza un modello computazionale differente dal classico modello \textbf{client-server},
si basa infatti sul sistema \textbf{peer-to-peer}.

In questo modello ogni nodo possiede una frazione del potere necessario a compiere determinate azioni all'interno
della blockchain. Il sistema nasce infatti per eliminare l'entità centrale che controlla e gestisce l'intero servizio.

Possiamo vedere una blockchain come un \textbf{registro} \emph{distribuito} e \emph{replicato} tra i nodi di una
rete \emph{peer-to-peer}. In altre parole ogni nodo della rete possiede una copia della blockchain uguale a quella
di tutti gli altri nodi.

Una proprietà fondamentale al fine di mantenere la blockchain consistente e \textbf{immutabile} è la
\textbf{tamper freeness} garantita tramite meccanismi crittografici e algoritmi basati sul consenso.


\section{Macchina di Turing}
Introduciamo il primo formalismo per esprimere algoritmi, ideato
da Alan Turing nel 1936 e che prende il nome di
\textbf{macchina di Turing}.

L'idea di base è quella di avere un \textbf{nastro} di lunghezza
infinita su cui disporre una stringa di cratteri, rappresentante
l'input dell'algoritmo. Tale macchina è dotata di un
\textbf{cursore} che, partendo dall'inizio della stringa, legge
i caratteri che incontra e, in base a cosa legge, cambia stato
e si muove. La computazione termina quando si giunge finalmente
in uno stato speciale di terminazione.

\begin{definition}
	Una \textbf{macchina di Turing} (MdT) è definita come una
	quadrupla di questo tipo
	\[
		M = \begin{pmatrix}
			Q, & \Sigma, & \delta, & q_0
		\end{pmatrix}
	\]
	dove
	\begin{itemize}
		\item $Q$ è l'\textbf{insieme finito degli stati} in
		      cui si può trovare la macchina. Tra gli stati
		      abbiamo uno stato speciale $h$, con cui
		      indicheremo la corretta terminazione del calcolo
		      della macchina $M$.
		\item $\Sigma = \{ \sigma, \sigma', \dots \}$ è
		      l'\textbf{insieme finito dei simboli}, ossia,
		      l'\textbf{alfabeto} utilizzato per esprimere
		      gli algoritmi. Il numero e e i simboli stessi
		      possono variare da una MdT all'altra, tuttavia
		      ci sono dei simboli considerato speciali che sono
		      sempre presenti:
		      \begin{itemize}
			      \item \textbf{Bianco}: indicato con $\#$ è
			            come un carattere nullo.
			      \item \textbf{Respingente}: indicato con
			            $\start$, simboleggia l'inizio delle
			            stringa. Il cursore della macchina non
			            può andare mai a sinistra di questo
			            simbolo.
			      \item \textbf{Spostamento}: simboli che non
			            appartengono a $\Sigma$ e indicano come
			            e in che direzione muovere il cursore.
			            Sono rispettivamnete $L$, $R$ e $-$ che
			            stanno per \verb|Left|, \verb|Right| e
			            \verb|Hold| (non muovere il cursore).
		      \end{itemize}
		\item \textbf{Stato iniziale}: indicato con $q_0 \in Q$
		      è lo stato iniziale in cui si trova la macchina.
		\item \textbf{Funzione di transizione}: indicata con
		      $\delta$ è la funzione che definisce la
		      transizione da uno stato all'altro della macchina
		      in base a quel che viene letto dal cursore e in
		      base a quel che è stato letto fino a quel momento.
		      \[
			      \delta \subseteq (Q \times \Sigma) \to
			      (Q \cup \{ h \}) \times \Sigma \times
			      \{ L, R, - \}
		      \]
		      che ci permette per l'appunto di cambiare lo
		      stato della macchina e progredire nel calcolo.
	\end{itemize}
\end{definition}

\subsection{Funzione di transizione}
La \textbf{funzione di transizione} $\delta$ è necessaria a
far progredire il calcolo. Come possiamo vedere dalla
definizione che ne abbiamo dato, questa prende in input una
coppia di valori.
\begin{itemize}
	\item Lo \textbf{stato corrente} $q$ della macchina.
	\item Un simbolo $\sigma$ dell'alfabeto
\end{itemize}

Notiamo inoltre che la funzione $\delta$ prende in input una
coppia di valori ma ritorna una tripla composta da uno stato
$q'$, che può essere anche $h$ in caso il calcolo sia terminato
con successo, un simbolo $\sigma'$ e una tra le 3 mosse $L$,
$R$ e $-$.

Ad ogni passo del calcolo, $\delta$ elabora la coppia in input
e restituisce una tripla contente il nuovo stato, il nuovo
simbolo da scrivere alla posizione attuale del cursore e come
ci si deve muovere al passo successivo.

Qualcuno potrebbe aver notato la similitudine con un automa a
stati finiti. In quanto vi è sempre uno stato corrente e, in
base a quello e ad una nuova parte dell'input ci si muove in
un nuovo stato o si rimane nello stesso. Non è tuttavia
corretto dire che una MdT è un automa.

\subsubsection{Considerazioni}
Ora che abbiamo le idee più chiare, facciamo qualche
considerazione in più su $\delta$. Essa è \textbf{iniettiva},
vale cioè che, prese due triple $(q', \sigma', D')$ e
$(q'', \sigma'', D'')$ tali che
\begin{gather*}
	\delta (q, \sigma) =  (q', \sigma', D') \\
	\delta (q, \sigma) = (q'', \sigma'', D'')
\end{gather*}
allora $q' = q''$, $\sigma' = \sigma''$ e $D' = D''$.
Questo ci dice sostanzialmente che, dato uno stato e un simbolo
c'è un solo altro stato in cui possiamo andare. Un'altra cosa
da specificare è che per $\delta$ vale sempre che se
\[ \delta(q, \start) = (q', \sigma, D) \]
allora $\sigma = \start$ e $D = R$. Questo ci dice che se ci
troviamo all'inizio del nastro possiamo andare solo a destra.

Ritornando velocemente
all'\hyperref[sec: algoritmo]{idea intuitiva di algoritmo},
possiamo facilmente verificare la prima e seconda condizione
sono verificate poiché, sia $Q$ che $\Sigma$ sono insiemi
finiti e di conseguenza anche $\delta$ contiene un numero
finito di elementi.

\subsection{Alfabeto}
I dati su cui opera una MdT sono stringhe $w$ di caratteri
appartenenti a $\Sigma$, più precisamente $w \in \Sigma^*$,
dove $\Sigma^*$ comprende anche la stringa vuota $\epsilon$.

Senza stare a incasinarsi con inutili formalismi matematici,
se $\Sigma$ è l'alfabeto, allora $\Sigma^*$ è l'insieme di
tutte le possibili stringhe generabili con quell'alfabeto e
la stringa vuota.

\begin{example}
	Prendiamo ad esempio l'alfabeto binario
	\begin{gather*}
		\Sigma = \{ 0, 1 \} \\
		\Downarrow \\
		\Sigma^* = \{ \epsilon, 0, 1, 01, 10, 11, \dots \}
	\end{gather*}
\end{example}

\subsection{Computazione}
Finalmente vediamo come opera una MdT su qualche problema di
esempio. La \textbf{configurazione corrente} di una MdT viene
identificata $(q, u, \sigma, v)$ dove
\begin{itemize}
	\item $q$ è lo stato corrente.
	\item $u$ è la stringa a sinistra del carattere attuale.
	\item $\sigma$ è il carattere attuale.
	\item $v$ è il resto della stringa che termina con un
	      carattere non nullo.
\end{itemize}
La situazione corrente di una MdT può essere espressa più
comodamente con $(q, u \underline{\sigma} v)$. Graficamente
una MdT appare in questo modo
\begin{center}
	\includegraphics[scale=0.225]{images/turing.png}
\end{center}
Tenendo a mente questa figura possiamo provare a costruire la
nostra prima MdT.

\begin{example} \label{ex: 11}
	Vogliamo costruire una MdT in grado di dirci se la stringa
	binaria in input contiene una sottosequenza composta da
	due $1$ consecutivi.

	Per costruire la nostra macchina abbiamo bisogno di due
	stati ($q_0$ e $q_1$), di due simboli ($0$ e $1$) e di
	una funzione di transizione $\delta$.

	Per capire se la stringa in input contiene almeno una
	sottostringa composta da due $1$ consecutivi, iniziamo
	con lo stato iniziale $q_0$, il quale indica sia lo stato
	inziale della macchina sia lo stato in cui la macchina
	deve transire ogni qual volta incontra uno $0$.

	Abbiamo poi bisogno di uno stato $q_1$ in cui la macchina
	transisce quando incontra un $1$ nella sequenza dopo essere
	stata in uno stato $q_0$.

	Se ci troviamo nello stato $q_1$ e incontriamo un altro $1$
	la macchina transisce nello stato $h$ di terminazione. La
	funzione $\delta$ di transizione che deriva dai seguenti
	ragionamenti è la seguente
	\begin{center}
		\begin{tabular}{|c|c|c|}
			\hline
			$q$   & $\sigma$ & $\delta(q, \sigma)$ \\
			\hline
			$q_0$ & $\start$ & $q_0, \start, R$    \\
			$q_0$ & $0$      & $q_0, \#, R$        \\
			$q_0$ & $1$      & $q_1, \#, R$        \\
			$q_1$ & $0$      & $q_0, \#, R$        \\
			$q_1$ & $1$      & $h, \#, -$          \\
			\hline
		\end{tabular}
	\end{center}
	Come possiamo notare, ogni volta che incontriamo un
	carattere lo "cancelliamo" scrivendo $\#$ ma avremmo potuto
	lasciare il numero che incontravamo. Vediamo una possibile
	simulazione di esecuzione con la seguente stringa binaria
	in input
	\[ 01011 \]
	La nostra configurazione iniziale è
	\[ q_0 / \underline{\start} 01011 \]
	dove il simbolo sottolineato è quello su cui si trova il
	cursore. La sequenza di operazioni sarà quindi la seguente
	\begin{align*}
		q_0 / & \underline{\start} 01011       \\
		q_0 / & \start \underline{0} 1011      \\
		q_0 / & \start \# \underline{1} 011    \\
		q_1 / & \start \#\# \underline{0} 11   \\
		q_0 / & \start \#\#\# \underline{1} 1  \\
		q_1 / & \start \#\#\#\# \underline{1}  \\
		h /   & \start \#\#\#\# \underline{\#}
	\end{align*}
	Il calcolo è dunque terminato con successo.
\end{example}

Per vedere una MdT in azione è possibile visitare questo
\href{https://turingmachinesimulator.com/}{sito} in cui è
possibile programmare una MdT oppure eseguire esempi già
proposti.

\subsubsection{Configurazione e computazione}
Come abbiamo detto precedentemente, una \textbf{configurazione}
è definita dalla quadrupla
\[ \gamma = (q, u, \sigma, v) \]
Ciò che non abbiamo detto è che gli elementi di $\gamma$
appartengono al seguente insieme
\[
	\gamma \in (Q \cup \{ h \}) \times
	\Sigma^* \times \Sigma \times \Sigma^F
\]
L'ultimo insieme ($\Sigma^F$) è un po' particolare, è infatti
definito come
\[
	\Sigma^F = \Sigma^* \cdot \left( \Sigma \backslash
	\{ \# \} \right) \cup \{ \epsilon \}
\]
quindi possiamo scrivere la stringa $v$ come
$\sigma_0 \sigma_1 \dots \sigma_n$, con $\sigma_n \neq \#$,
al posto della stringa infinita composta dai vari $\sigma$ e
con infiniti $\#$ alla fine.

Si noti però che un qualsiasi carattere $\sigma_i$ con $i < n$
può essere $\#$ ed inoltre la stringa $u$ può essere vuota
solo quando il carattere corrente è $\start$. La convenzione
impone quindi di scrivere la configurazione iniziale del primo
esempio fatto, in questo modo
\[ (q_0, \start 01011, \epsilon) \]

\begin{definition}
	Una \textbf{computazione} è una successione finita di
	passi
	\[ (q_0, w) \to^* (q', w') \]
	dove $\to^*$ è la chiusura riflessiva e transitiva di
	$\to$. Ovviamente se vi sono $n$ passi, la computazione
	è lunga $n$ e dunque scriveremo $\to^n$.
\end{definition}

Definiamo meglio cos'è un \textbf{passo di computazione},
procedendo per casi e considerando $a$, $b$ e $c$ come elementi
generici di $\Sigma$.
\begin{itemize}
	\item $(q, u \underline{a} v) \to (q', u \underline{b} v)$
	      se $\delta (q, a) = (q', b, -)$.
	\item $(q, u c \underline{a} v) \to
		      (q', u \underline{c} b v)$
	      se $\delta (q, a) = (q', b, L)$.
	\item \begin{enumerate}
		      \item $(q, u \underline{a} c v) \to
			            (q', u b \underline{c} v)$
		            se $\delta (q, a) = (q', b, R)$.
		      \item $(q, u \underline{a}) \to
			            (q', u b \underline{\#})$
		            se $\delta (q, a) = (q', b, R)$.
	      \end{enumerate}
\end{itemize}
In questo modo garantiamo che ciascun passo abbia effetto
limitato sulle configurazioni, come richiesto dalla seconda
parte del punto 2 nell'\hyperref[sec: algoritmo]{idea intuitiva
	dell'algoritmo}.

Allo stesso modo possiamo immaginarci che se partiamo da uno
stato iniziale $(q_0, \underline{\start} w)$, dopo un certo
numero di passi arriviamo in uno stato $(q', w')$.

\begin{definition} \label{def: convergenza}
	Diciamo che una computazione \textbf{termina} o
	\textbf{converge} ($\downarrow$) se e solo se lo stato
	finale è $h$. Nell'esempio \ref{ex: 11} fatto qui sopra
	$q' = h$.

	Diciamo invece che la computazione \textbf{non termina}
	o \textbf{diverge} ($\uparrow$) se e solo se per ogni
	$q'$ e $w'$ tali che
	\[ (q_0, w) \to^* (q', w') \]
	esistono $q''$ e $w''$ tali che
	\[ (q', w') \to^* (q'', w'') \]
	ovvero tali che è sempre possibile fare un nuovo passo di
	computazione.
\end{definition}

Come possiamo vedere, le MdT definite in questo modo,
rispettano l'idea intuitiva di algoritmo che abbiamo dato
all'inizio.

\begin{example}
	Facciamo un esempio di computazione che non termina mai
	tramite una macchine che semplicemente non contiene il
	simbolo $h$ di terminazione.
	\begin{center}
		\begin{tabular}{|c|c|c|}
			\hline
			$q$   & $\sigma$ & $\delta$             \\
			\hline
			$q_0$ & $\start$ & $q_0$, $\start$, $R$ \\
			$q_0$ & $a$      & $q_0$, $a$, $R$      \\
			$q_0$ & $\#$     & $q_0$, $\#$, $R$     \\
			\hline
		\end{tabular}
	\end{center}
\end{example}

Proviamo ora a fare un esempio più concreto di una MdT che
calcola qualcosa di più sensato rispetto agli esempi visti
fino ad ora.

\begin{example}
	Questa MdT si propone di calcolare la somma di due numeri
	$n$ ed $m$, dove $n$ ed $m$ sono rappresentati in notazione
	unaria tramite il simbolo $|$ ripetuto $n$ (o $m$) volte.
	La funzione $\delta$ è definita dalla seguente tabella
	\begin{center}
		\begin{tabular}{|c|c|c|}
			\hline
			$q$   & $\sigma$ & $\delta$         \\
			\hline
			$q_0$ & $\start$ & $q_0, \start, R$ \\
			$q_0$ & $|$      & $q_0, |, R$      \\
			$q_0$ & $+$      & $q_1, |, R$      \\
			$q_1$ & $|$      & $q_1, |, R$      \\
			$q_1$ & $\#$     & $q_2, \#, L$     \\
			$q_2$ & $|$      & $h, \#, -$       \\
			\hline
		\end{tabular}
	\end{center}
	Il simbolo $+$ deve essere tra le due sequenze di $|$.
	Proviamo ora ad eseguire la computazione per il calcolo
	di $1 + 2$, partendo dalla configurazione iniziale
	\[ q_0/ \quad \underline{|} + | | \]
	Svolgiamo quindi i seguenti passi
	\begin{gather*}
		(q_0, \; \underline{\start} | + | | ) \to
		(q_0, \; \start \underline{|} + | | ) \\
		(q_0, \; \start \underline{|} + | | ) \to
		(q_0, \; \start | \underline{+} | | ) \\
		(q_0, \; \start | \underline{+} | | ) \to
		(q_1, \; \start | | \underline{|} | ) \\
		(q_1, \; \start | | \underline{|} | ) \to
		(q_1, \; \start | | | \underline{|} ) \\
		(q_1, \; \start | | | \underline{|} ) \to
		(q_1, \; \start | | | | \underline{\#} ) \\
		(q_1, \; \start | | | | \underline{\#} ) \to
		(q_2, \; \start | | | \underline{|} \# ) \\
		(q_2, \; \start | | | \underline{|} \# ) \to
		(h, \; \start | | | \underline{\#} \# )
	\end{gather*}
	per concludere che il calcolo termina dato che siamo
	giunti nello stato speciale $h$. Come possiamo vedere il
	nostro risultato è dato dal numero di $|$ rimanente.
\end{example}

Un altro tipico esempio di utilizzo di queste macchine è
quello di decidere se una stringa è palindroma. In questo
caso descriviamo brevemente quale sarebbe l'idea.
\begin{enumerate}
	\item Si controlla il primo carattere della stringa, lo si
	      sovrascrive con $\start$ e si passa in uno stato
	      specifico per quel carattere (supponendo che il primo
	      carattere sia $a$, si passa in $q_a$).
	\item Si arriva in fondo alla stringa e si controlla che
	      l'ultimo carattere sia uguale al primo, se sì, lo
	      si sovrascrive con un $\#$.
	\item Si torna indietro fino al primo $\start$ che si
	      incontra.
\end{enumerate}
Si ripete il procedimento finché non si esaurisce tutta la
stringa o finché non fallisce.


\section{Linguaggi FOR e WHILE}
Introduciamo ora un formalismo sicuramente a noi più
familiare, che è quello di un semplice linguaggio
\textbf{imperativo}, che dunque contenga il concetto di
\textbf{memoria} e di \textbf{comando}.

\subsection{Sintassi astratta}
La sintassi proposta è \emph{ambigua}, abbiamo dunque diverse
possibili interpretazioni per la stessa stringa. In realtà
tale sintassi viene elaborata tramite alberi, che hanno proprio
il compito di disambiguare.

Come vedremo a breve si tratta di una sintassi molto semplice,
comprendente costrutti condizionali di base, la possibilità
di eseguire cicli e semplici operazioni aritmetiche e logiche.
\begin{align*}
	E \to                                 &
	n | x | E_1 + E_2 | E_1 \times E_2 \vert E_1 - E_2
	                                      &
	\text{Espressioni aritmetiche}          \\
	B \to                                 &
	b | E_1 < E_2 | \lnot B | B_1 \lor B_2
	                                      &
	\text{Espressioni booleane}             \\
	C\to                                  &
	\text{skip} | x := E | C_1 ; C_2 | \text{if } B
	\text{ then } C_1 \text{ else } C_2 | &
	\text{Comandi}                          \\
	                                      &
	\text{for } x = E_1 \text{ to } E_2 \text{ do } C |
	\text{while } B \text{ do } C
\end{align*}
dove $n \in \N$, $x \in \Var$ (insieme numerabile di
variabili) e $b \in \text{Bool} = \{ tt, ff \}$.

D'ora in poi chiameremo WHILE, il linguaggio descritto dalla
grammatica BNF di sopra. Chiameremo invece FOR, il linguaggio
risultante dall'omissione del comando \verb|while| della
stessa grammatica.

\subsection{Semantica}
Prima di addentrarci nelle dinamiche del linguaggio appena
definito, dobbiamo definire la \textbf{memoria}. Nel nostro
caso lo facciamo tramite la funzione
\[ \sigma : \Var \to \N \]
che molto semplicemente, data una variabile in input
restituisce il suo valore in memoria (operazione di lettura).
Noi però siamo interessati anche a scrivere in memoria,
definiamo quindi l'operazione di \textbf{aggiornamento}
tramite la funzione, o meglio il funzionale a tre argomenti
\[
	- [ - / - ] : (\Var \to \N) \times
	\N \times \Var \to (\Var \to \N)
\]
che è definita come
\[
	\sigma [n / x](y) = \begin{cases}
		n         & \text{se } y = x  \\
		\sigma(y) & \text{altrimenti}
	\end{cases}
\]
che non ho capito cosa faccia. Probabilmente copre i due casi
in cui vogliamo aggiungere una nuova variabile alla memoria
e assegnargli un valore, e il caso in cui vogliamo cambiare
il valore di una variabile già presente in memoria.

La \textbf{semantica} di un'espressione aritmetica è data
dalla seguente funzione di \textbf{valutazione}, in cui andremo
a scrivere il suo argomento principale, l'espressione aritmetica,
tra le parentesi $[[$ e $]]$, cui viene giustapposto il secondo
argomento, cioè la memoria in cui l'espressione va valutata.
\[ \xi [[-]] - : E \times (\Var \to \N) \to \N \]
Proviamo a capire come questa funzione si comporta quando
prende in input espressioni aritmetiche.
\begin{align*}
	\xi [[n]] \sigma              & = n         \\
	\xi [[x]] \sigma              & = \sigma(x) \\
	\xi [[E_1 + E_2]] \sigma      &
	= \xi [[E_1]] \text{ più } [[E_2]] \sigma   \\
	\xi [[E_1 \times E_2]] \sigma &
	= \xi [[ E_1 ]] \text{ per } [[E_2]] \sigma \\
	\xi [[E_1 - E_2]] \sigma      &
	= \xi [[ E_1 ]] \text{ meno } [[E_2]] \sigma
\end{align*}
Per adesso limitiamoci a notare che se diamo alla funzione di
valutazione un numero $n$, qualunque sia la memoria, questa
ci restituirà il numero stesso.

Se invece gli passiamo una variabile, questa ci restituirà
il valore della variabile in memoria, si fa infatti uso della
funzione $\sigma$.

Se invece passiamo una qualche operazione aritmetica, questa
verrà valutata come la somma (oppure sottrazione o prodotto)
delle valutazioni degli operandi.

\begin{example}
	Proviamo a valutare l'espressione
	\[ x \times 2 - ((y - 7) + 1) \]
	nella memoria $\sigma$ tale che
	\[ \sigma (x) = 3 \quad \text{e} \quad \sigma(y) = 5 \]
	I passaggi per risolvere l'esercizio sono i seguenti
	\begin{align*}
		  & \xi [[ x \times 2 - ((y - 7) + 1) ]] \sigma \\
		= & \xi [[ x \times 2 ]] \sigma \text{ meno }
		\xi [[ (y - 7) + 1 ]] \sigma
	\end{align*}
	come possiamo notare, quando c'è un'ambiguità tra le
	precedenze supponiamo di sapere quale sia l'ordine
	corretto. In questo caso valutiamo "prima" il meno.
	\begin{align*}
		= & (\xi [[ x ]] \sigma \text{ per } 2) \text{ meno }
		\xi [[ (y - 7) + 1 ]] \sigma                          \\
		= & (\sigma(x) \text{ per } 2) \text{ meno }
		\xi [[ (y - 7) + 1 ]] \sigma                          \\
		= & (3 \text{ per } 2) \text{ meno }
		\xi [[ (y - 7) + 1 ]] \sigma
	\end{align*}
	ora abbiamo semplicemente preso il valore $3$ dalla memoria
	per sostituirlo a $x$.

	\begin{align*}
		= & 6 \text{ meno } (\xi [[y - 7]] \sigma
		\text{ più } 1)                                  \\
		= & 6 \text{ meno } ((\xi [[y]] \text{ meno } 7)
		\text{ più } 1)                                  \\
		= & 6 \text{ meno } ((\sigma(y) \text{ meno } 7)
		\text{ più } 1)
	\end{align*}
	Di seguito $5 - 7 = 0$ perché stiamo utilizzando il
	\emph{meno ridotto}, il quale ritorna $0$ quando il
	minuendo è maggiore o uguale del sottraendo.
	\begin{align*}
		= & 6 \text{ meno } ((5 \text{ meno } 7)
		\text{ più } 1)                          \\
		= & 6 \text{ meno } (0 \text{ più } 1)   \\
		= & 6 \text{ meno } 1 = 5
	\end{align*}
	Il calcolo per il resto è abbastanza banale.
\end{example}

Passiamo ora alla semantica delle operazioni booleane, che
non si inventano nulla di diverso rispetto alle espressioni
aritmetiche. Semplicemente la funzione di valutazione è
definita in modo da restituire valori booleani.
\[ \B[[-]] - : \B \times (\Var \to \N) \to \text{Bool} \]
Come prima andiamo a vedere cosa succede per vari input a
tale funzione di valutazione.
\begin{align*}
	\B [[t]] \sigma            & = tt                           \\
	\B [[f]] \sigma            & = ff                           \\
	\B [[E_1 < E_2]] \sigma    & =
	\xi [[E_1]] \sigma \text{ minore } \xi [[E_2]] \sigma       \\
	\B [[\lnot B]] \sigma      & = \text{ not } \B [[B]] \sigma \\
	\B [[B_1 \lor B_2]] \sigma & = \B [[B_1]] \sigma
	\text{ or } \B [[B_2]] \sigma
\end{align*}
Lo stile di definizione seguito fino ad ora prende il nome di
stile \textbf{denotazionale} e si propone di associare una
funzione a ciascun operatore.

\section{Problemi e funzioni}
Per il momento abbiamo usato i nostri costrutti per calcolare
una funzione (la somma per esempio) o per decidere
l'appartenenza di un elemento ad un insieme (decidere se una
stringa è palindroma).

In questa prima parte del corso andremo a definire meglio il
concetto di \textbf{problema} e di \textbf{funzione} che, nel
nostro caso, non hanno l'accezione cui siamo abituati.

Un esempio di \emph{problema} è la domanda: "qual è il massimo
comun divisore tra $x$ e $y$?". Se sostituiamo a $x$ e a $y$
dei valori, per esempio 34 e 98, otteniamo un \textbf{caso}
del problema.

\begin{definition} \label{def: T-calcolabile}
	Siano $\Sigma$, $\Sigma_0$ e $\Sigma_1$ tali che
	\[ \#, \start \notin \Sigma_0 \cup \Sigma_1 \]
	e
	\[ ]\Sigma_0 \cup \Sigma_1 \subset \Sigma \]
	allora diciamo che una funzione
	\[ f : \Sigma_0 \to \Sigma_1 \]
	è \textbf{Turing calcolabile} o \textbf{T-calcolabile},
	se e solo se
	\begin{gather*}
		\forall w \in \Sigma_0^* : f(w) = z \\
		\Updownarrow \\
		(q_0, \underline{\start} w) \to_M^*
		(h, \start z \underline{\#})
	\end{gather*}
	Si dice anche che $f$ è T-calcolabile se esiste una MdT
	$M$ che la calcola.
\end{definition}

Ora che abbiamo la definizione precisa di cosa sia una
funzione T-calcolabile proviamo a fare una cosa analoga per
il linguaggio WHILE che abbiamo definito in precedenza.

\begin{definition} \label{def: while-calcolabile}
	Diciamo che una funzione
	\[ f : \Var \to \N \]
	è \textbf{WHILE-calcolabile} oppure diciamo che un comando
	$C$ \textbf{calcola} $f$, se e solo se
	\begin{gather*}
		\forall \sigma \in \Var \to \N : f(x) = n \\
		\Updownarrow \\
		(C, \sigma) \to^* \sigma' \quad \land \quad
		\sigma'(x) = n
	\end{gather*}
\end{definition}

Notiamo che la variabile $x$ di input è anche la variabile
di output, ossia quella che contiene il risultato.

\subsection{Codifiche}
Ci chiediamo ora se per una funzione $f$ che non opera su
dati sotto formato di stringa, memorie o numeri naturali,
le nozioni di calcolabilità che abbiamo definito fino ad ora
sono ancora valide.

Se così non fosse dovremmo ridefinire ogni volta tali nozioni
per ogni dominio di ogni funzione con un formato differente da
quelli che abbiamo già incontrato.

Per superare il problema si fa uso di opportune
\textbf{codifiche} dei dati, ossia funzioni che svolgono
il seguente compito.
\begin{enumerate}
	\item Dato $x$ in formato $A$, lo si codifica in un formato
	      $B$ con cui riusciamo a calcolare.
	\item Si applica la MdT a $y$ e si ottiene il risultato $z$
	      (se la computazione termina) in formato $B$.
	\item Si traduce $z$ dal formato $B$ al formato $A$.
\end{enumerate}
D'ora in avanti considereremo solo i numeri naturali come i
nostri dati. Abbiamo però bisogno che la funzione di codifica
sia \emph{biunivoca}.

\begin{example}
	La seguente funzione codifica coppie di naturali come un
	singolo naturale ed è detta codifica a
	\textbf{coda di colomba}.
	\[ (x, y) \to \frac{1}{2} (x^2 + 2 x y + y^2 + 3 x + y) \]
	la cui decodifica, ossia la funzione inversa è la seguente
	\[
		n \to (n - \frac{1}{2} k \cdot (k + 1),
		k - (n - \frac{1}{2} k \cdot (k + 1))
	\]
	dove $k=\lfloor \frac{1}{2}(\sqrt{1+8\cdot n}-1)\rfloor$.
\end{example}

A prescindere dall'esempio, possiamo dire che le proprietà
basilari dei formalismi e della classe di funzioni calcolate,
non cambiano al variare del formato dei dati su cui operano.

\begin{definition} \label{def: funzione totale}
	Diciamo che una funzione $f : A \to B$, sottoinsieme di
	$A \times B$ è una \textbf{funzione totale} se e solo se
	\begin{itemize}
		\item La funzione è \emph{definita ovunque}, ossia se
		      $\forall a \in A$, $\exists b \in B$ tale che la
		      coppia $(a, b) \in f$.
		\item Vi è \emph{unicità}, ossia se, date le coppie
		      $(a, b) \in f$ e $(a, c) \in f$, allora $b=c$.
	\end{itemize}
\end{definition}

Una funzione può essere calcolabile ma non totale, per esempio
la macchina di Turing che non termina mai vista all'inizio del
corso.

\begin{definition} \label{def: funzione parziale}
	Diciamo che una funzione $f : A \to B$ è \textbf{parziale}
	se è un sottoinsieme di $A \times B$ tale che
	\begin{itemize}
		\item Vi è \emph{unicità}, ossia se, date le coppie
		      $(a, b) \in f$ e $(a, c) \in f$, allora $b=c$.
		\item Esiste al più un $b \in B$ tale che $f(a) = b$.
	\end{itemize}
	e quindi non si richiede che $f$ sia definita ovunque.
\end{definition}

Introduciamo ora un po' di notazione utile a quello che faremo
più avanti. Data una funzione $f : A \to B$
\begin{itemize}
	\item Diremo che $f$ è \textbf{definita} o
	      \textbf{converge su $a$} ($f(a) \downarrow$) se
	      $\exists b$ tale che $(a, b) \in f$ (cioè
	      $f(a) = b$).
	\item Diremo che $f$ \textbf{non è definita} o che
	      \textbf{diverge} ($f(a) \uparrow$) se $\nexists b$
	      tale che $(a, b) \in f$.
\end{itemize}
Chiamiamo inoltre
\begin{itemize}
	\item \textbf{Dominio} di $f$ l'insieme
	      \[ \{ a \in A | f(a) \downarrow \} \]
	      che coincide con lo spazio di partenza ($A$) se e
	      solo se la funzione è totale.
	\item \textbf{Codominio} di $f$ l'insieme $B$.
	\item \textbf{Immagine} di $f$ l'insieme
	      \[ \{ b \in B | \exists a \in A : f(a) = b \} \]
\end{itemize}

Detto questo vogliamo capire qual è la relazione tra funzioni
e algoritmi. Una funzione possiamo vederla come un insieme di
coppie (\emph{argomento}, \emph{risultato}) (o (\emph{input},
\emph{output}) se preferiamo la notazione più informatica) ma
non ci dice come il risultato (o l'output) venga calcolato.

Di conseguenza non ci sono due funzioni diverse che per uno
stesso argomento restituiscono lo stesso risultato. In termini
insiemistici possiamo dire che non esistono due insiemi diversi
che hanno gli stessi elementi.
\begin{tcolorbox}
	Un algoritmo è invece una \textbf{rappresentazione finita}
	di una funzione, in quanto specifica come si calcola il
	risultato a partire dall'argomento. In questo caso possiamo
	certamente avere più algoritmi che calcolano la stessa
	funzione.
\end{tcolorbox}

\subsection{Funzioni calcolabili}
D'ora in avanti proveremo a capire
\begin{itemize}
	\item Quali sono le \textbf{funzioni calcolabili} e di
	      quali proprietà godono.
	\item Se esistono funzioni totali o parziali che non sono
	      calcolabili. Ovvero per cui si dimostra che non esiste
	      un algoritmo che le calcoli.
\end{itemize}

\begin{example}
	Prendiamo ora come esempio la
	\textbf{congettura di Goldbach}, la quale ci dice che ogni
	numero pari maggiore di 2 è esprimibile come somma di due
	numeri primi.

	Da questa congettura (mai dimostrata) nasce la
	\textbf{funzione di Goldbach}, definita come segue con
	$gb : \N \to \N$
	\[
		gb(n) = \begin{cases}
			0 & \text{se la congettura è vera} \\
			1 & \text{altrimenti}
		\end{cases}
	\]
	La congettura non è stata ancora dimostrata ma un algoritmo
	per calcolarla esiste, solo che non sappiamo quale sia.

	Se ad esempio volessimo decidere se la funzione è
	T-calcolabile, basterebbe prendere una MdT che ritorna
	sempre 0 se la congettura è vera o una MdT che ritorna
	sempre 1 se è falsa. Il problema è che fin tanto che la
	congettura non è dimostrata, non sappiamo quale delle due
	scegliere.
\end{example}


\section{Funzioni ricorsive primitive}
Iniziamo con il definire due delle funzioni definite per
\emph{ricorrenza} più popolari, il fattoriale e la successione
di Fibonacci.

Iniziamo con il \emph{fattoriale}, definito come una coppia
di equazioni, la prima per il caso base, ossia quando $x = 0$
e la seconda per tutti gli altri casi, ossia per ogni $x > 0$.
\[
	!(x) = \begin{cases}
		1        & \text{se } x = 0 \\
		!(n - 1) & \text{se } x > 0
	\end{cases}
\]
Proviamo a scrivere una versione WHILE e una versione FOR del
fattoriale
\begin{verbatim}
    fatt = 1;
    while x > 0 do
        fatt = fatt * x;
        x = x - 1;
\end{verbatim}
in questo caso salviamo il risultato nella variabile
\verb|fatt| che è la stessa che usiamo per ritornare il
risultato.

La \emph{successione di Fibonacci} presenta invece due casi
base e un caso definito per ricorrenza ed è definita come
segue
\[
	fib(x) = \begin{cases}
		0                       & \text{se } x = 0 \\
		1                       & \text{se } x = 1 \\
		fib(x - 1) + fib(x - 2) & \text{se } x > 1
	\end{cases}
\]
Ora vogliamo capire quali sono le regole per formare bene
delle formule ricorsive ed è qui che prenderemo un po' di
notazione in prestito dal $\lambda$-calcolo. Nel caso del
fattoriale possiamo scrivere
\[ \lambda x . !(x) \]
per dire che il fattoriale dipende solo da $x$. Possiamo anche
scrivere un qualcosa di questo tipo
\[ \lambda x . x + y \]
per definire una funzione che prende $x$ e restituisce una
funzione che dipende da $y$. Abbiamo quindi un modo per
\emph{costruire} delle funzioni specificando esattamente quali
sono i suoi argomenti.

\begin{definition}
	Le \textbf{funzioni ricorsive primitive} sono la minima
	classe $\C$ da $\N^n$, con $n \geq 0$, in $\N$ cui
	appartengono
	\begin{itemize}
		\item \textbf{Zero}: una funzione che prende $k \geq 0$
		      di argomenti e ritorna 0.
		      \[ \lambda x_1, \dots, x_k . 0 \]
		\item \textbf{Successore}: che prende un argomento solo
		      e restituisce il suo successore
		      \[ \lambda x . x + 1 \]
		\item \textbf{Identità}: che prende $k$ argomenti e
		      ritorna l'argomento $i$-esimo con $1\leq i\leq k$.
		      \[ \lambda x_1, \dots, x_k . x_i \]
		      Viene anche chiamata \textbf{proiezione}.
	\end{itemize}
	Questi sono anche detti \textbf{schemi primitivi di base}.
	La classe $\C$ che stiamo provando a definire è inoltre
	\emph{chiusa} per
	\begin{itemize}
		\item \textbf{Composizione}: Se $g_1, \dots, g_k \in \C$
		      sono funzioni in $m$ variabili, e $h \in \C$ è
		      una funzione in $k$ variabili, anche la loro
		      composizione
		      \[
			      \lambda x_1, \dots x_m .
			      h(g_1(x_1, \dots, x_m), \dots,
			      g_k(x_1, \dots, x_m)
			      )
		      \]
		      appartiene a $\C$.
		\item \textbf{Ricorsione primitiva}: Se $h \in \C$
		      è una funzione in $k+1$ variabili, $g \in \C$
		      è una funzione in $k-1$ variabili definita da
		      \[
			      \begin{cases}
				      f(0, x_2, \dots, x_k)       & =
				      g(x_2, \dots, x_k)              \\
				      f(x_1 + 1, x_2, \dots, x_k) & =
				      h(x_1, f(x_1, \dots, x_k),
				      x_2, \dots, x_k)
			      \end{cases}
		      \]
	\end{itemize}
\end{definition}

\begin{tcolorbox}
	Dato che $\C$ è la \emph{minima} classe che soddisfa le
	condizioni espresse sopra, affinché $f$ sia ricorsiva
	primitiva, occorre e basta che sia una successione finita,
	o \textbf{derivazione}, della seguente forma
	\[ f_1, \dots, f_n \]
	tale che $f = f_n$ e per ogni $i$ tale che
	$1 \leq i \leq n$ vale uno dei seguenti casi:
	\begin{itemize}
		\item $f_i \in C$ è uno \emph{Zero} o è
		      l'\emph{Identità}.
		\item $f_i$ è ottenibile tramite l'applicazione delle
		      regole di \emph{Combinazione} e
		      \emph{Ricorsione primitiva} da $f_j$ con $j < i$.
	\end{itemize}
\end{tcolorbox}

Tra i requisiti che abbiamo appena descritto perché una
funzione venga definita \emph{ricorsiva primitiva}, quello
meno intuitivo è sicuramente è quello che riguarda proprio
la ricorsione primitiva.

Come vediamo, noi possiamo definire una funzione di se stessa,
ma con delle limitazioni. Come possiamo vedere abbiamo un
primo caso in cui il primo argomento è $0$ e non c'è una
chiamata ricorsiva, siamo quindi davanti ad un caso base.

Più complesso è il secondo caso, in cui diciamo che la funzione
su $k$ argomenti ritorna il primo argomento decrementato,
effettua una chiamata ricorsiva su tutti gli argomenti e lascia
i rimanenti invariati.

Resistiamo per un esempio e proviamo a fare il punto della
situazione. Prima di fare l'esempio definiamo la somma tramite
le regole appena descritte.
\[
	\begin{array}{ll}
		f_1 & = \lambda x.x                    \\
		f_2 & = \lambda x.x + 1                \\
		f_3 & = \lambda x_1, x_2, x_3 . x_2    \\
		f_4 & = f_2 (f_3 (x_1, x_2, x_3))      \\
		f_5 & = \begin{cases}
			        f_5 (0, x_2)     & = f_1 (x_2) \\
			        f_5 (x + 1, x_2) & =
			        f_4 (x_1, f_5(x_1, x_2), x_2)
		        \end{cases}
	\end{array}
\]
Qui l'idea, avendo come unica funzione di somma la funzione
successore, è quella di calcolare il successore del primo
numero tante volte quanto è il secondo numero.

\begin{example}
	Proviamo a calcolare $2 + 3$ con la somma che abbiamo
	appena definito
	\[
		\begin{array}{l}
			f_5(2, 3) =                         \\
			f_4 (1, f_5(1, 3), 3) =             \\
			f_4 (1, f_4(0, f_5 (0, 3), 3), 3) = \\
			f_4 (1, f_4(0, f_1 (3), 3), 3) =    \\
			f_4 (1, f_4(0, 3, 3), 3) =          \\
			f_4 (1, f_2(f_3(0, 3, 3)), 3) =     \\
			f_4 (1, f_2(3), 3) =                \\
			f_4 (1, 4, 3) =                     \\
			f_2 (f_3 (1, 4, 3)) =               \\
			f_2 (4) = 5
		\end{array}
	\]
	Applicare la formula non è niente di difficile e forse
	non è più di tanto istruttivo.
\end{example}

Dunque quello che interessa principalmente sapere a noi è 
cosa sia una funzione ricorsiva primitiva e a quali regole 
sottostare per definirne una.



\end{document}
