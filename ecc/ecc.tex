\documentclass[12pt]{report}

\usepackage[T1]{fontenc}
\usepackage[italian]{babel}
\usepackage[hidelinks]{hyperref}
\usepackage{tikz, pgfplots}
\usepackage{graphicx}
\usepackage{tabularx}

\hypersetup {
    colorlinks=true,
    urlcolor=blue,
    linkcolor=blue
}

% --------------- STYLE ---------------
\usepackage[margin=1.25in]{geometry}
\usepackage{xcolor}

\usepackage{fancyhdr}
\usepackage[Sonny]{fncychap}
\usepackage[most]{tcolorbox}

% Font
\renewcommand{\familydefault}{\sfdefault}
\usepackage{sansmath}
\sansmath

\pagestyle{fancy}
\setlength{\headheight}{15pt}
\rhead{\thepage}


% --------------- MATH ---------------
\usepackage{amsmath, amssymb, amsthm, amsfonts, mathtools}

\usepackage{mdframed}
\newtheoremstyle{th_style}
{0pt}{0pt}
{\normalfont}
{}
{\color{green!40!black}}
{\;}{0.25em}
{\thmname{\textbf{#1}}\thmnumber{ \textbf{#2}}{\color{black}\thmnote{\textbf{ -- #3}}}}

\newmdenv[
	rightline=false,
	leftline=true,
	topline=false,
	bottomline=false,
	linecolor=green!40!black,
	innerleftmargin=5pt,
	innerrightmargin=5pt,
	innertopmargin=0pt,
	innerbottommargin=0pt,
	leftmargin=0cm,
	rightmargin=0cm,
	linewidth=4pt
]{dBox}

\newmdenv[
	rightline=false,
	leftline=true,
	topline=false,
	bottomline=false,
	linecolor=green!40!black,
	backgroundcolor=black!5,
	innerleftmargin=5pt,
	innerrightmargin=5pt,
	innertopmargin=5pt,
	innerbottommargin=5pt,
	leftmargin=0cm,
	rightmargin=0cm,
	linewidth=4pt
]{pBox}

\theoremstyle{th_style}
\newtheorem{theoremeT}{Teorema}[chapter]
\newtheorem{definitionT}{Definizione}[chapter]
\newtheorem{propositionT}{Proposizione}[chapter]
\newtheorem{corollary}{Corollario}[chapter]
\newtheorem{lemma}{Lemma}[chapter]
\newtheorem{observation}{Osservazione}[chapter]
\newtheorem{exampleT}{Esempio}[section]

\newenvironment{theorem}{\begin{pBox}\begin{theoremeT}}{\end{theoremeT}\end{pBox}}
\newenvironment{definition}{\begin{dBox}\begin{definitionT}}{\end{definitionT}\end{dBox}}
\newenvironment{proposition}{\begin{pBox}\begin{propositionT}}{\end{propositionT}\end{pBox}}
\newenvironment{example}{\begin{dBox}\begin{exampleT}}{\end{exampleT}\end{dBox}}


\pgfplotsset{compat=newest}

\newcommand{\N}{\mathbb{N}}
\newcommand{\Z}{\mathbb{Z}}
\newcommand{\R}{\mathbb{R}}
\newcommand{\RR}{\mathcal{R}}
\newcommand{\C}{\mathcal{C}}
\newcommand{\F}{\mathbb{F}}
\newcommand{\K}{\mathcal{K}}
\newcommand{\B}{\mathcal{B}}

\newcommand{\tx}{\tilde{x}}
\newcommand{\norm}[1]{\left\lVert#1\right\rVert}
\newcommand{\start}{\triangleright}

\DeclareMathOperator{\sign}{sign}
\DeclareMathOperator{\trn}{trn}
\DeclareMathOperator{\arr}{arr}
\DeclareMathOperator{\fl}{fl}
\DeclareMathOperator{\dist}{dist}
\DeclareMathOperator{\Ker}{Ker}
\DeclareMathOperator{\diag}{diag}
\DeclareMathOperator{\nnz}{nnz}
\DeclareMathOperator{\Var}{Var}

\title{Elementi di calcolabilità e complessità}
\author{Federico Bustaffa}
\date{15/04/2024}

\begin{document}

\maketitle

\begin{abstract}
	Lo scopo di questi appunti è rendere la comprensione dei
	concetti spiegati all'interno del corso il più semplice
	possibile fornendo esempi e spiegazioni quasi ad un livello
	divulgativo per quanto mi sia possibile.

	Non mancheranno il rigore e i formalismi spiegati a lezione
	una volta definite le basi per poterli comprendere.

	Ovviamente questi appunti rispecchiano la mia personale
	interpretazione dei concetti mostrati all'interno del corso
	e dunque anche i miei limiti di apprendimento degli stessi.

	Il corso sarà suddiviso in due parti: la prima tratterà la
	teoria della \textbf{calcolabilità}, mentre la seconda la
	teoria della \textbf{complessità}.
\end{abstract}

\tableofcontents

\chapter{Introduzione}

Il corso tratterà le modalità di reperimento e analisi di dati da blockchain tramite Python. Si compirà un'analisi
statistica sui dati raccolti tramite \emph{scraping} o API di vario genere per riuscire ad estrapolare informazioni
sulle blockchain prese in esame.

Saranno necessarie nozioni di statistica descrittiva ed inferenziale e si farà uso di strumenti per l'analisi di
grafi.

Le librerie Python necessarie per lavorare al meglio in questo ambito sono:
\begin{itemize}
	\item \verb|BeautifulSoup|, \verb|Scrapy|: web scraping
	\item \verb|Pandas|: gestione di tabelle
	\item \verb|Numpy|: algebra vettoriale e matriciale
	\item \verb|SciPy|: statistica
	\item \verb|Matplotlib|: visualizzazione di grafici
	\item \verb|NetworkX|: analisi di grafi
\end{itemize}
Per installare le librerie scaricare il package manager \verb|pip| di Python:
\begin{verbatim}
	> sudo apt install python3-pip
\end{verbatim}
Installazione delle librerie
\begin{verbatim}
	> pip install beautifulsoup4
	> pip install scrapy
	> pip install pandas
	> pip install numpy
	> pip install scipy
	> pip install matplotlib
	> pip install networkx
\end{verbatim}

\section{Blockchain}
Una \textbf{blockchain} utilizza un modello computazionale differente dal classico modello \textbf{client-server},
si basa infatti sul sistema \textbf{peer-to-peer}.

In questo modello ogni nodo possiede una frazione del potere necessario a compiere determinate azioni all'interno
della blockchain. Il sistema nasce infatti per eliminare l'entità centrale che controlla e gestisce l'intero servizio.

Possiamo vedere una blockchain come un \textbf{registro} \emph{distribuito} e \emph{replicato} tra i nodi di una
rete \emph{peer-to-peer}. In altre parole ogni nodo della rete possiede una copia della blockchain uguale a quella
di tutti gli altri nodi.

Una proprietà fondamentale al fine di mantenere la blockchain consistente e \textbf{immutabile} è la
\textbf{tamper freeness} garantita tramite meccanismi crittografici e algoritmi basati sul consenso.


\section{Algoritmo genetico}

Cerchiamo ora di definire un algoritmo genetico sequenziale che sia una buona
base per quello che andremo a fare dopo. Si vuole infatti costruire un modello
su cui basare le possibili implementazioni che andremo a studiare.

\subsection{API e utilizzo}

Per quanto riguarda l'API messa a disposizione dell'utente vogliamo un qualcosa
che sia il più semplice possibile e che prevenga errori o cattiva gestione
delle strutture dati, fondamentali per la corretta esecuzione dell'algoritmo.

L'idea sarebbe quella di istanziare l'algoritmo come un oggetto, al quale
verranno forniti i vari metodi che compongono un algoritmo genetico.

\begin{minted}{py}
from genetic import GeneticAlgorithm

if __name__ == "__main__":
	ga = GeneticAlgorithm(
		population_size
		gen_func,
		selection_func,
		crossover_func,
		mutation_func,
		fitness_func,
		replacement_func,
		convergence_func
	)

	ga.run()
	results = ga.get()
\end{minted}

In questo modo non si espongono le strutture dati all'esterno della classe se
non come copie o come \emph{viste} dell'originale. Questo rende anche possibile
il riutilizzo di strutture dati già allocate andando rendere più efficiente
l'algoritmo in un senso che sarà più chiaro andando avanti.

\subsection{Cromosomi}

Prima di definire le varie fasi dell'algoritmo, introduciamo brevemente come
vengono rappresentati i \textbf{cromosomi}.

\begin{minted}{py}
class Chromosome:
	def __init__(self, values, fitness) -> None:
		self.values = values
		self.fitness = fitness
\end{minted}

Ogni cromosoma rappresenta un individuo della popolazione, il quale è
generalmente identificato da un vettore di valori numerici e da un valore di
fitness.

Noi tratteremo solo casi in cui abbiamo vettori numerici, sarà poi compito del
programmatore mappare il cromosoma in ciò che gli serve per risolvere il
problema di interesse.

\subsection{Generazione della popolazione iniziale}

In questa prima fase andiamo a generare una popolazione iniziale di $N$
individui in modo del tutto casuale. Evitare di generare duplicati è buona
norma, almeno in questa fase, così da garantire un alto grado di
\emph{biodiversità} iniziale.

\begin{minted}{py}
def __generate(self) -> None:
	for i in range(N):
		values = self.gen_func()
		while values in population:
			values = self.gen_func()
		self.population[i].values = values
\end{minted}

Per la fase iniziale di generazione, lasciamo al programmatore il compito di
definire come viene generato il singolo cromosoma. Sarà poi il modulo ad
occupersi di gestire i duplicati e memorizzare la popolazione.

\subsection{Selezione}

Le possibili implementazioni della fase di selezione degli individui per
l'accoppiamento sono molto varie e potrebbero non avere parti in comune.

Prendiamo ad esempio la selezione a \emph{torneo} in cui si scelgono a due a
due degli individui e quello con fitness più alta viene scelto.

In una soluzione a \emph{roulette} invece gli individui vengono scelti
singolarmente e volta per volta. In ogni caso si lavora sempre sull'intera
popolazione e non ci sono parti facilmente \emph{generalizzabili}.

Ecco perché in questa fase si richiede al programmatore di implementare la fase
di selezione per intero.

\begin{minted}{py}
def __selection(self):
	self.selected = self.selection_func(self.population)
\end{minted}

Ciò che si richiede è che la funzione di selezione ritorni una lista di
puntatori agli individui selezionati.

\subsection{Crossover}

La fase di crossover si rivelerà essere una delle più costose ed è quindi
necessario implementare delle ottimizzazioni già nella progettazione
dell'algoritmo sequenziale.

Per la fase di crossover facciamo alcune assunzioni, necessarie per riuscire
a parallelizzare l'algoritmo in seguito. Vogliamo infatti lasciare al
programmatore solamente il compito di definire l'operatore di crossover, il
quale prende in input solo due individui e restituisce (in genere due) nuovi
individui. Il come vengono formate le coppie viene lasciato alle routine
interne della classe che stiamo definendo.

\begin{minted}{py}
def __crossover(self) -> None:
	for i in range(0, len(self.selected), 2):
		father, mother = random.choices(self.selected, k=2)

		offspring1, offspring2 = self.crossover_func(father, mother)
		self.offsprings[i] = offspring1
		self.offsprings[i+1] = offspring2

		self.selected.remove(father)
		try:
			selected.remove(mother)
		except ValueError:
			pass
\end{minted}

Altra assunzione che fa l'algoritmo in questione è che il numero di nuovi
individui generati non cambia mai. Questo ci permette di inizializzare una
lista \verb|offsprings| a cui poi possiamo accedere tramite indice e senza
quindi andare riallocare di continuo memoria per i nuovi individui generati.

\subsection{Mutazione}

Per quanto riguarda la fase di mutazione si lascia al programmatore il compito
di definire una funzione che prende in input un cromosoma e applica una qualche
tecnica di mutazione.

\begin{minted}{py}
def __mutation(self) -> None:
	for i in range(len(self.offsprings)):
		if random.random() < self.mutation_rate:
			self.offsprings[i] = self.mutation_func(self.offsprings[i])
\end{minted}

Il modulo si occupa di gestire la probabilità che ogni individuo ha di essere
mutato.

\subsection{Valutazione}

Fase abbastanza semplice ma che costituisce uno dei colli di bottiglia maggiori
per le prestazioni

\begin{minted}{py}
def __evaluation(self) -> None:
	for i in range(len(self.offsprings)):
		self.offsprings[i].fitness = self.fitness_func(self.offsprings[i].values)
\end{minted}

Anche qui l'algoritmo non fa altro che applicare la funzione di fitness definita
dall'utente ad ogni nuovo individuo della popolazione.

\section{Macchina di Turing}
Introduciamo il primo formalismo per esprimere algoritmi, ideato
da Alan Turing nel 1936 e che prende il nome di
\textbf{macchina di Turing}.

L'idea di base è quella di avere un \textbf{nastro} di lunghezza
infinita su cui disporre una stringa di cratteri, rappresentante
l'input dell'algoritmo. Tale macchina è dotata di un
\textbf{cursore} che, partendo dall'inizio della stringa, legge
i caratteri che incontra e, in base a cosa legge, cambia stato
e si muove. La computazione termina quando si giunge finalmente
in uno stato speciale di terminazione.

\begin{definition}
	Una \textbf{macchina di Turing} (MdT) è definita come una
	quadrupla di questo tipo
	\[
		M = \begin{pmatrix}
			Q, & \Sigma, & \delta, & q_0
		\end{pmatrix}
	\]
	dove
	\begin{itemize}
		\item $Q$ è l'\textbf{insieme finito degli stati} in
		      cui si può trovare la macchina. Tra gli stati
		      abbiamo uno stato speciale $h$, con cui
		      indicheremo la corretta terminazione del calcolo
		      della macchina $M$.
		\item $\Sigma = \{ \sigma, \sigma', \dots \}$ è
		      l'\textbf{insieme finito dei simboli}, ossia,
		      l'\textbf{alfabeto} utilizzato per esprimere
		      gli algoritmi. Il numero e e i simboli stessi
		      possono variare da una MdT all'altra, tuttavia
		      ci sono dei simboli considerato speciali che sono
		      sempre presenti:
		      \begin{itemize}
			      \item \textbf{Bianco}: indicato con $\#$ è
			            come un carattere nullo.
			      \item \textbf{Respingente}: indicato con
			            $\start$, simboleggia l'inizio delle
			            stringa. Il cursore della macchina non
			            può andare mai a sinistra di questo
			            simbolo.
			      \item \textbf{Spostamento}: simboli che non
			            appartengono a $\Sigma$ e indicano come
			            e in che direzione muovere il cursore.
			            Sono rispettivamnete $L$, $R$ e $-$ che
			            stanno per \verb|Left|, \verb|Right| e
			            \verb|Hold| (non muovere il cursore).
		      \end{itemize}
		\item \textbf{Stato iniziale}: indicato con $q_0 \in Q$
		      è lo stato iniziale in cui si trova la macchina.
		\item \textbf{Funzione di transizione}: indicata con
		      $\delta$ è la funzione che definisce la
		      transizione da uno stato all'altro della macchina
		      in base a quel che viene letto dal cursore e in
		      base a quel che è stato letto fino a quel momento.
		      \[
			      \delta \subseteq (Q \times \Sigma) \to
			      (Q \cup \{ h \}) \times \Sigma \times
			      \{ L, R, - \}
		      \]
		      che ci permette per l'appunto di cambiare lo
		      stato della macchina e progredire nel calcolo.
	\end{itemize}
\end{definition}

\subsection{Funzione di transizione}
La \textbf{funzione di transizione} $\delta$ è necessaria a
far progredire il calcolo. Come possiamo vedere dalla
definizione che ne abbiamo dato, questa prende in input una
coppia di valori.
\begin{itemize}
	\item Lo \textbf{stato corrente} $q$ della macchina.
	\item Un simbolo $\sigma$ dell'alfabeto
\end{itemize}

Notiamo inoltre che la funzione $\delta$ prende in input una
coppia di valori ma ritorna una tripla composta da uno stato
$q'$, che può essere anche $h$ in caso il calcolo sia terminato
con successo, un simbolo $\sigma'$ e una tra le 3 mosse $L$,
$R$ e $-$.

Ad ogni passo del calcolo, $\delta$ elabora la coppia in input
e restituisce una tripla contente il nuovo stato, il nuovo
simbolo da scrivere alla posizione attuale del cursore e come
ci si deve muovere al passo successivo.

Qualcuno potrebbe aver notato la similitudine con un automa a
stati finiti. In quanto vi è sempre uno stato corrente e, in
base a quello e ad una nuova parte dell'input ci si muove in
un nuovo stato o si rimane nello stesso. Non è tuttavia
corretto dire che una MdT è un automa.

\subsubsection{Considerazioni}
Ora che abbiamo le idee più chiare, facciamo qualche
considerazione in più su $\delta$. Essa è \textbf{iniettiva},
vale cioè che, prese due triple $(q', \sigma', D')$ e
$(q'', \sigma'', D'')$ tali che
\begin{gather*}
	\delta (q, \sigma) =  (q', \sigma', D') \\
	\delta (q, \sigma) = (q'', \sigma'', D'')
\end{gather*}
allora $q' = q''$, $\sigma' = \sigma''$ e $D' = D''$.
Questo ci dice sostanzialmente che, dato uno stato e un simbolo
c'è un solo altro stato in cui possiamo andare. Un'altra cosa
da specificare è che per $\delta$ vale sempre che se
\[ \delta(q, \start) = (q', \sigma, D) \]
allora $\sigma = \start$ e $D = R$. Questo ci dice che se ci
troviamo all'inizio del nastro possiamo andare solo a destra.

Ritornando velocemente
all'\hyperref[sec: algoritmo]{idea intuitiva di algoritmo},
possiamo facilmente verificare la prima e seconda condizione
sono verificate poiché, sia $Q$ che $\Sigma$ sono insiemi
finiti e di conseguenza anche $\delta$ contiene un numero
finito di elementi.

\subsection{Alfabeto}
I dati su cui opera una MdT sono stringhe $w$ di caratteri
appartenenti a $\Sigma$, più precisamente $w \in \Sigma^*$,
dove $\Sigma^*$ comprende anche la stringa vuota $\epsilon$.

Senza stare a incasinarsi con inutili formalismi matematici,
se $\Sigma$ è l'alfabeto, allora $\Sigma^*$ è l'insieme di
tutte le possibili stringhe generabili con quell'alfabeto e
la stringa vuota.

\begin{example}
	Prendiamo ad esempio l'alfabeto binario
	\begin{gather*}
		\Sigma = \{ 0, 1 \} \\
		\Downarrow \\
		\Sigma^* = \{ \epsilon, 0, 1, 01, 10, 11, \dots \}
	\end{gather*}
\end{example}

\subsection{Computazione}
Finalmente vediamo come opera una MdT su qualche problema di
esempio. La \textbf{configurazione corrente} di una MdT viene
identificata $(q, u, \sigma, v)$ dove
\begin{itemize}
	\item $q$ è lo stato corrente.
	\item $u$ è la stringa a sinistra del carattere attuale.
	\item $\sigma$ è il carattere attuale.
	\item $v$ è il resto della stringa che termina con un
	      carattere non nullo.
\end{itemize}
La situazione corrente di una MdT può essere espressa più
comodamente con $(q, u \underline{\sigma} v)$. Graficamente
una MdT appare in questo modo
\begin{center}
	\includegraphics[scale=0.225]{images/turing.png}
\end{center}
Tenendo a mente questa figura possiamo provare a costruire la
nostra prima MdT.

\begin{example} \label{ex: 11}
	Vogliamo costruire una MdT in grado di dirci se la stringa
	binaria in input contiene una sottosequenza composta da
	due $1$ consecutivi.

	Per costruire la nostra macchina abbiamo bisogno di due
	stati ($q_0$ e $q_1$), di due simboli ($0$ e $1$) e di
	una funzione di transizione $\delta$.

	Per capire se la stringa in input contiene almeno una
	sottostringa composta da due $1$ consecutivi, iniziamo
	con lo stato iniziale $q_0$, il quale indica sia lo stato
	inziale della macchina sia lo stato in cui la macchina
	deve transire ogni qual volta incontra uno $0$.

	Abbiamo poi bisogno di uno stato $q_1$ in cui la macchina
	transisce quando incontra un $1$ nella sequenza dopo essere
	stata in uno stato $q_0$.

	Se ci troviamo nello stato $q_1$ e incontriamo un altro $1$
	la macchina transisce nello stato $h$ di terminazione. La
	funzione $\delta$ di transizione che deriva dai seguenti
	ragionamenti è la seguente
	\begin{center}
		\begin{tabular}{|c|c|c|}
			\hline
			$q$   & $\sigma$ & $\delta(q, \sigma)$ \\
			\hline
			$q_0$ & $\start$ & $q_0, \start, R$    \\
			$q_0$ & $0$      & $q_0, \#, R$        \\
			$q_0$ & $1$      & $q_1, \#, R$        \\
			$q_1$ & $0$      & $q_0, \#, R$        \\
			$q_1$ & $1$      & $h, \#, -$          \\
			\hline
		\end{tabular}
	\end{center}
	Come possiamo notare, ogni volta che incontriamo un
	carattere lo "cancelliamo" scrivendo $\#$ ma avremmo potuto
	lasciare il numero che incontravamo. Vediamo una possibile
	simulazione di esecuzione con la seguente stringa binaria
	in input
	\[ 01011 \]
	La nostra configurazione iniziale è
	\[ q_0 / \underline{\start} 01011 \]
	dove il simbolo sottolineato è quello su cui si trova il
	cursore. La sequenza di operazioni sarà quindi la seguente
	\begin{align*}
		q_0 / & \underline{\start} 01011       \\
		q_0 / & \start \underline{0} 1011      \\
		q_0 / & \start \# \underline{1} 011    \\
		q_1 / & \start \#\# \underline{0} 11   \\
		q_0 / & \start \#\#\# \underline{1} 1  \\
		q_1 / & \start \#\#\#\# \underline{1}  \\
		h /   & \start \#\#\#\# \underline{\#}
	\end{align*}
	Il calcolo è dunque terminato con successo.
\end{example}

Per vedere una MdT in azione è possibile visitare questo
\href{https://turingmachinesimulator.com/}{sito} in cui è
possibile programmare una MdT oppure eseguire esempi già
proposti.

\subsubsection{Configurazione e computazione}
Come abbiamo detto precedentemente, una \textbf{configurazione}
è definita dalla quadrupla
\[ \gamma = (q, u, \sigma, v) \]
Ciò che non abbiamo detto è che gli elementi di $\gamma$
appartengono al seguente insieme
\[
	\gamma \in (Q \cup \{ h \}) \times
	\Sigma^* \times \Sigma \times \Sigma^F
\]
L'ultimo insieme ($\Sigma^F$) è un po' particolare, è infatti
definito come
\[
	\Sigma^F = \Sigma^* \cdot \left( \Sigma \backslash
	\{ \# \} \right) \cup \{ \epsilon \}
\]
quindi possiamo scrivere la stringa $v$ come
$\sigma_0 \sigma_1 \dots \sigma_n$, con $\sigma_n \neq \#$,
al posto della stringa infinita composta dai vari $\sigma$ e
con infiniti $\#$ alla fine.

Si noti però che un qualsiasi carattere $\sigma_i$ con $i < n$
può essere $\#$ ed inoltre la stringa $u$ può essere vuota
solo quando il carattere corrente è $\start$. La convenzione
impone quindi di scrivere la configurazione iniziale del primo
esempio fatto, in questo modo
\[ (q_0, \start 01011, \epsilon) \]

\begin{definition}
	Una \textbf{computazione} è una successione finita di
	passi
	\[ (q_0, w) \to^* (q', w') \]
	dove $\to^*$ è la chiusura riflessiva e transitiva di
	$\to$. Ovviamente se vi sono $n$ passi, la computazione
	è lunga $n$ e dunque scriveremo $\to^n$.
\end{definition}

Definiamo meglio cos'è un \textbf{passo di computazione},
procedendo per casi e considerando $a$, $b$ e $c$ come elementi
generici di $\Sigma$.
\begin{itemize}
	\item $(q, u \underline{a} v) \to (q', u \underline{b} v)$
	      se $\delta (q, a) = (q', b, -)$.
	\item $(q, u c \underline{a} v) \to
		      (q', u \underline{c} b v)$
	      se $\delta (q, a) = (q', b, L)$.
	\item \begin{enumerate}
		      \item $(q, u \underline{a} c v) \to
			            (q', u b \underline{c} v)$
		            se $\delta (q, a) = (q', b, R)$.
		      \item $(q, u \underline{a}) \to
			            (q', u b \underline{\#})$
		            se $\delta (q, a) = (q', b, R)$.
	      \end{enumerate}
\end{itemize}
In questo modo garantiamo che ciascun passo abbia effetto
limitato sulle configurazioni, come richiesto dalla seconda
parte del punto 2 nell'\hyperref[sec: algoritmo]{idea intuitiva
	dell'algoritmo}.

Allo stesso modo possiamo immaginarci che se partiamo da uno
stato iniziale $(q_0, \underline{\start} w)$, dopo un certo
numero di passi arriviamo in uno stato $(q', w')$.

\begin{definition} \label{def: convergenza}
	Diciamo che una computazione \textbf{termina} o
	\textbf{converge} ($\downarrow$) se e solo se lo stato
	finale è $h$. Nell'esempio \ref{ex: 11} fatto qui sopra
	$q' = h$.

	Diciamo invece che la computazione \textbf{non termina}
	o \textbf{diverge} ($\uparrow$) se e solo se per ogni
	$q'$ e $w'$ tali che
	\[ (q_0, w) \to^* (q', w') \]
	esistono $q''$ e $w''$ tali che
	\[ (q', w') \to^* (q'', w'') \]
	ovvero tali che è sempre possibile fare un nuovo passo di
	computazione.
\end{definition}

Come possiamo vedere, le MdT definite in questo modo,
rispettano l'idea intuitiva di algoritmo che abbiamo dato
all'inizio.

\begin{example}
	Facciamo un esempio di computazione che non termina mai
	tramite una macchine che semplicemente non contiene il
	simbolo $h$ di terminazione.
	\begin{center}
		\begin{tabular}{|c|c|c|}
			\hline
			$q$   & $\sigma$ & $\delta$             \\
			\hline
			$q_0$ & $\start$ & $q_0$, $\start$, $R$ \\
			$q_0$ & $a$      & $q_0$, $a$, $R$      \\
			$q_0$ & $\#$     & $q_0$, $\#$, $R$     \\
			\hline
		\end{tabular}
	\end{center}
\end{example}

Proviamo ora a fare un esempio più concreto di una MdT che
calcola qualcosa di più sensato rispetto agli esempi visti
fino ad ora.

\begin{example}
	Questa MdT si propone di calcolare la somma di due numeri
	$n$ ed $m$, dove $n$ ed $m$ sono rappresentati in notazione
	unaria tramite il simbolo $|$ ripetuto $n$ (o $m$) volte.
	La funzione $\delta$ è definita dalla seguente tabella
	\begin{center}
		\begin{tabular}{|c|c|c|}
			\hline
			$q$   & $\sigma$ & $\delta$         \\
			\hline
			$q_0$ & $\start$ & $q_0, \start, R$ \\
			$q_0$ & $|$      & $q_0, |, R$      \\
			$q_0$ & $+$      & $q_1, |, R$      \\
			$q_1$ & $|$      & $q_1, |, R$      \\
			$q_1$ & $\#$     & $q_2, \#, L$     \\
			$q_2$ & $|$      & $h, \#, -$       \\
			\hline
		\end{tabular}
	\end{center}
	Il simbolo $+$ deve essere tra le due sequenze di $|$.
	Proviamo ora ad eseguire la computazione per il calcolo
	di $1 + 2$, partendo dalla configurazione iniziale
	\[ q_0/ \quad \underline{|} + | | \]
	Svolgiamo quindi i seguenti passi
	\begin{gather*}
		(q_0, \; \underline{\start} | + | | ) \to
		(q_0, \; \start \underline{|} + | | ) \\
		(q_0, \; \start \underline{|} + | | ) \to
		(q_0, \; \start | \underline{+} | | ) \\
		(q_0, \; \start | \underline{+} | | ) \to
		(q_1, \; \start | | \underline{|} | ) \\
		(q_1, \; \start | | \underline{|} | ) \to
		(q_1, \; \start | | | \underline{|} ) \\
		(q_1, \; \start | | | \underline{|} ) \to
		(q_1, \; \start | | | | \underline{\#} ) \\
		(q_1, \; \start | | | | \underline{\#} ) \to
		(q_2, \; \start | | | \underline{|} \# ) \\
		(q_2, \; \start | | | \underline{|} \# ) \to
		(h, \; \start | | | \underline{\#} \# )
	\end{gather*}
	per concludere che il calcolo termina dato che siamo
	giunti nello stato speciale $h$. Come possiamo vedere il
	nostro risultato è dato dal numero di $|$ rimanente.
\end{example}

Un altro tipico esempio di utilizzo di queste macchine è
quello di decidere se una stringa è palindroma. In questo
caso descriviamo brevemente quale sarebbe l'idea.
\begin{enumerate}
	\item Si controlla il primo carattere della stringa, lo si
	      sovrascrive con $\start$ e si passa in uno stato
	      specifico per quel carattere (supponendo che il primo
	      carattere sia $a$, si passa in $q_a$).
	\item Si arriva in fondo alla stringa e si controlla che
	      l'ultimo carattere sia uguale al primo, se sì, lo
	      si sovrascrive con un $\#$.
	\item Si torna indietro fino al primo $\start$ che si
	      incontra.
\end{enumerate}
Si ripete il procedimento finché non si esaurisce tutta la
stringa o finché non fallisce.


\section{Linguaggi FOR e WHILE}
Introduciamo ora un formalismo sicuramente a noi più
familiare, che è quello di un semplice linguaggio
\textbf{imperativo}, che dunque contenga il concetto di
\textbf{memoria} e di \textbf{comando}.

\subsection{Sintassi astratta}
La sintassi proposta è \emph{ambigua}, abbiamo dunque diverse
possibili interpretazioni per la stessa stringa. In realtà
tale sintassi viene elaborata tramite alberi, che hanno proprio
il compito di disambiguare.

Come vedremo a breve si tratta di una sintassi molto semplice,
comprendente costrutti condizionali di base, la possibilità
di eseguire cicli e semplici operazioni aritmetiche e logiche.
\begin{align*}
	E \to                                 &
	n | x | E_1 + E_2 | E_1 \times E_2 \vert E_1 - E_2
	                                      &
	\text{Espressioni aritmetiche}          \\
	B \to                                 &
	b | E_1 < E_2 | \lnot B | B_1 \lor B_2
	                                      &
	\text{Espressioni booleane}             \\
	C\to                                  &
	\text{skip} | x := E | C_1 ; C_2 | \text{if } B
	\text{ then } C_1 \text{ else } C_2 | &
	\text{Comandi}                          \\
	                                      &
	\text{for } x = E_1 \text{ to } E_2 \text{ do } C |
	\text{while } B \text{ do } C
\end{align*}
dove $n \in \N$, $x \in \Var$ (insieme numerabile di
variabili) e $b \in \text{Bool} = \{ tt, ff \}$.

D'ora in poi chiameremo WHILE, il linguaggio descritto dalla
grammatica BNF di sopra. Chiameremo invece FOR, il linguaggio
risultante dall'omissione del comando \verb|while| della
stessa grammatica.

\subsection{Semantica}
Prima di addentrarci nelle dinamiche del linguaggio appena
definito, dobbiamo definire la \textbf{memoria}. Nel nostro
caso lo facciamo tramite la funzione
\[ \sigma : \Var \to \N \]
che molto semplicemente, data una variabile in input
restituisce il suo valore in memoria (operazione di lettura).
Noi però siamo interessati anche a scrivere in memoria,
definiamo quindi l'operazione di \textbf{aggiornamento}
tramite la funzione, o meglio il funzionale a tre argomenti
\[
	- [ - / - ] : (\Var \to \N) \times
	\N \times \Var \to (\Var \to \N)
\]
che è definita come
\[
	\sigma [n / x](y) = \begin{cases}
		n         & \text{se } y = x  \\
		\sigma(y) & \text{altrimenti}
	\end{cases}
\]
che non ho capito cosa faccia. Probabilmente copre i due casi
in cui vogliamo aggiungere una nuova variabile alla memoria
e assegnargli un valore, e il caso in cui vogliamo cambiare
il valore di una variabile già presente in memoria.

La \textbf{semantica} di un'espressione aritmetica è data
dalla seguente funzione di \textbf{valutazione}, in cui andremo
a scrivere il suo argomento principale, l'espressione aritmetica,
tra le parentesi $[[$ e $]]$, cui viene giustapposto il secondo
argomento, cioè la memoria in cui l'espressione va valutata.
\[ \xi [[-]] - : E \times (\Var \to \N) \to \N \]
Proviamo a capire come questa funzione si comporta quando
prende in input espressioni aritmetiche.
\begin{align*}
	\xi [[n]] \sigma              & = n         \\
	\xi [[x]] \sigma              & = \sigma(x) \\
	\xi [[E_1 + E_2]] \sigma      &
	= \xi [[E_1]] \text{ più } [[E_2]] \sigma   \\
	\xi [[E_1 \times E_2]] \sigma &
	= \xi [[ E_1 ]] \text{ per } [[E_2]] \sigma \\
	\xi [[E_1 - E_2]] \sigma      &
	= \xi [[ E_1 ]] \text{ meno } [[E_2]] \sigma
\end{align*}
Per adesso limitiamoci a notare che se diamo alla funzione di
valutazione un numero $n$, qualunque sia la memoria, questa
ci restituirà il numero stesso.

Se invece gli passiamo una variabile, questa ci restituirà
il valore della variabile in memoria, si fa infatti uso della
funzione $\sigma$.

Se invece passiamo una qualche operazione aritmetica, questa
verrà valutata come la somma (oppure sottrazione o prodotto)
delle valutazioni degli operandi.

\begin{example}
	Proviamo a valutare l'espressione
	\[ x \times 2 - ((y - 7) + 1) \]
	nella memoria $\sigma$ tale che
	\[ \sigma (x) = 3 \quad \text{e} \quad \sigma(y) = 5 \]
	I passaggi per risolvere l'esercizio sono i seguenti
	\begin{align*}
		  & \xi [[ x \times 2 - ((y - 7) + 1) ]] \sigma \\
		= & \xi [[ x \times 2 ]] \sigma \text{ meno }
		\xi [[ (y - 7) + 1 ]] \sigma
	\end{align*}
	come possiamo notare, quando c'è un'ambiguità tra le
	precedenze supponiamo di sapere quale sia l'ordine
	corretto. In questo caso valutiamo "prima" il meno.
	\begin{align*}
		= & (\xi [[ x ]] \sigma \text{ per } 2) \text{ meno }
		\xi [[ (y - 7) + 1 ]] \sigma                          \\
		= & (\sigma(x) \text{ per } 2) \text{ meno }
		\xi [[ (y - 7) + 1 ]] \sigma                          \\
		= & (3 \text{ per } 2) \text{ meno }
		\xi [[ (y - 7) + 1 ]] \sigma
	\end{align*}
	ora abbiamo semplicemente preso il valore $3$ dalla memoria
	per sostituirlo a $x$.

	\begin{align*}
		= & 6 \text{ meno } (\xi [[y - 7]] \sigma
		\text{ più } 1)                                  \\
		= & 6 \text{ meno } ((\xi [[y]] \text{ meno } 7)
		\text{ più } 1)                                  \\
		= & 6 \text{ meno } ((\sigma(y) \text{ meno } 7)
		\text{ più } 1)
	\end{align*}
	Di seguito $5 - 7 = 0$ perché stiamo utilizzando il
	\emph{meno ridotto}, il quale ritorna $0$ quando il
	minuendo è maggiore o uguale del sottraendo.
	\begin{align*}
		= & 6 \text{ meno } ((5 \text{ meno } 7)
		\text{ più } 1)                          \\
		= & 6 \text{ meno } (0 \text{ più } 1)   \\
		= & 6 \text{ meno } 1 = 5
	\end{align*}
	Il calcolo per il resto è abbastanza banale.
\end{example}

Passiamo ora alla semantica delle operazioni booleane, che
non si inventano nulla di diverso rispetto alle espressioni
aritmetiche. Semplicemente la funzione di valutazione è
definita in modo da restituire valori booleani.
\[ \B[[-]] - : \B \times (\Var \to \N) \to \text{Bool} \]
Come prima andiamo a vedere cosa succede per vari input a
tale funzione di valutazione.
\begin{align*}
	\B [[t]] \sigma            & = tt                           \\
	\B [[f]] \sigma            & = ff                           \\
	\B [[E_1 < E_2]] \sigma    & =
	\xi [[E_1]] \sigma \text{ minore } \xi [[E_2]] \sigma       \\
	\B [[\lnot B]] \sigma      & = \text{ not } \B [[B]] \sigma \\
	\B [[B_1 \lor B_2]] \sigma & = \B [[B_1]] \sigma
	\text{ or } \B [[B_2]] \sigma
\end{align*}
Lo stile di definizione seguito fino ad ora prende il nome di
stile \textbf{denotazionale} e si propone di associare una
funzione a ciascun operatore.

\section{Problemi e funzioni}
Per il momento abbiamo usato i nostri costrutti per calcolare
una funzione (la somma per esempio) o per decidere
l'appartenenza di un elemento ad un insieme (decidere se una
stringa è palindroma).

In questa prima parte del corso andremo a definire meglio il
concetto di \textbf{problema} e di \textbf{funzione} che, nel
nostro caso, non hanno l'accezione cui siamo abituati.

Un esempio di \emph{problema} è la domanda: "qual è il massimo
comun divisore tra $x$ e $y$?". Se sostituiamo a $x$ e a $y$
dei valori, per esempio 34 e 98, otteniamo un \textbf{caso}
del problema.

\begin{definition} \label{def: T-calcolabile}
	Siano $\Sigma$, $\Sigma_0$ e $\Sigma_1$ tali che
	\[ \#, \start \notin \Sigma_0 \cup \Sigma_1 \]
	e
	\[ ]\Sigma_0 \cup \Sigma_1 \subset \Sigma \]
	allora diciamo che una funzione
	\[ f : \Sigma_0 \to \Sigma_1 \]
	è \textbf{Turing calcolabile} o \textbf{T-calcolabile},
	se e solo se
	\begin{gather*}
		\forall w \in \Sigma_0^* : f(w) = z \\
		\Updownarrow \\
		(q_0, \underline{\start} w) \to_M^*
		(h, \start z \underline{\#})
	\end{gather*}
	Si dice anche che $f$ è T-calcolabile se esiste una MdT
	$M$ che la calcola.
\end{definition}

Ora che abbiamo la definizione precisa di cosa sia una
funzione T-calcolabile proviamo a fare una cosa analoga per
il linguaggio WHILE che abbiamo definito in precedenza.

\begin{definition} \label{def: while-calcolabile}
	Diciamo che una funzione
	\[ f : \Var \to \N \]
	è \textbf{WHILE-calcolabile} oppure diciamo che un comando
	$C$ \textbf{calcola} $f$, se e solo se
	\begin{gather*}
		\forall \sigma \in \Var \to \N : f(x) = n \\
		\Updownarrow \\
		(C, \sigma) \to^* \sigma' \quad \land \quad
		\sigma'(x) = n
	\end{gather*}
\end{definition}

Notiamo che la variabile $x$ di input è anche la variabile
di output, ossia quella che contiene il risultato.

\subsection{Codifiche}
Ci chiediamo ora se per una funzione $f$ che non opera su
dati sotto formato di stringa, memorie o numeri naturali,
le nozioni di calcolabilità che abbiamo definito fino ad ora
sono ancora valide.

Se così non fosse dovremmo ridefinire ogni volta tali nozioni
per ogni dominio di ogni funzione con un formato differente da
quelli che abbiamo già incontrato.

Per superare il problema si fa uso di opportune
\textbf{codifiche} dei dati, ossia funzioni che svolgono
il seguente compito.
\begin{enumerate}
	\item Dato $x$ in formato $A$, lo si codifica in un formato
	      $B$ con cui riusciamo a calcolare.
	\item Si applica la MdT a $y$ e si ottiene il risultato $z$
	      (se la computazione termina) in formato $B$.
	\item Si traduce $z$ dal formato $B$ al formato $A$.
\end{enumerate}
D'ora in avanti considereremo solo i numeri naturali come i
nostri dati. Abbiamo però bisogno che la funzione di codifica
sia \emph{biunivoca}.

\begin{example}
	La seguente funzione codifica coppie di naturali come un
	singolo naturale ed è detta codifica a
	\textbf{coda di colomba}.
	\[ (x, y) \to \frac{1}{2} (x^2 + 2 x y + y^2 + 3 x + y) \]
	la cui decodifica, ossia la funzione inversa è la seguente
	\[
		n \to (n - \frac{1}{2} k \cdot (k + 1),
		k - (n - \frac{1}{2} k \cdot (k + 1))
	\]
	dove $k=\lfloor \frac{1}{2}(\sqrt{1+8\cdot n}-1)\rfloor$.
\end{example}

A prescindere dall'esempio, possiamo dire che le proprietà
basilari dei formalismi e della classe di funzioni calcolate,
non cambiano al variare del formato dei dati su cui operano.

\begin{definition} \label{def: funzione totale}
	Diciamo che una funzione $f : A \to B$, sottoinsieme di
	$A \times B$ è una \textbf{funzione totale} se e solo se
	\begin{itemize}
		\item La funzione è \emph{definita ovunque}, ossia se
		      $\forall a \in A$, $\exists b \in B$ tale che la
		      coppia $(a, b) \in f$.
		\item Vi è \emph{unicità}, ossia se, date le coppie
		      $(a, b) \in f$ e $(a, c) \in f$, allora $b=c$.
	\end{itemize}
\end{definition}

Una funzione può essere calcolabile ma non totale, per esempio
la macchina di Turing che non termina mai vista all'inizio del
corso.

\begin{definition} \label{def: funzione parziale}
	Diciamo che una funzione $f : A \to B$ è \textbf{parziale}
	se è un sottoinsieme di $A \times B$ tale che
	\begin{itemize}
		\item Vi è \emph{unicità}, ossia se, date le coppie
		      $(a, b) \in f$ e $(a, c) \in f$, allora $b=c$.
		\item Esiste al più un $b \in B$ tale che $f(a) = b$.
	\end{itemize}
	e quindi non si richiede che $f$ sia definita ovunque.
\end{definition}

Introduciamo ora un po' di notazione utile a quello che faremo
più avanti. Data una funzione $f : A \to B$
\begin{itemize}
	\item Diremo che $f$ è \textbf{definita} o
	      \textbf{converge su $a$} ($f(a) \downarrow$) se
	      $\exists b$ tale che $(a, b) \in f$ (cioè
	      $f(a) = b$).
	\item Diremo che $f$ \textbf{non è definita} o che
	      \textbf{diverge} ($f(a) \uparrow$) se $\nexists b$
	      tale che $(a, b) \in f$.
\end{itemize}
Chiamiamo inoltre
\begin{itemize}
	\item \textbf{Dominio} di $f$ l'insieme
	      \[ \{ a \in A | f(a) \downarrow \} \]
	      che coincide con lo spazio di partenza ($A$) se e
	      solo se la funzione è totale.
	\item \textbf{Codominio} di $f$ l'insieme $B$.
	\item \textbf{Immagine} di $f$ l'insieme
	      \[ \{ b \in B | \exists a \in A : f(a) = b \} \]
\end{itemize}

Detto questo vogliamo capire qual è la relazione tra funzioni
e algoritmi. Una funzione possiamo vederla come un insieme di
coppie (\emph{argomento}, \emph{risultato}) (o (\emph{input},
\emph{output}) se preferiamo la notazione più informatica) ma
non ci dice come il risultato (o l'output) venga calcolato.

Di conseguenza non ci sono due funzioni diverse che per uno
stesso argomento restituiscono lo stesso risultato. In termini
insiemistici possiamo dire che non esistono due insiemi diversi
che hanno gli stessi elementi.
\begin{tcolorbox}
	Un algoritmo è invece una \textbf{rappresentazione finita}
	di una funzione, in quanto specifica come si calcola il
	risultato a partire dall'argomento. In questo caso possiamo
	certamente avere più algoritmi che calcolano la stessa
	funzione.
\end{tcolorbox}

\subsection{Funzioni calcolabili}
D'ora in avanti proveremo a capire
\begin{itemize}
	\item Quali sono le \textbf{funzioni calcolabili} e di
	      quali proprietà godono.
	\item Se esistono funzioni totali o parziali che non sono
	      calcolabili. Ovvero per cui si dimostra che non esiste
	      un algoritmo che le calcoli.
\end{itemize}

\begin{example}
	Prendiamo ora come esempio la
	\textbf{congettura di Goldbach}, la quale ci dice che ogni
	numero pari maggiore di 2 è esprimibile come somma di due
	numeri primi.

	Da questa congettura (mai dimostrata) nasce la
	\textbf{funzione di Goldbach}, definita come segue con
	$gb : \N \to \N$
	\[
		gb(n) = \begin{cases}
			0 & \text{se la congettura è vera} \\
			1 & \text{altrimenti}
		\end{cases}
	\]
	La congettura non è stata ancora dimostrata ma un algoritmo
	per calcolarla esiste, solo che non sappiamo quale sia.

	Se ad esempio volessimo decidere se la funzione è
	T-calcolabile, basterebbe prendere una MdT che ritorna
	sempre 0 se la congettura è vera o una MdT che ritorna
	sempre 1 se è falsa. Il problema è che fin tanto che la
	congettura non è dimostrata, non sappiamo quale delle due
	scegliere.
\end{example}


\section{Funzioni ricorsive primitive}
Iniziamo con il definire due delle funzioni definite per
\emph{ricorrenza} più popolari, il fattoriale e la successione
di Fibonacci.

Iniziamo con il \emph{fattoriale}, definito come una coppia
di equazioni, la prima per il caso base, ossia quando $x = 0$
e la seconda per tutti gli altri casi, ossia per ogni $x > 0$.
\[
	!(x) = \begin{cases}
		1        & \text{se } x = 0 \\
		!(n - 1) & \text{se } x > 0
	\end{cases}
\]
Proviamo a scrivere una versione WHILE e una versione FOR del
fattoriale
\begin{verbatim}
    fatt = 1;
    while x > 0 do
        fatt = fatt * x;
        x = x - 1;
\end{verbatim}
in questo caso salviamo il risultato nella variabile
\verb|fatt| che è la stessa che usiamo per ritornare il
risultato.

La \emph{successione di Fibonacci} presenta invece due casi
base e un caso definito per ricorrenza ed è definita come
segue
\[
	fib(x) = \begin{cases}
		0                       & \text{se } x = 0 \\
		1                       & \text{se } x = 1 \\
		fib(x - 1) + fib(x - 2) & \text{se } x > 1
	\end{cases}
\]
Ora vogliamo capire quali sono le regole per formare bene
delle formule ricorsive ed è qui che prenderemo un po' di
notazione in prestito dal $\lambda$-calcolo. Nel caso del
fattoriale possiamo scrivere
\[ \lambda x . !(x) \]
per dire che il fattoriale dipende solo da $x$. Possiamo anche
scrivere un qualcosa di questo tipo
\[ \lambda x . x + y \]
per definire una funzione che prende $x$ e restituisce una
funzione che dipende da $y$. Abbiamo quindi un modo per
\emph{costruire} delle funzioni specificando esattamente quali
sono i suoi argomenti.

\begin{definition}
	Le \textbf{funzioni ricorsive primitive} sono la minima
	classe $\C$ da $\N^n$, con $n \geq 0$, in $\N$ cui
	appartengono
	\begin{itemize}
		\item \textbf{Zero}: una funzione che prende $k \geq 0$
		      di argomenti e ritorna 0.
		      \[ \lambda x_1, \dots, x_k . 0 \]
		\item \textbf{Successore}: che prende un argomento solo
		      e restituisce il suo successore
		      \[ \lambda x . x + 1 \]
		\item \textbf{Identità}: che prende $k$ argomenti e
		      ritorna l'argomento $i$-esimo con $1\leq i\leq k$.
		      \[ \lambda x_1, \dots, x_k . x_i \]
		      Viene anche chiamata \textbf{proiezione}.
	\end{itemize}
	Questi sono anche detti \textbf{schemi primitivi di base}.
	La classe $\C$ che stiamo provando a definire è inoltre
	\emph{chiusa} per
	\begin{itemize}
		\item \textbf{Composizione}: Se $g_1, \dots, g_k \in \C$
		      sono funzioni in $m$ variabili, e $h \in \C$ è
		      una funzione in $k$ variabili, anche la loro
		      composizione
		      \[
			      \lambda x_1, \dots x_m .
			      h(g_1(x_1, \dots, x_m), \dots,
			      g_k(x_1, \dots, x_m)
			      )
		      \]
		      appartiene a $\C$.
		\item \textbf{Ricorsione primitiva}: Se $h \in \C$
		      è una funzione in $k+1$ variabili, $g \in \C$
		      è una funzione in $k-1$ variabili definita da
		      \[
			      \begin{cases}
				      f(0, x_2, \dots, x_k)       & =
				      g(x_2, \dots, x_k)              \\
				      f(x_1 + 1, x_2, \dots, x_k) & =
				      h(x_1, f(x_1, \dots, x_k),
				      x_2, \dots, x_k)
			      \end{cases}
		      \]
	\end{itemize}
\end{definition}

\begin{tcolorbox}
	Dato che $\C$ è la \emph{minima} classe che soddisfa le
	condizioni espresse sopra, affinché $f$ sia ricorsiva
	primitiva, occorre e basta che sia una successione finita,
	o \textbf{derivazione}, della seguente forma
	\[ f_1, \dots, f_n \]
	tale che $f = f_n$ e per ogni $i$ tale che
	$1 \leq i \leq n$ vale uno dei seguenti casi:
	\begin{itemize}
		\item $f_i \in C$ è uno \emph{Zero} o è
		      l'\emph{Identità}.
		\item $f_i$ è ottenibile tramite l'applicazione delle
		      regole di \emph{Combinazione} e
		      \emph{Ricorsione primitiva} da $f_j$ con $j < i$.
	\end{itemize}
\end{tcolorbox}

Tra i requisiti che abbiamo appena descritto perché una
funzione venga definita \emph{ricorsiva primitiva}, quello
meno intuitivo è sicuramente è quello che riguarda proprio
la ricorsione primitiva.

Come vediamo, noi possiamo definire una funzione di se stessa,
ma con delle limitazioni. Come possiamo vedere abbiamo un
primo caso in cui il primo argomento è $0$ e non c'è una
chiamata ricorsiva, siamo quindi davanti ad un caso base.

Più complesso è il secondo caso, in cui diciamo che la funzione
su $k$ argomenti ritorna il primo argomento decrementato,
effettua una chiamata ricorsiva su tutti gli argomenti e lascia
i rimanenti invariati.

Resistiamo per un esempio e proviamo a fare il punto della
situazione. Prima di fare l'esempio definiamo la somma tramite
le regole appena descritte.
\[
	\begin{array}{ll}
		f_1 & = \lambda x.x                    \\
		f_2 & = \lambda x.x + 1                \\
		f_3 & = \lambda x_1, x_2, x_3 . x_2    \\
		f_4 & = f_2 (f_3 (x_1, x_2, x_3))      \\
		f_5 & = \begin{cases}
			        f_5 (0, x_2)     & = f_1 (x_2) \\
			        f_5 (x + 1, x_2) & =
			        f_4 (x_1, f_5(x_1, x_2), x_2)
		        \end{cases}
	\end{array}
\]
Qui l'idea, avendo come unica funzione di somma la funzione
successore, è quella di calcolare il successore del primo
numero tante volte quanto è il secondo numero.

\begin{example}
	Proviamo a calcolare $2 + 3$ con la somma che abbiamo
	appena definito
	\[
		\begin{array}{l}
			f_5(2, 3) =                         \\
			f_4 (1, f_5(1, 3), 3) =             \\
			f_4 (1, f_4(0, f_5 (0, 3), 3), 3) = \\
			f_4 (1, f_4(0, f_1 (3), 3), 3) =    \\
			f_4 (1, f_4(0, 3, 3), 3) =          \\
			f_4 (1, f_2(f_3(0, 3, 3)), 3) =     \\
			f_4 (1, f_2(3), 3) =                \\
			f_4 (1, 4, 3) =                     \\
			f_2 (f_3 (1, 4, 3)) =               \\
			f_2 (4) = 5
		\end{array}
	\]
	Applicare la formula non è niente di difficile e forse
	non è più di tanto istruttivo.
\end{example}

Dunque quello che interessa principalmente sapere a noi è 
cosa sia una funzione ricorsiva primitiva e a quali regole 
sottostare per definirne una.


\chapter{Diagonalizzazione di endomorfismi lineari}
\section{Autovalori e autovettori}
Sia $T : V \to V$ un endomorfismo lineare dello spazio $V$ sul campo $\K$.

\begin{definition}
	Un vettore $v \in V - \{O\}$ si dice un \textbf{autovettore} di $T$ se
	\[
		T(v) = \lambda v
	\]
	per un certo $\lambda \in \K$.
\end{definition}

In altre parole un autovettore di $T$ è un vettore diverso da $O$ dello spazio $V$
che ha la seguente proprietà: la $T$ lo manda in un multiplo di se stesso.

\begin{definition}
	Se $v \in V - \{O\}$ è un autovettore di $T$ tale che
	\[
		T(v) = \lambda v
	\]
	allora lo scalare $\lambda \in \K$ si dice \textbf{autovalore} di $T$
	relativo a $v$ (e viceversa si dice che $v$ è un autovettore relativo a
	$\lambda$).
\end{definition}

Si noti che l'autovalore può essere $0 \in \K$: se per esempio $T$ non
è iniettiva, ossia $\Ker(T) \supsetneq \{O\}$, tutti gli elementi
$w \in (\Ker(T)) - \{O\}$ soddisfano
\[
	T(w) = O = 0w
\]
ossia sono autovettori relativi all'autovalore 0.

\begin{definition}
	Dato $\lambda \in \K$ chiamiamo l'insieme
	\[
		V_\lambda = \{v \in V \mid T(v) = \lambda v\}
	\]
	\textbf{autospazio} relativo a $\lambda$.
\end{definition}

\begin{observation}
	Possiamo notare dalla definizione precedente che $V_0 = \Ker(T)$.
\end{observation}

Anche se abbiamo definito l'autospazio $V_\lambda$ per qualunque
$\lambda \in \K$, in realtà $V_\lambda$ è sempre uguale a $\{O\}$ a
meno che $\lambda$ non sia un autovalore. Questo è dunque il caso interessante:
se $\lambda$ è un autovalore di $T$ allora $V_\lambda$ è costituito da $O$ e
da tutti gli autovettori relativi a $\lambda$.

Ma perché sono importanti autovettori e autovalori ?
Supponiamo che $V$ abbia dimensione $n$ e pensiamo a cosa succederebbe se
riuscissimo a trovare una base di $V$, $\{v_1, v_2, \dots, v_n\}$, composta solo
da autovettori di $T$.

Avremmo, per ogni $i = 1, 2, \dots, n$,
\[
	T(v_i) = \lambda_i v_i
\]
per certi autovalori $\lambda_i$.

Come sarebbe fatta la matrice
\[
	[T]_{\substack{
				v_1, v_2, \dots, v_n \\
				v_1, v_2, \dots, v_n
			}}
\]
associata a $T$ rispetto a questa base ?

Ricordandoci come si costruiscono le matrici osserviamo che la prima colonna
conterrebbe il vettore $T(v_1)$ scritto in termini della base $\{v_1, \dots v_n\}$,
ossia
\[
	T(v_1) = \lambda_1 v_1 + 0 v_2 + 0 v_3 + \cdots + 0 v_n
\]
la seconda il vettore $T(v_2) = 0 v_1 + \lambda_2 v_2 + 0 v_3 + \cdots + 0 v_n$ e
così via. Otterremo quindi una matrice diagonale.
\[
	[T]_{\substack{
				v_1, v_2, \dots, v_n \\
				v_1, v_2, \dots, v_n
			}} = \begin{pmatrix}
		\lambda_1 & 0         & 0     & 0         \\
		0         & \lambda_2 & 0     & 0         \\
		0         & 0         & \dots & 0         \\
		0         & 0         & 0     & \lambda_n
	\end{pmatrix}
\]

Da questa matrice possiamo ricavare a colpo d'occhio informazioni come
\begin{itemize}
	\item Il rango di $T$.
	\item La dimensione del nucleo.
	\item Quali sono i vettori di $\Ker(T)$.
	\item Quali sono (se esistono) i sottospazi in cui $T$ si comporta come l'identità,
	      ossia i sottospazi costituiti dai vettori di $V$ che $T$ lascia fissi.
\end{itemize}

Dunque l'obbiettivo di studiare autovalori e autovettori di $T$ è quello di
trovare basi "buone" che ci permettano di conoscere bene il comportamento di $T$.
Tuttavia non esistono sempre queste basi buone. E si dice che, se per un certo
endomorfismo $T$ esiste una base buona, questo è \textbf{diagonalizzabile}.

\begin{example}
	Consideriamo l'endomorfismo $R_{\theta} : \R^2 \to \R^2$ dato
	da una \emph{rotazione} di angolo $\theta$ con centro l'origine. Si verifica
	immediatamente che, rispetto alla base standard di $\R^2$, questo
	endomorfismo è rappresentato dalla matrice
	\[
		[R_\theta] = \begin{pmatrix}
			\cos{\theta} & -\sin{\theta} \\
			\sin{\theta} & \cos{\theta}
		\end{pmatrix}
	\]
	Per esempio nel caso di una rotazione di $60^\circ$ (ovvero $\frac{\pi}{3}$),
	abbiamo:
	\[
		R_{\frac{\pi}{3}} = \begin{pmatrix}
			\frac{1}{2}        & -\frac{\sqrt{3}}{2} \\
			\frac{\sqrt{3}}{2} & \frac{1}{2}
		\end{pmatrix}
	\]
	Nel caso in cui $0 < \theta < \pi$, non ci sono vettori $v \neq O$ che vengono
	mandati in un multiplo di se stessi, visto che tutti i vettori vengono ruotati
	di un angolo che non è nullo e non è di $180^\circ$. Dunque non ci sono
	autovalori e autovettori.

	Nel caso $\theta = 0$ la rotazione è l'identità, dunque tutti i vettori
	$v \neq O$ sono autovettori relativi all'autovalore 1, e $V_1 = \R^2$.

	Nel caso $\theta = \pi$ la rotazione è uguale a $-I$, dunque tutti i vettori
	$v \neq O$ sono autovettori relativi all'autovalore $-1$, e
	$V_{-1} = \R^2$.
\end{example}

\section{Polinomio caratteristico}
Vogliamo trovare dei criteri semplici per stabilire se un endomorfismo
è diagonalizzabile o no. Prima di tutto troviamo un metodo che, dato un
endomorfismo $T : V \to V$ e posto $n = \dim(V)$, ci permetta di decidere se
uno scalare $\lambda \in \K$ è o no un autovalore di $T$. Entrano
qui in gioco i polinomi e le loro radici.

Innanzitutto osserviamo che, perché $\lambda \in \K$ sia un
autovalore, secondo la definizione bisogna che esista un $v \in V - \{O\}$
tale che
\[
	T(v) = \lambda v
\]
Questo si può riscrivere anche come
\[
	T(v) - \lambda I(v) = O
\]
dove $I : V \to V$ è l'identità. Riscriviamo ancora:
\[
	(T - \lambda I)(v) = O
\]
Abbiamo scoperto che, se $T$ possiede un autovalore $\lambda$, allora
l'endomorfismo $T - \lambda I$ non è iniettivo: infatti manda il vettore
$v$ in $O$. Dunque, se scegliamo una base qualunque per $V$ e costruiamo la
matrice $[T]$ associata a $T$, la matrice $[T - \lambda I] = [T] - \lambda I$
dovrà avere determinante uguale a 0:
\[
	det([T] - \lambda I) = 0 = det(\lambda I - [T])
\]
dove come consuetudine abbiamo indicato con $I$ anche la matrice identità.

\begin{definition}
	Dato un endomorfismo $T : V \to V$ con $n = \dim(V)$, scegliamo una base
	per $V$ e costruiamo la matrice $[T]$ associata a $T$ rispetto a tale
	base. Il \textbf{polinomio caratteristico} $P_T(t) \in \K[t]$
	dell'endomorfismo $T$ è definito da:
	\[
		P_T(t) = det(t[I] - [T])
	\]
\end{definition}

\begin{observation}
	Prima di procedere dobbiamo fare un paio di considerazioni:
	\begin{enumerate}
		\item Perché la definizione precedente abbia senso si deve verificare
		      che \[det(t[I] - [T])\] sia veramente un polinomio. Questo si può
		      dimostrare facilmente per induzione sulla dimensione
		      $n$ di $V$.
		\item È fondamentale inoltre che la definizione appena data non dipenda
		      dalla base scelta di $V$: non sarebbe una definizione buona se con
		      la scelta di due basi diverse ottenessimo due polinomi
		      caratteristici diversi.
	\end{enumerate}
\end{observation}

Questo problema per fortuna non si verifica. Infatti se scegliamo due basi $b$ e
$b'$ di $V$, come sappiamo, le due matrici $[T]_{\substack{b \\ b}}$ e
$[T]_{\substack{b' \\ b'}}$ sono legate dalla seguente relazione: esiste una
matrice $[B]$ invertibile tale che
\[
	[T]_{\substack{b \\ b}} =
		[B]^{-1} [T]_{\substack{b' \\ b'}} [B]
\]

Usando il teorema di Binet a questo punto verifichiamo che
\begin{gather*}
	det \left(tI - [T]_{\substack{b \\ b}}\right) = \\
	det \left(tI - [B]^{-1} [T]_{\substack{b' \\ b'}} [B]\right) = \\
	det \left([B]^{-1} \left(tI - [T]_{\substack{b' \\ b'}}\right) [B] \right) = \\
	det \left([B]^{-1}\right) det \left(tI - [T]_{\substack{b' \\ b'}}\right)
	det \left([B]\right) = \\
	det \left(tI - [T]_{\substack{b' \\ b'}}\right)
\end{gather*}

Abbiamo dunque mostrato che $P_T(t) = det(tI - [T])$ non dipende dalla scelta della
base.

\begin{theorem}
	Considerato $T$ come sopra, vale che uno scalare $\lambda \in \K$ è un
	autovalore di $T$ se e solo se $\lambda$ è una radice di $P_T(t)$, ossia se e
	solo se $P_T(\lambda) = 0$.
\end{theorem}

\begin{example}
	Consideriamo l'endomorfismo $T : \mathbb{C}^2 \to \mathbb{C}^2$ che, rispetto
	alla base standard di $\mathbb{C}^2$, è rappresentato dalla matrice
	\[
		[T] = \begin{pmatrix}
			\frac{1}{2}        & -\frac{\sqrt{3}}{2} \\
			\frac{\sqrt{3}}{2} & \frac{1}{2}
		\end{pmatrix}
	\]
	Il suo polinomio caratteristico risulta $P_T(t) = t^2 - t + 1$. Questo polinomio
	ha due radici in $\mathbb{C}$, ovvero $\frac{1 - i\sqrt{3}}{2}$ e
	$\frac{1 + i\sqrt{3}}{2}$, che in effetti, come sappiamo, sono gli autovalori
	di $T$.
\end{example}


\section{Strategia per scoprire se un endomorfismo è diagonalizzabile}
Con il metodo descritto a breve si potrà scoprire se un endomorfismo
$T : V \to V$, dove $V$ è uno spazio vettoriale su $\K$ di dimensione $n$
è diagonalizzabile, e, in caso lo sia, si potrà trovare una base che lo diagonalizza,
ossia una base di $V$ fatta tutta di autovettori di $T$.

\begin{itemize}
	\item PASSO 1. Troviamo gli autovalori di un endomorfismo lineare $T$ calcolando il polinomio
	      caratteristico e le sue radici in $\K$.
	\item PASSO 2. Supponiamo dunque di aver trovato gli autovalori di $T$. A questo punto
	      vogliamo individuare gli autospazi relativi a tali autovalori.
	      Per questo basterà calcolare il $\Ker([T] - \lambda_i I)$. Prendiamo dunque la
	      matrice $[T] - \lambda_i I$ e risolviamo il sistema lineare omogeneo associato.
	\item PASSO 3. Per prima cosa enunciamo il seguente teorema
	      \begin{theorem}
		      Dato un endomorfismo $T : V \to V$, siano $\lambda_1, \dots, \lambda_k$
		      degli autovalori di $T$ distinti fra loro. Consideriamo ora degli autovettori
		      $v_1 \in V_{\lambda_1}, \dots, v_k \in V_{\lambda_k}$. Allora
		      $v_1, \dots, v_k$ è un insieme di vettori linearmente indipendenti.
	      \end{theorem}

	      Il seguente teorema è un rafforazamento del precedente.
	      \begin{theorem}
		      Dato un endomorfismo $T : V \to V$, siano $\lambda_1, \dots, \lambda_k$ degli
		      autovalori di $T$ distinti fra loro. Allora gli autospazi
		      $V_{\lambda_1}, \dots, V_{\lambda_k}$, sono in somma diretta.
	      \end{theorem}

	      Nelle ipotesi del teorema precedente sappiamo allora, che la dimensione della somma
	      degli autospazi è la massima possibile, ossia
	      \[
		      \dim(V_{\lambda_1} \oplus \cdots \oplus V_{\lambda_k}) = \dim(V_{\lambda_1}) +
		      \cdots + \dim(V_{\lambda_k})
	      \]

	      Osserviamo che abbiamo già un criterio per dire se $T$ è diagonalizzabile o no.
	      Ovvero, se
	      \[
		      \dim(V_{\lambda_1}) + \cdots + \dim(V_{\lambda_k}) = n = \dim(V)
	      \]
	      altrimenti se
	      \[
		      \dim(V_{\lambda_1}) + \cdots + \dim(V_{\lambda_k}) < n = \dim(V)
	      \]
	      $T$ non è diagonalizzabile. Infatti non è possibile trovare una base di
	      autovettori.
	\item PASSO 4. Se l'endomorfismo $T$ è diagonalizzabile, scegliamo allora una base di
	      autovettori nel modo descritto al Passo 3, e avremo una matrice associata $[T]$ che
	      risulterà diagonale. Mantenendo la notazione introdotta al Passo 3 troviamo sulla
	      diagonale $\dim(V_{\lambda_1})$ coefficienti uguali a
	      $\lambda_1$, ... e $\dim(V_{\lambda_k})$ coefficienti uguali a $\lambda_k$.
\end{itemize}
Il rango di $T$ sarà uguale al numero dei coefficienti non nulli che troviamo
sulla diagonale di $[T]$, la dimensione del nucleo sarà uguale al numero dei
coefficienti uguali a zero che troviamo sulla diagonale di $[T]$.

\begin{example}
	Consideriamo l'endomorfismo
	\[
		T \begin{pmatrix} x \\ y \end{pmatrix} =
		\begin{pmatrix}
			x + 2y \\
			-y
		\end{pmatrix}
	\]
	e la sua matrice associata rispetto alle basi standard
	\[
		[T] = \begin{pmatrix}
			1 & 2  \\
			0 & -1
		\end{pmatrix}
	\]
	Per prima cosa calcoliamo la matrice $[T] - tI$ che chiameremo $M$ per comodità
	\[
		M = \begin{pmatrix}
			t - 1 & -2    \\
			0     & t + 1
		\end{pmatrix}
	\]
	Troviamo il polinomio caratteristico calcolando il determinante di $M$ e otteniamo
	\[
		P_T(t) = (t - 1)(t + 1)
	\]
	Le radici di tale polinomio (e quindi gli autovalori di $T$) sono $t = 1$ e $t = -1$.
	Dobbiamo trovare quindi i relativi autospazi.
	\begin{itemize}
		\item Se $t = 1$ dobbiamo calcolare $\Ker([T] - 1I)$. Dobbiamo quindi risolvere il
		      sistema associato alla matrice
		      \[
			      \begin{pmatrix}
				      1 - 1 & -2    \\
				      0     & 1 + 1
			      \end{pmatrix} =
			      \begin{pmatrix}
				      0 & -2 \\
				      0 & 2
			      \end{pmatrix}
		      \]
		      ovvero
		      \[
			      \begin{cases}
				      -2y & = 0 \\
				      2y  & = 0
			      \end{cases} \quad \Rightarrow \quad
			      y = 0
		      \]
		      otteniamo dunque che l'autospazio $V_1$ è definito come segue
		      \[
			      V_1 = < \begin{pmatrix} 1 \\ 0 \end{pmatrix} > \quad \Rightarrow \quad
			      \dim(V_1) = 1
		      \]
		\item Se $t = -1$ procediamo in maniera analoga. Stavolta otteniamo il sistema
		      \[
			      \begin{cases}
				      -2x - 2y & = 0 \\
			      \end{cases} \quad \Rightarrow \quad
			      x = -y
		      \]
		      Ne deduciamo che l'autospazio $V_2$ sarà definito come segue
		      \[
			      V_2 = < \begin{pmatrix} -1 \\ 1 \end{pmatrix} > \quad \Rightarrow \quad
			      \dim(V_2) = 1
		      \]
	\end{itemize}
	Dato che $T$ è definta su $\R^2$ che ha dimensione 2 e dato che
	\[ \dim(V_1) + \dim(V_2) = 2 \] l'endomorfismo è diagonalizzabile.
\end{example}


\section{Strategia per scoprire se un endomorfismo è diagonalizzabile}
Con il metodo descritto a breve si potrà scoprire se un endomorfismo
$T : V \to V$, dove $V$ è uno spazio vettoriale su $\K$ di dimensione $n$
è diagonalizzabile, e, in caso lo sia, si potrà trovare una base che lo diagonalizza,
ossia una base di $V$ fatta tutta di autovettori di $T$.

\begin{itemize}
	\item PASSO 1. Troviamo gli autovalori di un endomorfismo lineare $T$ calcolando il polinomio
	      caratteristico e le sue radici in $\K$.
	\item PASSO 2. Supponiamo dunque di aver trovato gli autovalori di $T$. A questo punto
	      vogliamo individuare gli autospazi relativi a tali autovalori.
	      Per questo basterà calcolare il $\Ker([T] - \lambda_i I)$. Prendiamo dunque la
	      matrice $[T] - \lambda_i I$ e risolviamo il sistema lineare omogeneo associato.
	\item PASSO 3. Per prima cosa enunciamo il seguente teorema
	      \begin{theorem}
		      Dato un endomorfismo $T : V \to V$, siano $\lambda_1, \dots, \lambda_k$
		      degli autovalori di $T$ distinti fra loro. Consideriamo ora degli autovettori
		      $v_1 \in V_{\lambda_1}, \dots, v_k \in V_{\lambda_k}$. Allora
		      $v_1, \dots, v_k$ è un insieme di vettori linearmente indipendenti.
	      \end{theorem}

	      Il seguente teorema è un rafforazamento del precedente.
	      \begin{theorem}
		      Dato un endomorfismo $T : V \to V$, siano $\lambda_1, \dots, \lambda_k$ degli
		      autovalori di $T$ distinti fra loro. Allora gli autospazi
		      $V_{\lambda_1}, \dots, V_{\lambda_k}$, sono in somma diretta.
	      \end{theorem}

	      Nelle ipotesi del teorema precedente sappiamo allora, che la dimensione della somma
	      degli autospazi è la massima possibile, ossia
	      \[
		      \dim(V_{\lambda_1} \oplus \cdots \oplus V_{\lambda_k}) = \dim(V_{\lambda_1}) +
		      \cdots + \dim(V_{\lambda_k})
	      \]

	      Osserviamo che abbiamo già un criterio per dire se $T$ è diagonalizzabile o no.
	      Ovvero, se
	      \[
		      \dim(V_{\lambda_1}) + \cdots + \dim(V_{\lambda_k}) = n = \dim(V)
	      \]
	      altrimenti se
	      \[
		      \dim(V_{\lambda_1}) + \cdots + \dim(V_{\lambda_k}) < n = \dim(V)
	      \]
	      $T$ non è diagonalizzabile. Infatti non è possibile trovare una base di
	      autovettori.
	\item PASSO 4. Se l'endomorfismo $T$ è diagonalizzabile, scegliamo allora una base di
	      autovettori nel modo descritto al Passo 3, e avremo una matrice associata $[T]$ che
	      risulterà diagonale. Mantenendo la notazione introdotta al Passo 3 troviamo sulla
	      diagonale $\dim(V_{\lambda_1})$ coefficienti uguali a
	      $\lambda_1$, ... e $\dim(V_{\lambda_k})$ coefficienti uguali a $\lambda_k$.
\end{itemize}
Il rango di $T$ sarà uguale al numero dei coefficienti non nulli che troviamo
sulla diagonale di $[T]$, la dimensione del nucleo sarà uguale al numero dei
coefficienti uguali a zero che troviamo sulla diagonale di $[T]$.

\begin{example}
	Consideriamo l'endomorfismo
	\[
		T \begin{pmatrix} x \\ y \end{pmatrix} =
		\begin{pmatrix}
			x + 2y \\
			-y
		\end{pmatrix}
	\]
	e la sua matrice associata rispetto alle basi standard
	\[
		[T] = \begin{pmatrix}
			1 & 2  \\
			0 & -1
		\end{pmatrix}
	\]
	Per prima cosa calcoliamo la matrice $[T] - tI$ che chiameremo $M$ per comodità
	\[
		M = \begin{pmatrix}
			t - 1 & -2    \\
			0     & t + 1
		\end{pmatrix}
	\]
	Troviamo il polinomio caratteristico calcolando il determinante di $M$ e otteniamo
	\[
		P_T(t) = (t - 1)(t + 1)
	\]
	Le radici di tale polinomio (e quindi gli autovalori di $T$) sono $t = 1$ e $t = -1$.
	Dobbiamo trovare quindi i relativi autospazi.
	\begin{itemize}
		\item Se $t = 1$ dobbiamo calcolare $\Ker([T] - 1I)$. Dobbiamo quindi risolvere il
		      sistema associato alla matrice
		      \[
			      \begin{pmatrix}
				      1 - 1 & -2    \\
				      0     & 1 + 1
			      \end{pmatrix} =
			      \begin{pmatrix}
				      0 & -2 \\
				      0 & 2
			      \end{pmatrix}
		      \]
		      ovvero
		      \[
			      \begin{cases}
				      -2y & = 0 \\
				      2y  & = 0
			      \end{cases} \quad \Rightarrow \quad
			      y = 0
		      \]
		      otteniamo dunque che l'autospazio $V_1$ è definito come segue
		      \[
			      V_1 = < \begin{pmatrix} 1 \\ 0 \end{pmatrix} > \quad \Rightarrow \quad
			      \dim(V_1) = 1
		      \]
		\item Se $t = -1$ procediamo in maniera analoga. Stavolta otteniamo il sistema
		      \[
			      \begin{cases}
				      -2x - 2y & = 0 \\
			      \end{cases} \quad \Rightarrow \quad
			      x = -y
		      \]
		      Ne deduciamo che l'autospazio $V_2$ sarà definito come segue
		      \[
			      V_2 = < \begin{pmatrix} -1 \\ 1 \end{pmatrix} > \quad \Rightarrow \quad
			      \dim(V_2) = 1
		      \]
	\end{itemize}
	Dato che $T$ è definta su $\R^2$ che ha dimensione 2 e dato che
	\[ \dim(V_1) + \dim(V_2) = 2 \] l'endomorfismo è diagonalizzabile.
\end{example}
\chapter{Funzioni ricorsive generali}
Introduciamo ora le \textbf{funzioni ricorsive generali} come
un'estensione delle ricorsive primitive. In particolare vogliamo
aggiungere, agli schemi primitivi di base di \emph{combinazione}
e \emph{ricorsione primitiva} (\ref{def: ricorsive primitive}),
lo schema di \textbf{minimizzazione}.

Dobbiamo prima però introdurre un po' di notazione: l'operatore
$\mu$, detto \textbf{operatore di minimizzazione}, applicato
ad un insieme di numeri naturali, ne restituisce il minimo, se
presente.

\begin{definition} \label{def: mu ricorsive}
	La classe delle funzioni \textbf{$\mu$-ricorsive} (o
	\textbf{ricorsive generali}) è la minima classe
	$\RR$ tale che soddisfa le condizioni di
	\begin{itemize}
		\item \emph{Zero} e \emph{ricorsione primitiva}.
		\item \textbf{Minimizzazione}: se
		      $\varphi (x_1, \dots, x_n, y) \in \RR$ in $n+1$
		      variabili, allora la funzione $\psi$ in $n$
		      variabili è in $\RR$ se è definita dal minimo $y$
		      tale che
		      \begin{itemize}
			      \item $\varphi(x_1, \dots, x_n, y) = 0$
			      \item $\forall z \leq y$ vale $\varphi(x_1,
				            \dots, x_n, z) \downarrow$.
		      \end{itemize}
		      In forma più compatta la condizione di
		      minimizzazione è la seguente
		      \[
			      \psi (x_1, \dots, x_n) = \mu y [
					      \forall z \leq y, \;
					      \varphi(x_1, \dots, x_n, z)
					      \downarrow]
		      \]
	\end{itemize}
\end{definition}

Una funzione $\mu$-ricorsiva è intuitivamente calcolabile poiché
l'algoritmo \emph{intuitivo} che la calcola è composto da un
ciclo in cui si incrementa la variabile $y$ (inizialmente posta
a $0$), si calcola la $\varphi$ e si ripetono questi passi
finché il risultato non è $0$. I primi passi dell'esecuzione
di questo algoritmo potrebbero dunque essere:
\begin{enumerate}
	\item Calcolare $\varphi(x_1, \dots, x_n, 0)$. Se il
	      risultato è $0$, allora $\psi (x_1, \dots, x_n) = 0$.
	\item Altrimenti si calcola $\varphi(x_1, \dots, x_n, 1)$,
	      se il risultato è $0$, allora
	      $\psi(x_1, \dots, x_n)=1$.
	\item ...
\end{enumerate}
Intuitivamente l'algoritmo potrebbe non terminare mai perché
\begin{itemize}
	\item Per ogni valore di $y$ esiste un $m_y$ tale che
	      \[ \varphi (x_1, \dots, x_n, y) = m_y \neq 0 \]
	\item Per i primi $k$ numeri naturali vale che
	      \[
		      \varphi (x_1, \dots, x_n, z) = n_z \neq 0
		      \quad \land \quad
		      \varphi (x_1, \dots, x_n, k) \uparrow
	      \]
\end{itemize}
Nel primo caso infatti continuiamo a calcolare
$\varphi(x_1, \dots, x_n, y)$ per valori crescenti di $y$ senza
quindi terminare mai. Nel secondo caso non ci arrestiamo mai nel
calcolo di $\varphi (x_1, \dots, x_n, k) \uparrow$ poiché la
funzione diverge: da qui la parzialità di $\psi$.

\begin{example}
	Prendiamo ad esempio la seguente funzione
	\begin{align*}
		\varphi       & = \lambda x, y . 3 \\
		\psi_\uparrow & =
		\lambda x . (\mu y . \varphi(x, y) = 0)
	\end{align*}
	In questo caso possiamo facilmente notare che $\forall x$ il
	calcolo di $\varphi$ termina. Altrettanto facile è
	verificare che, per nessun $x$, esiste un $y$ tale che
	$\varphi (x, y) = 0$ e quindi la funzione $\psi$ è
	indefinita per ogni valore di $x$.
\end{example}

Dobbiamo quindi partire da una funzione ricorsiva primitiva e
applicargli l'operatore di minimizzazione $\mu$ per ottenere
una funzione $\mu$-ricorsiva. Tutto ciò a patto che la
condizione che per ogni $z \leq y$, la funzione
$\varphi (x_1, \dots, x_n)$ converge. In caso contrario la
funzione $\psi$ potrebbe non essere $\mu$-ricorsiva. Si noti
anche che
\[
	f(x) = \begin{cases}
		\mu y [y < g(x), \; h(x, y) = 0] &
		\text{se esiste tale } y                             \\
		0                                & \text{altrimenti}
	\end{cases}
\]
è ricorsiva primitiva se $g$ e $h$ lo sono. La ragione è che $g$
impone un limite ai tentativi di ricercare il minimo $y$, e
quindi o lo si trova in meno di $g(x)$ applicazioni di $h$ o
diamo risultato $0$.

\begin{theorem}[Tesi di Church-Turing] \label{th: church-turing}
	Le funzioni intuitivamente calcolabili sono tutte e sole
	le funzioni parziali T-calcolabili.
\end{theorem}

Più che un teorema, questo appena enunciato, è una tesi che è
stata dimostrata per i formalismi già definiti ma che non può
essere dimostrata per quelli non ancora definiti. Si tratta
comunque di una tesi talmente forte che viene trattata come un
teorema.

\section{Risultati classici}
Arriviamo finalmente al sodo di questa prima parte del corso in
cui abbiamo definito formalismi su formalismi senza capire bene
come, quando o perché applicarli.

Introdurremo quindi alcuni risultati della teoria della
calcolabilità che ci permetteranno di caratterizzare la classe
delle funzioni calcolabili, mediante alcuni teoremi di
\emph{"chiusura"}.

Prima di iniziare chiariamo che, grazie alla
\hyperref[th: church-turing]{tesi di Church-Turing}, possiamo
chiamare \emph{calcolabili} tutte le funzioni che rispettano
le cinque condizioni intuitive poste agli algoritmi definite
nell'\hyperref[sec: algoritmo]{idea intuitiva di algoritmo},
indipendentemente dal loro formalismo.

\subsection{Cardinalità delle funzioni calcolabili}
Iniziamo con un paio di teoremi che dovrebbe darci l'idea del
numero di funzioni calcolabili e non, dimostrando anche
l'esistenza di queste ultime.

\begin{theorem} \label{th: n calc}
	Le funzioni calcolabili sono $\#(\N)$, così come le funzioni
	calcolabili totali.
	\begin{proof}
		Costruiamo $\#(\N)$ MdT $M_i$ che svuotano il nastro
		dell'input e scrivono una stringa di tante $|$ quanto
		vale $i$ e si arrestano.
	\end{proof}
\end{theorem}

Che non siano più di $\#(\N)$ segue dal fatto che le $MdT$
si possono enumerare, come mostrato in precedenza
(\ref{ssec: enum MdT}).

\begin{theorem} \label{th: exists non calc}
	Esistono funzioni non calcolabili.
	\begin{proof}
		Con una costruzione analoga a quella di Cantor, in cui
		la classe dei sottoinsiemi di $\N$ non è numerabile,
		si vede che
		\[
			\# \left( \{ f \mid f : \N \to \N \} \right) =
			2^{\#(\N)}
		\]
		segue dal teorema \ref{th: n calc} che esistono dunque
		delle funzioni non calcolabili.
	\end{proof}
\end{theorem}

\subsection{Forma normale ed equivalenze}
Come abbiamo già visto, è possibile enumerare le MdT, associando
loro un indice. Analogamente è possibile enumerare le funzioni
ricorsive primitive e non è difficile pensare ad un'estensione
per l'enumerazione di funzioni $\mu$-ricorsive.

I due metodi hanno in comune il fatto che si basano solamente
sui simboli usati per definire gli algoritmi. Infatti, sotto
ragionevoli ipotesi, per i nostri scopi non c'è differenza tra
un metodo di enumerazione e l'altro, purché sia \emph{effettivo}.

Data un'enumerazione effettiva, indicheremo con $\varphi_i$ la
funzione parziale che la macchina, o meglio l'algoritmo, $M_i$
calcola e chiameremo $i$ \emph{indice}. L'indice è riferito
alla macchina e non alla funzione, infatti potrebbe darsi che
per $i \neq j$, valga $\varphi_i = \varphi_j$, ma sicuramente
vale $M_i \neq M_j$.

\begin{theorem}[Padding lemma] \label{th: padding lemma}
	Ogni funzione calcolabile $\varphi_x$ ha $\# (N)$ indici.
	Vale inoltre che $\forall x$ si può costruire, mediante una
	funzione ricorsiva primitiva, un insieme infinito $A_x$ di
	indici tale che $\forall y \in A_x$, vale
	\[ \varphi_y = \varphi_x \]
	\begin{proof}
		Per ogni macchina $M_x$, se $Q = \{ q_0, \dots, q_k \}$
		è l'insieme degli stati possibili di $M_x$. Aggiungendo
		lo stato $q_{k+1}$ e la quintupla
		\[ (q_{k+1}, \#, q_{k+1}, \#, -) \]
		a tale macchina, si ottiene la macchina $M_{x_1}$ con
		$x_1 \in A_x$. Possiamo continuare all'infinito
		aggiungendo stati e quintuple che di fatto sono inutili,
		o per meglio dire non vengono mai raggiunti dalla
		macchina e non cambiano quindi cosa essa calcola.
	\end{proof}
\end{theorem}

Questo teorema ci dice che esistono infiniti algoritmi
numerabili che calcolano la stessa funzione e che alcuni di
loro sono ottenibili \emph{facilmente} da un algoritmo dato.

\begin{theorem}[Forma normale] \label{th: fn}
	Esistono un predicato $T(i, x, y)$ e una funzione $U(y)$
	calcolabili totali tali che $\forall i,x$ vale
	\[ \varphi_i(x) = U(\mu y . T (i, x, y)) \]
	Inoltre $T$ e $U$ sono ricorsive primitive.
	\begin{proof}
		Definiamo $T(i,x,y)$, detto comunemente
		\textbf{predicato di Kleene}, vero se e solo se $y$ è
		la codifica di una computazione terminante di $M_i$
		con dato iniziale $x$. Per calcolare $T$ dato $i$,
		recuperiamo $M_i$ dalla lista e cominciamo a scandire
		i valori $y$. Decodifichiamo ognuno di essi e, avendo
		come ingresso $x$ controlliamo se il risultato è una
		computazione terminante della forma
		\[ M_i(x) = c_0, c_1, \dots, c_n \]
		Se lo è, allora $c_n = (h, \start z \underline{\#})$ e
		definiamo $U$ in modo che $U(y) = z$.

		Il procedimento è effettivo e quindi $T$ e $U$ sono
		calcolabili per la tesi di Church-Turing, inoltre tale
		procedimento termina sempre e dunque $T$ e $U$ sono
		totali. Abbiamo inoltre che $T$ e $U$ sono ricorsive
		primitive perché sai le codifiche usate, che i controlli
		effettuati lo sono.
	\end{proof}
\end{theorem}

Questo teorema ci dice che tra tutti gli algoritmi che calcolano
la stessa funzione, uno di questi ha una forma privilegiata,
ossia quella \emph{normale} e di conseguenza ogni funzione ha
una forma normale.

Proviamo ad uscire un minimo dal formalismo del teorema stesso
procedendo per step. Iniziamo dal predicato di Kleene: esso è
una funzione che semplicemente ritorna vero se $y$ è la codifica
(in questo caso possiamo considerare l'enumerazione di G\"odel)
di una computazione terminante della macchina $M_i$ che prende
in input $x$.

Come abbiamo già visto una computazione, vista come una sequenza
finita di passi è anch'essa enumerabile ed è quindi possibile
immaginare che valga
\[ M_i (x) = c_0, c_1, \dots, c_n \]
dove i vari $c_i$ sono le varie configurazioni raggiunte dalla
macchina durante la computazione. Se la computazione termina
allora abbiamo che $T$ ritorna vero e che sicuramente l'ultima
configurazione, ossia $c_n$ è equivalente a
\[ c_n = (h, \start z \underline{\#}) \]
poiché siamo sicuramente giunti nello stato speciale $h$. Ho
qualche dubbio sul fatto che il cursore debba essere perforza
posizionato su un valore bianco in quanto, da quel che abbiamo
visto fino ad ora, l'unico requisito affinché una computazione
venga considerata terminata è che giunga nello stato $h$.

Ciò che è importante capire è che $U$ è la parte di stringa
della configurazione finale privata del respingente.

\begin{example}
	Riprendiamo l'esempio della MdT in grado di riconoscere la
	presenza di una sequenza di due $1$ in una stringa data in
	input.

	In quell'esempio avevamo definito una funzione di
	transizione che sovrascriveva qualunque valore incontrasse
	con un $\#$. Per qualsiasi stringa in input che contenesse
	una sequenza di due $1$ consecutivi, la macchina si sarebbe
	sempre arrestata con la seguente configurazione finale
	\[ h / \start \# \dots \underline{\#} \]
	Per il teorema dobbiamo andare a prendere il \emph{minimo}
	$y$ che rappresenta una computazione terminante della
	macchina $i$-esima che prende un certo input $x$.

	Nel caso delle macchina di Turing ci sarà in realtà un solo
	$y$ del genere (discorso differente ad esempio per le
	funzioni ricorsive in cui possiamo avere diverse politiche
	di valutazione). Nel nostro specifico, tale $y$, sarà quello
	che codifica la computazione della macchina $i$ (quella che
	abbiamo scritto noi) che prende in input la stringa
	$x = 01011$ e che quindi avrà come configurazione finale
	\[ h / \start \# \# \# \# \underline{\#} \]
	Secondo il teorema vale che
	\[ \varphi_i (x) = \varphi (01011) = U(y) = \# \# \# \# \# \]
	che in realtà andrebbe convertito in nel formalismo
	ricorsivo perché abbiamo effettivamente più senso.
\end{example}

\begin{corollary}
	Le funzioni T-calcolabili sono $\mu$-ricorsive.
\end{corollary}

Questo corollario è una diretta conseguenza del teorema di
forma normale. Possiamo quindi dire che ogni funzione calcolata
da una MdT ammette una definizione $\mu$-ricorsiva.

\begin{lemma}
	Le funzioni $\mu$-calcolabili sono T-calcolabili.
\end{lemma}

Possiamo a questo punto concludere che l'equivalenza tra MdT
e funzioni $\mu$-ricorsive.

\begin{theorem}
	Una funzione è T-calcolabili se e solo se è
	$\mu$-calcolabile.
\end{theorem}

Il teorema di forma normale e quello d'equivalenza tra MdT e
funzioni $\mu$-ricorsive ha il seguente corollario interessante
dal punto di vista informatico. La sua rilevanza nel nostro
campo è legata al fatto che le funzioni primitive ricorsive
si possono rappresentare come un programma in linguaggio
\verb|FOR|, mentre le $\mu$-ricorsive con un programma in
linguaggio \verb|WHILE|.

\begin{corollary}
	Ogni funzione calcolabile parziale può essere può essere
	ottenuta da due funzioni ricorsive primitive e una sola
	applicazione di dell'operatore $\mu$.
\end{corollary}

\subsection{Formalismo universale}
Supponiamo ora di avere a disposizione un formalismo
\textbf{universale}, in grado cioè di esprimere \emph{tutte}
le funzione calcolabili. Questo sarebbe così potente da
riuscire ad esprimere l'interprete dei propri programmi.

\begin{theorem}[Enumerazione]
	Esiste una funzione numerabile parziale $\varphi_z(i, x)$
	tale che $\forall i,x$ vale
	\[ \varphi_i(x) = \varphi_z (i, x) \]
	\begin{proof}
		Poiché la funzione $U(\mu y . T(i, x, y))$ usata nel
		teorema \ref{th: fn} è definita per composizione e
		$\mu$-ricorsione a partire da funzioni ricorsive
		primitive, essa stessa è una una funzione calcolabile
		in due argomenti $i$ e $x$. Avrà quindi un indice che
		chiamiamo $z$ e per cui vale
		\[ \varphi_z (i, x) = U(\mu y. T(i, x, y)) \]
		Applichiamo quindi il teorema di forma normale per
		ottenere
		\[ U(\mu y . T(i, x, y)) = \varphi_i (x) \]
		da qui otteniamo la tesi per transitività
		dell'uguaglianza
		\[
			\varphi_z (i, x) = U(\mu y . T(i, x, y)) =
			\varphi_i (x)
		\]
		Più informalmente la macchina $M_z$ recupera la
		descrizione della macchina $M_i$ e la applica a $x$.
	\end{proof}
\end{theorem}

Il teorema, scritto in forma molto compatta, in sostanza ci
dice che esiste un algoritmo (o una MdT) $z$ che prende in
input un altro algoritmo (o MdT) $i$, i dati $x$ su cui lavora
$i$ e restituisce lo stesso risultato dell'algoritmo $i$-esimo
applicato a $x$. In genere la macchina di Turing $z$ viene
chiamata \textbf{macchina di Turing universale}.

Passiamo ora ad un teorema che possiamo vedere un po' come il
duale del teorema di enumerazione. Enunceremo una prima versione
semplificata e poi la sua forma generale.

\begin{theorem}[Teorema del parametro s-1-1]
	\label{th: s-1-1}
	Esiste una funzione $s$ calcolabile totale tale che per ogni
	$i$, $x$ e $y$ vale
	\[ \lambda y . \varphi_i (x, y) = \varphi_{s (i, x)} (y) \]
	\begin{proof}
		Da $i$ prendiamo l'$i$-esima MdT e scriviamo sul nastro
		di $M_i$ l'input $x$ e $y$. Questi due passi definiscono
		un algoritmo il quale ha un indice $j$ per la tesi di
		Church-Turing. Tale indice mi identifica la macchina che
		calcola
		\[ \varphi_{s(i,x)} \]
		A questo punto usiamo il padding lemma per trovare una
		$h$ maggiore di tutti i valori ottenuti fino ad ora.
		Otteniamo così una funzione strettamente crescente che
		è anche iniettiva.
	\end{proof}
\end{theorem}

Come possiamo notare nella prima funzione abbiamo due argomenti
e nella seconda uno solo. Il fatto è che quando $x$ viene
considerato come un \emph{parametro}, allora è possibile
calcolare la prima funzione utilizzando un altra macchina a
partire dall'indice $i$.

\begin{example}
	Prendiamo ad esempio la funzione
	\[ x + y \]
	Una volta fissata la $x$ ad un certo valore, supponiamo
	$2$, la $x$ a questo punto non è più un argomento ma
	diventa un parametro. A questo punto la funzione
	\[ 2 + y \]
	può essere calcolata da un'altra macchina che a regola
	dovrebbe essere un po' più efficiente di quella di partenza.
\end{example}

Questo sta alla base della \textbf{valutazione parziale} che è
molto utile in quei contesti in cui si vuole mantenere un alto
livello di generalizzazione.

\begin{theorem}[Teorema del parametro s-m-n]
	\label{th: s-m-n}
	Per ogni $m, n \geq 0$ esiste una funzione calcolabile
	totale (iniettiva) $s_n^m$ con $m+1$ argomenti, tale che
	per ogni $x, y_1, \dots, y_m$, vale
	\[
		\varphi_{s_n^m (x, y_1, \dots, y_m)}^{(n)} =
		\lambda z_1, \dots, z_n .
		\varphi_x^{(m+n)} (y_1, \dots, y_m, z_1, \dots, z_n)
	\]
\end{theorem}

Si noti come il teorema del parametro e quello di enumerazione
siano in un certo senso l'inverso l'uno dell'altro. Infatti
l'uno "abbassa" un argomento nella posizione di indice, mentre
l'altro "innalza" un indice nella posizione di argomento.

\begin{theorem}[Espressività]
	Un formalismo è \textbf{Turing-equivalente}, ossia calcola
	tutte e sole le funzioni T-calcolabili ed è universale, se
	e solo se
	\begin{itemize}
		\item Ha un algoritmo universale (vale cioè il teorema
		      di enumerazione).
		\item Vale il teorema del parametro.
	\end{itemize}
\end{theorem}

Grazie al teorema del parametro si dimostra un altro teorema
che ha un ruolo fondamentale sia in informatica che in teoria
della calcolabilità.

\begin{theorem}[Punto fisso]
	per ogni funzione $f$ calcolabile totale $\exists n$ tale
	che
	\[ \varphi_n = \varphi_{f(n)} \]
	\begin{proof}
		Definiamo la seguente funzione
		\[
			\psi (u, z) = \varphi_{d(u)} (z) =
			\begin{cases}
				\varphi_{\varphi_u (u)} (z) &
				\text{se } \varphi_u (u) \downarrow \\
				\text{indefinita}           &
				\text{altrimenti}
			\end{cases}
		\]
		Per il teorema del parametro, $d(u)$ è totale e
		iniettiva (e non dipende da $f$). Data $f$, allora
		anche $f \circ d$ è calcolabile e sia $v$ proprio
		un indice tale che
		\[ \varphi_v(x) = f(d(x)) \]
		Tale funzione è totale (perché sia $d$ che $f$ lo sono)
		e quindi $\varphi_v (v)$ converge. Pertanto, in accordo
		con la definizione data in precedenza di $\psi (u,z)$
		abbiamo che
		\[ \varphi_{d(v)} = \varphi_{\varphi_v(v)} \]
		Calcoliamo ora $d(v)$ e supponiamo che il risultato sia
		$n$, cioè poniamo
		\[ n = d(v) \]
		Dimostriamo che è un punto fisso di $f$. Infatti vale
		la seguente catena di eguaglianze
		\[
			\varphi_n  = \varphi_{d(v)}
			= \varphi_{\varphi_v (v)}
			= \varphi_{f(d(v))}
			= \varphi_{f(n)}
		\]
		Nell'eguaglianza più a sinistra si sfrutta l'iniettività
		che viene garantita dal teorema del parametro.
	\end{proof}
\end{theorem}

Un tale indice viene detto \textbf{punto fisso} di $f$. Il
teorema inoltre ci dice che la funzione $f$ trasforma algoritmi
in algoritmi, proprio come fa un compilatore.

In altre parole il punto fisso non cambia la funzione calcolata
ma trasforma l'algoritmo $P_n$ nell'algoritmo $P_{f(n)}$ con la
stessa semantica.

\begin{property}
	Nelle ipotesi del teorema di ricorsione
	\begin{itemize}
		\item Il punto fisso è calcolabile mediante una funzione
		      totale (iniettiva) $g$ a partire dall'indice di
		      $f$.
		\item Ci sono $\# (\N)$ punti fissi di $f$.
	\end{itemize}
	\begin{proof}
		Per dimostrare il primo punto prendiamo $h(x)$
		calcolabile totale tale che $\forall n$ vale
		\[ \varphi_{h(x)} (n) = \varphi_x (d(n)) \]
		Allora vale
		\[ g(x) = d(h(x)) \]
		Il secondo punto segue invece dal teorema
		\ref{th: n calc}.
	\end{proof}
\end{property}



\end{document}
