\section{Teorema di ricorsione (Kleene II)}

Per ogni funzione $f$ calcolabile totale esiste $n$ tale che
\[ \varphi_n = \varphi_{f(n)} \]

\subsection{Dimostrazione}

Dobbiamo quindi dimostrare che esiste questa $n$ per ogni $f$
calcolabile totale. Per dimostrare che esiste andiamo a definire
la funzione calcolabile
\[
	\psi (u, z) = \begin{cases}
		\varphi_{\varphi_u(u)} (z) & \text{se } \varphi_u (u) \downarrow \\
		\text{indefinita}          & \text{altrimenti}
	\end{cases}
\]
Tale funzione è calcolabile poiché per calcolarla prendiamo
$M_u$ e gli diamo $u$ in input
\begin{itemize}
	\item Se termina siamo nel primo ramo.
	\item Se non termina siamo nel secondo ramo.
\end{itemize}
La funzione è inoltre totale poiché è ovunque definita (il
calcolo di $\varphi_u (u)$ o termina o non termina).

Se $\psi$ è calcolabile, per Church-Turing ha un indice $i$
e dunque vale che
\[ \psi (u, z) = \varphi_i (u, z) \]
A questo punto possiamo applicare il teorema del parametro
perché $u$ è costante e dunque otteniamo
\[ \psi (u, z) = \varphi_i (u, z) = \varphi_{s(i, u)} (z) \]
Ma la $\psi$ è fissata e dunque anche l'indice $i$ lo è. Possiamo
allora scrivere che
\[
	\psi (u, z) = \varphi_i (u, z) = \varphi_{s(i, u)} (z) =
	\varphi_{f(u)} (z)
\]
Abbiamo quindi che se $\varphi_u (u) \downarrow$ vale che
\[
	\psi (u, z) = \varphi_i (u, z) = \varphi_{s(i, u)} (z) =
	\varphi_{f(u)} (z) = \varphi_{\varphi_u(u)} (z)
\]
da cui si deduce che
\[ f(u) = \varphi_u(u) \]