\section{Riduzioni}

\subsection{K si riduce secondo rec a CONST}

Per dimostrare che $K \leq_{rec} CONST$ ricordiamoci prima cosa
come sono definiti i vari insiemi

\begin{gather*}
	K = \{ x \mid \varphi_x (x) \downarrow \} \\
	CONST = \{ x \mid \varphi \text{ è totale e costante} \} \\
	rec = \{ \varphi_x \mid \forall y \in \N . \varphi_x (y) \downarrow \}
\end{gather*}

Dire che $K \leq_{rec} CONST$ significa che esiste una funzione
$f \in rec$ tale che
\[ \forall x \in K \implies f(x) \in CONST \]
Dire che $x \in K$ equivale a dire che $\varphi_x (x)$ converge,
dire invece che $f(x) \in CONST$ equivale a dire che la funzione
$\varphi_{f(x)}$ è totale e costante. L'obbiettivo della
dimostrazione è quindi quello di trovare la $f$ tale per cui sia
vera quest'ultima cosa. Iniziamo con il definire la funzione
\[
	\psi (x, y) = \begin{cases}
		1                 & \text{se } \exists z > y \mid
		\varphi_x(x) \downarrow \text{ in meno di $z$ passi} \\
		\text{indefinita} & \text{altrimenti}
	\end{cases}
\]
Ci chiediamo ora se $\psi$ è calcolabile. Intuitivamente possiamo
prendere l'$x$-esima macchina $M_x$ ed effettuare la più $z$
passi nel calcolo di $M_x(x)$.
\begin{itemize}
	\item Se converge in meno di $z$ passi allora cadiamo nel
	      primo ramo e $\psi(x, y) = 1$.
	\item Se invece non converge in meno di $z$ passi allora
	      cadiamo nel secondo ramo e $\psi(x,y)$ è indefinita.
\end{itemize}
Abbiamo quindi trovato una procedura che calcola $\psi$ che
termina sempre e dunque la funzione è calcolabile. Dato che la
funzione è calcolabile allora ha un indice $i$ e possiamo quindi
scrivere
\[ \psi(x, y) = \varphi_i (x, y) \]
A questo punto possiamo applicare il teorema del parametro per
ottenere
\[ \psi(x, y) = \varphi_i (x, y) = \varphi_{s(i,x)} (y) \]
Se notiamo che $i$ è costante (è fissato perché la funzione
$\psi$ è fissata), possiamo scrivere
\[
	\psi(x, y) = \varphi_i (x, y) =
	\varphi_{s(i,x)} (y) = \varphi_{f(x)} (y)
\]
Ecco che abbiamo ritrovato la stessa struttura che avevamo messo
in evidenza all'inizio. Avevamo detto che se $x \in K$ allora
$f(x) \in CONST$. Dire che $f(x) \in CONST$ equivale a dire che
$f(x)$ è l'indice di una funzione ($\varphi_{f(x)}$) totale e
costante.

Vogliamo quindi dimostrare che se $x \in K$, allora $f(x)$ è
l'indice di una funzione totale e costante. Vediamo quindi che
succede se $x \in K$
\[ x \in K \implies \varphi_x (x) \downarrow \]
e quindi
\[ \psi (x, y) = \varphi_{f(x)} (y) = 1 \]
e quindi come possiamo vedere $f(x)$ è l'indice di una funzione
$\varphi_{f(x)}$ che è totale e costante (in quanto uguale a
$1$). Possiamo quindi concludere che $f(x) \in CONST$. Per
terminare dobbiamo dimostrare che
\[ x \notin K \implies f(x) \notin CONST \]
Seguiamo la stessa catena di implicazioni:
\[ x \notin K \implies \varphi_x (x) \uparrow \]
quindi sicuramente supera gli $z$ passi di limite che avevamo
definito e dunque
\[ \psi (x, y) = \varphi_{f(x)} (y) = \text{indefinita} \]
E dunque in questo caso $f(x)$ è l'indice di una funzione
indefinita e dunque non totale (requisito necessario) affinché
$f(x) \in CONST$. Concludiamo quindi che
\[ x \notin K \implies f(x) \notin CONST \]
