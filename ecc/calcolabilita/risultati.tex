\section{Risultati classici}
Arriviamo finalmente al sodo di questa prima parte del corso in
cui abbiamo definito formalismi su formalismi senza capire bene
come, quando o perché applicarli.

Introdurremo quindi alcuni risultati della teoria della
calcolabilità che ci permetteranno di caratterizzare la classe
delle funzioni calcolabili, mediante alcuni teoremi di
\emph{"chiusura"}.

Prima di iniziare chiariamo che, grazie alla
\hyperref[th: church-turing]{tesi di Church-Turing}, possiamo
chiamare \emph{calcolabili} tutte le funzioni che rispettano
le cinque condizioni intuitive poste agli algoritmi definite
nell'\hyperref[sec: algoritmo]{idea intuitiva di algoritmo},
indipendentemente dal loro formalismo.

\begin{theorem} \label{th: n calc}
	Le funzioni calcolabili sono $\#(\N)$, così come le funzioni
	calcolabili totali.
	\begin{proof}
		Costruiamo $\#(\N)$ MdT $M_i$ che svuotano il nastro
		dell'input e scrivono una stringa di tante $|$ quanto
		vale $i$ e si arrestano. Che non siano più di $\#(\N)$
		segue dal fatto che le $MdT$ si possono enumerare, come
		mostrato in precedenza \hyperref[ssec: enum MdT]{qui}.
	\end{proof}
\end{theorem}

\begin{theorem}
	Esistono funzioni non calcolabili.
	\begin{proof}
		Con una costruzione analoga a quella di Cantor, in cui
		la classe dei sottoinsiemi di $\N$ non è numerabile,
		si vede che
		\[
			\# \left( \{ f | f : \N \to \N \} \right) =
			2^{\#(\N)}
		\]
        segue dal teorema \ref{th: n calc} che esistono dunque 
        delle funzioni non calcolabili.
	\end{proof}
\end{theorem}
