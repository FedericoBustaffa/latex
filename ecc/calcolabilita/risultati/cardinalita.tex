\section{Cardinalità delle funzioni calcolabili}
Iniziamo con un paio di teoremi che dovrebbe darci l'idea del
numero di funzioni calcolabili e non, dimostrando anche
l'esistenza di queste ultime.

\begin{theorem} \label{th: n calc}
	Le funzioni calcolabili sono $\#(\N)$, così come le funzioni
	calcolabili totali.
	\begin{proof}
		Costruiamo $\#(\N)$ MdT $M_i$ che svuotano il nastro
		dell'input e scrivono una stringa di tante $|$ quanto
		vale $i$ e si arrestano.
	\end{proof}
\end{theorem}

Che non siano più di $\#(\N)$ segue dal fatto che le $MdT$
si possono enumerare, come mostrato in precedenza
(\ref{ssec: enum MdT}).

\begin{theorem} \label{th: exists non calc}
	Esistono funzioni non calcolabili.
	\begin{proof}
		Con una costruzione analoga a quella di Cantor, in cui
		la classe dei sottoinsiemi di $\N$ non è numerabile,
		si vede che
		\[
			\# \left( \{ f \mid f : \N \to \N \} \right) =
			2^{\#(\N)}
		\]
		segue dal teorema \ref{th: n calc} che esistono dunque
		delle funzioni non calcolabili.
	\end{proof}
\end{theorem}
