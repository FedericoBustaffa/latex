\section{Problemi insolubili e riducibilità}
In quest'ultimo capitolo di questa prima parte di calcolabilità
andremo a trattare i problemi in cui si cerca di capire se un
elemento appartiene o no ad un dato insieme.

Fino ad ora ci siamo concentrati sulla risoluzione di problemi
tramite il calcolo di una funzione. In questa fase proviamo
invece un approccio alternativo, che però vedremo in seguito
essere correlato al metodo a cui siamo stati abituati fino ad
ora.

Un esempio molto semplice di correlazione tra i due metodi è
quello dato dalla correlazione tra una funzione e il suo dominio.
\[
	\lambda x . 2 x \leftrightarrow \{ \N \} \qquad
	\lambda x . x / 2 \leftrightarrow \{ 2n \mid n \in \N \}
\]
Oppure tra una funzione e la suna immagine
\[
	\lambda x . 2 x \leftrightarrow \{ 2 n \mid n \in \N \}
	\qquad \lambda x . x / 2 \leftrightarrow \{ \N \}
\]


