\chapter{Calcolabilità}
Lo scopo del corso è quello di stabilire le potenzialità e i
limiti del calcolo, formalizzando in modo esatto l'intuizione.

\section{Idea intuitiva di algoritmo} \label{sec: algoritmo}
Tralasciando i molti formalismi proposti per esprimere
algoritmi, in ognuno di essi, gli algoritmi devono soddisfare
i seguenti requisiti
\begin{itemize}
	\item Deve essere costituito da un insieme \emph{finito} di
	      istruzioni.
	\item Le istruzioni possibili sono in numero \emph{finito}
	      e hanno effetto \emph{limitato} su dati discreti,
	      esprimibili in maniera finita.
	\item Una computazione è eseguita per \emph{passi discreti},
	      senza ricorrere a sistemi analogici o metodi continui,
	      ciascuno dei quali impiega un tempo \emph{finito}.
	\item Ogni passo dipende solo dai precedenti e da una
	      porzione finita dei dati, in modo
	      \emph{deterministico}.
	\item Non c'è limite al numero di passi necessari
	      all'esecuzione di un algoritmo, né alla memoria
	      richiesta per contenere i dati iniziali, intermedi ed
	      eventualmente finali.
\end{itemize}
Sotto queste ipotesi, tutte le formulazioni fino ad ora
sviluppate sono equivalenti e si postula che lo saranno anche
tutte le future.

