\section{Analisi degli errori}
Parliamo ora delle tecniche per l'analisi dell'errore. La continuità della funzione implica che il problema sia
\emph{ben posto}. La relazione
\[
	\ein = \frac{f(\tx) - f(x)}{f(x)} =
	\frac{f(\tx) - f(x)}{\tx - x} \frac{x}{f(x)} \frac{\tx - x}{x}
\]
si ricava che la \emph{differenziabilità} di $f(x)$ è essenziale per il controllo dell'errore inerente. In
particolare se la funzione è \textbf{regolare}, ossia derivabile due volte con derivata continua, allora vale
lo sviluppo di Taylor
\[ f(\tx) = f(x) + f'(x) (\tx - x) + \frac{f''(\xi)}{2} (\tx - x)^2 \]
con $|\xi - x| \leq |\tx - x|$, da cui si ottiene
\[ \ein \frac{}{} \]