\section{Errori nel calcolo di funzioni non razionali}
Nel caso in cui si abbia a che fare con \textbf{funzioni non razionali} dobbiamo, prima di procedere al
calcolo in macchina, approssimarla con una funzione razionale.

Dobbiamo quindi trovare una funzione $h$ che approssima $f$, la quale introduce un nuovo tipo di errore, detto
\textbf{errore analitico} dato dalla formula
\[ \ean = \frac{h(x) - f(x)}{f(x)} \]
Un possibile modo di approssimare la funzione $f$ è tramite lo sviluppo di Taylor arrestato ad un certo ordine.

In questo corso non trattiamo l'errore analitico ma siamo comunque interessati all'analisi dell'errore inerente
e algoritmico di funzioni non razionali.

\begin{example}
	Studiamo il condizionamento della funzione
	\[ f(x) = \frac{e^x - 1}{x} \]
	per $x \to 0^+$. Come sappiamo l'errore inerente, in un'analisi al primo ordine, equivale a
	\[ \ein = \frac{f'(x)}{f(x)} \epsilon_x x \]
	Mentre l'errore di rappresentazione sul dato iniziale equivale a
	\[ \epsilon_x = \frac{\tx - x}{x} \]
	con $|\epsilon_x| \leq u$. Per prima cosa dobbiamo calcolare la derivata della funzione $f$
	\[ f'(x) = \frac{x e^x - 1 (e^x - 1)}{x^2} = \frac{x e^x - e^x + 1}{x^2} \]
	Possiamo ora calcolare il coefficienti di amplificazione
	\[
		c_x = \frac{f'(x)}{f(x)} x =
		\frac{x e^x - e^x + 1}{x^2} \frac{x}{e^x - 1} x =
		\frac{x e^x - e^x + 1}{e^x - 1}
	\]
	Il limite per $x \to 0^+$, come possiamo notare, è un caso di indeterminazione
	\[ \lim_{x \to 0^+} \frac{x e^x - e^x + 1}{e^x - 1} = \frac{0}{0} \]
	risolvibile con il teorema di de l'H\^ospital calcolando il limite
	\[ \lim_{x \to 0^+} \frac{e^x + x e^x - e^x}{e^x} = \lim_{x \to 0^+} x = 0 \]
	Concludiamo quindi che il problema per $x \to 0^+$ è perfettamente ben condizionato.
\end{example}

\begin{example}
	Calcoliamo l'errore algoritmico della funzione
	\[ f(x) = \frac{e^x - 1}{x} \]
	assumendo di avere a disposizione una funzione di macchina \verb|Exp| tale che
	\[ \text{Exp} (x) = e^x (1 + \epsilon) \]
	con $|\epsilon| \leq u$
\end{example}