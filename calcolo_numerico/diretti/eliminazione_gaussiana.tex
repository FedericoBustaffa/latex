\section{Eliminazione gaussiana}
Parliamo ora di come sia possibile calcolare una fattorizzazione $LU$ per la risoluzione di sistemi lineari
tramite eliminazione gaussiana.

\subsection{Matrici elementari di Gauss}
Fino ad ora non abbiamo parlato di fattorizzazione $LU$ ma non abbiamo un vero e proprio algoritmo per riuscire a
calcolarla. A tal proposito dobbiamo introdurre le \textbf{matrici elementari di Gauss}, ossia matrici elementari
espresse come
\[ E = I + u \cdot e_j^T \]
Tali matrici sono particolari matrici elementari nelle quali il vettore $e_j$ è il $j$-esimo vettore della base
canonica e il vettore $u$ ha le prime $j$ componenti nulle. Le matrici elementari di Gauss sono triangolari
inferiori con tutti 1 sulla diagonale principale. Per esempio per $j=1$ abbiamo
\[
	E = I + u \cdot e_1^T = I +
	\begin{bmatrix}
		0 \\ u_2 \\ u_3
	\end{bmatrix} \begin{bmatrix}
		1 \\ 0 \\ 0
	\end{bmatrix}^T = \begin{bmatrix}
		1   & 0 & 0 \\
		u_2 & 1 & 0 \\
		u_3 & 0 & 1
	\end{bmatrix}
\]
\begin{theorem}
	L'inversa di una matrice elementare di Gauss è equivalente alla stessa matrice dove si cambia segno agli
	elementi del vettore $u$.
	\[ E^{-1} = I - u \cdot e_j^T \]
	\begin{proof}
		Per dimostrarlo è sufficiente verificare la seguente uguaglianza
		\[ (I + u \cdot e_j^T) \cdot (I - u \cdot e_j^T) = I \]
		Sviluppando il prodotto otteniamo
		\[
			(I + u \cdot e_j^T) \cdot (I - u \cdot e_j^T) =
			I + u \cdot e_j^T - u \cdot e_j^T - u \cdot e_j^T \cdot u \cdot e_j^T
		\]
		Dato che il prodotto riga per colonna tra matrici è associativo è possibile svolgere per primo il prodotto
		riga per colonna
		\[ e_j^T \cdot u = 0 \]
		A questo punto l'ultimo termine dell'equazione si annulla e otteniamo l'identità.
	\end{proof}
\end{theorem}

Le matrici elementari di Gauss godono di una proprietà importante per riuscire a calcolare la fattorizzazione
$LU$ di una matrice $A$. In particolare questa proprietà è necessaria per riuscire a mettere degli zeri dove
necessario in modo da ottenere una matrice triangolare.

\begin{theorem}\label{th: eliminazione_gauss}
	Sia $x \in \R^n$ con $x_k \neq 0$ allora esiste una matrice elementare di Gauss $E$ tale che
	\[
		E \cdot x = \begin{bmatrix}
			x_1 \\ \vdots \\ x_k \\ 0 \\ \vdots \\ 0
		\end{bmatrix}
	\]
	\begin{proof}
		Prendiamo
		\[ E = I + u e_k^T \]
		con $u_1 = \cdots = u_k = 0$. Calcoliamo quindi $E \cdot x$
		\[
			(I + u e_k^T) x = x + u x_k =
			\begin{bmatrix}
				x_1 \\ \vdots \\ x_k \\ x_{k+1} \\ \vdots \\ x_n
			\end{bmatrix} + x_k
			\begin{bmatrix}
				0 \\ \vdots \\ 0 \\ u_{k+1} \\ \vdots \\ u_n
			\end{bmatrix} =
			\begin{bmatrix}
				x_1 \\ \vdots \\ x_k \\ x_{k+1} + x_k u_{k+1} \\ \vdots \\ x_n + x_k u_n
			\end{bmatrix}
		\]
		Dato che vogliamo le componenti dalla $(k+1)$-esima alla $n$-esima uguali a 0 basta imporre
		\begin{align*}
			u_{k+1} = & -\frac{x_{k+1}}{x_k} \\
			\vdots    &                      \\
			u_n =     & -\frac{x_n}{x_k}
		\end{align*}
		Le quali sono quantità ben definite dato che $x_k \neq 0$.
	\end{proof}
\end{theorem}

\subsection{Algoritmo per l'eliminazione gaussiana}
Passiamo ora al calcolo della fattorizzazione $LU$ tramite matrici elementari di Gauss. Supponiamo di avere una
generica matrice $A$ così definita
\[
	A = A_0 = \begin{bmatrix}
		a_{1,1}^{(0)} & \dots  & a_{1,n}^{(0)} \\
		\vdots        & \ddots & \vdots        \\
		a_{n,1}^{(0)} & \dots  & a_{n,n}^{(0)}
	\end{bmatrix}
\]
Per questo metodo supponiamo $a_{1,1}^{(0)} \neq 0$. Allora possiamo determinare $E_1$ tale che
\[
	E_1 \cdot \begin{bmatrix}
		a_{1,1}^{(0)} \\ \vdots \\ a_{n,1}^{(0)}
	\end{bmatrix} =
	\begin{bmatrix}
		a_{1,1}^{(0)} \\ 0 \\ \vdots \\ 0
	\end{bmatrix}
\]
Ricaviamo quindi che
\[ E_1 = I + u e_1^T \]
con
\[
	u = \begin{bmatrix}
		0 \\ -\frac{a_{2,1}^{(0)}}{a_{1,1}^{(0)}} \\ \vdots \\ -\frac{a_{n,1}^{(0)}}{a_{1,1}^{(0)}}
	\end{bmatrix}
\]
A questo punto, possiamo generare la matrice $A_1$ come segue
\[ A_1 = E_1 \cdot A_0 = E_1 \cdot A \]
La matrice $A_1$ ha la seguente forma
\[
	A_1 = \begin{bmatrix}
		a_{1,1}^{(1)} & \dots         & \dots  & a_{1,n}^{(1)} \\
		0             & a_{2,2}^{(1)} & \dots  & a_{2,n}^{(1)} \\
		\vdots        & \vdots        & \ddots & \vdots        \\
		0             & a_{n,2}^{(1)} & \dots  & a_{n,n}^{(1)}
	\end{bmatrix}
\]
dove
\[ a_{1,i}^{(1)} = a_{1,i}^{(0)} \]
e con
\[ a_{i,j}^{(1)} = a_{i,j}^{(0)} + \left( -\frac{a_{i,1}^{(0)}}{a_{1,1}^{(0)}} \right) \cdot a_{1,j}^{(0)} \]
Questo per ogni elemento $a_{i,j}^{(1)}$ con $2 \leq i,j \leq n$.

Gli elementi sulla prima colonna di $E_1$ eccetto l'1 sulla diagonale sono detti \textbf{moltiplicatori} di
Gauss. Per riuscire a calcolarli dobbiamo fare $n-1$ divisioni. Una volta ottenuti dobbiamo fare un prodotto
e un'addizione per $(n-1)^2$ elementi per aggiornare la matrice. Ne deduciamo che il passaggio da $A_0$ ad $A_1$
è costato $(n-1)^2$ operazioni moltiplicative.

Ora passiamo alla seconda colonna e supponiamo $a_{2,2}^{(1)} \neq 0$ allora possiamo determinare $E_2$ tale che
\[
	E_2 \cdot \begin{bmatrix}
		a_{1,2}^{(1)} \\ \vdots \\ a_{n,2}^{(1)}
	\end{bmatrix} =
	\begin{bmatrix}
		a_{1,2}^{(1)} \\ a_{2,2}^{(1)} \\ 0 \\ \vdots \\ 0
	\end{bmatrix}
\]
Per il teorema \ref{th: eliminazione_gauss} abbiamo che
\[ E_2 = I + u e_2^T \]
dove
\[
	u = \begin{bmatrix}
		0 \\ 0 \\ -\frac{a_{3,2}^{(1)}}{a_{2,2}^{(1)}} \\ \vdots \\ -\frac{a_{n,2}^{(1)}}{a_{2,2}^{(1)}}
	\end{bmatrix}
\]
Una volta ottenuto $u$ possiamo calcolare $A_2$ in questo modo
\[ A_2 = E_2 \cdot A_1 = E_2 \cdot E_1 \cdot A_0 \]
ottenendo una matrice con la seguente forma
\[
	A_2 = \begin{bmatrix}
		a_{1,1}^{(2)} & \dots         & \dots         & \dots & a_{1,n}^{(2)} \\
		0             & a_{2,2}^{(2)} & \dots         & \dots & a_{2,n}^{(2)} \\
		\vdots        & 0             & a_{3,3}^{(2)} & \dots & a_{3,n}^{(2)} \\
		\vdots        & \vdots        & \vdots        &       & \vdots        \\
		0             & 0             & a_{n,3}^{(2)} & \dots & a_{n,n}^{(2)}
	\end{bmatrix}
\]
A questo punto se $a_{3,3}^{(2)} \neq 0$ possiamo andare avanti fino a $a_{n,n}^{(n-1)}$ definendo così tutte
le matrici elementari di Gauss $E_i$ con $1 \leq i \leq n-1$ e ottenendo la matrice triangolare superiore $U$
in questo modo
\[ E_{n-1} \dots E_1 A_0 = U \]
Non ci rimane che calcolare $L$. Per farlo scriviamo
\[ A = A_0 = E_1^{-1} \dots E_{n-1}^{-1} U \]
da cui possiamo dedurre che
\[ L = E_1^{-1} \dots E_{n-1}^{-1} \]
Valutiamo ora il costo computazionale per il calcolo di una fattorizzazione $LU$ utilizzando il metodo appena
descritto. In generale, il passaggio da $A_i$ ad $A_j$ ha costo $O((n-j)^2)$ operazioni. Il costo complessivo
della computazione è quindi $O(n^3)$.

\begin{example}
	Calcoliamo la fattorizzazione $LU$ della matrice $A$ definita come segue
	\[ A = \begin{bmatrix}
			1 & 1 & 1 \\
			1 & 2 & 2 \\
			1 & 2 & 3
		\end{bmatrix}
	\]
	Prima di tutto vogliamo calcolarci $E_1$ tale che
	\[ E_1 \begin{bmatrix}
			1 \\ 1 \\ 1
		\end{bmatrix} = \begin{bmatrix}
			1 \\ 0 \\ 0
		\end{bmatrix}
	\]
	Per il teorema \ref{th: eliminazione_gauss} possiamo prendere $E_1$ tale che
	\[ E_1 = I + u e_1^T \]
	con
	\[ u = \begin{bmatrix} 0 \\ -1 \\ -1 \end{bmatrix} \]
	Ricaviamo quindi
	\[
		E_1 = \begin{bmatrix}
			1  & 0 & 0 \\
			-1 & 1 & 0 \\
			-1 & 0 & 1
		\end{bmatrix}
	\]
	Possiamo ora calcolare
	\[
		A_1 = E_1 A = \begin{bmatrix}
			1 & 1 & 1 \\
			0 & 1 & 1 \\
			0 & 1 & 2
		\end{bmatrix}
	\]
	Dato che $a_{2,2}^{(1)} \neq 0$ allora procediamo con il calcolo di $E_2$ tale che
	\[
		E_2 = I + u e_2^T = \begin{bmatrix}
			1 & 0  & 0 \\
			0 & 1  & 0 \\
			0 & -1 & 1 \\
		\end{bmatrix}
	\]
	A questo punto possiamo calcolare $A_2$
	\[
		A_2 = E_2 A_1 = \begin{bmatrix}
			1 & 1 & 1 \\
			0 & 1 & 1 \\
			0 & 0 & 1
		\end{bmatrix} = U
	\]
	Non ci rimane che trovare $L$ in questo modo
	\[
		E_1^{-1} E_2^{-1} = \begin{bmatrix}
			1 & 0 & 0 \\
			1 & 1 & 0 \\
			1 & 0 & 1
		\end{bmatrix} \begin{bmatrix}
			1 & 0 & 0 \\
			0 & 1 & 0 \\
			0 & 1 & 1
		\end{bmatrix} = \begin{bmatrix}
			1 & 0 & 0 \\
			1 & 1 & 0 \\
			1 & 1 & 1
		\end{bmatrix} = L
	\]
	Facciamo il prodotto $L \cdot U$ per verificare che la fattorizzazione sia corretta.
\end{example}

\subsection{Eliminazione gaussiana con pivoting parziale}
Supponiamo di avere ora una matrice $A$ definta come segue
\[
	A = \begin{bmatrix}
		1 & 1 & 1 \\
		1 & 1 & 2 \\
		1 & 2 & 3
	\end{bmatrix}
\]
In questo caso se provassimo ad applicare l'algoritmo appena visto otteniamo dopo il primo passaggio
\[
	A_1 = \begin{bmatrix}
		1 & 1 & 1 \\
		0 & 0 & 1 \\
		0 & 1 & 2
	\end{bmatrix}
\]
La quale non ci permette di compiere un ulteriore passo. Ma come possiamo notare, scambiando le ultime due righe
si giunge ad una forma triangolare. Questa \textbf{non} è una fattorizzazione $LU$ di $A$, vale infatti
\[ P_2 E_1 A = U \]
dove $P_2$ è detta \textbf{matrice di permutazione}, la quale scambia le ultime due righe ed è definita come
segue
\[
	P_2 = \begin{bmatrix}
		1 & 0 & 0 \\
		0 & 0 & 1 \\
		0 & 1 & 0
	\end{bmatrix}
\]
A questo punto per ricaviamo $A$ in questo modo
\[ A = E_1^{-1} P_2^{-1} U \]
L'inversa di una matrice di permutazione coincide con la trasposta.
\[ P^{-1} = P^T \]

Consideriamo ora la generica matrice $A$ e supponiamo che sia invertibile. Se la matrice è invertibile allora
sulla prima colonna c'è almeno un elemento non nullo.
\[
	A = \begin{bmatrix}
		a_{1,1}^{(0)} & \dots  & a_{1,n}^{(0)} \\
		\vdots        & \ddots & \vdots        \\
		a_{n,1}^{(0)} & \dots  & a_{n,n}^{(0)}
	\end{bmatrix}
\]
Se ci troviamo nell'ipotesi che $A$ sia invertibile l'algoritmo, all'$i$-esimo passo procede in questo modo:
\begin{enumerate}
	\item Per motivi di stabilità andiamo a scegliere l'elemento di modulo massimo della $i$-esima colonna.
	      Tale elemento è detto \textbf{pivot} e si trova sulla riga $k$ con $k$ tale che
	      \[
		      | a_{k,i}^{(i-1)} | = \norm{
			      \begin{bmatrix}
				      a_{1,i}^{(i-1)} \\ \vdots \\ a_{n,i}^{(i-1)}
			      \end{bmatrix}}_\infty
	      \]
	      L'elemento di modulo massimo garantisce che gli elementi di $L$ saranno più piccoli possibile. Per
	      trovarlo si ha un costo di $O(n)$ operazioni (scansione del vettore).
	\item Scambiamo l'$i$-esima riga con la $k$-esima determinando una matrice $P_i$ che effettua tale scambio
	      \[ P_i \cdot A_{i-1} \]
	\item Procediamo con l'eliminazione di gauss determinando $E_i$ e ricavando la matrice $A_i$ in questo modo
	      \[ A_i = E_i \cdot P_i \cdot A_{i-1} \]
\end{enumerate}
Una volta eseguiti $n-1$ passi dell'algoritmo otteniamo $U$ calcolando
\[ U = E_{n-1} P_{n-1} \dots E_1 P_1 A_0 \]
che possiamo riscrivere come
\[ A = E_{n-1} P_{n-1} \dots E_1 P_1 U = L U \]
Dove $U$ è una triangolare superiore e $L$ non è triangolare inferiore per effetto del prodotto con le matrici
di permutazione.

Questa fattorizzazione è calcolabile per una qualsiasi matrice invertibile e ha ottime proprietà di stabilità
numerica.

Dato che i pivot sono scelti prendendo l'elemento di modulo massimo della colonna che abbiamo come
riferimento, questa tecnica è detta \textbf{eliminazione gaussiana con pivoting parziale} che è anche l'algoritmo
di riferimento che viene usato in MatLab tramite l'operatore \verb|\| quando vogliamo risolvere un sistema lineare
$Ax = b$ e calcoliamo il vettore soluzione tramite
\begin{center}
	\verb|x = A \ b|
\end{center}
Se invece andassimo a scegliere l'elemento di modulo massimo su tutta la matrice dovremmo permutare sia righe
che colonne, si parlerebbe di \textbf{eliminazione gaussiana con pivoting totale}, la quale ha stabilità ancora
migliore ma intacca la complessità, in quanto la ricerca dell'elemento di modulo massimo dell'intera matrice ha
costo $O(n^2)$.

Per il calcolo del determinante teniamo sempre a mente che, per il teorema di Binet, vale
\[ \det(A) = \det(L) \cdot \det(U) \]
Per calcolare $\det(L)$ teniamo a mente che il determinante di una matrice elementare di Gauss è 1, mentre il
determinante di una matrice di permutazione può essere $\pm 1$. Dobbiamo quindi calcolare il prodotto tra tutti
i $\pm 1$ e moltiplicarlo per $\det(U)$. Dato che $U$ è una matrice triangolare il determinante equivale al
prodotto di tutti gli elementi sulla diagonale.

Per la risoluzione del sistema lineare $Ax = b$ non è necessario calcolare fattorizzazione di $A$. Possiamo
applicare le matrici di eliminazione e permutazione ad $A$ e $b$ passo dopo passo, ottenendo dei sistemi
equivalenti. Questo ci fa risparmiare memoria evitando di mantenere tutte le matrici di eliminazione e
permutazione.
