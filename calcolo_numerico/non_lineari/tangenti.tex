\section{Metodo delle tangenti}
Come avevamo anticipato in precedenza, risolvere il problema di trovare gli \emph{zeri} di $f$ può essere
riformulato andando a cercare i \emph{punti fissi} di $g$. Vale quindi la relazione
\[ f(x) = 0 \quad \Leftrightarrow \quad x = g(x) \]
per un'opportuna $g$. Come abbiamo visto però non c'era nessuna proprietà della funzione $g$ che ci garantisse
la convergenza del metodo.

\begin{theorem}[Punto fisso]\label{th: punto_fisso}
	Sia $g : [a, b] \to \R$ di classe $C^1 ([a, b])$ e sia $g(\alpha) = \alpha$ con $\alpha \in [a,b]$. Se esiste
	$\rho > 0$ tale che $|g'(x)| < 1$ per ogni $x \in [\alpha - \rho, \alpha + \rho] = I_\alpha$ allora la
	successione generata dal metodo
	\[ x_{k+1} = g(x_k) \]
	a partire da $x_0 \in I_\alpha$ soddisfa queste due proprietà
	\begin{itemize}
		\item $x_k \in I_\alpha$ per ogni $k$.
		\item $x_k \to \alpha$.
	\end{itemize}
	\begin{proof}
		La dimostrazione procede per induzione. Iniziamo col dire che noi sappiamo per ipotesi che esiste un
		intorno circolare dove $|g'(x)| < 1$. Dato che la derivata prima è una funzione continua, il modulo
		della derivata prima è ancora continua.

		Per il teorema di Weierstrass sappiamo che una funzione continua in un intervallo chiuso e limitato
		ammette massimo e minimo. Esiste quindi il massimo $\lambda$ di $|g'(x)|$ per $x \in I_\alpha$.
		Dato che $|g'(x)| < 1$ anche $\lambda < 1$.

		A questo punto noi vogliamo dimostrare per induzione che gli elementi della successione soddisfano
		questa proprietà:
		\[ |x_k - \alpha| \leq \lambda^k \rho \quad \forall k \geq 0 \]
		Dato che $\lambda < 1$ vale che
		\[ |x_k - \alpha| \leq \lambda^k \rho \leq \rho \quad \Rightarrow \quad x_k \in I_\alpha \]
		Consideriamo anche
		\[ 0 \leq |x_k - \alpha| \leq \lambda^k \rho \]
		Dato che
		\[ \lim_{k \to +\infty} \lambda^k \rho = 0\]
		allora, per il teorema dei carabinieri, anche
		\[ \lim_{k \to +\infty} |x_k - \alpha| = 0 \]
		e quindi $x_k \to \alpha$. Per riuscire a dimostrare la proprietà per induzione procediamo in questo modo:
		\begin{enumerate}
			\item Il passo base consiste nel dimostrare che
			      \[ |x_0 - \alpha| \leq \lambda^0 \rho = \rho \]
			      ma questo è vero per ipotesi poiché $x_0 \in I_\alpha$.
			\item Supponiamo ora di aver dimostrato fino all'ordine $k$ e cerchiamo di dimostrare che
			      \[ |x_{k+1} - \alpha| \leq \lambda^{k+1} \leq \rho \]
			      Dato che $x_{k+1} = g(x_k)$ e $\alpha$ è un punto fisso di $g$ possiamo scrivere
			      \[ |x_{k+1} - \alpha| = |g(x_k) - g(\alpha)| \]
			      A questo punto possiamo applicare il teorema di Lagrange, il quale ci dice che esiste un punto
			      $\xi_k$ tale che
			      \[ |g(x_k) - g(\alpha)| = |g'(\xi_k) (x_k - \alpha)| = |g'(\xi_k)| \cdot |x_k - \alpha| \]
			      dato che $|g'(\xi_k)| \leq \lambda$ e per ipotesi vale $|x_k - \alpha| \leq \lambda^k \rho$
			      allora vale
			      \[ |g'(\xi_k)| \cdot |x_k - \alpha| \leq \lambda \cdot \lambda^k \rho = \lambda^{k+1} \rho \]
		\end{enumerate}
	\end{proof}
\end{theorem}

Dato che la condizione $|g'(x)| < 1$ è \emph{difficile} da studiare su un intervallo, si da anche una versione
più debole del teorema il quale si occupa di verificare la proprietà nel punto $\alpha$.

\begin{corollary}
	Sia $g : [a, b] \to \R$ di classe $C^1 ([a, b])$ e sia $g(\alpha) = \alpha$ con $\alpha \in [a,b]$. Se
	$|g'(\alpha)| < 1$ allora esiste un intervallo $I_\alpha = [\alpha - \rho, \alpha + \rho]$ tale che per
	ogni $x_0 \in I_\alpha$ valgono le seguenti proprietà:
	\begin{itemize}
		\item $x_k \in I_\alpha$ per ogni $k$.
		\item $x_k \to \alpha$.
	\end{itemize}
	\begin{proof}
		Prendiamo $h(x) = |g'(x)| - 1$ che è una funzione continua. Dato che $h(\alpha) = |g'(\alpha)| - 1$ e
		per le ipotesi del teorema vale che
		\[ |g'(\alpha)| - 1 < 0 \]
		Per il teorema di permanenza del segno esiste un intervallo $I_\alpha$ tale che
		\[ h(x) < 0 \quad \forall x \in I_\alpha \]
		il quale ci dice che se una funzione continua è negativa in un punto possiamo trovare circolare del punto
		in cui la funzione è negativa. Nel nostro caso dire che la funzione è negativa equivale a dire
		\[ h(x) < 0 \quad \Leftrightarrow \quad |g'(x)| < 1 \]
		Ecco le ipotesi del teorema \ref{th: punto_fisso} sono soddisfatte.
	\end{proof}
\end{corollary}

Questo corollario ci dice che se $|g'(\alpha)| < 1$ allora il metodo è \textbf{localmente convergente} in $\alpha$.
Possiamo quindi trovare un intorno sufficientemente piccolo che permetta la convergenza.
