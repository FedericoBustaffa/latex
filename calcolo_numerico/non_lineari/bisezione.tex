\chapter{Equazioni non lineari}
In questo capitolo andremo a trattare la risoluzione di \textbf{equazioni non lineari}, tipicamente del tipo
\[ f : \R \to \R \quad \vee \quad f : [a, b] \to \R \]
Risolvere un'equazione non lineare significa trovare tutti i valori della variabile $x$ tali che
\[ f(x) = 0 \]
Chiameremo questi valori \textbf{radici} dell'equazione o \textbf{zeri} della funzione. In generale, possiamo
dire che una funzione non lineare è una qualsiasi funzione differente da
\[ f(x) = a x + b \]
ossia, dalla funzione che definisce una retta nel piano. Polinomi di grado superiore al primo nella variabile
$x$, logaritmi, esponenziali, radici, seno e coseno sono tutte funzioni non lineari.

L'obbettivo di questa parte del corso è definire dei metodi che siano in grado di risolvere equazioni non
lineari. Mentre per la risoluzione di sistemi lineari avevamo informazioni sulla risolvibilità di del sistema
come ad esempio l'\emph{invertibilità} della matrice stavolta non abbiamo garanzia che l'equazione abbia
delle soluzioni né quante siano queste soluzioni.

L'unico approccio che abbiamo per avere informazioni sulla funzione è di tipo grafico. Cerchiamo quindi di
costruire il grafico della funzione per definire meglio lo locazione delle soluzioni, di modo da poter procedere
in modo più mirato all'approssimazione di tali soluzioni.



\begin{example}
	Consideriamo la seguente funzione $f$
	\[ f = x^3 - 2x + 3 = 0 \]
	Tale equazione ha 3 soluzioni su $\C$ ma noi siamo interessati ad un'analisi su $\R$ e dunque possiamo dire
	se c'è una soluzione complessa allora anche la sua coniugata è soluzione dell'equazione e in quel caso
	avremmo una soluzione reale. Nel caso non ci siano soluzioni complesse allora ce ne sono 3 reali.

	Per capire se l'equazione ha 1 o 3 soluzioni reali proviamo a tracciare un grafico approssimativo della
	funzione. Per farlo possiamo prima di tutto fare i limiti agli estremi del campo di esistenza, che, in
	questo caso è tutto $\R$.
	\begin{align*}
		\lim_{x \to +\infty} x^3 - 2x + 3 = & +\infty \\
		\lim_{x \to -\infty} x^3 - 2x + 3 = & -\infty
	\end{align*}
	Altra cosa che possiamo fare è calcolare la derivata della funzione
	\[ f'(x) = 3x^2 - 2 \]
	Stavolta possiamo studiare facilmente il segno dato che ci troviamo davanti un'equazione di secondo grado
	\[ 3x^2 - 2 = 0 \quad \Leftrightarrow \quad x = \pm \sqrt{\frac{2}{3}} \]
	Ricaviamo quindi che la derivata è positiva per
	\[ x < -\sqrt{\frac{2}{3}} \quad \vee \quad x > \sqrt{\frac{2}{3}} \]
	mentre è negativa per
	\[ -\sqrt{\frac{2}{3}} < x < \sqrt{\frac{2}{3}} \]
	Ne segue che nell'intervallo di positività della derivata la funzione è crescente mentre in quello di
	negatività la funzione è decrescente. Abbiamo inoltre un massimo locale in $-\sqrt{\frac{2}{3}}$ e un
	minimo locale in $\sqrt{\frac{2}{3}}$.

	Possiamo ora provare a tracciare un grafico approssimativo. Per prima cosa possiamo valutare la funzione in
	$x = 0$ e vediamo che $f(0) = 3$. Dato che $-\sqrt{\frac{2}{3}} < 0 < \sqrt{\frac{2}{3}}$ possiamo dedurre
	la funzione valutata in $-\sqrt{\frac{2}{3}}$ avrà valore positivo. Valutiamo invece la funzione in
	$\sqrt{\frac{2}{3}}$ e otteniamo che
	\[ f \left( \sqrt{\frac{2}{3}} \right) = \frac{2}{3} \sqrt{\frac{2}{3}} - 2 \sqrt{\frac{2}{3}} + 3 > 0 \]
	Nel punto di minimo locale la funzione è positiva e quindi possiamo concludere che la funzione interseca
	l'asse delle ascisse solo in un punto minore di $-\sqrt{\frac{2}{3}}$. Concludiamo che l'equazione ha una
	sola soluzione $\alpha$ reale.

	A questo dobbiamo cercare di definire un intervallo nel quale sia compreso $\alpha$. Per farlo possiamo
	valutare la funzione in diversi punti finché non troviamo un $x$ tale che la funzione è negativa e uno
	tale che la funzione è positiva. La nostra soluzione $\alpha$ si troverà da qualche parte in mezzo a questo
	intervallo. Possiamo calcolare ad esempio
	\[ f(-1) = 4 \quad \quad f(-2) = -1 \]
	Possiamo quindi concludere che $\alpha \in [-2, -1]$. L'intervallo $[-2, -1]$ è detto
	\textbf{intervallo di separazione}.
\end{example}

Tutti i metodi che vedremo non saranno di tipo diretto ma \emph{iterativi}, andremo quindi a generare una
successione che, sotto le opportune ipotesi, andrà a convergere nella soluzione. Definiremo anche un criterio
di arresto per terminare l'algoritmo una volta raggiunto un certo livello di accuratezza dell'approssimazione.

\section{Metodo di bisezione}
Vediamo ora il metodo più semplice possibile per la risoluzione di equazioni non lineari, ossia il metodo di
\textbf{bisezione}. L'idea che sta alla base del metodo è la seguente
\begin{enumerate}
	\item Definire un intervallo di separazione che contiene una radice dell'equazione, quindi avente estremi
	      tali che la funzione calcolata sui due estremi abbia segno discorde. Definiamo quindi un intervallo
	      $[\alpha, \beta]$ tale che
	      \[ \sign(f(\alpha)) = -\sign(f(\beta)) \]
	\item Calcolare la funzione nel punto di mezzo $\gamma = \frac{\alpha + \beta}{2}$, ossia
	      \[ y = f \left( \frac{\alpha + \beta}{2} \right) = f(\gamma) \]
	\item Se $y$ è positivo allora $\gamma$ diventa il nuovo estremo tale che $f(\gamma) > 0$, viceversa se è
	      negativo diventa $\gamma$ diventa l'estremo tale che $f(\gamma) < 0$.
	\item Si va avanti finché l'intervallo non è talmente piccolo che uno dei due estremi non approssima alla
	      soluzione.
\end{enumerate}
