\documentclass[11pt, a4paper]{report}

\pdfpagewidth\paperwidth
\pdfpageheight\paperheight

\usepackage[utf8]{inputenc}
\usepackage[T1]{fontenc}
\usepackage[italian]{babel}
\usepackage[hidelinks]{hyperref}
\usepackage{amsmath, amssymb, amsthm, amsfonts, mathtools}
\usepackage{tikz}

\theoremstyle{definition}
\newtheorem{theorem}{Teorema}[chapter]
\newtheorem{definition}{Definizione}[chapter]
\newtheorem{proposition}{Proposizione}[chapter]
\newtheorem{corollary}{Corollario}[chapter]
\newtheorem{lemma}{Lemma}[chapter]
\newtheorem{observation}{Osservazione}[chapter]
\newtheorem{example}{Esempio}[chapter]

\newcommand{\N}{\mathbb{N}}
\newcommand{\Z}{\mathbb{Z}}
\newcommand{\R}{\mathbb{R}}
\newcommand{\F}{\mathbb{F}}

\newcommand{\ein}{\epsilon_\text{in}}
\newcommand{\ealg}{\epsilon_\text{alg}}
\newcommand{\etot}{\epsilon_\text{tot}}
\newcommand{\tx}{\tilde{x}}

\DeclareMathOperator{\sign}{sign}
\DeclareMathOperator{\trn}{trn}
\DeclareMathOperator{\arr}{arr}
\DeclareMathOperator{\fl}{fl}

\title{Calcolo numerico}
\author{Federico Bustaffa}
\date{13/02/2023}

\begin{document}

\maketitle
\tableofcontents

\chapter{Introduzione}

Il corso tratterà le modalità di reperimento e analisi di dati da blockchain tramite Python. Si compirà un'analisi
statistica sui dati raccolti tramite \emph{scraping} o API di vario genere per riuscire ad estrapolare informazioni
sulle blockchain prese in esame.

Saranno necessarie nozioni di statistica descrittiva ed inferenziale e si farà uso di strumenti per l'analisi di
grafi.

Le librerie Python necessarie per lavorare al meglio in questo ambito sono:
\begin{itemize}
	\item \verb|BeautifulSoup|, \verb|Scrapy|: web scraping
	\item \verb|Pandas|: gestione di tabelle
	\item \verb|Numpy|: algebra vettoriale e matriciale
	\item \verb|SciPy|: statistica
	\item \verb|Matplotlib|: visualizzazione di grafici
	\item \verb|NetworkX|: analisi di grafi
\end{itemize}
Per installare le librerie scaricare il package manager \verb|pip| di Python:
\begin{verbatim}
	> sudo apt install python3-pip
\end{verbatim}
Installazione delle librerie
\begin{verbatim}
	> pip install beautifulsoup4
	> pip install scrapy
	> pip install pandas
	> pip install numpy
	> pip install scipy
	> pip install matplotlib
	> pip install networkx
\end{verbatim}

\section{Blockchain}
Una \textbf{blockchain} utilizza un modello computazionale differente dal classico modello \textbf{client-server},
si basa infatti sul sistema \textbf{peer-to-peer}.

In questo modello ogni nodo possiede una frazione del potere necessario a compiere determinate azioni all'interno
della blockchain. Il sistema nasce infatti per eliminare l'entità centrale che controlla e gestisce l'intero servizio.

Possiamo vedere una blockchain come un \textbf{registro} \emph{distribuito} e \emph{replicato} tra i nodi di una
rete \emph{peer-to-peer}. In altre parole ogni nodo della rete possiede una copia della blockchain uguale a quella
di tutti gli altri nodi.

Una proprietà fondamentale al fine di mantenere la blockchain consistente e \textbf{immutabile} è la
\textbf{tamper freeness} garantita tramite meccanismi crittografici e algoritmi basati sul consenso.


\chapter{L'aritmetica del calcolatore}
I punti d'interesse del corso sono quindi
\begin{itemize}
	\item Capire come i numeri sono rappresentati in macchina.
	\item Capire come viene implementata l'aritmetica in macchina.
	\item Capire come il calcolatore approssima i calcoli eseguiti.
	\item Capire se tale approssimazione è accettabile per il problema che stiamo cercando di risolvere.
\end{itemize}
In questo capitolo andremo a trattare la \emph{rappresentazione} dei numeri in macchina, l'\emph{implementazione}
dell'aritmetica e quali \emph{errori} si possono generare.

\section{Rappresentazione dei numeri}
Cerchiamo come prima cosa di rappresentare i numeri in modo \textbf{univoco}, per farlo abbandoniamo la forma
frazionaria di rappresentazione. Come sappiamo, le quantità
\[ 0.1 \quad \frac{1}{10} \quad \frac{2}{20} \]
hanno tutte lo stesso valore ma, come possiamo facilmente intuire, la forma frazionaria fornisce infinite
rappresentazioni, tutte equivalenti per la stessa quantità $0.1$.

Considerando anche che la riduzione ai minimi termini di una frazione è un'operazione in generale costosa ma
necessaria al fine di avere espressioni più facili da calcolare, non ci possiamo permettere un tipo di
rappresentazione di questo tipo. Ecco perché rappresenteremo numeri reali come \emph{sequenza di cifre decimali}.

\begin{observation}
	Talvolta sarà necessario passare alla rappresentazione di numeri complessi ma, come sappiamo, possiamo vedere
	un numero complesso come una coppia di reali, dunque non ci saranno grosse differenze nella rappresentazione
	di questi ultimi.
\end{observation}

\begin{observation}
	La rappresentazione in cifre decimali potrebbe essere \textbf{infinita} (per esempio
	$\frac{1}{3} = 0.\overline{33}$), ma chiaramente non è possibile utilizzare una rappresentazione del genere
	su un calcolatore.
\end{observation}

Quest'ultima osservazione ci dice che un grado di approssimazione sarà necessario andando così a generare un
certo errore sulla rappresentazione.

Per una questione di \emph{minimizzazione} dello spazio utilizzato per la rappresentazione dei numeri in
macchina si farà ricorso ad esponenziali per rappresentare cifre non significative in forma \emph{compatta}.

\begin{example}
	Per esempio, per rappresentare 0.0000001, si usa la seguente notazione
	\[ 0.0000001 = 0.1 \cdot 10^{-6} \]
\end{example}

\subsection{Rappresentazione normalizzata in virgola mobile}
\begin{theorem}[Rappresentazione]\label{th: rappresentazione}
	Dato $x \in \R$ tale che $x \neq 0$, e sia $B \in \N$ tale che $B > 1$, la \textbf{base} di numerazione,
	esistono e sono univocamente determinati:
	\begin{itemize}
		\item Un intero $p \in \Z$, detto \textbf{esponente} della rappresentazione.
		\item Una serie di numeri naturali $\{ d_i \}_{i \geq 1}$ con
		      \begin{itemize}
			      \item $d_1 \neq 0$
			      \item $0 \leq d_i \leq B - 1$
			      \item Tutti i $d_i$ non definitivamente uguali a $B - 1$
		      \end{itemize}
		      dette \textbf{cifre} di rappresentazione.
	\end{itemize}
	tali che
	\[ x = \sign(x) \cdot B^p \cdot \sum_{i=1}^{+\infty} d_i B^{-i} \]
\end{theorem}

\begin{definition}
	La serie $d_1 B^{-1} + d_2 B^{-2} + \dots$ è detta \textbf{mantissa}.
\end{definition}

La rappresentazione \ref{th: rappresentazione} è detta \textbf{rappresentazione normalizzata in virgola mobile}
(\emph{floating point}) in quanto l'esponente $p$ è determinato in modo che il numero da rappresentare abbia
parte intera nulla e prima cifra dopo la virgola non nulla.

\subsubsection{Vincoli per la rappresentazione}
Andiamo ad analizzare meglio i vincoli imposti nel teorema \ref{th: rappresentazione} sulla rappresentazione.
\begin{itemize}
	\item Le condizioni $d_1 \neq 0$ (rappresentazione normalizzata) e $d_i$ non definitivamente uguale a
	      $B - 1$ ci garantiscono l'\emph{unicità} della rappresentazione
	\item Il numero $x = 0$ non ammette rappresentazione normalizzata e in macchina viene trattato e
	      memorizzato in modo speciale.
\end{itemize}

\begin{example}
	Rappresentiamo 0.1 tramite la rappresentazione \ref{th: rappresentazione}:
	\[ 0.1 = + 10^0 \cdot (1 \cdot 10^{-1} + 0 \cdot 10^{-2} + \dots) \]
	Se però volessimo rappresentare lo stesso numero in modo diverso, lo potremmo fare in questo modo
	\[ 0.1 = + 10^1 \cdot (0 \cdot 10^{-1} + 1 \cdot 10^{-2} + 0 \cdot 10^{-3} + \dots) \]
	Cambiando ogni volta l'esponente $p$ e adattando i $d_i$ è possibile trovare infinte rappresentazioni
	per 0.1. Dunque $d_1$ deve essere non nullo.
\end{example}

\begin{example}
	Come sappiamo $0.\overline{9} = 1$. Se volessimo rappresentare 1 otterremmo
	\[ 1 = + 10^1 \cdot (1 \cdot 10^{-1} + 0 \cdot 10^{-2} + \dots) \]
	mentre se rappresentassimo $0.\overline{9}$ otterremmo
	\[ 0.\overline{9} = + 10^0 \cdot (9 \cdot 10^{-1} + 9 \cdot 10^{-2} + \dots) \]
	Si ottiene così una rappresentazione normalizzata ($d_1 = 9$) e costituita da soli 9 dopo la virgola,
	ma dato che i due numeri sono equivalenti la rappresentazione utilizzata è la prima.
\end{example}

\subsection{Rappresentazione in macchina}
Per ora abbiamo dato una definizione ideale della rappresentazione in virgola mobile di un numero reale.
In macchina si ha però a disposizione un registro di lunghezza finita per memorizzarla. Tale registro è
suddiviso in tre parti:
\begin{itemize}
	\item Segno di $x$
	\item Esponente $p$
	\item Cifre di rappresentazione $\{ d_i \}$
\end{itemize}
Abbiamo quindi un numero finito di rappresentazioni possibili in macchina.

\begin{definition}\label{def: numeri_macchina}
	Si definisce \textbf{insieme dei numeri di macchina} in rappresentazione \emph{floating point} con $t$
	cifre, base $B$ e range $(-m, M)$, l'insieme dei numeri reali
	\[
		\F (B, t, m, M) = \{ 0 \} \cup
		\{ x \in \R : x = \sign(x) \cdot B^p \cdot \sum_{i=1}^t d_i B^{-i} \}
	\]
	con $0 \leq d_i \leq B-1$, $d_1 \neq 0$ e $-m \leq p \leq M$.
\end{definition}

\begin{observation}
	Guardando l'\emph{insieme dei numeri di macchina} possiamo osservare che
	\begin{itemize}
		\item Ha \textbf{cardinalità finita}
		      \[ N = 2 B^{t-1} (B - 1) (M + m + 1) + 1 \]
		      Otteniamo questo risultato dato che
		      \begin{itemize}
			      \item Il segno di $x$ può avere solo 2 valori possibili.
			      \item Avendo $t$ cifre disponibili per la rappresentazione abbiamo $B^t$ possibili
			            configurazioni. Ma dato che $d_1 \neq 0$ allora passiamo a $B^{t-1}$ configurazioni
			            effettive.
			      \item Ogni $d_i$ ha $B - 1$ possibili valori.
			      \item L'esponente è compreso tra $-m$ ed $M$ dunque abbiamo $m + M$ possibili valori ai quali
			            dobbiamo aggiungere lo zero.
			      \item Aggiungiamo lo zero.
		      \end{itemize}
		\item Se $x \in \F(B, t, m, M)$ e $x \neq 0$ allora $\omega = B^{-m - 1}$ è il più piccolo numero di
		      macchina positivo e $\Omega = B^M (1 - B^{-t})$ è il più grande numero di macchina positivo e
		      vale
		      \[ \omega \leq |x| \leq \Omega \]
		      Ne segue che non è possibile rappresentare esattamente numeri non nulli di modulo minore a
		      $\omega$. Ecco perché, per tali numeri, è prevista una rappresentazione \emph{denormalizzata}.

		      Quando $p = -m$ la condizione $d_1 \neq 0$ può essere abbandonata potendo quindi rappresentare
		      come numero più piccolo positivo $B^{-m-t}$ e dunque riuscendo a rappresentare numeri positivi
		      e negativi compresi in modulo tra
		      \[ [ B^{-m-t} \;;\; B^{-m-1} - B^{-m-t} ] \]

		      Analogamente se $p = M$ si introducono rappresentazioni speciali per i simboli $\pm \infty$
		      e NaN.
		\item L'insieme dei numeri di macchina $\F (B, t, m, M)$ è simmetrico rispetto all'origine.
		\item Posto $x = (-1)^s \cdot B^p \cdot \alpha$ appartente a $\F(B, t, m, M)$ allora il
		      successivo numero di macchina si ottiene sommando alle cifre di rappresentazione il più piccolo
		      numero di macchina rappresentabile in quell'ordine di grandezza, ossia
		      \[ B^p \cdot 1 \cdot B^{-t} = B^{p - t} \]
		      Otteniamo quindi che il numero di macchina successivo a $x$ è
		      \[ y = (-1)^s \cdot B^p \cdot (\alpha + B^{p-t}) \]
		      Questo ci dice che, in un sistema in virgola mobile, la distanza tra due numeri di macchina
		      \[ |y - x| = B^{p-t} \]
		      varia con $p$, ovvero con l'ordine di grandezza dei numeri considerati. In generale possiamo
		      immaginarci l'insieme dei numeri di macchina più denso vicino a $\omega$, al contrario andando
		      verso $\Omega$ la distanza tra i numeri di macchina sarà più grande.
	\end{itemize}
\end{observation}

\subsubsection{Standard IEEE per la doppia precisione}
Lo standard proposto dal IEEE per la rappresentazione in virgola mobile (base 2) prevede registri lunghi 64 bit.
In prima battuta si era pensato ad una suddivisione del registro di questo tipo:
\begin{itemize}
	\item 1 bit per il segno.
	\item 11 bit per l'esponente.
	\item 52 bit per le cifre di rappresentazione.
\end{itemize}
Il problema di questo modello di rappresentazione è la doppia rappresentazione dello 0 ($-0$ e $+0$) ed è stato
quindi scartato.

L'attuale standard rappresenta l'esponente in \textbf{traslazione}. Chiariamo meglio
il concetto procedendo con ordine:
\begin{enumerate}
	\item Dato che si hanno 11 bit per l'esponente si possono rappresentare numeri che vanno da 0 a 2047. Abbiamo
	      quindi 2048 configurazioni totali per l'esponente.
	\item Le due configurazioni di bit con tutti 0 e tutti 1 vengono tenute da parte e trattate in modo
	      \emph{speciale}. Ci rimangono dunque 2046 configurazioni.
	\item Lo standard definisce l'esponente nell'intervallo
	      \[ [ -1021, \; 1024 ] \]
	      Abbiamo dunque che $m = -1021$ e $M = 1024$.
\end{enumerate}
Notiamo inoltre che, in notazione normalizzata, $d_1$ deve essere $\neq 0$ ma dato che stiamo operando in
base 2 possiamo dedurre che $d_1$ deve essere sempre uguale a 1. Questo ci permette di non includere $d_1$
nelle cifre della rappresentazione dandoci la possibilità di usare, di fatto, 53 cifre di rappresentazione.

Come abbiamo detto poco fa, le due configurazioni dell'esponente con tutti 0 e tutti 1, sono configurazioni
\emph{speciali}. Nel caso in cui l'esponente sia rappresentato da tutti 1 distinguono due casi:
\begin{itemize}
	\item Nel caso in cui nella mantissa ci sia almeno una cifra diversa da 0 allora si rappresenta
	      $\pm \infty$
	\item Nel caso in cui tutte le cifre della mantissa siano uguali a 0 allora si rappresenta NaN.
\end{itemize}
Nel caso in cui l'esponente sia rappresentato da tutti 0 allora si usa la notazione \textbf{denormalizzata}
per riuscire a rappresentare anche un po' dei numeri che stanno tra 0 e $\omega$. Per riuscire a farlo
abbandoniamo il vincolo $d_1 \neq 0$. Nello specifico, si va rappresentare il numero in questo modo
\[ x = \sign(x) \cdot 2^{-1022} \cdot \sum_{i=1}^t d_i \cdot 2^{-i} \]
andando a rappresentare lo 0 nel caso in cui ci siano tutti 0 nella mantissa.

\section{Aritmetica di macchina}
Adesso rimane da capire come la macchina mappa i numeri reali in numeri di macchina. Il problema che ci
poniamo è il seguente: sia $x \in \R$, quale numero di macchina $\tilde{x} \in \F$ viene associato a $x$
per la rappresentazione?

Associare un numero di macchina $\tilde{x}$ a $x$ significa \textbf{approssimare} $x$ commettendo un
\textbf{errore relativo} di rappresentazione
\[ \epsilon_x = \frac{\tilde{x} - x}{x} = \frac{\eta_x}{x}, \quad x \neq 0 \]
quanto più piccolo possibile in valore assoluto. La quantità
\[ \eta_x = \tilde{x} - x \]
è detta \textbf{errore assoluto} di rappresentazione. Dato $x \in \R$ con $x \neq 0$, distinguiamo 2 casi:
\begin{itemize}
	\item $|x| < \omega$ (\textbf{underflow}) o $|x| > \Omega$ (\textbf{overflow})
	\item $\omega \leq |x| \leq \Omega$
\end{itemize}
Nel secondo caso le tecniche di approssimazione previste dallo standard IEEE 754 sono:
\begin{itemize}
	\item \textbf{Round to the nearest} (arrotondamento): il numero $x$ viene approssimato con il numero di
	      macchina $\tilde{x}$ più vicino.
	\item \textbf{Round toward zero} (troncamento): il numero $x$ viene approssimato con il più grande numero
	      di macchina $\tilde{x}$ il cui valore assoluto risulti minore o uguale al valore assoluto di $x$.
	\item \textbf{Round toward plus infinity}: il numero $x$ viene approssimato al più piccolo numero
	      rappresentabile maggiore del dato.
	\item \textbf{Round toward minus infinity}: il numero $x$ viene approssimato al più grande numero
	      rappresentabile minore del dato.
\end{itemize}
Assumiamo per semplicità di considerare una macchina che opera con troncamento sull'insieme $\F$. Per
convenzione indichiamo con $\trn(x) = \tilde{x}$ il risultato dell'approssimazione di $x$ con troncamento e
più generalmente $\fl(x)$ l'approssimazione in macchina del dato $x$ nel sistema \emph{floating point}
considerato. Il primo risultato fornisce una maggiorazione uniforme dell'errore di rappresentazione.

\begin{theorem}\label{th: prec}
	Sia $x \in \R$ con $\omega \leq |x| \leq \Omega$. Si ha
	\[ |\epsilon_x| = \left| \frac{\trn(x) - x}{x} \right| \leq u = B^{1-t} \]
	dove $u$ è detta \textbf{precisione di macchina}, la quale non dipende da $x$ e fornisce errori più grandi
	per numeri grandi ed errori piccoli per numeri piccoli.
	\begin{proof}
		Sia $x = (-1)^s \cdot B^p \alpha$. L'errore assoluto $|\trn(x) - x|$ risulta maggiorato dalla distanza
		tra i due numeri di macchina consecutivi e quindi
		\[ |\trn(x) - x| \leq B^{p-t} \]
		Inoltre $|x| \geq B^{p-1}$. Pertanto vale
		\[ |\epsilon_x| = \left| \frac{\trn(x) - x}{x} \right| \leq \frac{B^{p-t}}{B^{p-1}} = B^{1-t} = u \]
	\end{proof}
\end{theorem}

\begin{observation}
	Possiamo osservare che
	\begin{itemize}
		\item La \emph{precisione di macchina} è indipendente dalla grandezza del numero e caratteristica
		      dell'aritmetica \emph{floating point} implementata sulla macchina su cui stiamo operando. Se ad
		      esempio operiamo con arrotondamento, la distanza tra $x$ e la sua approssimazione di macchina, e
		      quindi la precisione di macchina, si dimezzano.
		\item Per valutare la precisione di macchina possiamo determinare il più piccolo numero di macchina
		      maggiore di 1. Sia infatti $x$ il numero che cerchiamo, abbiamo che $x - 1 = |x - 1| = B^{1-t}$,
		      essendo $1 = B^1 \cdot B^{-1}$ rappresentato con esponente $p = 1$.
		\item Dal teorema \ref{th: prec} si ricava che, dato $x \in \R$, in assenza di situazione di
		      \emph{underflow} o \emph{overflow}, per la sua rappresentazione in macchina vale che
		      \[ \fl(x) = x (1 + \epsilon_x), \quad |\epsilon_x| \leq u \]
		      Questa relazione esprime il modo in cui viene generalmente descritto il legame tra un numero reale
		      e la sua rappresentazione in macchina.
	\end{itemize}
\end{observation}

\subsection{Operazioni aritmetiche}

\chapter{Analisi degli errori}
In questo capitolo andiamo a trattare gli errori nel calcolo di funzioni razionali e non, andando a vedere
alcune tecniche per l'analisi tali errori.

\section{Errori nel calcolo di una funzione razionale}
Inizialmente andremo a considerare funzioni di questo tipo
\[ f : \R \to \R\]
anche se si tratta di una semplificazione in quanto, in casi più realistici si ha a che fare con funzioni del tipo
\[ f : \R^n \to \R \]
Come abbiamo già anticipato nel capitolo precedente, in generale non riusciamo a calcolare $f(x)$ ma, se abbiamo
fortuna, riusciamo a calcolare $f(\tx)$ dato che introduciamo un errore sul dato $x$ in ingresso.

\begin{definition}
	Definiamo \textbf{errore inerente} (o \textbf{inevitabile}) la quantità
	\[ \epsilon_{\text{in}} = \frac{f(\tx) - f(x)}{f(x)} \]
	La quale rappresenta l'errore che viene sempre commesso in quanto la macchina non riceve in ingresso
	il dato esatto ma quello approssimato.
\end{definition}

\begin{observation}
	Possiamo osservare che
	\begin{itemize}
		\item L'\emph{errore inerente} non dipende dall'algoritmo usato per la risoluzione del problema ma
		      misura la sensibilità della funzione $f$ rispetto alla perturbazione del dato iniziale.
		\item Se l'\emph{errore inerente} è qualitativamente elevato in valore assoluto diciamo che il problema
		      è \textbf{mal condizionato}. Al contrario, se è piccolo, diciamo che il problema è
		      \textbf{ben condizionato}.
	\end{itemize}
\end{observation}

\begin{definition}
	Definiamo \textbf{errore algoritmico} la quantità
	\[ \epsilon_{\text{alg}} = \frac{g(\tx) - f(\tx)}{f(\tx)} \]
	l'errore generato nel calcolo di $f(\tx) \neq 0$.
\end{definition}

\begin{observation}
	Possiamo osservare che
	\begin{itemize}
		\item La funzione $g$ dipende dall'algoritmo utilizzato per calcolare $f(x)$. In generale possiamo dire
		      che diversi algoritmi conducono a differenti errori algoritmici.
		\item Se l'\emph{errore algoritmico} è qualitativamente elevato in valore assoluto diciamo che l'algoritmo
		      è \textbf{numericamente instabile}. Al contrario, se è piccolo, diciamo che il problema è
		      \textbf{numericamente stabile}.
	\end{itemize}
\end{observation}

\begin{definition}
	Definiamo \textbf{errore totale} la quantità
	\[ \epsilon_{\text{tot}} = \frac{g(\tx) - f(x)}{f(x)} \]
	l'errore generato nel calcolo di $f(x) \neq 0$ mediante l'algoritmo specificato da $g$. L'\emph{errore totale}
	misura la differenza relativa tra l'output atteso e l'output effettivamente calcolato.
\end{definition}

\begin{theorem}
	In un'analisi al primo ordine vale
	\[ \etot = \ealg + \ein \]
	\begin{proof}
		Proviamo a dimostrarlo espandendo l'equazione per il calcolo dell'errore totale.
		\begin{align*}
			\epsilon_{\text{tot}} = & \frac{g(\tx) - f(x) + f(\tx) - f(\tx)}{f(x)}              \\
			=                       & \frac{g(\tx) - f(\tx)}{f(x)} + \frac{f(\tx) - f(x)}{f(x)} \\
			=                       & \frac{g(\tx) - f(\tx)}{f(x)} + \ein                       \\
			=                       & \frac{g(\tx) - f(\tx)}{f(\tx)} \frac{f(\tx)}{f(x)} + \ein
		\end{align*}
		Dato che
		\[ \ein = \frac{f(\tx) - f(x)}{f(x)} = \frac{f(\tx)}{f(x)} - 1 \]
		vale
		\[ \frac{f(\tx)}{f(x)} = \ein + 1 \]
		Possiamo quindi scrivere
		\begin{align*}
			\etot = & \ealg (\ein + 1) + \ein \\
			\doteq  & \ealg + \ein
		\end{align*}
	\end{proof}
\end{theorem}

Il teorema esprime il fatto che nel calcolo di una funzione razionale in un'analisi al primo ordine le due fonti
di errore individuate precedentemente forniscono contributi separati che possono essere analizzati
indipendentemente. L'obbiettivo dell'analisi numerica è individuare algoritmi numericamente \emph{stabili} per
problemi \emph{ben condizionati}.
\section{Analisi degli errori}
Parliamo ora delle tecniche per l'analisi dell'errore. La continuità della funzione implica che il problema sia
\emph{ben posto}. La relazione
\[
	\ein = \frac{f(\tx) - f(x)}{f(x)} =
	\frac{f(\tx) - f(x)}{\tx - x} \frac{x}{f(x)} \frac{\tx - x}{x}
\]
si ricava che la \emph{differenziabilità} di $f(x)$ è essenziale per il controllo dell'errore inerente. In
particolare se la funzione è \textbf{regolare}, ossia derivabile volte con derivate continue, allora vale
lo sviluppo di Taylor
\[ f(\tx) = f(x) + f'(x) (\tx - x) + \frac{f''(\xi)}{2} (\tx - x)^2 \]
con $|\xi - x| \leq |\tx - x|$, da cui si ottiene
\[
	\ein \frac{f(\tx) - f(x)}{f(x)} \doteq \frac{f'(x)}{f(x)} \cdot x \cdot \epsilon_x = c_x \cdot \epsilon_x
\]
\begin{definition}
	La quantità
	\[ c_x = c_x (f) = \frac{f'(x)}{f(x)} \cdot x \]
	è detta \textbf{coefficiente di amplificazione} e fornisce una misura del condizionamento del problema.
\end{definition}

Più in generale possiamo dire che se $f : \Omega \to \R$ è definita su un insieme aperto di $\R^n$, differenziabile
due volte su $\Omega$ ed il segmento di estremi $\tx$ e $x$ è contenuto in $\Omega$ allora vale
\[
	\ein = \frac{f(\tx) - f(x)}{f(x)} \doteq
	\frac{1}{f(x)} \sum_{i=1}^n \frac{\partial f}{\partial x_i} (x) x_i \epsilon_{x_i} =
	\sum_{i=1}^n c_{x_i} (f) \epsilon_{x_i}
\]
con
\[ c_{x_i} (f) = \frac{1}{f(x)} \frac{\partial f}{\partial x_i} (x) x_i \]
con $1 \leq i \leq n$, detti coefficienti di amplificazione della funzione $f$ rispetto alla variabile $x_i$.

\end{document}
