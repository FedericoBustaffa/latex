\chapter{Introduzione}\label{cap: introduction}

Il progetto di tesi verte sull'implementazione di una libreria di algoritmi
genetici in grado di sfruttare architetture multi-core per il calcolo parallelo.
La necessità di un algoritmo genetico con questa struttura nasce dal progetto
\textit{LORE}~\cite{guidotti2018LORE} di explainable AI, il quale utilizza
l'implementazione fornita dalla libreria di algoritmi genetici
\textit{DEAP}~\cite{fortin2012DEAP}.

\textit{LORE} si propone di generare spiegazioni a decisioni o predizioni fatte
da modelli di machine learning, spesso difficili da interpretare per via della
loro elevata complessità, come ad esempio \textit{deep neural networks} o
\textit{random forest}, che proprio per questo motivo prendono spesso il nome
di \textit{black-box}. In una prima fase, del metodo si sfrutta un algoritmo
genetico per la generazione di dati sintetici con determinate caratteristiche,
fondamentali in seguito per la produzione delle spiegazioni.

Tale approccio risulta però essere particolarmente dispendioso, soprattutto nei
casi in cui le predizioni da \quotes{spiegare} sono molte, la quantità di dati
sintetici che si desidera generare è molto grande oppure quando il modello è
particolarmente lento in fase di predizione (o una combinazione delle tre).

Da questa necessità nasce \textit{PPGA} (\textit{Parallel Processing for
	Genetic  Algorithms}), una libreria di algoritmi genetici in  grado di
produrre risultati qualitativamente simili a \textit{DEAP} ma più performante e
in grado di sfruttare al meglio architetture multi-core.

\textit{LORE} e \textit{DEAP}, così come la maggior parte delle librerie di
machine learning, sono implementati (o forniscono un'API) in Python, rendendo
difficoltosa l'implementazione di un modello di calcolo parallelo a causa del
\textit{GIL}, che limita il multithreading permettendo ad un solo thread alla
volta di essere eseguito. La prima fase del lavoro si è quindi focalizzata
sulla ricerca di possibili framework in grado di interfacciarsi e lavorare in
sinergia con Python, cercando allo stesso tempo di aggirare il problema che
introduce il \textit{GIL}.

Una volta implementata la libreria si è passati ad una fase di test per
valutarne la correttezza, impiegando l'algoritmo nella risoluzione di problemi
ben noti in letteratura, facilmente rappresentabili tramite grafici e che non
fossero computazionalmente troppo dispendiosi. Tra questi il problema dello
zaino, del commesso viaggiatore e un semplice caso di regressione lineare. Si è
infine testata la correttezza su una riproduzione semplificata del problema che
il metodo \textit{LORE} necessita di risolvere per generare spiegazioni.

Il lavoro si è concluso con due fasi di test, riguardanti rispettivamente la
qualità delle soluzioni prodotte e le prestazioni offerte dalla libreria.
Entrambe le tipologie di test sono state condotte sullo stesso problema di
generazione dati, implementando un algoritmo risolutivo sia con \textit{PPGA}
che con \textit{DEAP}, in modo da permettere un confronto tra le due librerie.

La tesi si sviluppa su quattro sezioni principali: la prima inquadra meglio il
contesto in cui si opera, trattando algoritmi genetici, calcolo parallelo ed
eXplainable AI. Di quest'ultimo verrà trattato più in dettaglio il metodo
\textit{LORE}, come sfrutta l'algoritmo genetico e quali sono le criticità che
\textit{PPGA} tenta di risolvere. Saranno inoltre trattati in profondità gli
algoritmi genetici, la loro struttura e come sia possibile ottimizzarli con il
calcolo parallelo. Si tratteranno infine le problematiche implementative che
comporta il \textit{GIL} e quali sono le sfide che pone nell'ambito degli
algoritmi genetici che operano in parallelo.

A seguire un resoconto di ciò che già esiste allo stato dell'arte, in particolar
modo verrà trattato il funzionamento della libreria \textit{DEAP}, dei costrutti
e dell'architettura parallela che propone, cercando quindi di evidenziarne
limiti e punti di forza.

Verrà trattata in seguito la metodologia che ha guidato lo sviluppo della
libreria, evidenziando quindi i requisiti che questa deve soddisfare in termini
di strutture dati, espressività e interazione con l'utente, sia nella versione
sequenziale che in quella parallela dell'algoritmo. Si sono messi inoltre in
evidenza vantaggi e problematiche dei vari framework esplorati per riuscire ad
implementare la versione parallela dell'algoritmo.

Infine un'analisi sulla sperimentazione effettuata al fine di giustificare
scelte implementative, valutare qualità e correttezza degli algoritmi e
confrontare le prestazioni di quanto implementato rispetto a \textit{DEAP}.
