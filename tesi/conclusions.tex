\chapter{Conclusioni}\label{cap: conclusions}

In conclusione, dai test effettuati, \textit{PPGA} ha mostrato un significativo
miglioramento nei tempi di esecuzione rispetto a \textit{DEAP}, in particolare
per problemi computazionalmente intensivi come il problema del commesso
viaggiatore e la generazione di dati sintetici per \textit{LORE}. L'approccio
parallelo adottato ha consentito di sfruttare in modo più efficiente le risorse
hardware disponibili, riducendo l'impatto del \textit{GIL} di Python grazie
all'uso del multiprocessing. Tuttavia, DEAP rimane una libreria più consolidata,
con una maggiore flessibilità nella definizione degli operatori genetici e un
supporto decisamente più completo.

Nonostante i risultati positivi, \textit{PPGA} presenta alcune limitazioni.
L'overhead introdotto dalla comunicazione tra processi può ridurre l'efficienza
in problemi di piccola scala, dove il costo di sincronizzazione tra processi
supera i benefici del parallelismo. Inoltre, la necessità di serializzare e
deserializzare gli oggetti per la comunicazione tra i processi può introdurre
un ulteriore collo di bottiglia. Per superare queste limitazioni, un futuro
sviluppo della libreria potrebbe includere:
\begin{itemize}
	\item L'implementazione di algoritmi e costrutti più complessi, in grado di
	      lavorare bene su un insieme più grande di problemi.
	\item L'uso di \verb|shared memory| per ridurre il costo della comunicazione
	      tra i worker, almeno per problemi che non richiedono strutture dati
	      particolarmente complesse.
\end{itemize}

In futuro, sarebbe interessante estendere il confronto con altri framework di
calcolo parallelo e valutare l'impatto di \textit{PPGA} in scenari reali.
La libreria offre comunque un punto di partenza solido per futuri sviluppi e
ottimizzazioni, contribuendo alla ricerca di metodi computazionali più veloci
ed efficienti nell'ambito degli algoritmi genetici e dell'Explainable AI.
