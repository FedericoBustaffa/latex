\section{Algoritmo genetico}

Cerchiamo ora di definire un algoritmo genetico sequenziale che sia una buona
base per quello che andremo a fare dopo. Si vuole infatti costruire un modello
su cui basare le possibili implementazioni che andremo a studiare.

\subsection{API e utilizzo}

Per quanto riguarda l'API messa a disposizione dell'utente vogliamo un qualcosa
che sia il più semplice possibile e che prevenga errori o cattiva gestione
delle strutture dati, fondamentali per la corretta esecuzione dell'algoritmo.

L'idea sarebbe quella di istanziare l'algoritmo come un oggetto, al quale
verranno forniti i vari metodi che compongono un algoritmo genetico.

\begin{minted}{py}
from genetic import GeneticAlgorithm

if __name__ == "__main__":
	ga = GeneticAlgorithm(
		population_size
		gen_func,
		selection_func,
		crossover_func,
		mutation_func,
		fitness_func,
		replacement_func,
		convergence_func
	)

	ga.run()
	results = ga.get()
\end{minted}

In questo modo non si espongono le strutture dati all'esterno della classe se
non come copie o come \emph{viste} dell'originale. Questo rende anche possibile
il riutilizzo di strutture dati già allocate andando rendere più efficiente
l'algoritmo in un senso che sarà più chiaro andando avanti.

\subsection{Cromosomi}

Prima di definire le varie fasi dell'algoritmo, introduciamo brevemente come
vengono rappresentati i \textbf{cromosomi}.

\begin{minted}{py}
class Chromosome:
	def __init__(self, values, fitness) -> None:
		self.values = values
		self.fitness = fitness
\end{minted}

Ogni cromosoma rappresenta un individuo della popolazione, il quale è
generalmente identificato da un vettore di valori numerici e da un valore di
fitness.

Noi tratteremo solo casi in cui abbiamo vettori numerici, sarà poi compito del
programmatore mappare il cromosoma in ciò che gli serve per risolvere il
problema di interesse.

\subsection{Generazione della popolazione iniziale}

In questa prima fase andiamo a generare una popolazione iniziale di $N$
individui in modo del tutto casuale. Evitare di generare duplicati è buona
norma, almeno in questa fase, così da garantire un alto grado di
\emph{biodiversità} iniziale.

\begin{minted}{py}
def __generate(self) -> None:
	for i in range(N):
		values = self.gen_func()
		while values in population:
			values = self.gen_func()
		self.population[i].values = values
\end{minted}

Per la fase iniziale di generazione, lasciamo al programmatore il compito di
definire come viene generato il singolo cromosoma. Sarà poi il modulo ad
occupersi di gestire i duplicati e memorizzare la popolazione.

\subsection{Selezione}

Le possibili implementazioni della fase di selezione degli individui per
l'accoppiamento sono molto varie e potrebbero non avere parti in comune.

Prendiamo ad esempio la selezione a \emph{torneo} in cui si scelgono a due a
due degli individui e quello con fitness più alta viene scelto.

In una soluzione a \emph{roulette} invece gli individui vengono scelti
singolarmente e volta per volta. In ogni caso si lavora sempre sull'intera
popolazione e non ci sono parti facilmente \emph{generalizzabili}.

Ecco perché in questa fase si richiede al programmatore di implementare la fase
di selezione per intero.

\begin{minted}{py}
def __selection(self):
	self.selected = self.selection_func(self.population)
\end{minted}

Ciò che si richiede è che la funzione di selezione ritorni una lista di
puntatori agli individui selezionati.

\subsection{Crossover}

La fase di crossover si rivelerà essere una delle più costose ed è quindi
necessario implementare delle ottimizzazioni già nella progettazione
dell'algoritmo sequenziale.

Per la fase di crossover facciamo alcune assunzioni, necessarie per riuscire
a parallelizzare l'algoritmo in seguito. Vogliamo infatti lasciare al
programmatore solamente il compito di definire l'operatore di crossover, il
quale prende in input solo due individui e restituisce (in genere due) nuovi
individui. Il come vengono formate le coppie viene lasciato alle routine
interne della classe che stiamo definendo.

\begin{minted}{py}
def __crossover(self) -> None:
	for i in range(0, len(self.selected), 2):
		father, mother = random.choices(self.selected, k=2)

		offspring1, offspring2 = self.crossover_func(father, mother)
		self.offsprings[i] = offspring1
		self.offsprings[i+1] = offspring2

		self.selected.remove(father)
		try:
			selected.remove(mother)
		except ValueError:
			pass
\end{minted}

Altra assunzione che fa l'algoritmo in questione è che il numero di nuovi
individui generati non cambia mai. Questo ci permette di inizializzare una
lista \verb|offsprings| a cui poi possiamo accedere tramite indice e senza
quindi andare riallocare di continuo memoria per i nuovi individui generati.

\subsection{Mutazione}

Per quanto riguarda la fase di mutazione si lascia al programmatore il compito
di definire una funzione che prende in input un cromosoma e applica una qualche
tecnica di mutazione.

\begin{minted}{py}
def __mutation(self) -> None:
	for i in range(len(self.offsprings)):
		if random.random() < self.mutation_rate:
			self.offsprings[i] = self.mutation_func(self.offsprings[i])
\end{minted}

Il modulo si occupa di gestire la probabilità che ogni individuo ha di essere
mutato.

\subsection{Valutazione}

Fase abbastanza semplice ma che costituisce uno dei colli di bottiglia maggiori
per le prestazioni

\begin{minted}{py}
def __evaluation(self) -> None:
	for i in range(len(self.offsprings)):
		self.offsprings[i].fitness = self.fitness_func(self.offsprings[i].values)
\end{minted}

Anche qui l'algoritmo non fa altro che applicare la funzione di fitness definita
dall'utente ad ogni nuovo individuo della popolazione.
