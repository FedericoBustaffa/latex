\chapter{Contesto}\label{cap: contesto}

Come anticipato nell'introduzione, il contesto in cui si va a collocare il
lavoro di tesi è molteplice. Abbiamo infatti a che fare con algoritmi genetici,
calcolo parallelo ed accenni di explainable AI.

\section{Algoritmi genetici}

Gli algoritmi genetici fanno parte delle possibili tecniche euristiche adottate
nell'ambito dell'ottimizzazione. Costituiscono infatti una valida scelta in
molti casi in cui non si conosce un algoritmo di ottimizzazione efficiente per
un problema specifico e non si è interessati alla ricerca di soluzioni ottime
per esso.

Il loro comportamento si ispira ad un modello evolutivo in cui gli individui
migliori di una popolazione sopravvivono e si riproducono, portando quindi la
popolazione iniziale ad uno stadio evolutivo migliore con il passare delle
generazioni.

Le possibili implementazioni sono molto varie ma una delle più comuni prevede
la rappresentazione di ogni individuo tramite un cromosoma. Questo è
generalmente strutturato come un vettore o una stringa, i cui elementi
rappresentano una possibile soluzione al problema o una sua codifica. Una delle
implementazioni più semplici è composta dalle seguenti fasi:
\begin{enumerate}
      \item Generazione: viene generata in modo casuale una popolazione i cui
            individui rappresentano possibili soluzioni al problema.
      \item Selezione: vengono selezionati $k$ individui utilizzando metodi
            basati sul valore di fitness.
      \item Crossover: i cromosomi di due o più individui vengono combinati per
            generare nuovi individui che condividono parte del materiale genetico
            dei genitori.
      \item Mutazione: vengono applicate piccole variazioni casuali al cromosoma
            di un individuo, permettendo così l'esplorazione di nuove soluzioni e
            favorendo così la diversità genetica all'interno della popolazione.
      \item Valutazione: ad ogni individuo viene attribuito un valore di
            \textit{fitness} in base alla qualità della soluzione che questo
            rappresenta.
      \item Rimpiazzo: viene attuata una politica di rimpiazzo (in alcuni casi
            chiamata \quotes{di sopravvivenza}), che prevede la scelta degli
            individui da mantenere da una generazione all'altra.
\end{enumerate}
Tutte le fasi tranne la prima sono eseguite in un ciclo che termina quando
viene soddisfatto un criterio di convergenza, che può essere definito in
termini dei valori di fitness della popolazione, della sua biodiversità o, più
semplicemente, definendo un limite massimo al numero delle iterazioni.
L'algoritmo risulta quindi essere strutturato come segue
\begin{figure}
      \centering
      \includesvg[scale=0.7]{immagini/genetic_algorithm.svg}
      \caption{Struttura di un algoritmo genetico semplice}
      \label{fig:simple_ga}
\end{figure}
Vista la natura stocastica dell'algoritmo, è inoltre buona norma avere a
disposizione una struttura dati in grado di memorizzare le migliori $k$
soluzioni mai registrate tra le varie generazioni.

\subsection{Calcolo parallelo}

L'obiettivo principale del lavoro svolto è stato quello di rimodellare la
struttura appena descritta, di modo da riuscire ad eseguire in parallelo le
fasi di crossover, mutazione e valutazione, le quali rappresentavano
(soprattutto la valutazione) il vero collo di bottiglia per le prestazioni nel
caso d'uso di riferimento. Il modello di calcolo ha quindi assunto la seguente
forma
\begin{figure}
      \centering
      \includesvg[scale=0.65]{immagini/parallel_ga.svg}
      \caption{Algoritmo genetico parallelo}
      \label{fig: parallel_ga}
\end{figure}
che mira ad implementare un paradigma di tipo \textit{map}. Queste fasi
si prestano bene a questo tipo di paradigma poiché consistono
nell'applicare una funzione in modo indipendente ad ogni elemento di una
lista. In realtà crossover e mutazione vengono applicate con una certa
probabilità, ma una volta terminate, i nuovi individui vengono
sicuramente tutti valutati.

\subsubsection{Calcolo parallelo in Python}

Implementare modelli di calcolo parallelo in Python può non essere così
immediato come in altri linguaggi come C, C++ o Java. In Python non è infatti
possibile sfruttare il multithreading per compiti computazionalmente pesanti,
a causa del \textit{GIL} (Global Interpreter Lock). Come suggerisce il nome,
si tratta di una variabile di mutua esclusione che ogni thread deve acquisire
per poter eseguire il proprio codice. Cercare quindi di sfruttare il
multithreading per questo tipo di compiti non porterà alcun beneficio poiché
i thread verranno eseguiti uno per volta. Potrebbe anzi verificarsi un
deterioramento delle prestazioni dovuto all'overhead di scheduling e
sincronizzazione che il multithreading comporta.

Al momento ciò che viene considerato lo standard per risolvere questo tipo di
problema è l'impiego di un paradigma multi-processo. In questo modo si hanno
più interpreti che operano indipendentemente l'uno dall'altro e su spazi di
memoria differenti. Questo approccio richiede però maggiori risorse sia in
termini di memoria, sia in termini di tempo poiché la condivisione dati avviene
tramite meccanismi di streaming di byte (come pipe o socket) e non in memoria
condivisa come per il multithreading.

Esistono però altri metodi che, sotto determinate ipotesi, permettono di
sfruttare il parallelismo senza dover ricorrere all'utilizzo di processi
multipli. Con le nuove versioni di Python sono state inoltre proposte nuove
funzionalità (ancora sperimentali) che mirano proprio a risolvere o mitigare il
problema. Parte del lavoro è stato infatti volto all'esplorazione di tali
opzioni per garantire la massima efficienza possibile, garantendo però un certo
livello di espressività.

\section{Explainable AI}

L'explainable AI, abbreviato XAI, è una branca della ricerca volta a fornire
metodi per interpretare e \textit{spiegare} predizioni o decisioni fatte da
determinati modelli di machine learning, chiamati \textit{black-box}. Di questi
si conoscono architettura e modello matematico, ma allo stato dell'arte possono
arrivare ad avere una quantità di parametri talmente elevata da rendere
impossibile per un essere umano capire come e quanto i singoli parametri
influiscano nelle decisioni prese dal modello.

Al fine di riuscire ad interpretare le scelte fatte dai modelli
\textit{black-box}, sono stati sviluppati vari metodi. Alcuni di questi cercano
di comprendere la struttura globale del modello predittivo, per esempio cercando
di capire la forma dei confini di classificazione. Altri metodi si concentrano
invece sul fornire spiegazioni alle singole decisioni prese dal modello su
specifiche istanze dei dati, fornendo quindi spiegazioni locali. I metodi di
XAI possono inoltre fornire \textit{controfattuali}, ossia spiegazioni su come
modificare i dati i dati in input per cambiare le decisioni prese dal modello.

% approfondimento possibili approcci e metodi...
