\documentclass[12pt]{article}

% % --------------- PACKAGES ---------------
\usepackage[utf8]{inputenc}
\usepackage[T1]{fontenc}
\usepackage[italian]{babel}
\usepackage[hidelinks]{hyperref}

\hypersetup{
	colorlinks=true,
	linkcolor=blue
}

% --------------- STYLE ---------------
\usepackage[margin=1.25in]{geometry}
\usepackage[most]{tcolorbox}

% Font
\usepackage{sansmath}

\renewcommand{\familydefault}{\sfdefault}
\sansmath

% page style
\usepackage{fancyhdr}
\usepackage[Sonny]{fncychap}

\pagestyle{fancy}
\setlength{\headheight}{15pt}
\rhead{\thepage}
\cfoot{\thepage}

% --------------- MATH ---------------
\usepackage{amsmath}
\usepackage{amssymb}
\usepackage{amsthm}
\usepackage{amsfonts}
\usepackage{mathtools}
\usepackage{mdframed}

\newcommand{\N}{\mathbb{N}}
\newcommand{\Z}{\mathbb{Z}}
\newcommand{\R}{\mathbb{R}}
\newcommand{\C}{\mathbb{C}}
\newcommand{\E}{\mathbb{E}}

\newcommand{\F}{\mathcal{F}}

\DeclareMathOperator{\Var}{Var}
\DeclareMathOperator{\Cov}{Cov}

% Boxes for theorem, definitions and examples
\newtheoremstyle{math_box}
{0pt}
{0pt}
{\normalfont}
{}
{\color{orange}}
{\;}
{0.25em}
{\thmname{\textbf{#1}}\thmnumber{ \textbf{#2}}{\color{black}\thmnote{\textbf{ -- #3.}}}}

\newmdenv[
	rightline=false,
	leftline=true,
	topline=false,
	bottomline=false,
	linecolor=orange!40,
	innerleftmargin=5pt,
	innerrightmargin=5pt,
	innertopmargin=0pt,
	innerbottommargin=0pt,
	leftmargin=0cm,
	rightmargin=0cm,
	linewidth=3pt
]{dBox}

\newmdenv[
	rightline=false,
	leftline=false,
	topline=false,
	bottomline=false,
	backgroundcolor=orange!15,
	innerleftmargin=5pt,
	innerrightmargin=5pt,
	innertopmargin=5pt,
	innerbottommargin=5pt,
	leftmargin=0cm,
	rightmargin=0cm,
]{pBox}

\theoremstyle{math_box}
\newtheorem{theoremeT}{Teorema}[section]
\newtheorem{definitionT}{Definizione}[section]
\newtheorem{propositionT}{Proposizione}[section]
\newtheorem{corollary}{Corollario}[section]
\newtheorem{lemma}{Lemma}[section]
\newtheorem{observation}{Osservazione}[section]
\newtheorem{exampleT}{Esempio}[subsection]

\newenvironment{theorem}{\begin{pBox}\begin{theoremeT}}{\end{theoremeT}\end{pBox}}
\newenvironment{definition}{\begin{dBox}\begin{definitionT}}{\end{definitionT}\end{dBox}}
\newenvironment{proposition}{\begin{pBox}\begin{propositionT}}{\end{propositionT}\end{pBox}}
\newenvironment{example}{\begin{dBox}\begin{exampleT}}{\end{exampleT}\end{dBox}}

\usepackage{tikz, pgfplots, pgf-pie}
\usepackage{caption, subcaption}
\usepackage{tikz}
\usepackage{scalerel}
\usepackage{pict2e}
\usepackage{tkz-euclide}
\usepackage{pgfplots, pgfplotstable, pgf-pie}

\usetikzlibrary{calc}
\usetikzlibrary{patterns, arrows}
\usetikzlibrary{shadows}
\usetikzlibrary{external}

\pgfplotsset{compat=newest}
\usepgfplotslibrary{statistics, fillbetween}

\usepackage[T1]{fontenc}
\usepackage[italian]{babel}
\usepackage[hidelinks]{hyperref}
\usepackage[margin=1in]{geometry}
\usepackage{minted}
\usepackage{diagbox}
\usepackage{svg}
\usepackage{wrapfig}

\definecolor{minted_bg}{rgb}{0.9, 0.9, 0.9}
\usemintedstyle{colorful}

\setminted[py]{
	tabsize=4,
	linenos=true,
	bgcolor=minted_bg,
	fontsize=\small,
	mathescape=true
}

\setminted[cpp]{
	tabsize=4,
	linenos=true,
	bgcolor=minted_bg,
	fontsize=\small,
	mathescape=true
}

\setminted[bash]{
	tabsize=4,
	% linenos=true,
	bgcolor=minted_bg,
	fontsize=\small,
	mathescape=true
}

\title{Parallelizzazione di algoritmi genetici}
\author{Federico Bustaffa}
\date{01/09/2024}

\begin{document}

\maketitle

\begin{abstract}
	L'obbiettivo di questo progetto è quello di andare a studiare varie alternative
	per rendere più efficienti algoritmi genetici, soprattutto andando a lavorare
	in ambito parallelo per migliorare le fasi che costituiscono un collo di
	bottiglia per le prestazioni. In generale un algoritmo genetico ha sei
	componenti fondamentali:
	\begin{enumerate}
		\item \textbf{Generazione}: si genera in modo casuale la popolazione
		      iniziale.
		\item \textbf{Selezione}: si selezionano gli individui per l'accoppiamento
		      e la generazione di nuovi individui.
		\item \textbf{Crossover}: gli individui selezionati vengono fatti
		      accoppiare e se ne generano di nuovi.
		\item \textbf{Mutazione}: ogni nuovo individuo ha un certa probabilità di
		      subire una mutazione.
		\item \textbf{Valutazione}: si valuta il valore di fitness degli individui.
		\item \textbf{Rimpiazzo}: per mantenere omogeneo il numero di individui
		      nella popolazione si adottano politiche di rimpiazzo per scartare
		      alcuni individui.
	\end{enumerate}
	Non entriamo nel merito di quali siano possibili tecniche per implementare un
	algoritmo genetico. Quello che ci interessa è individuare la struttura di base.

	Le fasi che più ci preme ottimizzare andando a lavorare in parallelo sono
	quelle di crossover, mutazione e valutazione.

	La mutazione non è un passo computazionalmente dispendioso ma si presta bene ad
	essere parallelizzato. Le altre due fasi invece potrebbero richiedere molto
	tempo e costituire un grosso limite per le performance.
\end{abstract}

\tableofcontents

\section{Modello di calcolo parallelo}

Cerchiamo ora di definire lo scheletro del modello di calcolo parallelo, così
da avere un riferimento per le possibili implementazioni che andremo a studiare.

\subsection{API e utilizzo}

Per quanto riguarda l'API messa a disposizione dell'utente vogliamo un qualcosa
che sia il più semplice possibile e che prevenga errori o cattiva gestione
delle strutture dati, fondamentali per la corretta esecuzione dell'algoritmo.

L'idea sarebbe quella di istanziare l'algoritmo come un oggetto, al quale
verranno forniti i vari metodi che compongono un algoritmo genetico.

\begin{minted}{py}
from genetic import GeneticAlgorithm

if __name__ == "__main__":
	ga = GeneticAlgorithm(
		population_size
		gen_func,
		selection_func,
		crossover_func,
		mutation_func,
		fitness_func,
		replacement_func,
		convergence_func
	)

	ga.run()
	results = ga.get()
\end{minted}

In questo modo non si espongono le strutture dati all'esterno della classe se
non come copie o come \emph{viste} dell'originale. Questo rende anche possibile
il riutilizzo di strutture dati già allocate andando rendere più efficiente
l'algoritmo in un senso che sarà più chiaro andando avanti.

\subsection{Cromosomi}

Prima di definire l'algoritmo, introduciamo brevemente come vengono
rappresentati i \textbf{cromosomi}. Ogni cromosoma rappresenta un individuo
della popolazione, il quale è generalmente identificato da un vettore di valori
numerici e da un valore di fitness.

\begin{minted}{py}
class Chromosome:
	def __init__(self, values, fitness) -> None:
		self.values = values
		self.fitness = fitness
\end{minted}

Noi tratteremo solo casi in cui abbiamo vettori numerici, sarà poi compito del
programmatore mappare il cromosoma in ciò che gli serve per risolvere il suo
problema.

In questo modo abbiamo una rappresentazione chiara di cosa sia un cromosoma e
anche un codice più pulito.

\subsection{Generazione della popolazione iniziale}

In questa prima fase andiamo a generare una popolazione iniziale di $N$
individui in modo del tutto casuale. Evitare di generare duplicati è buona
norma, almeno in questa fase, così da garantire un alto grado di
\emph{biodiversità} iniziale.

Per la fase iniziale di generazione, lasciamo al programmatore il compito di
definire come viene generato il singolo cromosoma. Sarà poi il modulo ad
occupersi di gestire i duplicati e memorizzare la popolazione.

\begin{minted}{py}
	def __generate(self) -> None:
		for i in range(N):
			values = self.gen_func()
			while values in population:
				values = self.gen_func()
			self.population[i].values = values
\end{minted}



\end{document}
