\documentclass[12pt]{report}
% Template Tesi in Informatica, ispirata al template "Tesi di laurea - Universit� di Pisa" di Simone Schirinzi (https://www.overleaf.com/latex/templates/tesi-di-laurea-universita-di-pisa/rwdcqtqwftpg). 

% Carattere dimensione 12

% Per la stampa fronte-retro sostituire con:
% \documentclass[12pt, twoside]{report}

% Margini (4cm a sx, 2.5cm a dx, 2.5cm in alto, 2.5cm in basso)
\usepackage[top=2.5cm, bottom=2.5cm, left=4cm, right=2.5cm, head=21.75pt, centering]{geometry}

% Per la stampa fronte-retro sostituire con: 
% \usepackage[top=2.5cm, bottom=2.5cm, inner=4cm, outer=4cm, right=2.5cm, centering]{geometry}

% Interlinea
\linespread{1.5}

% Librerie utili
\usepackage[italian]{babel} % applicazione regole di scrittura per la lingua italiana 
\usepackage[utf8]{inputenc} % codifica UTF-8
\usepackage{scrlayer-scrpage} % stili pagina per il frontespizio
\usepackage[hidelinks]{hyperref} % collegamenti ipertestuali
\ifoot[]{}
\cfoot[]{}
\ofoot[\pagemark]{\pagemark}
\pagestyle{scrplain}
% \usepackage{mathptmx} % font Times New Roman (simile)
\usepackage{booktabs}
\usepackage{tabularx}
\usepackage{caption}
\usepackage{graphicx} % inserimento di immagini
\usepackage{csquotes} % per le citazioni "in blocco"
\usepackage{svg}
\usepackage{tikz}
\usepackage{pgfplots}
\pgfplotsset{compat=1.18}

% \usepackage[backend=biber, sorting=nty, ]{biblatex} % bibliografia con pacchetto biblatex (https://ctan.org/pkg/biblatex?lang=en)
\appto{\bibsetup}{\raggedright}

\newcommand{\quotes}[1]{``#1''}

\usepackage{titlesec} % per la formattazione dei titoli delle sezioni, capitoli etc.
\usepackage{float} % per il posizionamento delle immagini

% Fonte https://en.wikibooks.org/wiki/LaTeX/Source_Code_Listings. Per la lista di sintassi riconosciute.
\usepackage{listings} % per il codice di programmazione
\usepackage{algpseudocodex}
\usepackage{algorithm}
\renewcommand{\lstlistingname}{Snippet}% Listing -> Codice
\renewcommand{\lstlistlistingname}{Codice}% Listing -> Codice
\usepackage{xcolor}  % stile del codice
\definecolor{mygreen}{rgb}{0,0.6,0}
\definecolor{mygray}{rgb}{0.5,0.5,0.5}
\definecolor{mymauve}{rgb}{0.58,0,0.82}
\definecolor{darkgray}{rgb}{.4,.4,.4}
\definecolor{navy}{HTML}{000080}
\definecolor{purple}{rgb}{0.65, 0.12, 0.82}
\definecolor{codepurple}{rgb}{0.58,0,0.82}
\definecolor{backcolour}{rgb}{0.95,0.95,0.92}

% Stili configurabili del codice (lslisting) 
\lstset{ %
  belowcaptionskip=0.5em,
  backgroundcolor=\color{backcolour}, % choose the background color; you must add \usepackage{color} or \usepackage{xcolor}
  basicstyle=\footnotesize, % the size of the fonts that are used for the code
  breakatwhitespace=false, % sets if automatic breaks should only happen at whitespace
  breaklines=true, % sets automatic line breaking
  captionpos=b, % sets the caption-position to bottom
  commentstyle=\color{mygray}, % comment style
  deletekeywords={...}, % if you want to delete keywords from the given language
  escapeinside={\%*}{*)}, % if you want to add LaTeX within your code
  extendedchars=true, % lets you use non-ASCII characters; for 8-bits encodings only, does not work with UTF-8
  frame=single, % adds a frame around the code
  keepspaces=true, % keeps spaces in text, useful for keeping indentation of code (possibly needs columns=flexible)
  keywordstyle=\color{mygreen}, % keyword style
  language=Python, % the language of the code
  morekeywords={*,...}, % if you want to add more keywords to the set
  numbers=left, % where to put the line-numbers; possible values are (none, left, right)
  numbersep=2pt, % how far the line-numbers are from the code
  numberstyle=\tiny\color{mygray}, % the style that is used for the line-numbers
  rulecolor=\color{black}, % if not set, the frame-color may be changed on line-breaks within not-black text (e.g. comments (green here))
  showspaces=false, % show spaces everywhere adding particular underscores; it overrides 'showstringspaces'
  showstringspaces=false, % underline spaces within strings only
  showtabs=false, % show tabs within strings adding particular underscores
  stepnumber=1, % the step between two line-numbers. If it's 1, each line will be numbered
  stringstyle=\color{mymauve}, % string literal style
  tabsize=2, % sets default tabsize to 2 spaces
  title=\lstname % show the filename of files included with \lstinputlisting; also try caption instead of title
}

% END of listing package 
\lstset{
  language=Python,
  extendedchars=true,
  basicstyle=\footnotesize\ttfamily,
  morekeywords={as, with, nogil, cdef, cpdef},
  showstringspaces=false,
  showspaces=false,
  numbers=left,
  numberstyle=\footnotesize,
  numbersep=9pt,
  tabsize=2,
  breaklines=true,
  showtabs=false,
  captionpos=b
}

% Formato delle intestazioni
\titleformat{\chapter}[block]
  {\normalfont\LARGE\bfseries}{\thechapter.}{0.5em}{\LARGE}
\titlespacing*{\chapter}{0pt}{-20pt}{25pt}

\begin{document}

% Frontespizio
\begin{titlepage}
  \begin{figure}
    \centering\includegraphics[scale=0.5]{immagini/cherubino_pant541.png}
  \end{figure}

  \begin{center}
    {\LARGE{ Corso di Laurea in Informatica \\ }}
    \vspace{2cm}
    {\Large { TESI DI LAUREA }}\\
    \vspace{2cm}
    {\Large { Implementazione Parallela ed Efficiente di Algoritmi Genetici per Explainable AI }}
  \end{center}

  \vspace{2cm}

  \begin{minipage}[t]{0.47\textwidth}
    {\large{Relatori:\\ Riccardo Guidotti}}
    \vspace{0.5cm}
    {\large{\\Correlatore:\\ Marco Danelutto}}
  \end{minipage}\hfill\begin{minipage}[t]{0.47\textwidth}\raggedleft
    {\large{Candidato: \\ Federico Bustaffa\\ }}
  \end{minipage}

  \vspace{25mm}

  \centering{\large{\bf ANNO ACCADEMICO 2024/2025 }}
\end{titlepage}
% Fine frontespizio

\tableofcontents
\thispagestyle{empty}

\listoffigures
\lstlistoflistings

\thispagestyle{empty}
\clearpage
\setcounter{page}{1}
\addtocontents{toc}{\protect\thispagestyle{empty}}
% \addcontentsline{toc}{chapter}{Introduzione} % Capitolo non numerato

\chapter{Introduzione}\label{cap: introduction}

Il progetto di tesi verte sull'implementazione di una libreria di algoritmi
genetici in grado di sfruttare architetture multi-core per il calcolo parallelo.
La necessità di un algoritmo genetico con questa struttura nasce dal progetto
\textit{LORE}~\cite{guidotti2018LORE} di explainable AI, il quale utilizza
l'implementazione fornita dalla libreria di algoritmi genetici
\textit{DEAP}~\cite{fortin2012DEAP}.

\textit{LORE} si propone di generare spiegazioni a decisioni o predizioni fatte
da modelli di machine learning, spesso difficili da interpretare per via della
loro elevata complessità, come ad esempio \textit{deep neural networks} o
\textit{random forest}, che proprio per questo motivo prendono spesso il nome
di \textit{black-box}. In una prima fase, del metodo si sfrutta un algoritmo
genetico per la generazione di dati sintetici con determinate caratteristiche,
fondamentali in seguito per la produzione delle spiegazioni.

Tale approccio risulta però essere particolarmente dispendioso, soprattutto nei
casi in cui le predizioni da \quotes{spiegare} sono molte, la quantità di dati
sintetici che si desidera generare è molto grande oppure quando il modello è
particolarmente lento in fase di predizione (o una combinazione delle tre).

Da questa necessità nasce \textit{PPGA} (\textit{Parallel Processing for
	Genetic  Algorithms}), una libreria di algoritmi genetici in  grado di
produrre risultati qualitativamente simili a \textit{DEAP} ma più performante e
in grado di sfruttare al meglio architetture multi-core.

\textit{LORE} e \textit{DEAP}, così come la maggior parte delle librerie di
machine learning, sono implementati (o forniscono un'API) in Python, rendendo
difficoltosa l'implementazione di un modello di calcolo parallelo a causa del
\textit{GIL}, che limita il multithreading permettendo ad un solo thread alla
volta di essere eseguito. La prima fase del lavoro si è quindi focalizzata
sulla ricerca di possibili framework in grado di interfacciarsi e lavorare in
sinergia con Python, cercando allo stesso tempo di aggirare il problema che
introduce il \textit{GIL}.

Una volta implementata la libreria si è passati ad una fase di test per
valutarne la correttezza, impiegando l'algoritmo nella risoluzione di problemi
ben noti in letteratura, facilmente rappresentabili tramite grafici e che non
fossero computazionalmente troppo dispendiosi. Tra questi il problema dello
zaino, del commesso viaggiatore e un semplice caso di regressione lineare. Si è
infine testata la correttezza su una riproduzione semplificata del problema che
il metodo \textit{LORE} necessita di risolvere per generare spiegazioni.

Il lavoro si è concluso con due fasi di test, riguardanti rispettivamente la
qualità delle soluzioni prodotte e le prestazioni offerte dalla libreria.
Entrambe le tipologie di test sono state condotte sullo stesso problema di
generazione dati, implementando un algoritmo risolutivo sia con \textit{PPGA}
che con \textit{DEAP}, in modo da permettere un confronto tra le due librerie.

La tesi si sviluppa su quattro sezioni principali: la prima inquadra meglio il
contesto in cui si opera, trattando algoritmi genetici, calcolo parallelo ed
eXplainable AI. Di quest'ultimo verrà trattato più in dettaglio il metodo
\textit{LORE}, come sfrutta l'algoritmo genetico e quali sono le criticità che
\textit{PPGA} tenta di risolvere. Saranno inoltre trattati in profondità gli
algoritmi genetici, la loro struttura e come sia possibile ottimizzarli con il
calcolo parallelo. Si tratteranno infine le problematiche implementative che
comporta il \textit{GIL} e quali sono le sfide che pone nell'ambito degli
algoritmi genetici che operano in parallelo.

A seguire un resoconto di ciò che già esiste allo stato dell'arte, in particolar
modo verrà trattato il funzionamento della libreria \textit{DEAP}, dei costrutti
e dell'architettura parallela che propone, cercando quindi di evidenziarne
limiti e punti di forza.

Verrà trattata in seguito la metodologia che ha guidato lo sviluppo della
libreria, evidenziando quindi i requisiti che questa deve soddisfare in termini
di strutture dati, espressività e interazione con l'utente, sia nella versione
sequenziale che in quella parallela dell'algoritmo. Si sono messi inoltre in
evidenza vantaggi e problematiche dei vari framework esplorati per riuscire ad
implementare la versione parallela dell'algoritmo.

Infine un'analisi sulla sperimentazione effettuata al fine di giustificare
scelte implementative, valutare qualità e correttezza degli algoritmi e
confrontare le prestazioni di quanto implementato rispetto a \textit{DEAP}.


\chapter{Background}\label{cap: background}

Per comprendere le motivazioni dietro all'implementazione della libreria è
necessario analizzare il metodo \textit{LORE} di eXplainable AI (XAI), come
questo sfrutti un algoritmo genetico per la generazione di dati sintetici e
la natura dietro al problema di prestazioni riscontrato.

Si sono quindi messe a fuoco la struttura e le operazioni tipiche di un
algoritmo genetico, per riuscire a implementare una versione sequenziale che
fosse facilmente parallelizzabile. Durante lo sviluppo della versione parallela
dell'algoritmo ci si è però scontrati con il problema introdotto dal
\textit{GIL}, che in Python limita molto l'esecuzione simultanea di più thread,
almeno per quanto riguarda compiti CPU-bound. Si sono quindi individuate
possibili soluzioni al problema, che comportano però alcuni compromessi in
quanto a prestazioni o epsressività del codice.

\section{Algoritmi Genetici}\label{sec: gen_alg}

Gli algoritmi genetici fanno parte delle possibili tecniche euristiche adottate
nell'ambito dell'ottimizzazione. Costituiscono infatti una valida scelta in
molti casi in cui non si conosce un algoritmo efficiente per un problema
specifico e non si è interessati alla ricerca di soluzioni ottime per esso.

Il loro comportamento si ispira ad un modello evolutivo in cui gli individui
migliori di una popolazione sopravvivono e si riproducono, portando quindi la
popolazione iniziale ad uno \quotes{stadio superiore} con il passare delle
generazioni.

Le possibili strutture sono molto varie ma una delle più comuni prevede la
rappresentazione di ogni individuo tramite un cromosoma. Questo è generalmente
strutturato come un vettore o una stringa, i cui elementi rappresentano una
possibile soluzione al problema o una sua codifica. Una delle strutture più
semplici e comuni degli algoritmi genetici è composta dalle seguenti fasi:
\begin{enumerate}
	\item \textbf{Generazione}: viene generata in modo casuale una popolazione
	      di individui il cui cromosoma rappresenta una possibile soluzione al
	      problema.
	\item \textbf{Selezione}: viene selezionato un certo numero di individui
	      utilizzando metodi basati sul valore di \textit{fitness}, il quale
	      indica la qualità della soluzione.
	\item \textbf{Crossover}: i cromosomi di due o più individui vengono
	      combinati per generarne di nuovi, i quali condividono parte del
	      materiale genetico dei genitori.
	\item \textbf{Mutazione}: vengono applicate variazioni casuali al cromosoma
	      di un individuo, permettendo così l'esplorazione di nuove soluzioni e
	      favorendo così la diversità genetica all'interno della popolazione.
	\item \textbf{Valutazione}: ogni individuo viene valutato e gli viene
	      attribuito un valore di \textit{fitness} in base alla qualità della
	      soluzione che questo rappresenta.
	\item \textbf{Rimpiazzo}: viene attuata una politica di rimpiazzo (in
	      alcuni casi chiamata \quotes{di sopravvivenza}), che prevede la
	      scelta degli individui da mantenere da una generazione all'altra.
\end{enumerate}
Tutte le fasi tranne la prima sono eseguite in un ciclo che termina quando
viene soddisfatto un criterio di convergenza, che può essere definito in
termini dei valori di fitness della popolazione, della sua biodiversità o, più
semplicemente, definendo un limite massimo al numero delle iterazioni.

\begin{figure}[H]
	\centering
	\includesvg[width=0.95\linewidth]{immagini/simple_ga.svg}
	\caption{Struttura di un algoritmo genetico semplice}
	\label{fig:simple_ga}
\end{figure}

Data inoltre la natura stocastica dell'algoritmo, è possibile che le migliori
soluzioni vengano trovate dopo poche iterazioni. Se il metodo di convergenza
si basa semplicemente sul raggiungimento di un numero massimo di iterazioni e
la fase di rimpiazzo sostituisce completamente la vecchia generazione con la
nuova, si rischia di perdere alcune delle soluzioni migliori tra le varie
generazioni. È infatti buona norma avere una struttura dati in grado di
mantenere le migliori soluzioni mai generate, di modo da evitarne la perdita.

\section{Modello di calcolo parallelo}

Cerchiamo ora di definire lo scheletro del modello di calcolo parallelo, così
da avere un riferimento per le possibili implementazioni che andremo a studiare.

\subsection{API e utilizzo}

Per quanto riguarda l'API messa a disposizione dell'utente vogliamo un qualcosa
che sia il più semplice possibile e che prevenga errori o cattiva gestione
delle strutture dati, fondamentali per la corretta esecuzione dell'algoritmo.

L'idea sarebbe quella di istanziare l'algoritmo come un oggetto, al quale
verranno forniti i vari metodi che compongono un algoritmo genetico.

\begin{minted}{py}
from genetic import GeneticAlgorithm

if __name__ == "__main__":
	ga = GeneticAlgorithm(
		population_size
		gen_func,
		selection_func,
		crossover_func,
		mutation_func,
		fitness_func,
		replacement_func,
		convergence_func
	)

	ga.run()
	results = ga.get()
\end{minted}

In questo modo non si espongono le strutture dati all'esterno della classe se
non come copie o come \emph{viste} dell'originale. Questo rende anche possibile
il riutilizzo di strutture dati già allocate andando rendere più efficiente
l'algoritmo in un senso che sarà più chiaro andando avanti.

\subsection{Cromosomi}

Prima di definire l'algoritmo, introduciamo brevemente come vengono
rappresentati i \textbf{cromosomi}. Ogni cromosoma rappresenta un individuo
della popolazione, il quale è generalmente identificato da un vettore di valori
numerici e da un valore di fitness.

\begin{minted}{py}
class Chromosome:
	def __init__(self, values, fitness) -> None:
		self.values = values
		self.fitness = fitness
\end{minted}

Noi tratteremo solo casi in cui abbiamo vettori numerici, sarà poi compito del
programmatore mappare il cromosoma in ciò che gli serve per risolvere il suo
problema.

In questo modo abbiamo una rappresentazione chiara di cosa sia un cromosoma e
anche un codice più pulito.

\subsection{Generazione della popolazione iniziale}

In questa prima fase andiamo a generare una popolazione iniziale di $N$
individui in modo del tutto casuale. Evitare di generare duplicati è buona
norma, almeno in questa fase, così da garantire un alto grado di
\emph{biodiversità} iniziale.

Per la fase iniziale di generazione, lasciamo al programmatore il compito di
definire come viene generato il singolo cromosoma. Sarà poi il modulo ad
occupersi di gestire i duplicati e memorizzare la popolazione.

\begin{minted}{py}
	def __generate(self) -> None:
		for i in range(N):
			values = self.gen_func()
			while values in population:
				values = self.gen_func()
			self.population[i].values = values
\end{minted}


\section{Explainable AI}\label{sec: bg_xai}

L'explainable AI, è una branca della ricerca volta a fornire metodi per
interpretare e \textit{spiegare} predizioni o decisioni fatte da determinati
modelli di machine learning, chiamati \textit{black-box}. Di questi si
conoscono architettura e modello matematico, ma allo stato dell'arte possono
arrivare ad avere una quantità di parametri talmente elevata da rendere
impossibile per un essere umano capire come e quanto i singoli parametri
influiscano nelle decisioni prese dal modello.

Al fine di riuscire ad interpretare le scelte fatte dai modelli
\textit{black-box}, sono stati sviluppati vari metodi. Alcuni di questi cercano
di comprendere la struttura globale del modello predittivo, per esempio cercando
di definire i confini di classificazione. Altri metodi si concentrano invece
sul fornire spiegazioni alle singole decisioni prese dal modello su specifiche
istanze dei dati, fornendo quindi spiegazioni locali. I metodi di XAI possono
inoltre fornire \textit{controfattuali}, ossia spiegazioni su come modificare i
dati in input per cambiare le decisioni prese dal modello.

\subsection*{LORE}

Il metodo \textit{LORE} rientra nella categoria dei metodi di XAI locali, in
quanto opera sulle singole istanze dei dati e lavora sotto l'assunzione che
determinare i confini di classificazione locali sia in generale meno complesso
di determinare l'intero confine. In questo modo riesce ad ottenere spiegazioni
più accurate, ma per l'appunto locali e dunque valide solo per l'istanza
analizzata.

Il metodo prevede la generazione di dati sintetici tramite un algoritmo
genetico; questi verranno in seguito usati per allenare un albero di decisione
in grado di produrre regole fattuali e controfattuali.

L'algoritmo genetico viene eseguito due volte per ognuna delle istanze dei dati
classificate dal modello e di cui si vogliono produrre spiegazioni; la prima
volta generando dati classificati allo stesso modo dell'istanza analizzata,
producendo l'insieme $Z_=$, la seconda volta generando invece dati classificati
diversamente, producendo l'insieme $Z_{\neq}$. In entrambi i casi l'algoritmo
genetico cerca di minimizzare la distanza tra i punti generati e l'istanza
analizzata, scartando di generazione in generazione i dati sintetici più lontani
e/o non classificati correttamente.

I dati hanno tipicamente una struttura vettoriale, in cui ogni elemento
corrisponde a una caratteristica specifica. Un esempio banale è un vettore di
due elementi, i quali definiscono rispettivamente altezza e peso di un
individuo. Tra questi però è anche possibile definire dati dalla natura
categoriale, come per esempio il colore degli occhi di un individuo, che non ha
un significato matematico e per il quale non è possibile stabilire un ordine di
importanza.

Dato che \textit{LORE} utilizza un algoritmo genetico per generare individui
sintetici che siano simili alle istanze classificate dal modello, la forma dei
cromosomi è esattamente quella di un vettore di feature. Questo perché le
feature di un individuo sintetico rappresentano intrinsecamente una possibile
soluzione ai problemi di minimizzazione della distanza, in quanto costituiscono
le coordinate dell'individuo nello spazio di lavoro e dunque la sua distanza
dall'istanza di riferimento. In questo modo è inoltre possibile classificare
gli individui sintetici per riuscire a scartare quelli non classificati
correttamente.

La funzione di valutazione ritorna quindi come unico valore la distanza tra
l'individui sintetico e l'istanza di riferimento nel caso quello sintetico sia
classificato correttamente. In caso contrario ritorna una distanza di $+\infty$.

\begin{algorithm}[H]
	\caption{LORE evaluation function}
	\begin{algorithmic}
		\Function{evaluate\_same}{chromosome, istance, model}
		\State $C_i$ $\gets$ model.predict(istance)
		\State $C_c$ $\gets$ model.predict(chromosome)
		\If{$C_i \neq C_c$}
		\State \Return $+\infty$
		\Else
		\State \Return distance (chromosome, istance)
		\EndIf
		\EndFunction
	\end{algorithmic}
	\label{alg: lore_eval}
\end{algorithm}

Nel caso di dati puramente numerici è possibile calcolare la distanza tramite
funzioni come la norma vettoriale euclidea, ma nel caso di dati categoriali
come quelli descritti in precedenza sarebbe necessario definire un metodo o un
ordine di importanza tra di essi per riuscire a calcolare la distanza. Per
semplicità i test sono stati condotti solo su dataset a valori reali, calcolando
la distanza euclidea tra i punti.

Una volta generati, i dati sintetici vengono usati per allenare un albero di
decisione, in grado di produrre regole fattuali e controfattuali, che andranno
poi a costituire la spiegazione della scelta fatta e come sia possibile
modificare i dati per ottenere una predizione diversa. Le diramazioni di un
albero decisionale corrispondono infatti a valori soglia delle feature che
determinano una scelta piuttosto che un'altra. Seguendo quindi i cammini che
portano alla stessa decisione effettuata dal modello si ottiene la spiegazione
di tale scelta, seguendo invece cammini alternativi è possibile capire come
cambiare tale scelta.


\chapter{Lavori Correlati}\label{cap: related_works}

Allo stato dell'arte, il metodo \textit{LORE} sfrutta l'implementazione fornita
dalla libreria \textit{DEAP} di algoritmi genetici, la quale offre grande
flessibilità, permettendo all'utente di definire le strutture dati necessarie
a rappresentare cromosomi e popolazione e le varie fasi dell'algoritmo.
Mette inoltre a disposizione un metodo per riuscire a sfruttare il parallelismo
tramite il modulo \verb|multiprocessing|, riuscendo così a mantenere un alto
livello di espressività e massima compatibilità con moduli di terze parti.

\section*{DEAP}

La libreria \textit{DEAP} fornisce un'API molto flessibile ma richiede una fase
di inizializzazione in cui vengono definiti tipi, strutture dati e funzioni
necessarie al corretto funzionamento dell'algoritmo.

Come prima cosa è necessario definire la struttura di un generico cromosoma, il
tipo di ciascun gene e i pesi da assegnare a ciascun parametro da ottimizzare.
A tal fine vengono messi a disposizione il modulo \verb|creator| e la classe
\verb|Toolbox|.

Tramite \verb|creator| è possibile definire nuove classi con caratteristiche
personalizzate e che si adattano alle proprie esigenze.

\begin{lstlisting}[caption={Definizione fitness e strutture dati con DEAP}]
from deap import creator

creator.create("FitnessMin", base.Fitness, weights=(-1.0,))
creator.create("Individual", list, fitness=creator.FitnessMin)
\end{lstlisting}

La classe \verb|Toolbox| serve invece ad immagazzinare funzioni e relativi
parametri, di modo da poter essere utilizzati con facilità in seguito. Tramite
tale classe è infatti possibile definire operatori genetici e metodi di
inizializzazione, selezione e valutazione degli individui, ognuno dei quali
viene registrato e legato ad una stringa che ne definisce il nome.

\begin{lstlisting}[caption={Registrazione metodi e operatori con DEAP}]
import random
from deap import base, tools

toolbox = base.Toolbox()
toolbox.register("indices", random.sample, range(n), n)
toolbox.register("individual", tools.initIterate, creator.Individual, toolbox.indices)
toolbox.register("population", tools.initRepeat, list, toolbox.individual)
toolbox.register("mate", tools.cxOrdered)
toolbox.register("mutate", tools.mutShuffleIndexes)
toolbox.register("select", tools.selTournament, tournsize=2)
toolbox.register("evaluate", evaluate)
\end{lstlisting}

Ad eccezione della funzione di valutazione, che è strettamente legata al
problema di interesse e deve essere quindi definita dall'utente caso per caso,
\textit{DEAP} fornisce implementazioni per tutte le componenti appena citate,
lasciando quindi il solo compito di scegliere quale si addice di più al
problema in questione. Rimane comunque la possibilità di definire ciascuna
delle componenti dell'algoritmo come meglio si crede in caso di esigenze
particolari.

\subsection*{Algoritmo}\label{ssec: rw_algos}

Una volta definite tutte le componenti dell'algoritmo, è possibile invocare
direttamente i metodi registrati nell'istanza della classe \verb|Toolbox|,
andando a definire un comportamento personalizzato per esso. Una seconda opzione
consiste invece nel ricorrere alle implementazioni fornite da \textit{DEAP}
tramite il modulo \verb|algorithms|, che presenta diverse opzioni possibili,
tra cui \verb|eaSimple|, che ha una struttura molto simile a quella definita in
figura \ref{fig:simple_ga} ed è l'algoritmo attualmente impiegato da
\textit{LORE}.

L'algoritmo prende in input l'istanza della classe \verb|Toolbox|
precedentemente definita, una popolazione iniziale, il numero massimo di
iterazioni da compiere e le probabilità di crossover e mutazione. \`E possibile
passare anche parametri opzionali per registrare dati statistici sui valori di
fitness e una struttura dati per memorizzare i migliori $k$ individui mai
prodotti durante le varie generazioni, la \verb|HallOfFame|.

\begin{lstlisting}[caption={DEAP \lstinline|eaSimple|}]
import numpy as np
from deap import algorithms, tools

stats = tools.Statistics(key=lambda ind: ind.fitness.values)
stats.register("min", np.min)
stats.register("mean", np.mean)
stats.register("max", np.max)

hall_of_fame = tools.HallOfFame(20)
population = toolbox.population(n=100)
population, logbook = algorithms.eaSimple(
	population=population, toolbox=toolbox,
	cxpb=0.7, mutpb=0.3, ngen=50,
	stats=stats, halloffame=hall_of_fame
)
\end{lstlisting}

A differenza dello schema \ref{fig:simple_ga}, \verb|eaSimple| effettua una
prima valutazione degli individui subito dopo aver generato la popolazione
iniziale e prima di entrare nel ciclo.

\begin{figure}[H]
	\centering
	\includesvg[width=0.95\linewidth]{immagini/deap_ga.svg}
	\caption{Struttura dell'algoritmo eaSimple implementato da DEAP}
	\label{fig: deap_ga}
\end{figure}

In fase di selezione vengono scelti sempre $n$ individui senza alcun controllo
sui duplicati, rendendo possibile scegliere lo stesso individuo più volte.
L'operatore di crossover viene applicato con una certa probabilità \verb|cxpb|,
provocando la generazione di nuovi individui, nel caso venga applicato e la
clonazione dei genitori in caso contrario.

Segue poi la fase di mutazione che introduce una variazione nel cromosoma di
ciascun individuo con una probabilità \verb|mutpb| e, a seconda del metodo di
mutazione, potrebbe essere necessario specificare la probabilità di mutare
ciascun gene del singolo cromosoma.

I nuovi individui vengono infine valutati tramite la funzione di valutazione
registrata, la \textit{HallOfFame} viene aggiornata e la nuova generazione
prende il posto della vecchia, rimpiazzandola totalmente.

\subsection*{Multi-Processing}

La classe \verb|Toolbox| ha anche dei metodi predefiniti tra cui \verb|map|, la
cui implementazione è data dall'omonima funzione built-in di Python e che viene
utilizzata per applicare la funzione di valutazione a tutti gli individui. Per
riuscire a valutare gli individui in parallelo sarà quindi sufficiente
registrare nella classe \verb|Toolbox|, la funzione \verb|map| della classe
\verb|Pool|, offerta dal modulo \verb|multiprocessing|.

\begin{lstlisting}[caption={DEAP Multiprocessing}]
import multiprocessing as mp
pool = mp.Pool()
toolbox.register("map", pool.map)
\end{lstlisting}

Il metodo \verb|map| della classe \verb|Pool| applica una certa funzione a
tutti gli elementi di una struttura dati iterabile passata come argomento. Se
non vengono specificati ulteriori argomenti, gli elementi dell'iterabile
verranno distribuiti uno per volta ai processi non appena questi sono liberi.
Se invece viene specificato il parametro \verb|chunksize| è possibile definire
il numero di elementi che un processo deve elaborare prima di poterne ricevere
altri.

In questo modo l'algoritmo utilizza quest'ultima per valutare gli individui
sfruttando più worker e modificando quindi la sua struttura dell'algoritmo come
segue:

\begin{figure}[H]
	\centering
	\includesvg[width=0.9\linewidth]{immagini/parallel_eaSimple}
	\caption{Versione parallela di \lstinline|eaSimple| con 4 worker}
	\label{fig: parallel_eaSimple}
\end{figure}

In questo modo si abilita facilmente la computazione parallela ma ci si imbatte
in tutti i problemi di comunicazione tipici del multiprocessing, discussi in
precedenza.


\chapter{Metodologia}\label{cap: methods}

La prima fase di sviluppo della libreria si è incentrata sull'individuazione
e definizione di requisiti e struttura, mirando ad un'implementazione
efficiente, senza però rinunciare ad un alto livello di espressività.

La seconda fase ha invece riguardato l'esplorazione delle possibili soluzioni
al problema introdotto dal \textit{GIL}, in cui sono stati testati e valutati
diversi metodi per ottenere l'esecuzione.

Una volta terminata la fase di sperimentazione dei vari framework si è passati
allo sviluppo della libreria vera e propria e con essa, le versioni sequenziale
e parallela dell'algoritmo.

La fase finale del lavoro è stata di sperimentazione ed è stata suddivisa in
tre parti: la prima, utilizzata anche in fase di sviluppo, è stata necessaria
per testare la correttezza di quanto implementato. Si sono infatti considerati
problemi ben noti in letteratura, tra cui il problema del commesso viaggiatore,
il problema dello zaino e un caso di regressione lineare. Per le ultime due
fasi è stato invece riprodotto il problema specifico del metodo \textit{LORE},
valutando qualità e prestazioni tramite un'analisi comparativa con \textit{DEAP}.

\section{Struttura e Requisiti}

I primi requisiti individuati sono quelli imposti dalla natura stessa degli
algoritmi genetici nella loro forma più comune. Primo fra tutti la struttura
del cromosoma, tipicamente vettoriale o simile ad una lista, rendendo così più
semplice l'implementazione degli operatori genetici classici, che operano bene
su strutture di questo tipo. L'utente deve poter assumere di lavorare su
cromosomi con tale forma, senza che debba essere lui a specificarla.

Il secondo requisito consiste nel permettere all'utente di definire tutte le
fasi dell'algoritmo, permettendogli quindi di implementare operatori genetici e
altre funzionalità specifiche per il suo caso d'uso, se quelle fornite
dalla libreria non siano in grado di soddisfare i suoi bisogni. Da questo
nasce anche la necessità di garantire un livello minimo di espressività nel
definire tali metodi. In particolare la funzione di valutazione è ciò che più
contraddistingue il problema di riferimento ed è infatti sempre richiesto che
sia l'utente a definirla. Non gli si può quindi precludere l'impiego di moduli
o librerie necessarie alla valutazione degli individui, come ad esempio
\textit{Numpy} e \textit{Scikit-Learn}, fondamentali alla valutazione degli
individui nel caso del metodo \textit{LORE}.

Quest'ultimo requisito ha avuto ripercussioni sulla scelta del framework per
l'implementazione della versione parallela dell'algoritmo. Come anticipato,
alcuni di questi richiedono compromessi sulla forma del codice, mentre altri,
ancora sperimentali, potrebbero non essere pienamente supportati dalle librerie
di cui l'utente potrebbe aver bisogno, rendendoli inadeguati.

In quanto a struttura della libreria si è cercato di riprodurre API e costrutti
simili a \textit{DEAP}, semplificando però alcuni costrutti e specializzandoli
in base alla tipologia dei casi d'uso trattati. Come per \textit{DEAP} viene
fornita una classe \verb|ToolBox| la quale ha il compito di registrare metodi,
operatori genetici e parametri necessari alla generazione e valutazione di
individui. Un costrutto simile rende la fase di test e l'effettivo utilizzo
finale molto comodi, in particolar modo quando sono previste esecuzioni
multiple, in cui si potrebbero dover variare parametri o metodi a seconda
dell'input o dell'effetto che si vuole ottenere. Sarebbe anche possibile avere
più istanze di \verb|ToolBox|, ognuna impostata con parametri e metodi anche
molto diversi, così da poter cambiare l'intera struttura dell'algoritmo
sostituendo semplicemente l'istanza della classe con cui si intende lavorare.

Sempre in modo simile a \textit{DEAP} è stato implementato un algoritmo simile
a \verb|eaSimple|, il quale accetta come parametri anche l'istanza della classe
\verb|ToolBox| precedentemente definita. Questo approccio permette grande
modularità, mantenendo la logica dell'algoritmo separata da metodi e parametri
con cui opera.

\section{Test dei Framework per il Parallelismo}\label{sec: frameworks}

Per riuscire ad implementare una versione parallela dell'algoritmo genetico,
il primo ostacolo da superare è quello imposto dal \textit{GIL}, il quale non
permette l'esecuzione contemporanea di più thread. Durante la prima fase dello
sviluppo della libreria si sono quindi esplorate le seguenti soluzioni per
provare a risolvere il problema:
\begin{itemize}
	\item \textit{Compilazione Python}: framework come \textit{Numba} e
	      \textit{Cython}, i quali permettono di rilasciare il \textit{GIL}
	      compilando il codice Python.
	\item \textit{Subinterpreters}: framework sperimentale che istanzia dei
	      \textit{sotto-intepreti} all'interno dello stesso processo, ognuno
	      dotato del proprio \textit{GIL}.
	\item \textit{Free-Threading}: versione sperimentale dell'interprete
	      \textit{CPython} con \textit{GIL} disabilitato.
	\item \textit{Multiprocessing}: uso di processi multipli, ciascuno con il
	      proprio interprete e quindi con il proprio \textit{GIL}.
\end{itemize}
Quasi subito però ci si è resi conto dell'inadeguatezza di alcuni di loro,
fornendo però spunto per la definizione di ulteriori vincoli che il framework
scelto avrebbe dovuto soddisfare:
\begin{enumerate}
	\item Esecuzione in parallelo di task CPU-bound.
	\item Esecuzione in parallelo di funzioni definite dall'utente.
	\item Esecuzione in parallelo di funzioni definite dall'utente che
	      impiegano routine e strutture dati di librerie standard di Python o
	      moduli di terze parti.
\end{enumerate}
Se anche uno solo di questi non venisse soddisfatto, la soluzione in esame
verrebbe scartata, in quanto non è in grado di garantire esecuzione parallela,
o di soddisfare alcuni requisiti fondamentali definiti in precedenza e validi
anche per la versione sequenziale.

Rifacendosi inoltre a quanto discusso nella sezione \ref{sec: bg_xai}, il metodo
\textit{LORE} definisce una funzione di valutazione simile a quella descritta
dall'algoritmo \ref{alg: lore_eval}, che prende come parametro il modello di
classificazione di cui si vogliono spiegare le predizioni. Dato che il metodo
vuole essere indipendente da tipologia e implementazione del modello di machine
learning, è necessario garantire la possibilità di definire una funzione di
valutazione per modelli provenienti da librerie di machine learning differenti
come \textit{Scikit-Learn}, \textit{Tensorflow} o \textit{PyTorch}, giusto per
citare alcune delle più popolari. L'impossibilità di usufruire di tali librerie
o di effettuare predizioni in parallelo decreterebbe quindi l'inadeguatezza del
framework. Il discorso in realtà si estende potenzialmente ad altre librerie
largamente usate in moltissimi progetti Python, come ad esempio \textit{Numpy}
o \textit{Pandas}.

\subsection{Compilazione e Rilascio del GIL}

Al fine di rilasciare il \textit{GIL} per riuscire a lavorare in multithreading
sono state esplorate due opzioni che prevedono la compilazione del codice
Python, rendendo quindi non più necessario l'interprete, che si occuperà solo
di invocare codice già compilato.

Sono necessarie tuttavia alcune premesse: compilare codice Python o avere a che
fare con codice compilato in altri linguaggi non rilascia automaticamente il
\textit{GIL}; è necessario infatti specificando le sezioni di codice che possono
operare senza \textit{GIL}. L'altra premessa da fare è che non è possibile
compilare codice contenente costrutti nativi di Python, per esempio funzioni
builtin, liste o dizionari. Queste strutture dati sono molto flessibili e
possono contenere elementi di qualunque; in una stessa lista infatti è possibile
inserire numeri, stringhe o qualsiasi altro tipo arbitrariamente complesso, i
quali però hanno bisogno di essere manipolati tramite l'interprete. Se invece
consideriamo un array \textit{Numpy}, si ha sempre a che fare con un oggetto
Python, ma la sua implementazione in C contiene un buffer con un layout di
memoria ben preciso, in cui sono contenuti gli elementi che lo compongono. Le
funzioni che fornisce \textit{Numpy} per le operazioni numeriche spesso
rilasciano il \textit{GIL} poiché la loro implementazione in C agisce
direttamente sul buffer di memoria e non su una qualche struttura dati Python,
rendendo quindi l'interprete superfluo.

\subsection*{Cython}

L'approccio che propone \textit{Cython} ha invece lo scopo di generare un modulo
compilato prima dell'esecuzione, sfruttando dei decoratori oppure scrivendo il
codice in linguaggio \textit{Cython}, ossia un'estensione di Python che supporta
nativamente la tipizzazione statica, incentivando uno stile di programmazione
simile al C.

\begin{lstlisting}[caption={Esempio utilizzo Cython}]
cpdef float task(float[:] a) nogil:
	cdef float s = 0
	cdef int i = 0
	with nogil:
		for i in range(len(a)):
			s += a[i]

	return s
\end{lstlisting}

Si può subito notare un codice decisamente più verboso rispetto a prima;
è inoltre necessario notare la \textit{keyword} \verb|nogil| che compare sia
nella \textit{signature} che nel corpo della funzione. La prima non comporta il
rilascio del \textit{GIL} ma segnala al compilatore che l'intera funzione deve
rispettare i parametri per il rilascio; in caso contrario la compilazione
fallisce. Nel secondo caso invece, il codice all'interno del blocco \verb|with|
è effettivamente compilato con \textit{GIL} rilasciato. La \textit{keyword}
sulla signature non è obbligatoria ma se la compilazione ha successo è indice
del fatto che sia possibile inserire il blocco \verb|with nogil| in qualsiasi
punto della funzione.

Prima di poter sfruttare il codice prodotto è necessario compilarlo tramite
uno script simile al seguente:

\begin{lstlisting}[caption={Script di compilazione Cython}]
from Cython.Build import cythonize
from setuptools import setup
setup(ext_modules=cythonize("module.pyx"))
\end{lstlisting}

\subsection*{Numba}

Come prima opzione è stato valutato il framework \textit{Numba}, il quale opera
molto bene in sinergia con \textit{Numpy} e il cui utilizzo più comune prevede
l'impiego di decoratori sulle funzioni che si intende compilare.

\begin{lstlisting}[caption={Esempio di utilizzo Numba}]
from numba import njit
import numpy as np

@njit(nogil=True)
def task(a: np.ndarray):
	return np.sum(a)
\end{lstlisting}

Così facendo si abilita una compilazione di tipo \textit{Just-In-Time} che
avviene a tempo d'esecuzione la prima volta che si invoca la funzione. Eventuali
chiamate future verranno reindirizzate al codice binario prodotto in precedenza.

La libreria mette a disposizione due decoratori, \verb|jit| e \verb|njit|, in
grado di compilare la funzione. Il secondo però agisce in quella che viene
definita modalità \quotes{nopython}, necessaria per il rilascio del \textit{GIL}
e che fa in modo che la compilazione fallisca se nella funzione sono presenti
costrutti Python nativi, non compilabili per le ragioni elencate in precedenza.


\subsection*{Considerazioni}

Entrambi i framework danno la possibilità di compilare codice Python nativo,
effettuando alcune ottimizzazioni e cercando di inferire il tipo delle
variabili quando possibile. Per riuscire però a generare codice capace di
rilasciare il \textit{GIL} è necessario adottare uno stile di programmazione
più rigido, definendo il tipo di ciascun oggetto coinvolto, senza utilizzare
strutture dati Python come liste o dizionari per i motivi precedentemente
evidenziati. Il primo lo fa incentivando l'uso di \textit{Numpy}, l'altro
fornendo invece un'estensione del linguaggio simile al C.

In generale si può dire che entrambi i framework soddisfino il primo requisito,
in quanto danno la possibilità di produrre codice parallelizzabile, a patto che
si adotti uno stile di programmazione più rigido e orientato a una tipizzazione
statica delle variabili.

Assumendo che l'utente sia in grado di produrre codice seguendo le linee guida
necessarie per la compilazione e successivo rilascio del \textit{GIL}, sono
entrambe opzioni che soddisfano anche il secondo requisito, poiché l'utente può
definire funzioni in Python (o \textit{Cython}) eseguibili in parallelo.

Entrambi i framework presentano però problemi quando si prova ad implementare
la funzione di valutazione specifica per il caso d'uso di \textit{LORE},
descritto in precedenza. L'introduzione del modello usato per la classificazione
dei dati comporta l'impossibilità di compilare tale funzione rilasciando il
\textit{GIL}. Questo perché i modelli di \textit{Scikit-Learn} sono oggetti
Python su cui non è possibile rilasciare il \textit{GIL} in fase di predizione e
dunque incompatibili per essere compilati con \textit{Numba} o \textit{Cython}.

Nel caso di \textit{Cython} sarebbe possibile escludere la fase di predizione
del modello dalla parte con \textit{GIL} rilasciato, comportando però
l'acquisizione di quest'ultimo e l'attesa degli altri thread.

\subsection{Framework Sperimentali}

Un secondo possibile approccio è stato quello di considerare due funzionalità
sperimentali, che mirano a risolvere il problema del \textit{GIL} più in
profondità e senza tutti i compromessi che si dovrebbero fare con i framework
precedentemente descritti. La prima sfrutta i \textit{Subinterpreters}, mentre
l'altra fa uso di una versione sperimentale dell'interprete Python ma con il
\textit{GIL} sempre rilasciato.

\subsection*{Subinterpreters}

I \textit{Subinterpreters} si basano sull'idea di istanziare più
\textit{sotto-interpreti}, ciascuno dotato del proprio \textit{GIL} e
coordinabili tramite thread multipli. Tuttavia l'API attualmente disponibile
non supporta una metodo per assegnare i task ai \textit{subinterpreters} in
modo simile al multithreading, che sfrutta puntatori a funzione. È infatti
necessario sottomettere il codice da eseguire sotto forma di stringa come
riportato nell'esempio seguente.

\begin{lstlisting}[caption={Subinterpreters API Python 3.12}]
from test.support import interpreters

def handle():
	interp = intepreters.create()
	interp.run("print('Hello World')")
	interp.close()

worker = threading.Thread(target=handle)
worker.start()
worker.join()
\end{lstlisting}

Questo framework è presente da diversi anni ma disponibile solo per lo sviluppo
di estensioni tramite l'API in C. Di recente è stata sviluppata una prima API
utilizzabile direttamente da Python, ma che ancora fornisce poche funzionalità
e che molte librerie non supportano. Si tratta inoltre di una soluzione in
continuo sviluppo, le cui funzionalità variano da una versione Python all'altra
e di cui si è analizzata solo la versione presente nei moduli standard di
Python 3.12.

Il framework presenta tuttavia alcune criticità in quanto gli interpreti,
sebbene siano tutti all'interno dello stesso spazio di memoria, sono isolati
gli uni dagli altri. Non è infatti possibile accedere a strutture dati in
memoria condivisa o usare meccanismi di comunicazione che non prevedano fasi
di serializzazione e stream dati come nel caso del multiprocessing quando si
cerca di condividere oggetti arbitrariamente complessi. Per risolvere il
problema, sembra che gli sviluppatori sia intenti a sviluppare un sistema ad
alte prestazioni basato su code, in grado di sfruttare la memoria condivisa. In
fase di sviluppo questa funzionalità non era però disponibile e non si è stati
in grado di testarla. % controllare

Il problema principale è tuttavia il mancato supporto delle librerie necessarie
come \textit{Numpy} e \textit{Scikit-Learn}, che provocano errori e fallimenti
anche solo dopo essere state importarte da un \textit{subinterpreter}. Risulta
quindi chiaro che si tratti di una funzionalità non compatibile né per il caso
d'uso d'interesse né per tanti altri.

\subsection*{Free-Threading}

Dalla versione 3.13 è possibile utilizzare una versione dell'interprete Python
che rimuove il \textit{GIL}, risolvendo il problema alla radice anche per
librerie che non lo rilasciano esplicitamente.

Tramite questa soluzione diventa possibile sviluppare codice in Python nativo
senza dover adottare alcuno stile di programmazione particolare come con
\textit{Numba} o \textit{Cython} ma non è garantito il corretto funzionamento
dei moduli coinvolti.

L'implementazione degli algoritmi è esattamente la stessa che si avrebbe per la
versione con il \textit{GIL} abilitato, con la differenza che in questo modo si
ottiene vero parallelismo anche per compiti CPU-bound.

\begin{lstlisting}[caption={Esempio multithreading}]
from threading import Thread
import numpy as np
import queue

def task(q: queue.Queue):
	a = q.get()
	q.put(np.sum(a))

queues = [queue.Queue() for _ in range(8)]
workers = [Thread(target=task, args=(q,)) for q in queues]
for w, q in zip(workers, queues):
	w.start()
	q.put(np.random.random(size=(512,)))

for w, q in zip(workers, queues):
	print(q.get())
	w.join()
\end{lstlisting}


Al momento dei test che hanno decretato la scelta del framework alcune delle
librerie come \textit{Scipy} (dipendenza di \textit{Scikit-Learn}) non
fornivano supporto per tale versione e non è stato possibile installarle e
testarle, portando ad escludere subito l'opzione. Ad oggi è invece presente un
supporto dichiarato dagli sviluppatori delle librerie stesse ed è quindi
diventato possibile eseguire codice contenente modelli di \textit{Scikit-Learn}
in multithreading.

Si tratta tuttavia di una funzionalità ancora sperimentale e che viene in
generale sconsigliata, se non per effettuare test; senza considerare che alcune
delle librerie come \textit{Tensorflow} e \textit{PyTorch} non forniscono ancora
una versione per Python 3.13 (con o senza \textit{GIL}).

\subsection{Multiprocessing}

L'ultima opzione consiste nell'impiego del modulo \verb|multiprocessing|. In
questo modo è possibile sfruttare più processi, ciascuno dotato del proprio
interprete e quindi del proprio \textit{GIL}, permettendo così di eseguire
codice effettivamente in parallelo.

\begin{lstlisting}[caption={Esempio multiprocessing}]
import multiprocessing as mp
import numpy as np

def task(q: mp.Queue):
	a = q.get()
	q.put(np.sum(a))

queues = [mp.Queue() for _ in range(8)]
processes = [mp.Process(target=task, args=(q,)) for q in queues]
for p, q in zip(processes, queues):
	p.start()
	q.put(np.random.random(size=(512,)))

for p, q in zip(processes, queues):
	print(q.get())
	p.join()
\end{lstlisting}

Questa soluzione presenta diversi vantaggi, offre infatti una API praticamente
identica al modulo \verb|threading|, rendendo l'implementazione semplice se si
ha già esperienza in ambito multithreading.

Un altro vantaggio, come possiamo vedere anche dall'esempio, è la compatibilità
con codice Python nativo e con moduli di terze parti, evitando quindi di
influenzare o limitare scelte architetturali o di design, sia della libreria che
delle possibili implementazioni fornite dall'utente.

Si tratta quindi di una soluzione semplice e flessibile, in grado di adattarsi
bene ad eventuali sviluppi futuri; impiega però processi multipli, portando con
sé le problematiche discusse nella sezione \ref{ssec: parallel_mp}.

Tra tutte è l'unica opzione che soddisfa pienamente tutti i requisiti, senza
alcun tipo di compromesso in termini di espressività e che permette l'esecuzione
in parallelo di qualsiasi tipo di codice, comprendendo tutte le librerie Python
standard e di terze parti. La scelta finale è quindi ricaduta su questa opzione,
che sicuramente garantisce flessibilità ma pecca in efficienza per diversi
aspetti. Riesce comunque ad apportare miglioramenti significativi, soprattutto
su carichi di lavoro grandi, i quali rendono trascurabile il tempo necessario a
condividere i dati tra i worker.
\section{Implementazione}

L'implementazione della libreria cerca di riprodurre quella proposta da
\textit{DEAP}, fornendo alcuni dei suoi costrutti fondamentali come una classe
\verb|ToolBox| e un modulo \verb|algorithms| contenente un algoritmo simile a
\verb|eaSimple|.

\subsection{Inizializzazione e Strutture Dati}

A differenza di \textit{DEAP} non è presente un modulo \verb|creator| per
definire il tipo di strutture dati che vanno a formare cromosomi e popolazione.
I cromosomi sono infatti forzati ad essere array \textit{Numpy}, mentre la
popolazione è contenuta dentro una lista Python.

L'inizializzazione avviene interamente tramite la classe \verb|ToolBox|, con
cui è possibile definire la tupla \verb|weights| dei pesi da attribuire a
ciascuno dei parametri che si intende ottimizzare. La tupla contiene numeri
reali il cui segno indica l'intenzione di minimizzare un parametro (segno
negativo) o di massimizzarlo (segno positivo).

È inoltre necessario definire la funzione di generazione, in cui si specifica
come generare un generico cromosoma. Ogni individuo è un'istanza della classe
\verb|Individual|, formata da tre componenti principali:
\begin{itemize}
	\item \verb|chromosome|: il cromosoma che identifica l'individuo e codifica
	      la soluzione al problema di riferimento. Si tratta di un array
	      \textit{Numpy} i cui elementi possono essere oggetti arbitrariamente
	      complessi.
	\item \verb|values|: una tupla contenente i valori grezzi dei parametri che
	      si intende ottimizzare. Corrispondono al valore di ritorno della
	      funzione di valutazione definita dall'utente.
	\item \verb|fitness|: un valore reale che definisce il valore di fitness di
	      un individuo. Viene calcolata tramite una somma pesata dei valori
	      contenuti nella tupla \verb|values|, moltiplicati per i pesi definiti
	      nella tupla \verb|weights| della classe \verb|ToolBox|.
\end{itemize}
Come è possibile dedurre dalla definizione del valore di fitness, le tuple
\verb|weights| e \verb|values| devono contenere lo stesso numero di elementi.

Una volta definiti i pesi dei parametri e la funzione di generazione dei
cromosomi, è necessario definire quali metodi e operatori utilizzare.

\begin{lstlisting}[caption={Utilizzo classe ToolBox PPGA}]
from ppga import base, tools

toolbox = base.ToolBox()
toolbox.set_weights(weights=(2.0, -1.0))
toolbox.set_generation(tools.gen_repetition, [0, 1], length=10)
toolbox.set_selection(tools.sel_tournament, tournsize=3)
toolbox.set_crossover(tools.cx_uniform)
toolbox.set_mutation(tools.mut_bitflip)
toolbox.set_evaluation(evaluate)
\end{lstlisting}

In modo simile a \textit{DEAP}, ognuno di essi verrà registrato in un'istanza
della classe \verb|ToolBox|, attraverso uno specifico metodo per ognuna delle
fasi possibili. In particolare sarà necessario definire il metodo di selezione,
gli operatori genetici di crossover e mutazione e la funzione di valutazione e,
per ognuno di essi, eventuali argomenti necessari al loro corretto funzionamento.

\subsection{Algoritmo}

Completati tutti i passi descritti nel paragrafo precedente è possibile
utilizzare la \verb|ToolBox| e i metodi registrati al suo interno per dare
forma ad un algoritmo genetico personalizzato. Nel modulo \verb|algorithms| è
tuttavia presente l'implementazione di un algoritmo genetico generazionale
simile a quella offerta da \textit{DEAP}.

\begin{lstlisting}[caption={Utilizzo algoritmo PPGA}]
from ppga import tools, algorithms

hof = tools.HallOfFame(50)
population, stats = algorithms.simple(
	toolbox=toolbox,
	population_size=100,
	keep=0.1, cxpb=0.7, mutpb=0.3,
	max_generations=50,
	hall_of_fame=hof,
)
\end{lstlisting}

L'algoritmo riproduce lo schema di algoritmo genetico riportato in figura
\ref{fig:simple_ga} introducendo qualche modifica rispetto all'implementazione
fornita da \textit{DEAP}. Quest'ultima genera infatti la popolazione iniziale e
la valuta subito dopo, così da entrare nel ciclo con individui già valutati e
permettendo una fase di selezione più significativa fin dalla prima iterazione.

L'implementazione di \textit{PPGA} non valuta subito gli individui, rendendo la
prima selezione del tutto casuale, poiché tutti gli individui avranno un valore
di fitness pari a zero. Dai test condotti su diversi problemi questo non sembra
però essere rilevante in quanto \textit{PPGA} riesce a produrre soluzioni di
qualità simile o superiore a \textit{DEAP}.

L'altra differenza con \textit{DEAP} è data dall'introduzione di un parametro
\verb|keep|, il quale introduce un possibile comportamento \textit{elitista}.
Questo può avere valori compresi tra 0 e 1 (inclusi) e indica la percentuale
della vecchia generazione che si intende provare a far sopravvivere fino alla
generazione successiva. Se il parametro ha valore 0 il comportamento della fase
di rimpiazzo è uguale a quello di \verb|eaSimple|, mentre per valori maggiori
di 0 viene prelevata la percentuale di popolazione indicata dal parametro dei
migliori individui, per poi aggiungerli alla nuova generazione. Questo permette
in alcuni casi una convergenza più rapida dell'intera popolazione, ma un valore
troppo alto potrebbe portare ad una convergenza prematura e ad un calo troppo
brusco della diversità genetica complessiva, con la possibilità di bloccarsi in
un minimo (o massimo) locale.

\subsection{Versione Parallela}

La versione parallela dell'algoritmo si basa invece sulla struttura di algoritmo
genetico parallelo riportato in figura \ref{fig: parallel_ga} e, a differenza
di \textit{DEAP}, che esegue in parallelo solo la fase di valutazione, questo
parallelizza anche le fasi di crossover e mutazione.

\begin{lstlisting}[caption={Utilizzo algoritmo parallelo PPGA}]
from ppga import tools, algorithms

hof = tools.HallOfFame(50)
population, stats = algorithms.simple(
	toolbox=toolbox,
	population_size=100,
	keep=0.1, cxpb=0.7, mutpb=0.3,
	max_generations=50,
	hall_of_fame=hof,
	workers_num=-1, # uses all physical cores
)
\end{lstlisting}

Per riuscire a sfruttare il parallelismo è sufficiente specificare il valore
del parametro \verb|workers_num|, impostato di norma a 0, che come suggerisce
il nome, indica il numero di \textit{worker} che si desidera impiegare nelle
fasi parallelizzate. I valori 0 e 1 producono un'esecuzione sequenziale, mentre
il valore -1 permette l'impiego di un numero di worker pari al numero di core
fisici della macchina su cui viene eseguito l'algoritmo. Valori maggiori di 1
permettono di definire il numero esatto di worker che si intende impiegare, ma
non sarà comunque possibile superare il numero di core fisici della macchina su
cui si esegue il calcolo.

Per implementare il modello parallelo sono state utilizzate unicamente le classi
\verb|Process| e \verb|Queue| fornite dal modulo \verb|multiprocessing|, con le
quali è stata implementata una classe \verb|Pool|, specializzata per il contesto
d'interesse.

Quando si decide di operare in parallelo, ogni \textit{worker} viene creato e
rimane attivo fino a quando non viene invocato il metodo \verb|join|, con il
quale si attende il termine di eventuali task rimasti in esecuzione, con la
conseguente distruzione dei processi. Fino ad allora questi rimangono in attesa
di nuovi task, i quali vengono sottomessi tramite la funzione \verb|map| dal
processo principale. Tale funzione accetta una struttura dati iterabile, la
funzione da applicare a ciascun elemento ed eventuali parametri aggiuntivi
necessari alla funzione passata come argomento.

\subsection{Distribuzione del Carico}

Nel caso d'interesse è richiesta l'implementazione di un paradigma di tipo
\textit{map}, che prevede l'applicazione della funzione a tutti gli elementi di
una lista, che in questo caso contiene la popolazione di individui. A
prescindere quindi dall'impiego o meno di un modello di calcolo parallelo, la
quantità di individui elaborata dall'insieme delle fasi di crossover, mutazione
e valutazione è fissata. Dai test risulta anche che il tempo di elaborazione di
un singolo individuo sia poco variabile, a tal punto da ritenere non necessario
un bilanciamento del carico dinamico.

Si è quindi optato per una distribuzione del carico statica e omogenea tra i
\textit{worker}, suddividendo la struttura dati in un numero di \textit{chunk}
pari al numero di \textit{worker} impiegati.

\begin{figure}[H]
	\centering
	\includesvg[width=0.95\linewidth]{immagini/workload.svg}
	\caption{Distribuzione del carico su 4 worker}
	\label{fig: workload}
\end{figure}

La dimensione dei blocchi è calcolata come il rapporto tra la lunghezza della
struttura dati totale e il numero di worker impiegati. Qualora il risultato non
fosse intero, il resto della divisione verrebbe distribuito omogeneamente sui
vari worker, producendo quindi alcuni blocchi più grandi di un'unità rispetto
ad altri.

\subsection{Comunicazione}

Per la comunicazione tra processi \textit{worker} e processo principale sono
state esplorate diverse opzioni per riuscire ad avere un overhead più contenuto
possibile. Oltre a questo si è cercato di garantire la possibilità di operare
con tipologie di dato arbitrariamente complesse. Si ricorda infatti che le
strutture dati iterabili inviate sono liste di istanze della classe
\verb|Individual|, le quali contengono cromosomi i cui elementi possono essere
a loro volta arbitrariamente complessi.

Il metodo di comunicazione che costituisce l'attuale implementazione prevede
l'impiego di code con politica \textit{FIFO} fornite dal modulo
\verb|multiprocessing|, le quali offrono diversi vantaggi e permettono di
implementare in modo semplice un paradigma \textit{produttore-consumatore} per
l'invio e la ricezione di dati. L'attuale implementazione sfrutta due code per
ogni \textit{worker}, una per l'invio e una per la ricezione di messaggi.

\begin{figure}[H]
	\centering
	\includesvg{immagini/queue.svg}
	\caption{Sistema di comunicazione con due code}
	\label{fig: double_queue}
\end{figure}

La scelta di una doppia coda è stata fatta per evitare eventuali conflitti
che si potrebbero presentare dall'impiego di una sola coda; una singola coda
implica infatti che questa sia una via di comunicazione bidirezionale tra i
processi e dato che sia il processo worker che il processo principale si
alternano i ruoli di produttore e consumatore, si potrebbero verificare
situazioni imprevedibili. In sintesi, affinché tutto proceda senza errori è
necessario che uno dei due processi, dopo aver inviato un messaggio, non ne
invii altri prima di aver ricevuto la risposta.

Un caso d'errore tipico si verifica quando, dopo aver terminato il task
assegnatogli, il worker inserisce i risultati sulla coda su cui si mette in
attesa subito dopo.

\begin{figure}[H]
	\centering
	\includesvg[width=0.95\linewidth]{immagini/deadlock.svg}
	\caption{Possibile situazione di deadlock con coda singola}
	\label{fig: deadlock}
\end{figure}

Come è possibile vedere nella parte in basso a destra della figura, se il
processo principale non estrae il contenuto della coda prima che il worker si
rimetta in attesa, sarà quest'ultimo ad estrarre i risultati. Questo li tratterà
come se fossero il un nuovo task in input, provocando un errore nel flusso
d'esecuzione del worker e deadlock nel processo principale che rimane in attesa
di un messaggio che è già stato consumato.

La soluzione più semplice e probabilmente una delle più efficienti è proprio
quella delle due code. Avere due code, ciascuna con una \quotes{direzione}
diversa, evita il problema appena descritto, non richiede alcun tipo di
sincronizzazione particolare tra mittente e ricevitore e permette di non creare
conflitti qualora si intenda sottomottere più task, senza attendere che i
precedenti siano terminati.

Una volta terminato il lavoro è possibile fermare l'esecuzione del \textit{pool}
di processi tramite il metodo \verb|join|. Questo invia un \verb|None| a tutti i
worker, che lo riconoscono come segnale di terminazione ed escono dal ciclo di
attesa, terminando la loro esecuzione.

Tramite questa architettura di comunicazione e la politica di distribuzione del
carico descritta in precedenza, si riesce a limitare al minimo il numero di
invii e ricezioni, che producono un overhead non sempre trascurabile,
soprattutto per carichi di lavoro medi o più piccoli. Questo di norma non
costituisce un problema in quanto i task più leggeri sono anche quelli che meno
necessitano un'ottimizzazione pesante. In un caso come quello di \textit{LORE}
però l'algoritmo genetico potrebbe essere eseguito centinaia o migliaia di
volte, aumentando significativamente il tempo d'esecuzione totale.

% \subsection*{Invio e ricezione asincroni}

% Per riuscire quindi a sfruttare al meglio il parallelismo sono state esplorate
% alcune varianti del modello per riuscire mitigare l'overhead di comunicazione.
% L'idea per riuscire a ridurre il tempo di comunicazione consiste nel dividere
% i \textit{chunk} di struttura dati da inviare a ciascun \textit{worker}, in
% ulteriori \textit{sub-chunk}. A questo punto sarebbe possibile inserire i
% \textit{sub-chunk} nella coda in modo separato. Affinché questo metodo funzioni
% è necessario che il \textit{worker} estragga i \textit{sub-chunk} nella coda
% mentre elabora quelli inviati in precedenza. Per riuscire ad implementare un
% meccanismo del genere è necessario che ogni worker utilizzi un thread adibito
% all'estrazione di elementi dalla coda, e quindi alla deserializzazione degli
% stessi, fornendoli poi al processo \textit{worker}, per esempio tramite una
% coda \textit{locale} che non necessita serializzazione, deserializzazione o
% stream di dati.

% \begin{figure}[H]
% 	\centering
% 	\includesvg[scale=0.6]{immagini/recv_pipeline.svg}
% 	\caption{Pipeline di comunicazione}
% 	\label{fig: queue_pipeline}
% \end{figure}

% In questo modo si cerca di sfruttare il fatto che il multithreading in Python
% riesce ad eseguire istruzioni di tipo I/O (come lo stream di byte tra processi)
% in parallelo, per esempio sfruttando tecnologie come il DMA. Il problema di
% questa implementazione è che le fasi di serializzazione e deserializzazione
% sono significativamente più dispendiose dello stream dati ed inoltre sono
% operazioni di tipo CPU-bound. Ecco che tentare di ottimizzare la comunicazione
% in questo modo non garantisce alcun tipo di vantaggio e anzi, potrebbe
% introdurre un ulteriore overhead dovuto allo scheduling e alla sincronizzazione
% dei thread.

\subsection*{Memoria condivisa}

Una terza possibilità prevede l'utilizzo del modulo \verb|shared_memory|, che
permette di allocare blocchi di memoria condivisa tra i worker ma che non
fanno parte dello spazio di memoria di nessuno di essi.

Questa soluzione offre vantaggi considerevoli per quanto riguarda le performance
ma porta con sé alcune problematiche da tenere di conto. La prima è che la
dimensione del blocco di memoria è fissa, rendendo quindi necessaria una
riallocazione completa, con conseguente copia dei dati contenuti, qualora lo
spazio si esaurisse.

La seconda è relativa ai tipi di dato che possono essere inseriti all'interno
del blocco. È infatti possibile inserire array di byte, tipi primitivi come
\verb|int| e \verb|float| o array numerici. Non è possibile inserire oggetti
complessi a meno che questi non siano prima serializzati, comportando quindi
una necessaria deserializzazione ogni volta che si intende leggerli, tornando
al problema di performance evidenziato nelle implementazioni tramite code.

In definitiva, questa rimane una valida opzione per ottenere prestazioni elevate
quando si tratta di condividere dati, ma è necessario fare diversi compromessi
sulla complessità dei cromosomi, sulla dimensione della popolazione e sul tipo
di architettura della libreria.

Si è quindi preferito avere un'implementazione più semplice e che permettesse
maggiore flessibilità nella definizione di tipi e strutture dati.

\chapter{Sperimentazione}\label{cap: experiments}

In fase di sperimentazione si è prima testata la correttezza di metodi e
algoritmi offerti dalla libreria, confrontando i risultati prodotti con
\textit{DEAP} e con algoritmi specializzati nella risoluzione di un problema
specifico. Una volta testata la correttezza si è passati a test approfonditi
sulla qualità delle soluzioni prodotte per il problema di generazione dati,
descritto nel caso d'uso di \textit{LORE}. Si sono infine valutate le
prestazioni della libreria utilizzando le versioni sequenziale e parallela
dell'algoritmo. Anche per queste due ultime fasi si è implementato un algoritmo
per la risoluzione del problema con \textit{DEAP} per effettuare un confronto
tra i risultati prodotti.

\section{Test di Correttezza}

In fase di test si sono valutate correttezza, qualità e prestazioni della
libreria tramite il confronto con \textit{DEAP} nella risoluzione di problemi
ben noti come quello del commesso viaggiatore, dello zaino e alcuni casi di
regressione lineare. Per questi tre problemi si è prima trovata una soluzione
tramite un algoritmo specializzato e in seguito la si è confrontata con i
risultati prodotti da \textit{DEAP} e \textit{PPGA}.

\subsection{Problema del Commesso Viaggiatore}

Tra i problemi presi in esame abbiamo una versione del commesso viaggiatore
implementata tramite un grafo geometrico completo. I nodi rappresentano le
città ed ognuno di essi è connesso a tutti gli altri tramite un arco
bidirezionale, la cui lunghezza equivale alla distanza euclidea tra i nodi
stessi. Non sono stati inoltre selezionati i punti di origine e destinazione;
l'unico obiettivo è trovare il cammino di costo minimo e che attraversi tutte
le città una sola volta, senza tornare al punto di partenza.

\begin{figure}[H]
	\centering
	\includesvg[width=0.75\linewidth]{immagini/tsp.svg}
	\caption{Possibile soluzione commesso viaggiatore con 50 città}
	\label{fig: deap_tsp}
\end{figure}

L'algoritmo specializzato è quello di Christofides, fornito dalla libreria
\textit{NetworkX}, la quale è specializzata per lavorare su grafi. Le versioni
\textit{DEAP} e \textit{PPGA} codificano invece una possibile soluzione in una
sequenza di indici, rappresentante l'ordine con cui visitare le città.

\begin{figure}[H]
	\centering
	\begin{minipage}{0.45\linewidth}
		\centering
		\begin{tabular}{rrrr}
			\toprule
			Città & NX    & PPGA   & DEAP   \\
			\midrule
			25    & 3.78  & 3.39   & 4.03   \\
			50    & 5.72  & 5.51   & 9.34   \\
			100   & 8.53  & 13.57  & 26.34  \\
			200   & 11.98 & 44.52  & 67.69  \\
			400   & 16.16 & 109.12 & 147.49 \\
			\bottomrule
		\end{tabular}
	\end{minipage}
	\hfill
	\begin{minipage}{0.5\linewidth}
		\centering
		\includesvg[width=\linewidth]{immagini/tsp_trend.svg}
	\end{minipage}
	\caption{Risultati commesso viaggiatore}
	\label{fig: tsp}
\end{figure}

Come possiamo vedere dalla tabella, i risultati di \textit{PPGA} spesso superano
\textit{DEAP} e si avvicinano alla soluzione trovata da \textit{NetworkX}. Il
grafico riporta gli stessi valori in tabella mostrando chiaramente come le
soluzioni trovate da \textit{PPGA} siano sempre migliori di \textit{DEAP}, anche
se sembra che entrambe seguano lo stesso andamento peggiorativo con il crescere
della dimensione dell'input.

\subsection{Problema dello Zaino}

Un altro caso interessante per capire come la libreria si comporti su problemi
multi-obbiettivo è stato il problema dello zaino. Per questo problema è stata
implementata un'euristica di tipo \textit{greedy} che ordina gli oggetti secondo
il rapporto tra il loro valore e il loro peso in modo decrescente. Le versioni
genetiche sono state invece implementate trattando valore e peso come due
parametri da ottimizzare; in particolare l'obiettivo è stato quello di
massimizzare il valore totale degli oggetti, cercando allo stesso tempo di
minimizzare il peso complessivo. Ognuno dei cromosomi è strutturato come una
lista di booleani, i quali indicano se un oggetto è stato preso o meno.

\begin{figure}[H]
	\centering
	\begin{minipage}{0.45\linewidth}
		\centering
		\begin{tabular}{rrrr}
			\toprule
			Oggetti & Greedy & PPGA   & DEAP   \\
			\midrule
			25      & 13.25  & 13.25  & 13.26  \\
			50      & 22.22  & 22.50  & 22.50  \\
			100     & 37.63  & 37.66  & 37.66  \\
			200     & 82.39  & 82.38  & 82.39  \\
			400     & 155.47 & 155.44 & 155.37 \\
			\bottomrule
		\end{tabular}
	\end{minipage}
	\hfill
	\begin{minipage}{0.5\linewidth}
		\centering
		\includesvg[width=\linewidth]{immagini/knapsack_trend.svg}
	\end{minipage}
	\caption{Risultati problema dello zaino}
	\label{fig: knapsack}
\end{figure}

Anche in questo caso \textit{PPGA} riesce ad eguagliare \textit{DEAP}, ottenendo
risultati praticamente identici. La libreria non dispone tuttavia di costrutti
come il fronte di Pareto o routine ottimizzate per lavorare a problemi di questo
tipo. Pone comunque le basi per futuri sviluppi e implementazioni in questa
direzione.

\subsection{Regressione Lineare}

Come ultimo problema è stato trattato un caso di regressione lineare semplice
di modo da testare il comportamento di alcuni operatori genetici su cromosomi
a valori reali. La retta di regressione di riferimento è stata prima calcolata
tramite il modulo \verb|statistics| offerto da Python e in seguito tramite gli
algoritmi genetici. In questo caso un cromosoma è formato da due soli elementi,
ossia coefficiente angolare e intercetta della retta, mentre la funzione di
valutazione calcola l'errore quadratico medio.


\begin{figure}[H] % mettere figura con DEAP e PPGA
	\centering
	\includesvg[width=0.75\linewidth]{immagini/regresssion.svg}
	\caption{Regressione lineare genetica}
	\label{fig: reg}
\end{figure}

In figura \ref{fig: reg} in verde la retta di regressione calcolata tramite
l'algoritmo offerto da \textit{Numpy} tramite la funzione \verb|polyfit|,
mentre in rosso e blu le rette calcolate rispettivamente con \textit{DEAP} e
\textit{PPGA}. I test sono stati infine ripetuti con un numero crescente di
sample.

\begin{figure}[H]
	\centering
	\begin{minipage}{0.45\linewidth}
		\centering
		\begin{tabular}{rrrr}
			\toprule
			Punti & Numpy & DEAP & PPGA \\
			\midrule
			50    & 3.01  & 3.01 & 3.01 \\
			100   & 3.15  & 3.15 & 3.15 \\
			200   & 3.55  & 3.55 & 3.55 \\
			400   & 3.54  & 3.54 & 3.54 \\
			800   & 3.55  & 3.55 & 3.55 \\
			\bottomrule
		\end{tabular}
	\end{minipage}
	\hfill
	\begin{minipage}{0.5\linewidth}
		\centering
		\includesvg[width=\linewidth]{immagini/regression_trend.svg}
	\end{minipage}
	\caption{Risultati regressione lineare}
	\label{fig: regression}
\end{figure}

In modo simile al problema dello zaino, i risultati sono molto simili per tutti
e tre gli algoritmi, anche all'aumentare del numero di campioni.

\subsection{Generazione Dati per Explainable AI}

Una volta terminata la fase di test sui problemi appena descritti si è passati
alla generazione di un vicinato sintetico in modo simile a quanto fa il metodo
\textit{LORE} per produrre spiegazioni. L'obiettivo è minimizzare la distanza
tra i punti generati e un punto di riferimento specifico, facendo però in modo
che questi siano classificati in una classe target specificata.

Al fine di rendere più semplice l'implementazione dell'algoritmo e la verifica
delle soluzioni si è lavorato, almeno inizialmente, su istanze di dato composte
da sole 2 feature a valori reali, così che fossero facilmente rappresentabili
graficamente.

\begin{figure}[H]
	\centering
	\includesvg[width=0.8\linewidth]{immagini/points.svg}
	\caption{Dataset generazione dati}
	\label{fig: xai_test}
\end{figure}

Come è possibile notare, in figura \ref{fig: xai_test} sono presenti due classi
di punti, in questo caso identificate rispettivamente dai colori blu e rosso.
Per ciascuno di essi sono necessarie due esecuzioni dell'algoritmo genetico per
generare due tipi di dato sintetico.

Le figure di seguito mostrano il risultato delle due esecuzioni dell'algoritmo
genetico: contrassegnato da una croce, il dato di riferimento, mentre in giallo
i punti generati dalle due esecuzioni dell'algoritmo genetico, rispettivamente
per la generazione dell'insieme $Z_=$ (a sinistra) e $Z_{\neq}$ (a destra).

\begin{figure}[H]
	\centering
	\includesvg[width=0.925\linewidth]{immagini/synth_points.svg}
	\caption{Generazione dati sintetici}
	\label{fig: synth_points}
\end{figure}

Come è possibile notare, nella prima immagine i punti si dispongono attorno al
punto di riferimento senza però sovrapporsi totalmente. Nel secondo caso invece
si nota come i punti generati si dispongano sul confine di classificazione che
il modello ha individuato in fase di allenamento.

Per ottenere questo risultato, la funzione di valutazione è stata definita in
modo tale che ogni punto classificato diversamente dalla classe target fosse
scartato. La funzione di valutazione inoltre fa in modo che anche i punti uguali
all'istanza di riferimento vengano scartati, evitando così la completa
sovrapposizione con l'istanza di riferimento.

\begin{lstlisting}[caption={Funzione di valutazione}]
def evaluate(chromosome, point, target, model):
	synth_point_class = model.predict(chromosome.reshape(1, -1))
	if synth_point_class != target:
		return (np.inf,)

	distance = np.linalg.norm(chromosome - point, ord=2)
	distance = distance if distance > 0.0 else np.inf

	return (distance,)
\end{lstlisting}

Per assicurarsi che i due gruppi di dati sintetici fossero classificati
correttamente si è calcolato il rapporto tra il numero di punti classificati
nella classe target e il numero totale di punti. Affinché la generazione sia
considerata corretta, tale rapporto deve essere sempre pari a 1.

\section{Test Qualitativi}

Per il problema di generazione dati si sono svolti test qualitativi approfonditi
con l'obiettivo di determinare quanto i risultati delle due librerie fossero
simili su una larga varietà di input. A tal fine sono stati generati diversi
dataset tramite \textit{Scikit-Learn}, che mette a disposizione la funzione
\verb|make_classification|, in grado di produrre dataset con caratteristiche
specifiche.

\begin{lstlisting}[caption={Generazione dataset con \lstinline|make_classification|}]
from sklearn.datasets import make_classification

X, y = make_classification(
	n_samples=1000,
	n_features=32,
	n_informative=32,
	n_redundant=0,
	n_repeated=0,
	n_classes=2,
	n_clusters_per_class=1,
	random_state=0,
)
\end{lstlisting}

Il codice appena presentato genera un dataset da $1000$ istanze, ciascuna
composta da $32$ feature a valori reali e un vettore contenente la classe
assegnata ad ogni istanza. Per i test si è utilizzata la funzione per generare
dataset con un numero di feature crescente e, per ognuno di essi, si è eseguito
l'algoritmo genetico generando sempre più individui così da studiare e
confrontare i risultati delle due librerie al variare di questi due parametri.

Una volta generati i dataset sono state effettuate simulazioni impiegando tre
dei modelli di machine learning che offre \textit{Scikit-Learn}:
\textit{MultiLayer Perceptron}, \textit{Support Vector Machine} e
\textit{Random Forest}. Per ogni dataset sono state valutate dieci istanze dei
dati, per le quali sono stati generati sia l'insieme $Z_=$ che l'insieme
$Z_{\neq}$, in tutti i casi con un massimo di 100 iterazioni e con probabilità
di crossover e mutazione rispettivamente del 70\% e 30\%. Al fine di valutare
robustezza e variabilità dei risultati, ogni simulazione è stata ripetuta dieci
volte, calcolando quindi media e deviazione standard dei risultati. Per ogni
simulazione si sono quindi registrati i valori di distanza minima, media e
massima di ogni popolazione sintetica generata, separando i risultati in base
alla classe target.

In più del 60\% delle simulazioni \textit{PPGA} ha ottenuto valori di distanza
medi e massimi migliori rispetto a \textit{DEAP}. Per sapere invece a quanto
ammonta il miglioramento di una libreria rispetto all'altra si sono considerati
solo i valori medi di distanza, andando a calcolare il miglioramento minimo,
medio e massimo nelle simulazioni in cui una libreria prevale sull'altra.

\begin{table}[H]
	\centering
	\begin{tabular}{lrr}
		\toprule
		Miglioramento (\%) & PPGA    & DEAP    \\
		\midrule
		Minimo             & 0.0028  & 0.0102  \\
		Medio              & 33.0630 & 18.2321 \\
		Massimo            & 93.1545 & 84.8844 \\
		\bottomrule
	\end{tabular}
	\caption{Percentuale di qualità relativa}
	\label{tab: improvement}
\end{table}

Come si può vedere in tabella \ref{tab: improvement}, quando \textit{PPGA} su
ottiene risultati migliori a \textit{DEAP}, questi sono mediamente migliori del
33\%, mentre per \textit{DEAP} si arriva ad un miglioramento medio del 18\%.

Si sono infine sommati tutti i valori di distanza medi ottenuti su ognuna delle
simulazioni effettuate con entrambe le librerie, così da ottenere un punteggio
complessivo per ognuna di esse. Anche in questo caso \textit{PPGA} prevale con
un punteggio complessivo migliore di circa l'$11.74\%$ rispetto a \textit{DEAP}.

Sono seguite analisi più approfondite per riuscire a individuare differenze
qualitative su diverse tipologie di input, suddividendo i risultati ottenuti
per modello di machine learning impiegato e classe target per la generazione
dei dati sintetici. Come prima cosa si sono analizzate le distribuzioni dei
valori di distanza ottenuti con le due librerie, che come si può dedurre dai
grafici sottostanti sono molto simili per diverse tipologie di simulazioni.

\begin{figure}[H]
	\includesvg[width=0.95\linewidth]{immagini/hist.svg}
	\caption{Istogramma delle frequenze dei valori di distanza}
	\label{fig: hist_distance}
\end{figure}

In figura \ref{fig: hist_distance} l'istogramma delle frequenze per ogni
tipologia di simulazione effettuata; in blu la distribuzione delle distanze
relativa a \textit{PPGA}, mentre in arancio quella relativa a \textit{DEAP}.

Sebbene l'istogramma delle frequenze sia molto simile per le due librerie, non
confronta i risultati ottenuti su una specifica simulazione. In altre parole,
si vuole valutare la differenza di qualità fissando il dataset, il modello di
machine learning utilizzato e il numero di individui generati dall'algoritmo
genetico. Per farlo è stata calcolata la differenza tra i valori medi di
distanza ottenuti dalle due librerie sulle simulazioni eseguite con la stessa
configurazione di input e parametri iniziali, ottenendo quindi la distribuzione
della differenza dei risultati prodotti dalle due librerie su una stessa
simulazione. Si è quindi contato il numero di simulazioni in cui il risultato
delle due librerie avesse una differenza che superasse due deviazioni standard
della distribuzione delle differenze, ricavando che neanche il 7\% supera tale
valore.

Si è poi studiato più in dettaglio come queste si comportano al variare del
numero delle feature e del numero di individui sintetici generati.

\begin{figure}[H]
	\centering
	\includesvg[width=0.95\linewidth]{immagini/quality_feature.svg}
	\caption{Distanza media al variare del numero di feature}
	\label{fig: quality_feature}
\end{figure}

I grafici mostrano l'andamento dei valori di distanza al variare del numero di
feature che compongono i dati. In blu le curve relative all'andamento di
\textit{PPGA}, mentre in rosso quelle relative a \textit{DEAP}. Le diverse
tonalità di blu e rosso indicano simulazioni in cui sono stati generati un
diverso numero di individui.

Si nota come l'andamento dei valori di distanza sia molto simile tra le due
librerie, sia al variare del numero di feature che di individui generati.
L'unico caso in cui sembra esserci una differenza più marcata è nel caso del
\textit{Random Forest} in cui si generano dati classificati diversamente da
quello di riferimento, dove tutte le curve relative a \textit{PPGA} stanno al
di sotto di quelle relative a \textit{DEAP}.

Negli altri casi le librerie sembrano produrre risultati molto simili per la
stessa configurazione di parametri in input, mantenendo un andamento crescente
dei valori di distanza, in cui \textit{PPGA} ottiene risultati leggermente
migliori, soprattutto con un numero di feature più elevato. In modo analogo si
è studiato l'andamento dei valori medi di distanza ottenuti all'aumentare del
numero di individui generati.

\begin{figure}[H]
	\centering
	\includesvg[width=0.95\linewidth]{immagini/quality_pop.svg}
	\caption{Distanza media al variare del numero di dati sintetici generati}
	\label{fig: quality_pop}
\end{figure}

Come prima le curve relative a \textit{PPGA} hanno sempre valori di distanza
inferiori rispetto alle corrispondenti curve relative a \textit{DEAP}. Sebbene
i valori siano differenti, per entrambe le librerie si denota uno sviluppo
decrescente dei valori di distanza all'aumentare del numero di individui
generati, soprattutto quando il numero di feature è maggiore.

In definitiva, dalle analisi condotte sulla qualità dei risultati, sembra che
le due librerie producano risultati e si comportino in modo simile al variare
dei diversi parametri e dati in input.

\section{Prestazioni}

Nella fase finale di test si sono messe a confronto le prestazioni delle due
librerie, misurando i tempi d'esecuzione dei due algoritmi nelle loro versioni
sequenziale e parallela. Tutti i test sono stati effettuati su una macchina con
doppio processore AMD EPYC 7313, ciascuno dei quali con 16 core fisici, con una
frequenza di clock massima di 3.7 GHz.

Il problema di riferimento è ancora quello di generazione dati impiegato per i
test qualitativi, in cui è stato eseguito l'algoritmo genetico utilizzando i
tre modelli e andando a variare parametri come il numero di feature del dataset,
numero di individui generati, e numero di worker impiegati. Come per i test
precedenti, per ogni istanza di dato, l'algoritmo genetico è stato eseguito due
volte andando a cambiare il target di classificazione nella funzione di
valutazione.

Tutti i test sono stati effettuati su 10 istanze, tutte composte da 64 feature,
dello stesso dataset, per le quali è stato generato sia l'insieme $Z_=$ che
l'insieme $Z_{\neq}$. Per ogni test sono state effettuate 20 iterazioni con una
probabilità di crossover del 70\% e una probabilità di mutazione del 30\%,
ripetendo ogni l'esecuzione 10 volte per poi calcolare media e deviazione
standard dei tempi registrati.

Per entrambi gli algoritmi sono stati misurati sia il tempo totale d'esecuzione
che quello parziale delle fasi che i due algoritmi eseguono in parallelo.
Quest'ultimo prende in considerazione, nel caso delle esecuzioni in parallelo
anche delle frazioni seriali che il processo principale impiega per le fasi di
invio e ricezione.

\subsection{Tempo d'esecuzione}

La prima analisi riguarda i tempi d'esecuzione al variare del numero
di individui generati dall'algoritmo, partendo da un minimo di 1.000 individui
generati, fino ad arrivare ad un massimo di 16.000. Ogni simulazione su una
stessa istanza è stata ripetuta variando anche il numero di worker, partendo
dalla versione sequenziale, fino ad arrivare alla versione parallela con un
massimo di 32 worker.

In più del 90\% delle simulazioni \textit{PPGA} è riuscito a terminare prima di
\textit{DEAP}, con un miglioramento medio variabile a seconda del modello di
machine learning impiegato. Nella tabella di seguito sono riportate le
percentuali di miglioramento medio di \textit{PPGA} su \textit{DEAP},
calcolato sia sul tempo d'esecuzione totale che sulla sola fase parallelizzata.

\begin{table}[H]
	\centering
	\begin{tabular}{lrr}
		\toprule
		Modello        & Totale  & Fase Parallela \\
		\midrule
		Neural Network & 31.40\% & 50.19\%        \\
		Random Forest  & 21.56\% & 25.09\%        \\
		SVM            & 32.01\% & 48.32\%        \\
		\bottomrule
	\end{tabular}
	\caption{Miglioramento medio rispetto a DEAP}
	\label{tab: time_improv_DEAP}
\end{table}

Non si sono notate invece particolari differenze nel tempo relativo che i due
algoritmi, nella loro versione sequenziale, passano nella fase parallelizzabile.

\begin{table}[H]
	\centering
	\begin{tabular}{lrr}
		\toprule
		Modello        & PPGA    & DEAP    \\
		\midrule
		Neural Network & 88.93\% & 91.18\% \\
		Random Forest  & 99.58\% & 99.68\% \\
		SVM            & 90.32\% & 92.88\% \\
		\bottomrule
	\end{tabular}
	\caption{Percentuale di tempo passato nella fase parallelizzabile}
\end{table}

Questo fornisce un indicazione di quanto l'impiego di una paradigma di calcolo
parallelo possa incidere.

I grafici di seguito mostrano l'andamento del tempo d'esecuzione dei due
algoritmi all'aumentare del numero di individui generati. In rosso le
simulazioni riguardanti \textit{DEAP}, mentre in blu quelle riguardanti
\textit{PPGA}; le linee con tonalità più chiara indicano le simulazioni che
hanno impiegato un minor numero di worker.

\begin{figure}[H]
	\centering
	\includesvg[width=\linewidth]{immagini/time_pop2.svg}
	\caption{Runtime al variare del numero di individui generati}
	\label{fig: time_pop}
\end{figure}

Nelle simulazioni in cui è stata impiegata la rete neurale e la \textit{SVM},
si osserva che in entrambi i casi \textit{PPGA} tende ad avere tempi
d'esecuzione migliori, anche per input di dimensione differente. Nel caso in
cui invece è stato impiegato un modello \textit{Random Forest}, la differenza
tra i due algoritmi si appiana ma \textit{PPGA} tende ad avere mediamente dei
tempi d'esecuzione inferiori.

Da considerare anche la scalabilità dei due algoritmi all'aumentare del numero
di worker, i quali potrebbero creare un overhead di comunicazione più elevato
del calcolo effettivo.

\begin{figure}[H]
	\centering
	\includesvg[width=\linewidth]{immagini/time_pop.svg}
	\caption{Scalabilità}
	\label{fig: scalability}
\end{figure}

Per entrambe le librerie sembra esserci un andamento simile quando si
considerano le simulazioni eseguite con \textit{Random Forest}, mentre sembra
esserci una differenza più accentuata quando vengono impiegati gli altri
modelli, più performanti ma che riescono a beneficiare maggioramente del più
alto grado di parallelismo.

Come si può notare dalla tabella seguente, anche il guadagno ottenuto dalle
versioni parallele dei due algoritmi è significativamente diverso, a seconda
del modello di machine learning impiegato.

\begin{table}[H]
	\centering
	\begin{tabular}{lrrrrr}
		\toprule
		Modello        & Workers & PPGA    & $\mathrm{PPGA}_p$ & DEAP    & $\mathrm{DEAP}_p$ \\
		\midrule
		Neural Network & 1       & 4.684   & 4.177             & 5.299   & 4.831             \\
		Neural Network & 32      & 1.764   & 0.575             & 3.251   & 2.712             \\
		Random Forest  & 1       & 130.852 & 130.314           & 172.356 & 171.804           \\
		Random Forest  & 32      & 8.458   & 7.109             & 8.350   & 7.793             \\
		SVM            & 1       & 5.434   & 4.920             & 6.772   & 6.292             \\
		SVM            & 32      & 1.686   & 0.535             & 2.897   & 2.328             \\
		\bottomrule
	\end{tabular}
	\caption{Tempo d'esecuzione per la generazione di $8.000$ individui}
	\label{tab: runtime}
\end{table}

Nel caso di \textit{Random Forest}, \textit{PPGA} riesce infatti a diminuire il
tempo d'esecuzione, passando da più di 2 minuti a soli 8 secondi impiegando 32
worker. Nel caso invece degli altri modelli più performanti in fase di
predizione, il miglioramento è meno accentuato ma sicuramente significativo in
un caso d'uso come quello di riferimento, in cui sono necessarie molte
esecuzioni dell'algoritmo genetico.

\subsection{Speed-up}

Per riuscire ad avere un indicatore del livello di parallelismo raggiunto dai
due algoritmi, sono stati calcolati i valori di \textit{speed-up} ottenuti
sulle fasi parallelizzate dalle due librerie.

\begin{table}[H]
	\centering
	\begin{tabular}{lrrr}
		\toprule
		Modello        & Popolazione & PPGA   & DEAP   \\
		\midrule
		Neural Network & 8000        & 7.269  & 1.781  \\
		Neural Network & 16000       & 8.859  & 1.958  \\
		Random Forest  & 8000        & 18.331 & 22.046 \\
		Random Forest  & 16000       & 20.315 & 23.407 \\
		SVM            & 8000        & 9.192  & 2.703  \\
		SVM            & 16000       & 9.049  & 2.716  \\
		\bottomrule
	\end{tabular}
	\caption{Speed-up con 32 worker}
	\label{speed-up}
\end{table}

Dai test e similmente a quanto riportato in tabella, i valori di speed-up
raggiunti da \textit{PPGA} sono superiori a quelli di \textit{DEAP} in più
del 65\% delle simulazioni effettuate. Da notare per che nel caso di
\textit{Random Forest} è \textit{DEAP} ad avere i valori migliori, soprattutto
su input grandi. Facendo però riferimento ai tempi d'esecuzione riportati in
tabella \ref{tab: runtime}, è possibile notare come, nella sua versione
sequenziale, \textit{DEAP} impieghi più tempo di \textit{PPGA} e come entrambi
ottengano un tempo d'esecuzione praticamente identico per le loro versioni
parallele con 32 worker.

Per quanto riguarda invece le simulazioni con rete neurale e \textit{SVM}, i
valori di speed-up ottenuti da \textit{PPGA} sono generalmente superiori
rispetto a \textit{DEAP}.

\begin{figure}[H]
	\centering
	\includesvg[width=\linewidth]{immagini/speedup_pop.svg}
	\caption{Speed-up}
	\label{fig: speedup_pop}
\end{figure}

Per le simulazioni effettuate con rete neurale o \textit{SVM}, i valori di
speed-up crescono più lentamente, anche con un aumento esponenziale della
dimensione dell'input. Come si è visto infatti per l'analisi dei tempi
d'esecuzione, generando $8.000$ individui, l'algoritmo sequenziale di entrambe
le librerie impiegava pochi secondi per terminare, rendendo più difficile
ottenere uno speed-up significativo. Questo è dovuto anche all'elevato costo di
comunicazione che richiede il multi-processing, che tende a crescere con il
numero di worker. Per quanto riguarda invece le simulazioni con
\textit{Random Forest} i valori di speed-up si discostano dalla retta ideale
solo quando il numero di worker cresce significativamente o quando l'input è
troppo piccolo, non a caso è il modello più lento in fase di predizione rispetto
agli altri e che quindi trae maggior beneficio dalla parallelizzazione.

% Possiamo quindi affermare che \textit{DEAP} sia stato in grado di ridurre in
% modo più efficiente il tempo d'esecuzione quando questo diventa molto grande.
% È necessario però tenere in conto che tanto maggiore è il tempo d'esecuzione
% sequenziale, tanto più \quotes{facile} sarà ottenere uno speed-up elevato;
% questo perché il tempo di comunicazione e sincronizzazione tra i worker passa
% in secondo piano rispetto al tempo di calcolo.

\subsection{Efficienza}

Per confrontare quanto venga sfruttata bene l'architettura multi-core è stata
calcolata l'\textit{efficienza} dei due algoritmi come il rapporto tra il valore
di speed-up ottenuto e il numero di worker impiegati per una certa simulazione.
Anche in questo \textit{PPGA} prevale con valori di efficienza migliori in più
dell'80\% delle simulazioni effettuate.

\begin{figure}[H]
	\centering
	\includesvg[width=\linewidth]{immagini/efficiency_pop.svg}
	\caption{Efficienza}
	\label{fig: efficiency}
\end{figure}

Come è possibile vedere dai grafici, l'efficienza dei due algoritmi decresce
con l'aumentare del numero di worker ma, nel caso di simulazioni con
\textit{Random Forest}, questa rimane relativamente stabile su valori superiori
all'80\% per input medi o grandi e un massimo di 16 worker. Quando si fa uso
degli altri due modelli la decrescita è invece costante, ma si può vedere come
i valori di efficienza di \textit{PPGA} siano sempre superiori a quelli di
\textit{DEAP} per molte dimensioni dell'input.

\begin{table}[H]
	\centering
	\begin{tabular}{lrrr}
		\toprule
		Modello        & Workers & PPGA  & DEAP  \\
		\midrule
		Neural Network & 8       & 0.593 & 0.234 \\
		Neural Network & 16      & 0.387 & 0.115 \\
		Neural Network & 32      & 0.227 & 0.056 \\
		Random Forest  & 8       & 0.946 & 0.949 \\
		Random Forest  & 16      & 0.888 & 0.853 \\
		Random Forest  & 32      & 0.573 & 0.689 \\
		SVM            & 8       & 0.643 & 0.304 \\
		SVM            & 16      & 0.417 & 0.156 \\
		SVM            & 32      & 0.287 & 0.084 \\
		\bottomrule
	\end{tabular}
	\caption{Efficienza per la generazione di 8.000 individui}
	\label{tab: efficiency}
\end{table}

Come si può vedere sia dalla tabella riportata di sopra, sia dai grafici, nelle
simulazioni in cui sono stati impiegati rete neurale e \textit{SVM}, i valori
di efficienza tendono ad essere superiori per \textit{PPGA}, nella quasi
totalità dei casi. Per quanto riguarda invece le simulazioni con
\textit{Random Forest}, il valore di efficienza decresce molto più lentamente
per entrambe le librerie, raggiungendo un'efficienza migliore di quasi 10 punti
percentuali per \textit{DEAP} quando vengono impiegati 32 worker, riflettendo
lo speed-up più elevato evidenziato nella sezione precedente con lo stesso
numero di worker.


\chapter{Conclusioni}\label{cap: conclusions}

In conclusione, dai test effettuati, \textit{PPGA} ha mostrato un significativo
miglioramento nei tempi di esecuzione rispetto a \textit{DEAP}, in particolare
per problemi computazionalmente intensivi come il problema del commesso
viaggiatore e la generazione di dati sintetici per \textit{LORE}. L'approccio
parallelo adottato ha consentito di sfruttare in modo più efficiente le risorse
hardware disponibili, riducendo l'impatto del \textit{GIL} di Python grazie
all'uso del multiprocessing. Tuttavia, DEAP rimane una libreria più consolidata,
con una maggiore flessibilità nella definizione degli operatori genetici e un
supporto decisamente più completo.

Nonostante i risultati positivi, \textit{PPGA} presenta alcune limitazioni.
L'overhead introdotto dalla comunicazione tra processi può ridurre l'efficienza
in problemi di piccola scala, dove il costo di sincronizzazione tra processi
supera i benefici del parallelismo. Inoltre, la necessità di serializzare e
deserializzare gli oggetti per la comunicazione tra i processi può introdurre
un ulteriore collo di bottiglia. Per superare queste limitazioni, un futuro
sviluppo della libreria potrebbe includere:
\begin{itemize}
	\item L'implementazione di algoritmi e costrutti più complessi, in grado di
	      lavorare bene su un insieme più grande di problemi.
	\item L'uso di \verb|shared memory| per ridurre il costo della comunicazione
	      tra i worker, almeno per problemi che non richiedono strutture dati
	      particolarmente complesse.
\end{itemize}

In futuro, sarebbe interessante estendere il confronto con altri framework di
calcolo parallelo e valutare l'impatto di \textit{PPGA} in scenari reali.
La libreria offre comunque un punto di partenza solido per futuri sviluppi e
ottimizzazioni, contribuendo alla ricerca di metodi computazionali più veloci
ed efficienti nell'ambito degli algoritmi genetici e dell'Explainable AI.


\nocite{*}
\bibliographystyle{plain}
\bibliography{bibliography} % Entries are in the refs.bib file

\end{document}