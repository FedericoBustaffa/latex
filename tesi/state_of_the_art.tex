\chapter{Stato dell'arte}

% 1. Iniziare a definire qui le possibili opzioni per l'implementazione di
% algoritmi paralleli in Python? Come multiprocessing, Cython, C/C++ extensions,
% e subinterpreters

% 2. PARALLELISMO: Spiegare solo il modello di parallelismo offerto dalla
% combinazione di DEAP e multiprocessing o anche di parallelismo generale (map)?

Allo stato dell'arte, un'implementazione di quanto citato nel capitolo
precedente è già presente. Abbiamo infatti \textit{LORE}~\cite{guidotti2018LORE}
come metodo di XIA, un'implementazione di algoritmo genetico fornita da
\textit{DEAP}~\cite{fortin2012DEAP}, e un metodo per parallelizzare tale
algoritmo, tramite la combinazione di \textit{DEAP} e del modulo
\verb|multiprocessing| di Python.

\section{LORE}

Il metodo proposto da \textit{LORE}~\cite{guidotti2018LORE} rientra nella
categoria dei metodi locali. Opera infatti sulle singole istanze dei dati,
lavorando sotto l'assunzione che i confini di classificazione locali siano
generalmente meno complessi di quelli globali, riuscendo così ad ottenere % verificare
spiegazioni più accurate.

Il modo in cui opera \textit{LORE} prevede l'esecuzione di un algoritmo genetico
su ognuna delle istanze classificate dal modello e di cui si vogliono ottenere
fattuali e controfattuali. In una prima fase si vuole quindi generare un
vicinato sintetico i cui individui siano classificati allo stesso modo
dell'istanza di riferimento, si cerca poi di generare individui sintetici
classificati diversamente da essa, minimizzando in entrambi i casi la distanza
tra i vicini sintetici e l'istanza stessa.

Una volta generati, i dati sintetici vengono usati per allenare un albero di
decisione contenente delle regole che andranno poi a costituire la spiegazione
della scelta fatta e le possibili modifiche sui dati, di modo da cambiare tale
scelta. Le diramazioni di un albero decisionale corrispondono infatti a valori
soglia delle feature che determinano una scelta piuttosto che un'altra. Seguendo
quindi i cammini che portano alla stessa decisione effettuata dal modello si
ottiene la spiegazione di tale scelta, seguendo invece cammini alternativi è
possibile capire come cambiare tale scelta.

\section{DEAP}

La libreria \textit{DEAP} fornisce un'API molto flessibile ma richiede un fase
di inizializzazione in cui vengono definiti tipi, strutture dati e funzioni
necessarie al corretto funzionamento dell'algoritmo.

\subsection{Inizializzazione}

Come prima cosa è necessario definire la struttura di un generico cromosoma, il
tipo di ciascun gene e i pesi da assegnare a ciascun parametro da ottimizzare.
A tal fine vengono messi a disposizione il modulo \verb|creator| e la classe
\verb|Toolbox|.

Tramite \verb|creator| è possibile definire nuove classi con caratteristiche
personalizzate e che si adattano alle proprie esigenze. La classe \verb|Toolbox|
serve invece per immagazzinare funzioni e relativi parametri, di modo da poter
essere utilizzati con facilità in seguito. Tramite tale classe è infatti
possibile definire operatori genetici e metodi di inizializzazione, selezione e
valutazione degli individui. Ad eccezione però della funzione di valutazione,
che è strettamente legata al problema di interesse e deve essere quindi definita
dall'utente caso per caso, \textit{DEAP} fornisce implementazioni per tutte le
componenti appena citate, lasciando quindi il solo compito di scegliere quale
si addice di più al problema in questione. Rimane comunque la possibilità di
definire ciascuna delle componenti dell'algoritmo come meglio si crede nel caso
ci siano esigenze particolari.

\subsection{Algoritmi}

Una volta definite tutte le componenti dell'algoritmo, è possibile invocare
direttamente i metodi registrati nell'istanza della classe \verb|Toolbox|,
andando a definire un comportamento personalizzato per esso. Una seconda opzione
consiste invece nel ricorrere alle implementazioni fornite da \textit{DEAP},
tramite il modulo \verb|algorithms|, che presenta diverse implementazioni
possibili, tra cui \verb|eaSimple|, che ha una struttura molto simile a quella
definita in figura \ref{fig:simple_ga} ed è l'algoritmo attualmente impiegato
da \textit{LORE}.

L'algoritmo prende in input l'istanza della classe \verb|Toolbox| precedentemente
definita, una popolazione iniziale, il numero massimo di iterazioni da compiere
e le probabilità di crossover e mutazione. \`E possibile passare anche parametri
opzionali per registrare dati statistici sui valori di fitness e una struttura
dati per registrare i migliori $k$ individui mai prodotti durante le varie
generazioni, ossia la \verb|HallOfFame|.

\subsection{Multi-processing}

La classe \verb|Toolbox| ha anche dei metodi predefiniti, tra cui il metodo
\verb|map|, la cui implementazione è data dall'omonima funzione standard di
Python. Nel caso dell'algoritmo \verb|eaSimple|, viene usata la funzione
\verb|map| registrata nell'istanza delle \verb|Toolbox| per applicare la
funzione di valutazione a ciascun individuo della popolazione.

Come per gli altri metodi è possibile registrare una qualunque funzione sotto
il nome "map" e \textit{DEAP}, al fine di ottenere l'esecuzione in parallelo,
suggerisce di registrare la funzione \verb|map| della classe \verb|Pool| del
modulo \verb|multiprocessing|.

Il modulo \verb|multiprocessing| è considerato lo standard per il calcolo
parallelo in Python quando si ha del lavoro computazionalmente intensivo da
compiere. Come suggerisce il nome sfrutta processi multipli per aggirare il
problema posto dal GIL, aumentando però l'overhead di comunicazione tra i
processi e il consumo complessivo di memoria.

Al fine di gestire i processi e la distribuzione del carico in modo più semplice
è disponibile una classe \verb|Pool|, che implementa un pool di processi in
grado di bilanciare automaticamente il carico e comunicare dati input e output
per ogni operazione. Il metodo \verb|map| della classe \verb|Pool| applica una
certa funzione a tutti gli elementi di una struttura dati iterabile passata come
argomento. Se non si specificano ulteriori argomenti, gli elementi dell'iterabile
verranno inviati uno per volta ai processi una volta che questi sono pronti. Al
contrario se si specifica il parametro \verb|chunksize| è possibile definire il
numero di elementi che un processo deve elaborare prima di poterne ricevere
altri.