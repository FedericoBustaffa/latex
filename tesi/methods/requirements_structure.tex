\chapter{Metodologia}\label{cap: methods}

La prima fase di sviluppo della libreria si è incentrata sull'individuazione
e definizione di requisiti e struttura, mirando ad un'implementazione
efficiente, senza però rinunciare ad un alto livello di espressività.

La seconda fase ha invece riguardato l'esplorazione delle possibili soluzioni
al problema introdotto dal \textit{GIL}, in cui sono stati testati e valutati
diversi metodi per ottenere l'esecuzione.

Una volta terminata la fase di sperimentazione dei vari framework si è passati
allo sviluppo della libreria vera e propria e con essa, le versioni sequenziale
e parallela dell'algoritmo.

La fase finale del lavoro è stata di sperimentazione ed è stata suddivisa in
tre parti: la prima, utilizzata anche in fase di sviluppo, è stata necessaria
per testare la correttezza di quanto implementato. Si sono infatti considerati
problemi ben noti in letteratura, tra cui il problema del commesso viaggiatore,
il problema dello zaino e un caso di regressione lineare. Per le ultime due
fasi è stato invece riprodotto il problema specifico del metodo \textit{LORE},
valutando qualità e prestazioni tramite un'analisi comparativa con \textit{DEAP}.

\section{Struttura e Requisiti}

I primi requisiti individuati sono quelli imposti dalla natura stessa degli
algoritmi genetici nella loro forma più comune. Primo fra tutti la struttura
del cromosoma, tipicamente vettoriale o simile ad una lista, rendendo così più
semplice l'implementazione degli operatori genetici classici, che operano bene
su strutture di questo tipo. L'utente deve poter assumere di lavorare su
cromosomi con tale forma, senza che debba essere lui a specificarla.

Il secondo requisito consiste nel permettere all'utente di definire tutte le
fasi dell'algoritmo, permettendogli quindi di implementare operatori genetici e
altre funzionalità specifiche per il suo caso d'uso, se quelle fornite
dalla libreria non siano in grado di soddisfare i suoi bisogni. Da questo
nasce anche la necessità di garantire un livello minimo di espressività nel
definire tali metodi. In particolare la funzione di valutazione è ciò che più
contraddistingue il problema di riferimento ed è infatti sempre richiesto che
sia l'utente a definirla. Non gli si può quindi precludere l'impiego di moduli
o librerie necessarie alla valutazione degli individui, come ad esempio
\textit{Numpy} e \textit{Scikit-Learn}, fondamentali alla valutazione degli
individui nel caso del metodo \textit{LORE}.

Quest'ultimo requisito ha avuto ripercussioni sulla scelta del framework per
l'implementazione della versione parallela dell'algoritmo. Come anticipato,
alcuni di questi richiedono compromessi sulla forma del codice, mentre altri,
ancora sperimentali, potrebbero non essere pienamente supportati dalle librerie
di cui l'utente potrebbe aver bisogno, rendendoli inadeguati.

In quanto a struttura della libreria si è cercato di riprodurre API e costrutti
simili a \textit{DEAP}, semplificando però alcuni costrutti e specializzandoli
in base alla tipologia dei casi d'uso trattati. Come per \textit{DEAP} viene
fornita una classe \verb|ToolBox| la quale ha il compito di registrare metodi,
operatori genetici e parametri necessari alla generazione e valutazione di
individui. Un costrutto simile rende la fase di test e l'effettivo utilizzo
finale molto comodi, in particolar modo quando sono previste esecuzioni
multiple, in cui si potrebbero dover variare parametri o metodi a seconda
dell'input o dell'effetto che si vuole ottenere. Sarebbe anche possibile avere
più istanze di \verb|ToolBox|, ognuna impostata con parametri e metodi anche
molto diversi, così da poter cambiare l'intera struttura dell'algoritmo
sostituendo semplicemente l'istanza della classe con cui si intende lavorare.

Sempre in modo simile a \textit{DEAP} è stato implementato un algoritmo simile
a \verb|eaSimple|, il quale accetta come parametri anche l'istanza della classe
\verb|ToolBox| precedentemente definita. Questo approccio permette grande
modularità, mantenendo la logica dell'algoritmo separata da metodi e parametri
con cui opera.
