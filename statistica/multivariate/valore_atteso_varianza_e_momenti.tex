\section{Valore atteso, varianza e momenti}
Abbiamo definito la media campionaria di un insieme di dati come la loro media aritmetica
\[ \tilde{x} = \frac{1}{n} \sum_{i=1}^n x_i \]
per dare una misura del \emph{centro} della distribuzione dei dati. Abbiamo poi definito la
varianza campionaria
\[ \Var(x) = \frac{1}{n-1} \sum_{i=1}^n (x_i - \tilde{x})^2 \]
come misura della \emph{dispersione} dei dati. Le stesse idee sono alla base dei concetti di
\textbf{valore atteso} e \textbf{varianza} di una variabile aleatoria, quest'ultimo inteso come
oggetto matematico che si propone di descrivere la \emph{casualità} dell'esito di un esperimento.

\subsection{Valore atteso}
Possiamo vedere il valore atteso come la quantità che indica attorno a quale valore la variabile
aleatoria è centrata.

Per una variabile aleatoria discreta che assume valori $x_1, x_2, \dots$ consideriamo come valore
atteso una media pesata di tali valori, dando come peso ad ogni valore la probabilità a questo
associata. Nel caso di variabile discrete che assumono infiniti valori, la media pesata diventa
una serie, di cui bisogna controllare la convergenza.

Per una variabile aleatoria con densità, rimpiazziamo la somma pesata della funzione di massa con
un integrale pesato della densità. Le considerazioni fatte per variabili discrete che possono
assumere infiniti valori sono analoghe per variabili con densità, per le quali dobbiamo effettuare
un controllo sui valori assumibili dall'integrale che definisce il valore atteso della variabile.

\begin{definition}
	Sia $X$ una variabile discreta con funzione di massa $p$, si dice che $X$ ha
	\textbf{valore atteso} se $\sum_i |x_i| p(x_i) < +\infty$, e in tal caso si chiama valore
	atteso il numero
	\[ \E[X] = \sum_i x_i \cdot p(x_i) \]
	Sia poi $X$ con densità $f$, essa ha \textbf{valore atteso} se
	$\int_{-\infty}^{+\infty} |x| \cdot f(x) dx < +\infty$ e in tal caso si chiama valore atteso
	il numero
	\[ \E[X] = \int_{-\infty}^{+\infty} x \cdot f(x) dx \]
\end{definition}

Notiamo che il valore atteso dipende solo dalla funzione di massa (nel caso discreto) o dalla
densità (nel caso con densità), quindi il valore atteso di $X$ dipende solo dalla legge $P_X$
di $X$.

\begin{example}
	Lanciamo un dado equilibrato e supponiamo di ricevere 2 euro se esce 6 e 1 euro se esce 4 o 5,
	0 altrimenti. Qual è il valore atteso del denaro che riceviamo?

	Prima di tutto definiamo una variabile aleatoria $X$ che può assumere come valori la quantità
	di denaro ricevuto in corrispondenza all'esito del lancio del dado. Dato che $X$ è concentrata
	su $\{ 0, 1, 2 \}$ dobbiamo calcolare la probabilità associata ad ognuno di questi valori
	\begin{gather*}
		P_X(0) = 1 / 2 \\[1ex]
		P_X(1) = 1 / 3 \\[1ex]
		P_X(2) = 1 / 6
	\end{gather*}
	A questo punto possiamo applicare la formula per il calcolo del valore atteso dalla quale
	otteniamo
	\[ \E[X] = 0 \cdot \frac{1}{2} + 1 \cdot \frac{1}{3} + 2 \cdot \frac{1}{6} = \frac{2}{3} \]
	ossia 0,67 euro vinti in media ad ogni lancio ripetuto del dado.
\end{example}

Vediamo ora come calcolare il valore atteso di una trasformazione di una variabile aleatoria $X$,
ossia di una nuova variabile $Y$ della forma $g(X)$ con $g : \R \to \R$.

\begin{proposition}
	Sia $X$ discreta, la variabile $g(X)$ ammette valore atteso se
	$\sum_i |g(x_i)| \cdot p(x_i) < +\infty$, e in tal caso
	\[ \E[g(X)] = \sum_i g(x_i) \cdot p(x_i) \]
	Sia $X$ con densità $f$, la variabile $g(X)$ ha valore atteso se
	$\int_{-\infty}^{+\infty} |g(x)| \cdot f(x) dx < +\infty$, e in tal caso
	\[ \E[g(X)] = \int_{-\infty}^{+\infty} g(x) \cdot f(x) dx \]
	\begin{proof}
		Forniamo la dimostrazione per il caso discreto partendo da un caso particolare, ossia il
		caso in cui $g(x_i)$ sono tutti valori distinti. Allora la variabile aleatoria $g(X)$ ha
		come valori $g(x_1), g(x_2), \dots$ con probabilità rispettivamente $p(x_1), p(x_2), \dots$
		e quindi
		\[ \E[g(x)] = \sum_i g(x_i) \cdot p(x_i) \]
		Per dimostrare il caso generale partiamo da un sottocaso, ossia quello in cui $g \geq 0$.
		Abbiamo che $Y = g(x)$ assume valori $y_1, y_2, \dots$. Consideriamo ora gli insiemi
		\[ A_i = \{ j | g(x_j) = y_i \} \]
		e osserviamo che
		\[ P_Y(y_i) = P(g(X) = y_i) = P(X = x_j) \]
		per qualche $j \in A_i$. Questa probabilità la si riscrive come
		\[ P(X = x_j) = \sum_{j \in A_i} P_X (x_j) \]
		quindi
		\begin{multline*}
			\E[Y] = \sum_i y_i p_Y(y_i) = \sum_i y_i \left( \sum_{j \in A_i} p_X(x_j) \right) \\
			= \sum_i \left( \sum_{j \in A_i} g(x_j) p_X (x_j) \right) = \sum_j g(x_j) p_X (x_j)
		\end{multline*}
		Dato che $A_i$ è una partizione di tutti gli indici $j$. Il caso più generale si ottiene
		osservando che $g = g^+ - g^-$ dove
		\begin{align*}
			g^+ = & \max \{g,0\} \\
			g^- = & \max\{-g,0\}
		\end{align*}
	\end{proof}
\end{proposition}

\begin{example}
	Calcolare il valore atteso del quadrato dell'esito del lancio di un dado equilibrato. Prima di
	tutto definiamo la variabile aleatoria $X$ che definisce l'esito del lancio del dado. Abbiamo
	quindi $P_X(k) = 1/6$ per ogni $k \in \{1, \dots, 6\}$. Dobbiamo calcolare quindi $\E[X^2]$.
	\[
		\E[X^2] = \sum_i x_i^2 P(X = x_i) = \sum_{k=1}^6 k^2 \cdot \frac{1}{6}
		= \frac{91}{6} = 15.16
	\]
\end{example}

\begin{example}
	Sia $X \sim U([0,1])$ e $Y = X^2$, calcolare la densità $f_Y$ di $Y$ e $\E[Y]$. Notiamo che
	$Y = g(X)$ con $g(x) = x^2$ e $X$ ha densità
	\[
		f_X(x) = \begin{cases}
			\dfrac{1}{1-0} = 1 & x \in (0,1)    \\
			0                  & x \notin (0,1)
		\end{cases}
	\]
	In particolare $g : (0,1) \to (0,1)$ è $C^1$ invertibile con inversa
	\[ h(y) = \sqrt{y} \]
	che è $C^1$ e la sua derivata è $h'(x) = \frac{1}{2 \sqrt{y}}$ con $y \in (0,1)$. Quindi
	applicando la formula del cambio di variabile ottenendo che $Y$ ha densità
	\begin{align*}
		f_Y (y) = & \begin{cases}
			            f_X (h(y)) |h'(y)| & y \in (0,1)    \\
			            0                  & y \notin (0,1)
		            \end{cases}                  \\
		=         & \begin{cases}
			            \dfrac{1}{2 \sqrt{y}} & y \in (0,1)    \\
			            0                     & y \notin (0,1)
		            \end{cases}
	\end{align*}
	Calcoliamo quindi il valore atteso di $Y$. Il primo metodo è il seguente
	\[ \E[Y] = \int_{-\infty}^{+\infty} x^2 f_X (x) dx = \int_0^1 x^2 dx = \frac{1}{3} x^3 \]
	da valutare tra 1 e 0 e che quindi fa $1/3$.
\end{example}

\begin{observation}
	Quando $X \geq 0$, allora si può sempre definire $\E[X] \in [0, +\infty)$ come
	\[
		\E[X] = \begin{cases}
			\displaystyle\sum_i x_i \cdot p(x_i)            & X \text{ discreta}    \\[3ex]
			\displaystyle\int_{-\infty}^{+\infty} x f(x) dx & X \text{ con densità}
		\end{cases}
	\]
\end{observation}