\chapter{Calcolo combinatorio}
Il calcolo combinatorio è un elemento fondamentale alla base del calcolo delle probabilità. Non ci
soffermiamo più di tanto su questo aspetto ma cerchiamo di fare nostro qualche concetto utile.

\section{Distribuzione uniforme}
Se $\Omega$ è finito e tutti i suoi punti sembrano ragionevolmente equiprobabili, allora vale
\[ P(\omega_i) = \frac{1}{n} \]
Questa è chiamata \textbf{distribuzione uniforme} di probabilità e vale
\[ P(A) = \frac{\# A}{\# \Omega} = \frac{\text{casi favorevoli}}{\text{casi possibili}} \]
Se $\Omega$ è infinito non possiamo parlare di distribuzione uniforme.

\subsection{Nozioni di calcolo combinatorio}
A questo punto è necessario introdurre alcuni concetti fondamentali necessari a comprendere meglio
ciò che andremo a fare più avanti. Il calcolo combinatorio si occupa di proporre dei metodi per
raggruppare gruppi di elementi e, per ogni metodo, vuole contare il numero di possibili
raggruppamenti.

I metodi che andiamo a trattare possono tenere conto dell'ordine con il quale gli elementi sono
raggruppati oppure no. Ogni metodo ha una sua versione in cui si tiene conto delle possibili 
ripetizioni degli elementi.

\subsubsection{Combinazioni}
Le \textbf{combinazioni} contano il numero di sottoinsiemi di $k$ elementi estratti da un insieme di
$n$ elementi senza tener conto dell'ordine.

\begin{definition}
	Definiamo la \textbf{combinazione senza ripetizione} come il numero di sottoinsiemi di $k$ elementi,
	ognuno estratto una sola volta, da un insieme di $n$ elementi.
	\[ \binom{n}{k} = \frac{n!}{k! (n-k)!} = \frac{n (n-1) \dots (n-k+1)}{k!} \]
	La notazione $\binom{n}{k}$ indica il \textbf{coefficiente binomiale}.
\end{definition}

\begin{definition}
	Definiamo la \textbf{combinazione con ripetizione} come il numero di sottoinsiemi di $k$ elementi,
	dove ogni elemento può essere estratto $k$ volte da un insieme di $n$ elementi.
	\[ \binom{n+k-1}{k} = \frac{(n + k - 1)!}{k! (n-1)!} \]
\end{definition}

\subsubsection{Disposizioni}
Le \textbf{disposizioni} contano il numero di possibili sequenze ordinate di $k$ elementi estratte da 
un insieme di $n$ elementi.

\begin{definition}
	Definiamo la \textbf{disposizione senza ripetizione} come il numero di sequenze ordinate di $k$
	elementi, estratta da un insieme di $n$ elementi. In questo caso i $k$ elementi sono tutti diversi
	ma vengono contati tutti i possibili modi di riordinare la stessa sequenza.
	\[ \frac{n!}{(n - k)!} \]
\end{definition}

\begin{definition}
	Definiamo la \textbf{disposizione con ripetizione} come il numero di sequenze ordinate di $k$
	elementi, estratta da un insieme di $n$ elementi. In questo caso i $k$ elementi possono ripetersi
	fino a $k$ volte
	\[ n^k \]
\end{definition}

\subsubsection{Permutazioni}
Le \textbf{permutazioni} contano in quanti possibili modi si può riordinare un insieme di $n$ elementi.

\begin{definition}
	Definiamo la \textbf{permutazione senza ripetizione} come il numero di modi in cui si possono ordinare
	$n$ elementi tutti diversi.
	\[ n! \]
\end{definition}

\begin{definition}
	Definiamo la \textbf{permutazione con ripetizione} come il numero di modi in cui è possibile riordinare
	$n$ elementi nel caso in cui $k$ di essi siano ripetuti.
	\[ \frac{n!}{n_1! \cdot n_2! \cdot ... \cdot n_k!} \]
	In questa formula $n_i$ è il numero di volte che l'elemento $i$ è ripetuto all'interno della sequenza.
\end{definition}

\subsubsection{Binomio di Newton}
Uno strumento interessante di cui non facciamo la dimostrazione ma che ci sarà utile più avanti è il
\textbf{binomio di Newton}.

\begin{definition}
	Definiamo il \textbf{binomio di Newton} come
	\[ (a + b)^n = \sum_{k=0}^n \binom{n}{k} \cdot a^k b^{n-k} \]
\end{definition}

