\chapter{Calcolo combinatorio}
Il calcolo combinatorio è un elemento fondamentale alla base del calcolo delle probabilità. Non ci
soffermiamo più di tanto su questo aspetto ma cerchiamo di fare nostro qualche concetto utile.

\section{Distribuzione uniforme}
Se $\Omega$ è finito e tutti i suoi punti sembrano ragionevolmente equiprobabili, allora vale
\[ P(\omega_i) = \frac{1}{n} \]
Questa è chiamata \textbf{distribuzione uniforme} di probabilità e vale
\[ P(A) = \frac{\# A}{\# \Omega} = \frac{\text{casi favorevoli}}{\text{casi possibili}} \]
Se $\Omega$ è infinito non possiamo parlare di distribuzione uniforme.

\subsection{Nozioni di calcolo combinatorio}
A questo punto è necessario introdurre 3 concetti fondamentali
\begin{definition}
	Definiamo le \textbf{disposizioni} come il numero di applicazioni da $\{ 1, \dots, k \}$ a
	$\{ 1, \dots, n \}$
	\[ n^k \]
\end{definition}

\begin{definition}
	Definiamo le \textbf{permutazioni} come il numero di modi in cui si possono ordinare $n$ elementi.
	\[ n! \]
\end{definition}

\begin{definition}
	Definiamo il \textbf{coefficiente binomiale} o \textbf{combinazione} come il numero di sottoinsiemi
	di $k$ elementi da un insieme di $n$ elementi.
	\[ \binom{n}{k} = \frac{n!}{k! (n-k)!} = \frac{n (n-1) \dots (n-k+1)}{k!} \]
\end{definition}

\begin{definition}
	Definiamo il \textbf{binomio di Newton} come
	\[ (a + b)^n = \sum_{k=0}^n \binom{n}{k} a^k b^{n-k} \]
\end{definition}

Questi concetti sono strettamente legati al calcolo delle probabilità e alla distribuzione uniforme.