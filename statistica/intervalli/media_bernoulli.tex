\section{Media di un campione Bernoulliano}
Sia $X \sim B(p)$ che rappresenta il verificarsi o meno di un \emph{successo} dove $p$ è la
probabilità del successo. Consideriamo un campione $X_1, \dots, X_n$ i.i.d. di $X$, abbiamo che
\[ X_1 + \dots + X_n \]
è il numero successi nel campione e che
\[ \overline{X_n} = \frac{X_1 + \dots + X_n}{n} \]
è la frequenza relativa del successo nel campione. Ricordiamo inoltre che
\[ X_1 + \dots + X_n \sim B(n,p) \quad \Rightarrow \quad \E[X] = \E[X_i] = p \]
In particolare $\overline{X_n}$ è uno stimatore di $p$. A questo punto è ragionevole cercare un
intervallo di fiducia per $p$ del tipo $[\overline{X_n} \pm d]$. In questo caso, dovremo usare i
quantili della distribuzione binomiale, i quali dipendono da $n$. Questo è possibile ma essendo
la formula complessa, offriamo un'alternativa che è valida per $n$ grande (in genere $n \geq 80$),
la quale fa uso del \hyperref[th: tlc]{teorema del limite centrale} poiché la variabile aleatoria
\[
	\frac{X_1 + \dots + X_n - np}{\sqrt{n \cdot p \cdot (1-p)}} =
	\sqrt{n} \cdot \frac{\overline{X_n} - p}{\sqrt{p \cdot (1-p)}}
\]
è approssimativamente una Gaussiana standard $N(0,1)$. Dato che la varianza
$\sigma^2 = p \cdot (1-p)$ dipende da $p$ che incognita possiamo stimare $p$ con $\overline{X_n}$
e quindi $p \cdot (1-p)$ con $\overline{X_n} \cdot (1 - \overline{X_n})$. Si può dimostrare, come
conseguenza del teorema centrale del limite, che
\[
	\frac{X_1 + \dots + X_n - np}{\sqrt{n \cdot \overline{X_n} \cdot (1 - \overline{X_n})}} =
	\sqrt{n} \cdot \frac{\overline{X_n} - p}{\sqrt{\overline{X_n} \cdot (1 - \overline{X_n})}}
\]
converge in legge ad una Gaussiana standard per $n \to +\infty$.

\begin{proposition}
	Dato $\alpha \in (0,1)$, l'intervallo aleatorio
	\[
		\left[
			\overline{X_n} \pm \sqrt{\frac{\overline{X_n} \cdot (1 - \overline{X_n})}{n}}
			\cdot q_{1 - \frac{\alpha}{2}}
			\right]
	\]
	è un intervallo di fiducia per $p$ con livello di fiducia approssimativamente $1-\alpha$, più
	precisamente, si ha
	\[
		\lim_{n \to +\infty} P \left( p \in \left[ \overline{X_n} \pm
			\sqrt{\frac{\overline{X_n} \cdot (1 - \overline{X_n})}{n}} \cdot
			q_{1 - \frac{\alpha}{2}} \right] \right) = 1 - \alpha
	\]
\end{proposition}

\begin{example}
	Si vuole condurre un sondaggio per determinare la percentuale di gradimento nei confronti del
	governo. Qual è il numero minimo di telefonta da effettuare affinché la precisione $d$ della
	stima sia inferiore all'1\%, con livello di fiducia del 95\%? La precisione è
	\[
		d = \sqrt{\frac{\overline{X_n} \cdot (1 - \overline{X_n})}{n}} \cdot
		q_{1 - \frac{\alpha}{2}} = \sqrt{\frac{\overline{X_n} \cdot (1 - \overline{X_n})}{n}} \cdot
		1.96
	\]
	che dipende da $\overline{X_n}$, non noto a priori. Sappiamo però che la funzione che associa
	$x \in [0,1]$ a $x \cdot (1-x)$ ha un massimo in $x=1/2$ e il massimo vale $1/4$, quindi
	\[ \overline{X_n} \cdot (1 - \overline{X_n}) \leq \frac{1}{4} \]
	e di conseguenza
	\[
		d \leq \sqrt{\frac{1/4}{n}} \cdot q_{1 - \frac{\alpha}{2}} =
		\frac{1}{2} \cdot \frac{1}{\sqrt{n}} \cdot q_{1 - \frac{\alpha}{2}}
	\]
	A questo punto otteniamo
	\[
		\frac{1}{2} \cdot \frac{1}{\sqrt{n}} \cdot q_{1 - \frac{\alpha}{2}} \leq 0.01
		\quad \Leftrightarrow \quad
		n \geq \frac{1.96^2}{4 \cdot 0.01^2} = 9604
	\]
	Dato che $n$ è grande vale l'approssimazione Gaussiana.
\end{example}