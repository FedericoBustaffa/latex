\section{Variabili aleatorie notevoli}
In questa sezione trattiamo le distribuzione di probabilità su $\R$ indispensabili per trattare
ogni applicazione. Si tratta sempre di variabili discrete o definite tramite densità.

\subsection{Variabili binomiali}
Consideriamo come primo caso le \textbf{variabili binomiali} prendendo $n$ prove ripetute, con
esito (per ciascuna prova) successo o insuccesso (schema di Bernoulli) e sia $p$ la probabilità di
successo (nella singola prova).
\[ P(a_1, \dots, a_n) p^{\# \{i | a_i=1\}} \cdot (1-p)^{\# \{ i | a_i=0 \}} \]
Sia $X$ la variabile aleatoria che conta il numero di successi, ossia
\[ X(a_1, \dots, a_n) = \sum_{i=1}^n a_i \]
Come possiamo notare $X$ è discreta a valori in $\{0,1,2,\dots,n\}$ e ha funzione di massa
\begin{equation}\label{eq: binom}
	P_X(h) = P(X = h) = \binom{n}{h} \cdot p^h \cdot (1-p)^{n-h}
\end{equation}
con $h \in \{ 0, 1, \dots, n \}$. Possiamo tradurre tutto questo nella probabilità che abbiamo di
avere $h$ successi.

Una variabile binomiale di parametri $n \in \N^+$ e $p \in (0,1)$ è una variabile aleatoria avente
come funzione di massa \ref{eq: binom}

\begin{observation}
	Per $n=1$ si parla di variabile aleatoria di Bernoulli e si indica con $B(1,p) = B(p)$.
\end{observation}

\begin{example}
	Su 5 lanci di un dado equilibrato, qual è la probabilita che il 6 appaia almeno 2 volte?
\end{example}

\subsection{Variabili geometriche}