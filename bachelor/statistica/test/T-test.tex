\section{T-test}
Vogliamo ora formulare un test per la media di una popolazione Gaussiana con varianza non nota. Sia
$X \sim N(\mu, \sigma^2)$ con $\sigma^2$ non noto e sia $X_1, \dots, X_n$ un campione i.i.d. di
$X$. Il ragionamento è analogo allo $Z$-test ma considerando come statistica
\[ \sqrt{n} \cdot \frac{\bar{X} - \mu}{S} \sim T_{n-1} \]
che, come abbiamo visto in precedenza, si tratta di una distribuzione $t$ di Student a $n-1$ gradi
di libertà, con
\[ S^2 = \frac{1}{n-1} \cdot \sum_{i=1}^n \left( X_i - \bar{X} \right)^2 \]
varianza campionaria. Una volta definito questo ci basta usare i quantili della distribuzione $t$
di Student invece di quelli della Gaussiana.

\subsection{Test bilatero}
Poniamoci nell'ipotesi
\[ H_0: \mu = \mu_0 \qquad H_1: \mu \neq \mu_0 \]

\subsubsection{Formulazione del test}
Cerchiamo una regione critica della forma
\[ C = \left\{ |\bar{X} - \mu_0| > d \right\} \]
Per farlo imponiamo quindi il livello $\alpha$
\begin{align*}
	P_{\mu_0} (C) =                     & \alpha \\
	P_{\mu_0} (|\bar{X} - \mu_0| > d) = &
	P_{\mu_0} \left( \sqrt{n} \cdot \frac{|\bar{X} - \mu_0|}{S} >
	\frac{\sqrt{n}}{S} \cdot d \right)           \\
	=                                   &
	P \left( |T_{n-1}| > \frac{\sqrt{n}}{S} \cdot d \right)
\end{align*}
e otteniamo $\frac{\sqrt{n}}{S} \cdot d = \tau_{1 - \frac{\alpha}{2}, n-1}$ che è il quantile di
ordine $1 - \frac{\alpha}{2}$ di una $t$ di Studenti a $n-1$ gradi di libertà. La regione critica
è dunque
\[
	C = \left\{ \sqrt{n} \cdot \frac{|\bar{X} - \mu_0|}{S} >
	\tau_{1 - \frac{\alpha}{2}, n-1} \right\}
\]

\subsubsection{Calcolo del p-value}
Consideriamo $(y_1, \dots, y_n)$ come dati più estremi sotto $H_0 : \mu = \mu_0$ tali che
\[ \frac{|\bar{y} - \mu_0|}{S_y} > \frac{|\bar{x} - \mu_0|}{S_x} \]
con $S_x^2$ e $S_y^2$ varianze campionarie rispettivamente di $(x_1, \dots, x_n)$ e
$(y_1, \dots, y_n)$. A questo punto calcoliamo il $p$-value di $(x_1, \dots, x_n)$ come segue
\begin{align*}
	\bar{\alpha}(x_1, \dots, x_n) = & P_{\mu_0} \left( \frac{|\bar{X} - \mu_0|}{S} >
	\frac{|\bar{x} - \mu_0|}{s} \right)                                                \\
	=                               & P_{\mu_0} \left( \sqrt{n} \cdot
	\frac{|\bar{X} - \mu_0|}{S} > \sqrt{n} \cdot \frac{|\bar{x} - \mu_0|}{s} \right)   \\
	=                               & P \left( |T_{n-1}| >
	\sqrt{n} \cdot \frac{|\bar{x} - \mu_0|}{s} \right)                                 \\
	=                               & 2 \cdot \left( 1 - F_{n-1} \left( \sqrt{n} \cdot
	\frac{|\bar{x} - \mu_0|}{s} \right) \right)
\end{align*}
con $F_{n-1}$ che è la funzione di ripartizione della $t$ di Studenti a $n-1$ gradi di libertà.

\subsubsection{Calcolo della curva opertiva}
In questo caso la curva operativa non è ben definita, infatti se proviamo a calcolarla otteniamo
\[
	\beta (\mu) = P_\mu \left( \sqrt{n} \cdot \frac{|\bar{X} - \mu_0|}{S}
	\leq \tau_{1 - \frac{\alpha}{2}, n-1} \right)
\]
La variabile aleatoria a sinistra della disequazione non è una $t$ di Student ma possiamo
riscriverla come
\[
	\sqrt{n}  \cdot \frac{\bar{X} - \mu_0}{S} =
	\underbrace{\sqrt{n} \cdot \frac{\bar{X} - \mu}{S}}_{T_{n-1}} +
	\underbrace{\sqrt{n}  \cdot \frac{\mu - \mu_0}{S}}_{??}
\]
L'ultima variabile aleatoria dipende da $\sigma^2$ e non la vedremo anche se si può collegare alla
distribuzione $\chi^2$.

\subsection{Test unilatero}
Poniamoci nell'ipotesi
\[ H_0 : \mu \leq \mu_0 \qquad H_1 : \mu > \mu_0 \]

\subsubsection{Formulazione del test}
Identifichiamo la regione critica imponendo
\[ C = \left\{ \sqrt{n} \cdot \frac{\bar{X} - \mu_0}{S} > \tau_{1 - \alpha, n-1} \right\} \]

\subsubsection{Calcolo del p-value}
Il $p$-value si ottiene calcolando
\begin{align*}
	P_{\mu_0} \left( \sqrt{n} \cdot \frac{\bar{X} - \mu_0}{S} >
	\sqrt{n} \cdot \frac{\bar{x} - \mu_0}{s} \right) = &
	P \left( T_{n-1} > \sqrt{n} \cdot \frac{\bar{x} - \mu_0}{s} \right) \\
	=                                                  &
	1 - F_{n-1} \left( \sqrt{n} \cdot \frac{\bar{x} - \mu_0}{s} \right)
\end{align*}

\begin{example}
	Il peso di una popolazione di salmoni ha distribuzione Gaussiana, di media e varianza non nota.
	Viene rilevato il peso su un campione di 16 esemplari e vogliamo formulare un test di livello
	$\alpha = 0.05$ per verificare se l'ipotesi $H_0 : \mu = 3$ è plausibile oppure no e applicarlo
	nel caso le misurazioni diano $\bar{x} = 3.2$ e $s = 0.5$. La regione critica è del tipo
	\[
		C = \left\{ \sqrt{n} \cdot \frac{|\bar{X} - \mu_0|}{S}
		> \tau_{1 - \frac{\alpha}{2}, n-1} \right\} =
		\left\{ 4 \cdot \frac{|\bar{X} - 3|}{S} > 2.13 \right\}
	\]
	Applichiamo il test e otteniamo
	\[ 4 \cdot \frac{|\bar{x} - 3|}{s} = 4 \cdot \frac{0.2}{0.5} = 1.6 \leq 2.13 \]
	quindi si accetta l'ipotesi $H_0$. Calcoliamo ora il $p$-value dei dati
	\begin{align*}
		P_{\mu_0} \left( \sqrt{n} \cdot \frac{|\bar{X} - \mu_0|}{S} >
		\sqrt{n} \cdot \frac{|\bar{x} - \mu_0|}{s} \right) = &
		P \left( |T_{n-1}| > \sqrt{n} \cdot \frac{|\bar{x} - \mu_0|}{s} \right)                 \\
		=                                                    &
		2 \cdot \left( 1 - F_{n-1} \left( \sqrt{n} \cdot
		\frac{|\bar{x} - \mu_0|}{s} \right) \right)                                             \\
		=                                                    &
		2 \cdot \left( 1 - F_{15} \left( 4 \cdot \frac{|3.2 - 3|}{0.5} \right) \right)          \\
		=                                                    & 2 \cdot \left( 1 - 0.935 \right)
		= 0.13
	\end{align*}
	In questo caso accettiamo per $\alpha < 0.13$.
\end{example}

Come possiamo notare, a differenza dello $Z$-test, nonostante il valore della varianza campionaria
sia lo stesso della varianza teorica, il $p$-value è più alto poiché c'è maggiore incertezza.
Questo si traduce in una regione di accettazione più ampia per il $T$-test rispetto allo $Z$-test.