\section{Test per la media di un campione di taglia grande}
Come per gli intervalli di fiducia, per il \hyperref[th: tcl]{teorema centrale del limite}, si può
formulare un test approssimato per la media di una popolazione con distribuzione non nota se il
campione ha taglia $n$ grande.

\subsection{Test per la media con varianza nota}
Sia $X$ una distribuzione di media $\mu$ non nota e varianza $\sigma^2$ nota e sia
$X_1, \dots, X_n$ un campione i.i.d. di $X$ con $n$ grande, allora per il TCL
\[ Z = \sqrt{n} \cdot \frac{\bar{X} - \mu}{\sigma} \sim N(0,1) \]
Come possiamo vedere ci siamo ricondotti nella casistica in cui è possibile applicare uno $Z$-test.

\subsubsection{Test bilatero}
Come al solito ci poniamo nelle ipotesi
\[ H_0 : \mu = \mu_0 \qquad H_1 : \mu \neq \mu_0 \]
La regione critica approssimata equivale a
\[ C = \left\{ \sqrt{n} \cdot \frac{|\bar{X} - \mu_0|}{\sigma} > q_{1 - \frac{\alpha}{2}} \right\} \]
e il $p$-value equivale a
\begin{align*}
	P_{\mu_0} \left( \sqrt{n} \cdot \frac{|\bar{X} - \mu_0|}{\sigma} >
	\sqrt{n} \cdot \frac{|\bar{x} - \mu_0|}{\sigma} \right) = &
	P \left( |Z| > \sqrt{n} \cdot \frac{|\bar{x} - \mu_0|}{\sigma} \right) \\
	=                                                         &
	2 \cdot \left( 1 - \Phi \left( \sqrt{n} \cdot \frac{|\bar{x} - \mu_0|}{\sigma} \right) \right)
\end{align*}

\subsection{Test per la media con varianza non nota}
Nel caso in cui $X$ abbia varianza non nota, sempre con $n$ grande, per la proposizione
\ref{prop: tcl} del TCL, la statisitca
\[ Z = \sqrt{n} \cdot \frac{\bar{X} - \mu_0}{S} \sim N(0,1) \]
quindi, anche in questo caso, possiamo usare uno $Z$-test per campioni gaussiani a varianza nota
ma sostituendo $S$ a $\sigma$.

\begin{example}
	Una ditta che produce aste dichiara una lunghezza media di $2.3$m. Viene rilevata la lunghezza
	di 100 aste prodotte la cui media e varianza campionarie risultano $2.317$m e $0.1$m.
	L'affermazione dichiarata dalla ditta è plausibile o no? Per capirlo formuliamo un test per
	la media di un campione grande con varianza non nota, usando la statistica
	\[ Z = \sqrt{n} \cdot \frac{\bar{X} - \mu}{S} \sim N(0, 1) \]
	Dato che non ci viene fornito un $\alpha$ ci conviene calcolare direttamente il $p$-value
	\begin{align*}
		P_{\mu_0} \left( \sqrt{n} \cdot \frac{|\bar{X} - \mu_0|}{S} >
		\sqrt{n} \cdot \frac{|\bar{x} - \mu_0|}{s} \right) = &
		P \left( |Z| > \sqrt{n} \cdot \frac{|\bar{x} - \mu_0|}{s} \right)                         \\
		=                                                    &
		P \left( |Z| > \sqrt{100} \cdot \frac{|2.317 - 2.3|}{0.1} \right)                         \\
		=                                                    & P \left( |Z| > 1.7 \right)         \\
		=                                                    & 2 \cdot (1 - \Phi (1.7) ) = 0.0892
	\end{align*}
	Quindi per $\alpha > 0.0892$ rifiutiamo $H_0$ e invece accettiamo $H_0$ per $\alpha < 0.0892$.
\end{example}