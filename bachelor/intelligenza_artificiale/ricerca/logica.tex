\chapter{Agenti logici}
Fino ad ora abbiamo trattato
\begin{itemize}
	\item Agenti con stato obbiettivo in mondi osservabili con stati atomici e azioni
	      descrivibili in maniera semplice.
	\item Problemi con struttura fattorizzata, i cui stati sono definiti come un insieme di caratteristiche
	      rilevanti.
\end{itemize}
Ora l'obbiettivo \`e quello di migliorare la \textbf{capacit\`a di ragionamento} degli agenti dotandoli di
rappresentazioni di mondi pi\`u \textbf{complessi} e \textbf{astratti}, non descrivibili semplicemente.

D'ora in poi tratteremo gli \textbf{agenti basati su conoscenza}, dotati di una \textbf{Knowledge Base} con
conoscenza espressa in maniera \textbf{esplicita} e \textbf{dichiarativa}.

I problemi pi\`u comuni dell'IA sono tipicamente basati su conoscenza poich\'e il mondo \`e tipicamente complesso
e ci serve una rappresentazione \textbf{parziale} e \textbf{incompleta} del mondo utile agli scopi dell'agente.

Per ambienti parzialmente osservabili e complessi ci servono linguaggi di rappresentazione della conoscenza pi\`u
espressivi della conoscenza \textbf{pi\`u espressivi} e \textbf{capacit\`a inferenziali}.

La conoscenza pu\`o essere codificata a mano ma anche estratta dai testi, appresa dall'\textbf{esperienza} o
\textbf{estratta} dagli esperti.

La KB racchiude tutta la conoscenza necessaria a decidere l'azione da compiere in forma \textbf{dichiarativa}.
L'alternativa \`e scrivere un programma che implementa il processo decisionale, una volta per tutte.

\section{Agenti basati su conoscenza}
Un agente basato su conoscenza mantiene una \textbf{base di conoscenza} (KB), ovvero un insieme di
\textbf{enunciati} espressi in un \textbf{linguaggio di rappresentazione}.

Interagisce con la KB mediante un'interfaccia funzionale \verb|Tell_Ask|:
\begin{itemize}
	\item \verb|Tell|: per aggiungere nuovi enunciati a KB.
	\item \verb|Ask|: per interrogare la KB.
	\item \verb|Retract|: per eliminare enunciati.
\end{itemize}
Gli enunciati nella KB rappresentano le \textbf{opinioni} o \textbf{credenze} dell'agente.

Le risposte $\alpha$ devono essere tali che $\alpha$ discende necessariamente dalla KB.

Dunque un agente logico, data una KB, contenente rappresentazioni dei fatti che si \textbf{ritengono veri},
vuole sapere se un certo fatto $\alpha$ \`e vero di conseguenza. Vuole quindi sapere se
\[ KB \models \alpha \quad \text{(conseguenza logica)} \]

Il processo seguito da un KB agent \`e questo:
\begin{enumerate}
	\item L'agente riceve una \textbf{percezione}.
	\item Costruisce una \textbf{formula logica} che rappresenta la percezione.
	\item Memorizza la formula nella KB.
	\item Costruisce una domanda opportuna per l'azione che dovrebbe essere compiuta.
	\item Chiede alla KB l'azione da compiere.
	\item Aggiorna di nuovo la KB in base all'azione compiuta.
	\item Ritorna l'azione.
\end{enumerate}

\begin{lstlisting}[style=pseudo-style]
KB_Agent(perception)
	KB = KnowledgeBase();
	t = 0;
	KB.Tell(new_statement(perception, t));
	action = KB.Ask(new_query(t));
	KB.Tell(new_statement(action, t));
	t = t + 1;
	return action;
\end{lstlisting}
Chiariamo che una KB \textbf{non \`e una base di dati}. Una base di dati memorizza solo i dati che gli vengono passati. Ci\`o che
caratterizza una KB \`e la \textbf{capacit\`a inferenziale}, ossia derivare nuovi fatti da quelli memorizzati esplicitamente.

\subsection{Rappresentazione della conoscenza}
Sfortunatamente, pi\`u il linguaggio \`e espressivo, meno efficiente \`e li meccanismo inferenziale. Dunque se deve trovare il giusto
compromesso tra:
\begin{itemize}
	\item \textbf{Espressivit\`a} del linguaggio di rappresentazione.
	\item \textbf{Complessit\`a} del meccanismo inferenziale.
\end{itemize}

\subsubsection{Formalismi per RC}
Un formalismo per la rappresentazione della conoscenza ha tre componenti:
\begin{itemize}
	\item \textbf{Sintassi}: linguaggio composto da un vocabolario e regole per la formazione di frasi.
	\item \textbf{Semantica}: corrispondenza fra gli enunciati e i fatti del mondo trattato.
	\item \textbf{Meccanismo inferenziale}: metodo per inferire nuovi dati.
\end{itemize}

\section{Calcolo proposizionale - PROP}
\subsection{Sintassi}
La sintassi per PROP \`e definita come segue:
\[
	\begin{array}{rl}
		operator \rightarrow         & \lnot \mid \wedge \mid \vee \mid \Rightarrow \mid \Leftrightarrow \\
		formula \rightarrow          & formulaAtomica \mid formulaComplessa                              \\
		formulaAtomica \rightarrow   & True \mid False \mid simbolo                                      \\
		simbolo \rightarrow          & P \mid Q \mid R \mid \dots                                        \\
		formulaComplessa \rightarrow & \lnot formula \mid                                                \\
		                             & (formula \wedge formula) \mid                                     \\
		                             & (formula \vee formula) \mid                                       \\
		                             & (formula \Rightarrow formula) \mid                                \\
		                             & (formula \Leftrightarrow formula)
	\end{array}
\]
Definiamo anche una \textbf{precedenza tra gli operatori} per riuscire a omettere in larga parte le
parentesi. In ordine di precedenza abbiamo:
\[ \lnot, \wedge, \vee, \Rightarrow, \Leftrightarrow \]

\subsection{Semantica}
La \textbf{semantica} per una proposizione PROP definisce se un enunciato è vero o falso rispetto a
un'\textbf{interpretazione}. Un'interpretazione definisce un valore di verit\`a per tutti i
simboli proposizionali.

Un \textbf{modello} \`e un'interpretazione che \emph{rende vera} una formula o un insieme di formule.

La semantica nel calcolo proposizionale \`e di questo tipo:
\begin{itemize}
	\item $True$ \`e sempre vero, $False$ \`e sempre falso.
	\item $P \wedge Q$ \`e vero se $P$ e $Q$ sono veri.
	\item $P \vee Q$ \`e vero se $P$ oppure $Q$ o entrambi sono veri.
	\item $\lnot P$ \`e vero se $P$ \`e falso.
	\item $P \Rightarrow Q$ \`e falso solo quando $P$ \`e vero e $Q$ \`e falso.
	\item $P \Leftrightarrow Q$ \`e vero se sono entrambi veri o entrambi falsi.
\end{itemize}

\subsection{Inferenza}
Prima di definire un meccanismo inferenziale per PROP dobbiamo mettere a fuoco ancora qualche concetto.

\subsubsection{Conseguenza logica}
Una formula $\alpha$ \`e \textbf{conseguenza logica} di un insieme di formule KB se e solo se in ogni modello di KB,
anche $\alpha$ \`e vera ($KB \models \alpha$). Se indichiamo con $M(KB)$ i modelli dell'insieme di formule in KB e
con $M(\alpha)$ l'insieme delle interpretazioni che rendono vera $\alpha$:
\[ KB \models \alpha \quad \Leftrightarrow \quad M(KB) \subseteq M(\alpha) \]

\subsubsection{Equivalenza logica}
Un'\textbf{equivalenza logica} si verifica quando due o pi\`u formule hanno lo stesso valore di verit\`a ma sono scritte
in modo diverso:
\[ A \equiv B \quad \Leftrightarrow \quad A \models B \wedge B \models A \]
Ne sono esempio le leggi di De Morgan e la commutativit\`a di $\wedge$.

\subsubsection{Validit\`a e soddisfacibilit\`a}
\begin{itemize}
	\item Una formula \`e \textbf{valida} se e solo se \`e vera per tutte le interpretazioni (\textbf{tautologia}).
	\item Una formula \`e \textbf{soddisfacibile} se e solo se esiste almeno un'interpretazione che la rende vera.
\end{itemize}
Ne segue che $A$ \`e valida se e solo se $\lnot A$ \`e insoddisfacibile.

\subsubsection{Inferenza per PROP}
Ora possiamo andare a parlare delle strategie per fare inferenza nel calcolo proposizionale.
\begin{itemize}
	\item \textbf{Model Checking}: \`e una forma di inferenza che fa riferimento alla definizione di conseguenza logica
	      in cui si enumerano i possibili modelli e corrisponde alla tecnica delle tabelle di verit\`a.
	\item \textbf{Algoritmi per la soddisfacibilit\`a (SAT)}: la conseguenza logica pu\`o essere ricondotta a un problema di
	      soddisfacibilit\`a. Per farlo sar\`a indispensabile il \textbf{teorema di refutazione}.
\end{itemize}

\begin{theorem}[Teorema di refutazione]
	Sia KB una \emph{Knowledge Base} e sia $A$ una formula. $A$ \`e conseguenza logica di KB se e solo se
	\[ KB_\wedge \wedge \lnot A \]
	\`e insoddisfacibile.
\end{theorem}

\subsection{TT-entails}
L'algoritmo \textbf{TT-entails} procede come segue:
\begin{enumerate}
	\item Enumera tutte le possibili interpretazioni di KB (con $k$ simboli avremmo $2^k$ possibili interpretazioni).
	\item Per ciascuna interpretazione controlla se soddisfa o meno KB. Se non soddisfa KB, si procede, se soddisfa KB
	      controlla che soddisfi anche $\alpha$.
	\item Se si trova anche una sola interpretazione che soddisfa KB e non $\alpha$ allora $KB \not\models \alpha$.
\end{enumerate}

\begin{lstlisting}[style=pseudo-style]
TT_entails(KB, alpha)
	symbols = symbolsFrom(KB, alpha);
	return TT_check_all(KB, alpha, symbols, {});
\end{lstlisting}

\begin{lstlisting}[style=pseudo-style]
TT_check_all(KB, alpha, symbols, model)
	if symbols.isEmpty() then
		if truthCheck(KB, model) then
			return truthCheck(alpha, model);
		return true;
	P = symbols.pop();
	return
		TT_check_all(KB, alpha, symbols, model.add(P = true))
		and
		TT_check_all(KB, alpha, symbols, model.add(P = false));
\end{lstlisting}

\subsection{Forma a clausole}
Introduciamo ora una KB in \textbf{forma a clausole}, ovvero un insieme di letterali.
Qualcosa di questo tipo
\[ \{ A, B \} \quad \{ \lnot B, C, D \} \quad \{ \lnot A, F \} \]
La forma a clausole \`e la \textbf{forma normale congiuntiva}: ovvero una congiunzione di disgiunzioni di letterali.
\[ \{ A, B \} \{ C, \lnot D \} = (A \vee B) \wedge (C \vee \lnot D) \]
Non \`e restrittiva poich\'e \`e sempre possibile ottenerla con trasformazioni che preservano l'equivalenza logica.

\subsubsection{Trasformazione in forma a clausole}
Data una proposizione qualsiasi, per ottenere la forma a clausole, i passi sono i seguenti:
\begin{enumerate}
	\item Eliminazione $\Leftrightarrow$: \[ (A \Leftrightarrow B) \equiv (A \Rightarrow B) \wedge (B \Rightarrow A) \]
	\item Eliminazione $\Rightarrow$: \[ (A \Rightarrow B) \equiv (\lnot A \vee B) \]
	\item Negazioni all'interno: De Morgan
	      \begin{gather*}
		      \lnot (A \vee B) \equiv (\lnot A \wedge \lnot B) \\
		      \lnot (A \wedge B) \equiv (\lnot A \vee \lnot B)
	      \end{gather*}
	\item Distribuzione di $\vee$ su $\wedge$: \[ (A \vee (B \wedge C)) \equiv (A \vee B) \wedge (A \vee C) \]
\end{enumerate}

\subsection{DPLL}
\`E un algoritmo SAT che compie un'enumerazione in profondit\`a di tutte le possibili interpretazioni alla ricerca
di un modello.
Rispetto a TT-entails ha tre punti che lo migliorano di molto:
\begin{itemize}
	\item Terminazione anticipata.
	\item Euristica dei simboli puri.
	\item Euristica delle clausole unitarie.
\end{itemize}

\subsubsection{Terminazione anticipata}
Si pu\`o decidere sulla verit\`a di una clausola anche con interpretazioni parziali:
\begin{itemize}
	\item Se c'\`e un letterale vero la clausola sar\`a vera indipendetemente dal valore degli altri letterali.
	\item Se anche una sola clausola \`e falsa l'interpretazione non pu\`o essere un modello dell'insieme di clausole.
\end{itemize}

\subsubsection{Euristica dei simboli puri}
Un \textbf{simbolo puro} \`e un simbolo che appare con lo stesso segno in tutte le clausole.
\begin{itemize}
	\item Nel determinare se un simbolo \`e puro o meno si possono trascurare le occorrenze in clausole gi\`a rese vere.
	\item I simboli puri possono essere assegnati a \verb|True| se il letterale \`e positivo e a \verb|False| se \`e negativo.
	\item Non si eliminano modelli utili: se le clausole hanno un modello continuano ad averlo dopo questo assegnamento.
	      L'assegnamento \`e obbligato.
\end{itemize}

\subsubsection{Euristica delle clausole unitarie}
Una \textbf{clausola unitaria} \`e una clausola con un solo letterale \textbf{non assegnato}.

In questo caso conviene assegnare prima valori al letterale in clausole unitarie. Assegnamo \verb|True| se il letterale \`e
positivo, \verb|False| se \`e negativo.

\subsubsection{Algoritmo}
L'algoritmo segue questi passi:
\begin{enumerate}
	\item Costruisce delle clausole a partire dalle formule in input.
	\item Estrae tutti i simboli dalle clausole.
	\item Se tutte le clausole sono vere ho trovato un modello: termino con successo.
	\item Se trovo anche solo una clausola falsa: termino con fallimento.
	\item Cerco un simbolo puro. Se lo trovo lo tolgo dall'insieme dei simboli, lo metto nel modello e torno al punto 3.
	\item Cerco una clausola unitaria. Se la trovo tolgo il simbolo nella clausola dall'insieme dei simboli,
	      lo metto nel modello e torno al punto 3.
	\item Se non ho trovato n\'e simboli puri n\'e clausole unitarie rimuovo il primo simbolo che c'\`e nell'insieme dei simboli
	      e lo metto nel modello una volta assegnandogli valore \verb|True| e poi assegnandogli valore \verb|False|.
	\item Concludo con una doppia chiamata ricorsiva, una per ognuno dei due valori assegnati al simbolo. Dato che non ho avuto modo
	      di effettuare una terminazione anticipata devo considerare tutte le possibilit\`a.
\end{enumerate}

\begin{lstlisting}[style=pseudo-style]
DPLL_SAT(prop)
	clauses = prop.clauses();
	symbols = prop.symbols();
	return DPLL(clauses, symbols, {});
\end{lstlisting}

\begin{lstlisting}[style=pseudo-style]
DPLL(clauses, symbols, model)
	if model.truthTest(clauses) == false then
		return false;
	if model.truthTest(clauses) == true then
		return true;
	
	// truthTest puo' ritornare un terzo valore neutro
	p = findPureSymbol(symbols, clauses, model);
	if p != null then 
		symbols.remove(p);
		model.add(p);
		return DPLL(clauses, symbols, model);
	
	p = findUnitClause(clauses, model);
	if p != null then
		symbols.remove(p);
		model.add(p);
		return DPLL(clauses, symbols, model);
	
	p = symbols.pop();
	return
		DPLL(clauses, symbols, model.add(p = true))
		or
		DPLL(clauses, symbols, model.add(p = false));
\end{lstlisting}

\subsubsection{Miglioramenti}
Anche se DPLL \`e \textbf{completo} e termina sempre si possono fare dei miglioramenti:
\begin{itemize}
	\item Analisi di sotto-problemi indipendenti se le variabili possono essere suddivise in sotto-insiemi disgiunti
	      (senza simboli in comune).
	\item Ordinamento di variabili e valori: scegliere prima la variabile che compare in pi\`u clausole.
	\item Backtracking intelligente.
\end{itemize}

\subsection{WalkSAT}
\begin{enumerate}
	\item Il modello \`e un assegnamento completo scelto casualmente.
	\item Se il modello soddisfa le clausole allora termino con successo.
	\item Prendo una clausola qualsiasi non ancora soddisfatta.
	\item Prendo un simbolo da modificare (\textbf{flip}). La modalit\`a di scelta del simbolo viene scelta con
	      una probabilit\`a $p$ (di solito $0.5$) tra:
	      \begin{itemize}
		      \item A caso.
		      \item Il simbolo che rende pi\`u clausole soddisfatte.
	      \end{itemize}
	\item Ci si arrende dopo un certo numero di flip.
\end{enumerate}

\begin{lstlisting}[style=pseudo-style]
WalkSAT(clauses, p, max_flips)
	model = randomAssignment(clauses);
	for i = 1 to max_flips
		if model.satisfies(clauses) then
			return model;
		clause = clauses.getRandomFalseClause(model);
		symbol = clause.getSymbol(p);
		flip(symbol);
	return failure;
\end{lstlisting}

\subsubsection{Analisi WalkSAT}
\begin{itemize}
	\item \`E incompleto per un numero limitato di flip.
	\item Va bene per cercare un modello sapendo che esiste.
	\item Non pu\`o essere usato per verificare l'insoddisfacibilit\`a.
\end{itemize}

\subsection{Problemi SAT difficili}
Se un problema ha molte soluzioni, (problema \textbf{sotto-vincolato}) \`e pi\`u probabile che WalkSAT trovi una
soluzione in tempi brevi.

In generale \`e molto significativo il rapporto $m / n$ dove $m$ \`e il numero di clausole (vincoli) e $n$ \`e il
numero di simboli. Pi\`u grande \`e questo rapporto pi\`u vincolato \`e il problema.

La probabilit\`a che un problema sia soddisfacibile \`e inversamente proporzionale a questo rapporto. Pi\`u un
problema \`e vincolato meno sar\`a la probabilit\`a che questo sia soddisfacibile.

Esiste una fascia intermedia in cui si trovano problemi soddisfacibili ma difficili. In questa fascia WalkSAT \`e molto
pi\`u efficiente di DPLL a parit\`a di rapporto $m/n$

\subsection{Inferenza come deduzione}
Un altro modo per stabilire se $KB \models A$ \`e usare un \textbf{sistema di deduzione}.
Introduciamo la notazione $KB \vdash A$ che significa che $A$ \`e deducibile da KB.

La deduzione avviene semplificando delle \textbf{regole di inferenza}.
In un sistema di inferenza le regole dovrebbero derivare
\begin{itemize}
	\item solo formule che sono conseguenza logica.
	\item tutte le formule che sono conseguenza logica.
\end{itemize}

\subsubsection{Correttezza e completezza}
\begin{itemize}
	\item \textbf{Correttezza}: Se $KB \vdash A$ allora $KB \models A$. Tutto ci\`o che \`e derivabile \`e conseguenza logica.
	\item \textbf{Completezza}: Se $KB \models A$ allora $KB \vdash A$. Tutto ci\`o che \`e conseguenza logica \`e ottenibile
	      tramite il sistema deduttivo.
\end{itemize}

\subsubsection{Regole di inferenza}
Le regole di inferenza sono \textbf{schemi deduttivi} del tipo:
\[ \frac{\alpha \Rightarrow \beta, \quad \alpha}{\beta} \quad \text{Modus Ponens} \]
dove quello che al di sopra della linea sono premesse o assiomi dimostrati e quello che c'\`e sotto la linea \`e la conseguenza,
ovvero la \textbf{deduzione}.

\subsubsection{Ricerca}
In problemi come la dimostrazione di teoremi conviene procedere all'indietro.
\begin{itemize}
	\item \textbf{Completezza}: Le regole della deduzione naturale sono un insieme di regole di inferenza completo, se anche l'algoritmo
	      \`e completo allora non c'\`e problema.
	\item \textbf{Efficienza}: La complessit\`a \`e alta poich\'e ci troviamo davanti a un problema decidibile ma NP-completo.
\end{itemize}

\subsection{Regola di risoluzione}
Questa regola presuppone che il problema sia in forma a clausole ed \`e preferita la notazione di tipo
insiemistico cos\`i da eliminare duplicati.
\[
	\frac{ \{ p_1, \dots, p_i, \dots, p_m \} \{ q_1, \dots, q_j, \dots, q_n \} }
	{ \{ p_1, \dots, p_{i-1}, p_{i+1}, \dots, p_{m}, q_1, \dots, q_{j-1}, q_{j+1}, \dots, q_n \} }
\]
dove $p_i = \lnot q_j$ o viceversa.


Un caso particolare \`e quello della \textbf{clausola vuota} o \textbf{contraddizione}.
\[ \frac{ \{ P \}\{ \lnot P \} }{ \{ \} } \]

\subsubsection{Analisi}
Non \`e un metodo completo per dimostrare la soddisfacibilit\`a di KB. Il fatto che $\alpha$ sia conseguenza
logica di KB non implica sempre che $\alpha$ sia deducibile da KB. Si possono trovare controesempi.

\begin{theorem}[Teorema di risoluzione]
	KB \`e insoddisfacibile se e solo se \[ KB \vdash_{\text{res}} \{ \} \]
\end{theorem}
Dunque se utilizzo la regola per \textbf{dimostrare l'insoddisfacibilit\`a} di KB allora il metodo \`e completo.

Un metodo alternativo consiste nel negare $\alpha$, aggiungerla alla KB e dimostrare che si giunge ad una
contraddizione.
\begin{theorem}[Teorema di refutazione]
	$\alpha$ \`e conseguenza logica di KB se e solo se
	\[ KB \cup \{ \lnot \alpha \} \]
	\`e insoddisfacibile.
\end{theorem}

\section{Logica del primo ordine - FOL}
Nella logica del primo ordine abbiamo assunzioni logiche pi\`u ricche con le quali rappresentiamo un mondo
pi\`u complesso tramite delle entit\`a e relazioni fra esse.

Si inizia con una \textbf{concettualizzazione} in cui si decide quali sono le cose di cui si vuole parlare.
Abbiamo tre categorie principali:
\begin{itemize}
	\item \textbf{Oggetti}: sono le entit\`a.
	\item \textbf{Propriet\`a}: sono le caratteristiche delle entit\`a.
	\item \textbf{Relazioni fra gli oggetti}: sono relazioni che legano entit\`a diverse.
\end{itemize}

\subsection{Sintassi}
La sintassi per la logica del primo ordine \`e definita come segue:
\[
	\begin{array}{rl}
		connettivo \rightarrow     & \wedge \mid \vee \mid \lnot \mid \Rightarrow \mid \Leftrightarrow \mid
		\Leftarrow                                                                                          \\
		quantificatore \rightarrow & \forall \mid \exists                                                   \\
		variabile \rightarrow      & x \mid y \mid \dots \quad \text{lettere minuscole}                     \\
		costante \rightarrow       & A \mid B \mid \dots \quad \text{lettere maiuscole}                     \\
		funzione \rightarrow       & + \mid - \mid \dots                                                    \\
		predicato \rightarrow      & \leq \mid \geq \mid \dots
	\end{array}
\]
Di seguito, in ordine decrescente di precedenza, gli operatori logici:
\[ = \quad \lnot \quad \wedge \quad \vee \quad \Rightarrow, \Leftrightarrow \quad \forall, \exists \]
\subsubsection{Sintassi per i termini}
\[
	\begin{array}{rl}
		termine \rightarrow & costante \mid variabile \mid funzione(termine, \dots)
	\end{array}
\]
\subsubsection{Sintassi per le formule}
\[
	\begin{array}{rl}
		formulaAtomica \rightarrow & True \mid False \mid                              \\
		                           & termine = termine \mid                            \\
		                           & predicato(termine, \dots)                         \\
		formula \rightarrow        & formulaAtomica \mid                               \\
		                           & formula \quad connettivo \quad formula \mid       \\
		                           & quantificatore \quad variabile \quad formula \mid \\
		                           & \lnot formula \mid (formula)
	\end{array}
\]

Si dice che le variabili che appaiono nella formula associata ad un quantificatore sono nel suo \textbf{ambito}

Una variabile si dice \textbf{legata} se \`e utilizzata nell'ambito dei quantificatori. Si dice \textbf{libera}
altrimenti.

Le formule si dividono in tre categorie che dipendono dalla propriet\`a delle variabili precedentemente descritta.
\begin{itemize}
	\item \textbf{Chiusa}: non contiene occorrenze di variabili libere.
	\item \textbf{Aperta}: contiene almeno un'occorrenza di variabili libere.
	\item \textbf{Ground}: non contiene variabili.
\end{itemize}

\subsection{Semantica}
Vedremo vari tipi di semantica per FOL.

La \textbf{semantica dichiarativa} consiste nello \textbf{stabilire una corrispondenza} tra
\begin{itemize}
	\item i termini del linguaggio e gli oggetti del mondo.
	\item le formule chiuse e i valori di verit\`a.
\end{itemize}
Un'interpretazione stabilisce una corrispondeza precisa tra elementi atomici del linguaggio ed elementi della
concettualizzazione. Tale interpretazione interpreta:
\begin{itemize}
	\item Le costanti come elementi del dominio.
	\item Le funzioni come funzioni da $n$-uple di $D$ in $D$.
	\item I predicati come insiemi di $n$-uple.
\end{itemize}

Nella \textbf{semantica composizionale} il significato di un termine o di una formula composta \`e determinato
in funzione del significato dei suoi componenti.

Nel caso di $\forall$ la formula \`e vera se \`e vera per ciascuno degli elementi del dominio indicato.

Nel caso invece di $\exists$ la formula \`e vera se esiste almeno un elemento del dominio indicato per cui la
formula \`e vera.

In rappresentazione della conoscenza viene spesso usata la \textbf{semantica dei database} che \`e pi\`u semplice
rispetto alla semantica standard della logica. Questa semantica ha le seguenti caratteristiche:
\begin{itemize}
	\item Ipotesi dei nomi unici: simboli distinti per oggetti distinti.
	\item Ipotesi del mondo chiuso: tutto ci\`o di cui non si sa se \`e vero \`e falso.
	\item Chiusura del dominio: esistono solo gli oggetti di cui si parla.
\end{itemize}

\subsection{Inferenza}
Andiamo ora a vedere qualche regola di inferenza per i quantificatori.
\subsubsection{Istanziazione dell'Universale}
Detta \textbf{eliminazione del $\forall$} \`e definita come segue:
\[ \frac{\forall x \quad A[x]}{A[g]} \]
dove $g$ \`e un termine \emph{ground} e $A[g]$ \`e il risultato della sostituzione di $g$ per $x$ in $A$.

\subsubsection{Istanziazione dell'esistenziale}
Detta anche \textbf{eliminazione dell'$\exists$} \`e definita come segue:
\[ \frac{\exists x \quad A[x]}{A[k]} \]
\begin{itemize}
	\item Se $\exists$ non compare nell'ambito di $\forall$, $k$ \`e una costante nuova (\textbf{costante di Skolem}).
	\item Altrimenti va introdotta una funzione (\textbf{funzione di Skolem}) nelle variabili quantificate
	      universalmente.
\end{itemize}
A questo punto potrei effettuare un processo di \textbf{proposizionalizzazione}
\begin{itemize}
	\item Creando tante istanze delle formule quantificate universalmente quanti sono gli oggetti menzionati.
	\item Eliminando gli esistenziali skolemizzando.
\end{itemize}
A questo potremmo pensare di trattare la KB come proposizionale e applicare gli algoritmi visti per
il calcolo proposizionale. Questo \textbf{non \`e possibile} dato che se ci sono funzioni il numero di istanze
da creare \`e \textbf{infinito}.

Per fortuna possiamo comunque trattare il problema. Per farlo facciamo riferimento al seguente teorema.
\begin{theorem}[Teorema di Herbrand]
	Se $KB \models A$ allora c'\`e una dimostrazione che coinvolge solo un sotto-insieme finito della KB
	proposizionalizzata.
\end{theorem}
Possiamo procedere in modo incrementale:
\begin{enumerate}
	\item Creare istanze con le costanti.
	\item Creare poi quelle con un solo livello di annidamento utilizzando le funzioni.
	\item Poi quelle con due livelli di annidamento e cos\`i via.
\end{enumerate}
Se $KB \not\models A$ il processo non termina. Il problema \`e \textbf{semidecidible}.

\subsection{Forma a clausole}
Abbiamo gi\`a visto una regola di risoluzione per PROP. Per riuscire ad estenderla a FOL dobbiamo riuscire a
trasformare il problema in forma a clausole.

Per farlo seguiamo le seguenti regole:
\begin{itemize}
	\item Costanti, funzioni e predicati sono come definiti, ma escludiamo nel seguito formule atomiche.
	\item Una clausola \`e un insieme di letterali, che rappresenta la loro disgiunzione.
	\item Una KB \`e un insieme di clausole.
\end{itemize}

\subsubsection{Trasformazione in forma a clausole}
\begin{theorem}
	Per ogni formula chiusa $\alpha$ del FOL \`e possibile trovare in maniera \textbf{effettiva} un insieme
	di clausole $FC(\alpha)$ che \`e soddisfacibile se e sole se $\alpha$ (formula di partenza) era soddisfacibile.
\end{theorem}

Vediamo ora nel dettaglio il processo di trasformazione:
\begin{enumerate}
	\item Eliminazione delle implicazioni:
	      \[
		      \begin{array}{rcl}
			      A \Rightarrow B     & \rightarrow & \lnot A \vee B                         \\
			      A \Leftrightarrow B & \rightarrow & (\lnot \vee B) \wedge (\lnot B \vee A)
		      \end{array}
	      \]
	\item Negazioni all'interno:
	      \[
		      \begin{array}{rclr}
			      \lnot \lnot A      & \rightarrow & A                      &                    \\
			      \lnot (A \wedge B) & \rightarrow & \lnot A \vee \lnot B   & (\text{De Morgan}) \\
			      \lnot (A \vee B)   & \rightarrow & \lnot A \wedge \lnot B & (\text{De Morgan}) \\
			      \lnot \forall x A  & \rightarrow & \exists x \lnot A      &                    \\
			      \lnot \exists x A  & \rightarrow & \forall x \lnot A      &                    \\
		      \end{array}
	      \]
	\item Standardizzazione delle variabili: si fa in modo che ogni quantificatore usi una variabile diversa.
	\item Skolemizzazione: eliminazione dei quantificatori esistenziali.
	\item Eliminazione dei quantificatori universali.
	      \begin{enumerate}
		      \item Si portano tutti davanti.
		            \[
			            \begin{array}{rcl}
				            (\forall x A) \vee B   & \rightarrow & \forall x (A \vee B)   \\
				            (\forall x A) \wedge B & \rightarrow & \forall x (A \wedge B)
			            \end{array}
		            \]
		      \item Si eliminano usando la convenzione delle variabili libere, le quali sono quantificate
		            universalmente.
	      \end{enumerate}
	\item Forma normale congiuntiva (congiunzione di disgiunzione di letterali):
	      \[
		      \begin{array}{rcl}
			      A \vee (B \wedge C) & \rightarrow & (A \vee B) \wedge (A \vee C)
		      \end{array}
	      \]
	\item Notazione a clausole.
	\item Separazione delle variabili: clausole diverse, varibili diverse.
\end{enumerate}
\textbf{NOTA}: Tutti i passi meno la Skolemizzazione preservano l'equivalenza delle formule. \`E comunque
preservata la soddisfacibilit\`a.

\subsection{Unificazione}
Parliamo ora del processo di \textbf{unificazione}: un processo mediante il quale si determina se due espressioni
possono essere rese \textbf{identiche} mediante una \textbf{sostituzione} di termini a variabili.

Il risultato \`e la \textbf{sostituzione} che rende le due espressioni identiche, detta \textbf{unificatore},
oppure \textbf{FAIL}, se le espressioni non sono unificabili.

\subsubsection{Sostituzione}
Una \textbf{sostituzione} \`e un insieme finito di \textbf{associazioni} tra variabili e termini, in cui ogni
variabile compare una sola volta sulla sinistra.

\textbf{NOTA}: sulla sinistra possono apparire solo variabili, sulla destra pu\`o apparire un qualsiasi
termine, con la restrizione che non pu\`o essere una variabile che appare gi\`a sulla sinistra.

Sia $\sigma$ una sostituzione e $A$ un'espressione: \verb|Subst(sigma, A)| \`e l'istanza generata dalla
sostituzione.

In questo processo le variabili vengono sostituite simultaneamente e viene eseguito un solo passo di sostituzione.

\subsubsection{Espressioni unificabili}
Un espressione \`e \textbf{unificabile} se esiste una sostituzione che le rende \textbf{identiche}.

In generale non esiste un solo unificatore per una certa espressione, vorremmo per\`o considerare
\textbf{il pi\`u generale} (MGU), ovvero quello che sostituisce solo le variabili necessarie.

\begin{theorem}
	L'unificatore pi\`u generale \`e \textbf{unico}, a parte i nomi delle variabili (l'ordine non conta).
\end{theorem}

\subsubsection{Algoritmo di unificazione}
L'algoritmo di unificazione prende in input due espressioni $x$ e $y$ e restituisce $\theta$ (un MGU), se esiste.

Ci\`o che fa \`e esplorare in parallelo le due espressioni e costruisce l'unificatore strada facendo. Appena
trova due espressioni non unificabili fallisce.
\begin{enumerate}
	\item Se $\theta$ \`e un fallimento ritorno fallimento.
	\item Se $x = y$ ritorno l'unificatore $\theta$.
	\item Se $x$ \`e una variabile ritorno il risultato dell'unificazione con $x$ a sinistra.
	\item Se $y$ \`e una variabile ritorno il risultato dell'unificazione con $y$ a sinistra.
	\item Se $x$ e $y$ sono entrambe espressioni composte ritorno l'unificazione degli argomenti di $x$ e $y$
	      usando l'unificazione degli operandi (funzioni o predicati) come unificatore.
	\item Se $x$ e $y$ sono entrambe liste unifico i primi due elementi di ognuna e uso il risultato come
	      unificatore per unificare il resto della lista.
	\item Altrimenti ritorno fallimento.
\end{enumerate}

\begin{lstlisting}[style=pseudo-style]
Unify(x, y, theta)
	if theta == FAIL then
		return FAIL;
	else if x == y then
		return theta;
	else if x.isVariable() then
		return UnifyVar(x, y, theta);
	else if y.isVariable() then
		return UnifyVar(y, x, theta);
	else if x.Compound() and y.Compound() then
		return Unify(x.args, y.args, Unify(x.OP, y.OP, theta));
	else if x.isList() and y.isList() then
		x_first = x.pop();
		y_first = y.pop();
		return Unify(x, y, Unify(x_first, y_first, theta));
	
	return FAIL;
\end{lstlisting}

\begin{lstlisting}[style=pseudo-style]
UnifyVar(var, x, theta)
	if var.value is in theta then
		return Unify(var.value, x, theta);
	else if x.value is in theta then
		return Unify(var, x.value, theta);
	else if OccurCheck(var, x) then 
		return FAIL;
	
	return Extend(var, x, theta);
\end{lstlisting}

\verb|OccurCheck| controlla se \verb|var| occorre all'interno dell'espressione \verb|x|. In tal caso fallisce.

\verb|Extend| non aggiunge semplicemente un nuovo legame ma applica questa sostituzione in $\theta$,
diversamente la sostituzione \textbf{non sarebbe normalizzato}.

\subsection{Regola di risoluzione}
Possimo ora definre una regola di risoluzione per FOL come segue.

Date due clausole $\phi$ e $\psi$, con $A$, un letterale in $\phi$ e $\lnot B$ un letterale di segno opposto in
$\psi$. Se $A$ e $B$ sono unificabili e $\gamma$ \`e il loro MGU. Posso ottenere una nuova clausola
\[ ((\phi / \{ A \}) \cup (\psi / \{ \lnot B \})) \gamma \]
detta \textbf{risolvente}

\subsubsection{Problema dei fattori}
Se un sottoinsieme dei letterali di una clausola pu\`o essere unificato allora la clausola ottenuta dopo tale
unificazione si dice \textbf{fattore} della clausola originaria.

Il metodo di risoluzione va applicato ai \emph{fattori} delle clausole.

\subsubsection{Analisi}
La deduzione per risoluzione
\begin{itemize}
	\item \textbf{\`e corretta} sotto questa ipotesi:
	      \[ KB \vdash_{\text{res}} A \Rightarrow KB \models A \]
	\item \textbf{non \`e completa}: pu\`o succedere che
	      \[ KB \models A \]
	      ma
	      \[ KB \not\vdash_{\text{res}} A \]
\end{itemize}

\subsubsection{Risoluzione per refutazione}
Il teorema di refutazione ci suggerisce un metodo alternativo \textbf{completo}

\begin{theorem}[Teorema di refutazione]
	\[ KB \cup \{ \lnot A \} \]
	\`e insoddisfacibile se e solo se
	\[ KB \models A \]
\end{theorem}

\begin{theorem}
	$KB$ \`e insoddisfacibile se e solo se $KB \vdash_{\text{res}} \{ \}$.
\end{theorem}

Abbiamo ora un metodo \textbf{meccanizzabile}, \textbf{corretto} e \textbf{completo}: basta aggiungere il negato
della formula da dimostrare e provare a generare la clausola vuota ossia una \textbf{contraddizione}.

\section{Sistemi a regole}
Il metodo di risoluzione visto per FOL \`e un metodo completo, rimane tuttavia computazionalmente molto
complesso. In questa sezione ci occuperemo di introdurre delle strategie di risoluzione che rendono pi\`u
efficiente tale metodo senza perdere completezza.

Tali strategie ci autorizzano a giustificare una classe di problemi che fanno uso di una contrazione della
logica del primo ordine.

\subsection{Strategie di risoluzione}
Queste strategie servono a restringere lo spazio del problema ad un sotto-insieme di tutte le possibili
clausole. Ne vedremo di tre tipologie:
\begin{itemize}
	\item \textbf{Cancellazione}: eliminazione di clausole superflue.
	\item \textbf{Restrizione}: ottimizzazione del numero di clausole utilizzate ad ogni passo.
	\item \textbf{Ordinamento}: risoluzione dei letterali secondo un ordine specifico.
\end{itemize}

\subsubsection{Strategie di cancellazione}
Le diverse tipologie di clausola eliminabile per questo tipo di strategia sono:
\begin{itemize}
	\item Clausole \textbf{con letterali puri}: elimino le clausole con dei letterali puri ovvero letterali
	      che compaiono in tutte le clausole con lo stesso segno.
	\item \textbf{Tautologie}: elimino le clausole contenenti due letterali complementari e di segno opposto.
	\item Clausole \textbf{sussunte}: elimino clausole \emph{sussunte} dove una clausola $\alpha$
	      \emph{sussume} una clausola $\beta$ se e solo se esiste una sostituzione $\sigma$ tale che, se
	      applicata ad $\alpha$ ottengo un sottoinsieme di $\beta$. In questo caso posso quindi eliminare la
	      clausola $\beta$.
\end{itemize}

\subsubsection{Strategie di restrizione}
Ad ogni passo si sceglie un sottoinsieme delle possibili clausole:
\begin{itemize}
	\item \textbf{Risoluzione unitaria}: unifico due clausole di cui una almeno una \textbf{unitaria}, ovvero,
	      contenente un solo letterale. Cerco di unificare con clausole che contengono il letterale contenuto
	      nella clausola unitaria ma che abbia segno opposto.

	      In generale non \`e una strategia completa. \`E tuttavia completa per \textbf{clausole Horn}.
	      \begin{definition}
		      Una clausola contenente al pi\`u un letterale positivo \`e chiamata \textbf{clausola Horn}.
	      \end{definition}
	\item \textbf{Risoluzione lineare}: unifico l'ultima clausola generata con una delle clausole iniziali oppure
	      con una \textbf{clausola antenata}, ovvero generata precedentemente.

	      \`E una strategia completa.
	\item \textbf{Risoluzione guidata dal goal}: per questo tipo di strategia si usa un \textbf{insieme di supporto},
	      ovvero un sottoinsieme della KB responsabile dell'insoddisfacibilit\`a.

	      Dato che di norma si assume che la KB sia consistente e che l'aggiunta dello stato \emph{goal} negato la
	      renda inconsistente, tipicamente l'insieme di supporto \`e proprio lo stato \emph{goal} negato.

	      \`E una strategia completa per il teorema di refutazione.
\end{itemize}

\subsubsection{Strategie di ordinamento}
In questo caso abbiamo una sola strategia: la \textbf{risoluzione ordinata}. In questo caso, ogni clausola \`e un insieme
ordinato di letterali e si possono unificare solo i primi letterali delle clausole. L'ordinamento deve essere rispettato
nel \textbf{risolvente}.

Si tratta di una strategia completa per clausole Horn.

\subsection{Sottoinsieme a regole del FOL}
I sottoinsiemi a regole non sono altro che KB formati da \textbf{clausole Horn definite}

\begin{definition}
	Una clausola con \textbf{esattamente} un letterale positivo \`e detta \textbf{clausola Horn definita}.
\end{definition}

Tali clausole possono essere di due tipi:
\begin{itemize}
	\item \textbf{Fatti}: clausole che hanno un solo letterale positivo (in totale).
	\item \textbf{Regole}: clausole con una serie di letterali negativi e uno positivo.
\end{itemize}
Se la KB possiede solo clausole Horn definite abbiamo meccanismi inferenziali molto pi\`u semplici senza rinunciare
alla completezza.

Si hanno persino metodi di risoluzione in tempo lineare per il caso proposizionale ma \`e restrittivo per il FOL completo.

\subsubsection{Direzione nell'uso delle regole}
Se utilizzo le regole con concatenazione all'indietro (\textbf{Backward Chaining}) applico un ragionamento guidato
dall'obiettivo. In questo caso le regole sono applicate al contrario. \`E questo il caso della
\textbf{programmazione logica} (\textbf{PROLOG}).

Se invece utilizzo una concatenazione (\textbf{Forward Chaining}) applico un ragionamento guidato dai dati. \`E questo il
caso delle \textbf{basi di dati deduttive} e \textbf{sistemi di produzione}.

\subsection{Programmazione logica}
I \textbf{programmi logici} sono KB, costituiti di clausole Horn definite, espressi come fatti e regole, con una sintassi
alternativa. Scrivo la regola con in testa il \emph{conseguente}.
\[
	\begin{array}{ll}
		A.                        & \text{fatto}  \\
		A :- B_1, B_2, \dots, B_n & \text{regola}
	\end{array}
\]
Un'altra convenzione per la programmazione logica \`e che si scrivono le variabili con lettere maiuscole e le costanti
con lettere minuscole.

\subsubsection{Rappresentazione del goal}
Se il goal \`e
\[ B_1 \wedge B_2 \wedge \dots \wedge B_n \]
per riscriverlo negato posso scrivere
\[ \lnot (B_1 \wedge B_2 \wedge \dots \wedge B_n) \vee False \]
che equivale alla seguente implicazione
\[ B_1 \wedge B_2 \wedge \dots \wedge B_n \Rightarrow False \]
che convertito nella sintassi per la programmazione logica diventa
\[ :- B_1, B_2, \dots, B_n \]
ovvero si scrive solo l'antecedente ommettendo il conseguente.

\subsubsection{Interpretazioni per la programmazione logica}
Nei programmi logici si fa uso di due tipi di interpretazione
\begin{itemize}
	\item \textbf{Dichiarativa}: Scrivo il conseguente in testa e dopo il corpo:
	      \[ A :- B_1, B_2, \dots, B_n \]
	      Questa sintassi significa che $A$ \`e vero se sono veri tutti i $B_i$ con $1 < i < n$.
	\item \textbf{Procedurale}: La testa pu\`o essere vista come una chiamata procedurale e il corpo come una
	      serie di procedure da eseguire in sequenza.
\end{itemize}

\subsection{Risoluzione SLD}
La \textbf{risoluzione SLD} (\textbf{Selection Linear Definte-clauses}) \`e un metodo di risoluzione per i programmi
logici di tipo \textbf{ordinato}, \textbf{lineare} e  basato su un \textbf{insieme di supporto} (lo stato goal).

Questo metodo di risoluzione \`e completo per clausole Horn.

\subsubsection{Alberi di risoluzione SLD}
Dato un programma logico $P$, l'albero SLD per un goal $G$ \`e definito come segue:
\begin{itemize}
	\item Ogni nodo dell'albero corrisponde a un goal (congiuntivo).
	\item La radice \`e $:- G_1, G_2, \dots, G_k$, il goal.
	\item Sia $:- G_1, G_2, \dots, G_k$ un nodo dell'albero. Il nodo ha tanti discendenti quanti sono i fatti e le
	      regole in $P$ la cui testa \`e unificabile con $G_1$.

	      Se $A :- B_1, \dots, B_k$ e $A$ \`e unificabile con $G_1$ e $\gamma = MGU(A, G_1)$ allora un discendente \`e
	      il goal
	      \[ :- (B_1, \dots, B_k, G_2, \dots, G_k) \gamma \]
	\item I nodi che sono clausole vuote sono successi.
	\item I nodi che non hanno successori sono fallimenti.
\end{itemize}
La risoluzione SLD \`e completa per clausole Horn definite, non \`e restrittivo andare in ordine nel risolvere i
sottogoal in and. La sostituzione corrispondente \`e chiamata \textbf{risposta calcolata}.

In PROLOG l'albero di risoluzione viene visitato in profondit\`a con \emph{backtracking} in caso di fallimento e le
regole vengono applicate nell'ordine in cui sono immesse. Per PROLOG tale strategia non \`e completa e si omette inoltre
l'\verb|OccurCheck| per motivi di efficienza.

\subsubsection{Liste}
PROLOG ammette anche liste come strutture dati espressa come segue
\begin{center}
	\texttt{[E|L]}
\end{center}
Dove \verb|E| indica la testa della lista e \verb|L| il resto della lista. \`E anche definita la concatenazione in questo
modo:
\begin{itemize}
	\item Se concateno una lista vuota ad un'altra lista ottengo la lista stessa:
	      \begin{center}
		      \verb|concatena([], Y, Y)|
	      \end{center}
	\item Se concateno due liste non vuote ho una situazione di questo tipo:
	      \begin{center}
		      \texttt{concatena([A|X], Y, [A|Z])}
	      \end{center}
	      dove \verb|Z| \`e il risultato della concatenazione di \verb|X| e \verb|Y|
\end{itemize}

\subsection{Sistemi a regole con concatenazione in avanti}
Anche questi sistemi a regole si basano su una regola di inferenza, il \textbf{Modus Ponens generalizzato}:
\[ \frac{p_1', p_2', \dots, p_n' \quad (p_1 \wedge p_2 \wedge \dots \wedge p_n \Rightarrow q)}{(q)\theta} \]
dove $\theta$ \`e l'$MGU(p_i', p_i)$ per ogni $i$.

La regola, se instanziata come segue, \`e corretta:
\begin{enumerate}
	\item Si istanziano gli universali.
	\item Si istanziano le regole.
	\item Si applica il Modus Ponens classico.
\end{enumerate}
\`E una regola pi\`u generale del Modus Ponens ma pi\`u limitata per il conseguente.

\subsection{FOL-FC-Ask}
Si tratta di un semplice processo inferenziale applica ripetutamente il Modus Ponens generalizzato per ottenere
nuovi fatti fino a che
\begin{itemize}
	\item Si dimostra ci\`o che si desidera dimostrare.
	\item Nessun nuovo fatto pu\`o essere aggiunto.
\end{itemize}

\subsubsection{Analisi FOL-FC-Ask}
\begin{itemize}
	\item \`E completa per KB con clausole Horn definite.
	\item \`E completa e convergente per il calcolo proposizionale e per KB di tipo DATALOG (senza funzioni) perch\'e
	      la chiusura deduttiva \`e un insieme finito.
	\item \`E completa anche con funzioni ma il processo potrebbe non terminare (semidecidibile).
	\item \`E corretta perch\'e il Modus Ponens generalizzato \`e corretto.
	\item \`E sistematica ma non molto efficiente.
\end{itemize}

\subsubsection{FC efficiente}
\begin{itemize}
	\item Un metodo per migliorare il metodo appena visto \`e quello di compiere un ordinamento dei congiunti.
	      In particolare conviene soddisfare prima i congiunti con meno istanze nella KB e che compaiono in pi\`u regole.
	      Questo \`e approccio molto simile all'euristica MRV per i problemi CSP.
	\item \textbf{Rete di discriminazione}: possiamo assumere che regole diverse possano condividere molte delle
	      precondizioni. Sotto quest'ipotesi possiamo codificare gli antecedenti di tali regole sotto forma di
	      \emph{rete di discriminazione} cos\`i da permetterci di testare una sola volta le precondizioni comuni a
	      pi\`u regole.
	\item \textbf{Approccio incrementale}:
	      \begin{itemize}
		      \item Ogni nuovo fatto inferito al tempo $t$ deve essere stato dedotto usando almeno un fatto dedotto al tempo $t-1$.
		      \item Si possono guardare solo regole che hanno premesse unificabili con fatti aggiunti nell'ultima iterazione.
		      \item Indicizzazione delle regole sui fatti.
		      \item Evitare di ricalcolare le unificazioni.
	      \end{itemize}
	\item \textbf{Riduzione di deduzioni irrilevanti}:
	      \begin{enumerate}
		      \item Si fa una sorta \emph{pre-processing} per individuare le regole che servono procedendo all'indietro
		            dal goal e marcando le regole utili.
		      \item Si procede in avanti utilizzando solo le regole marcate.
	      \end{enumerate}
\end{itemize}

