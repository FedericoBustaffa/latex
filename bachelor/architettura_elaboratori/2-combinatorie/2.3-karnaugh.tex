\section{Mappe di Karnaugh}
Come abbiamo appena visto, non sempre ridurre la complessità della nostra formula in modo
\emph{monotòno} ci porta alla migliore ottimizzazione. A volte conviene aumentare la complessità
per poi giungere ad un modello migliore.

Le \textbf{mappe di Karnaugh} forniscono un metodo grafico per riuscire a semplificare le formule
booleane senza però garantire la miglior minimizzazione di quest'ultime. Nell'esempio di prima
abbiamo una funzione booleana con la seguente tabella di verità
\begin{center}
	\begin{tabular}{c c c | c}
		$a$ & $b$ & $c$ & $f(a,b,c)$ \\ \hline
		0   & 0   & 0   & 1          \\
		0   & 0   & 1   & 0          \\
		0   & 1   & 0   & 0          \\
		0   & 1   & 1   & 0          \\
		1   & 0   & 0   & 1          \\
		1   & 0   & 1   & 1          \\
		1   & 1   & 0   & 0          \\
		1   & 1   & 1   & 0
	\end{tabular}
\end{center}
Da questa tabella possiamo ricavare una mappa di Karnaugh prendendo tutti i possibili valori di $a$
e mettendoli nella prima colonna e poi prendendo tutti i possibili valori della coppia $bc$ e
mettendoli sulla prima riga, disponendoli in modo che ogni valore differisca dal precedente al più
di un bit.

Il nostro obbiettivo è quello di individuare gli \textbf{implicanti}, ossia quei quadrati o
rettangoli contenenti un numero di 1 pari ad una potenza di 2 e raggrupparli. Per tale
raggruppamento è possibile
\begin{itemize}
	\item Uscire dalla tabella e rientrare dall'altra parte se ho degli 1 agli estremi.
	\item Includere degli 1 già raccolti in un precedente raggruppamento.
\end{itemize}
Nel nostro caso abbiamo due rettangoli da due 1:
\begin{center}
\begin{karnaugh-map}[4][2][1][$c$][$b$][$a$]
\maxterms{1, 2, 3, 6, 7}
\minterms{0, 4, 5}
\implicant{0}{4}
\implicant{4}{5}
\end{karnaugh-map}
\end{center}
A questo punto siamo
in grado di semplificare la formula di partenza
\begin{enumerate}
	\item Mettendo in \verb|AND| le variabili facenti parte dello stesso raggruppamento che
	      rimangono costanti e negando quelle con valore 0.
	\item Sommando tra di loro i raggruppamenti.
\end{enumerate}
Otteniamo così la formula ottenuta in precedenza con le proprietà dell'algebra booleana
\[ \bar{b} \bar{c} + a \bar{b} \]
in modo meccanico. Il primo termine della somma è ottenuto prendendo in considerazione il
raggruppamento verticale di 1 e considerando che $b$ e $c$ non variano ed essendo a 0 vengono
negati. Il secondo termine si ottiene similmente notando che $a$ e $b$ sono la parte costante del
raggruppamento ed inoltre $b$ è a 0 e dunque deve essere negato.

Prendiamo ora come esempio un \textbf{addizionatore} di 2 bit con riporto, il cui funzionamento
dipende da tre parametri di ingresso: $x_1$ e $x_2$, i bit che vogliamo sommare, e $r_0$, il
possibile riporto da aggiungere. Abbiamo inoltre due uscite: il risultato $s$ della somma e il
possibile riporto $r_1$ generato da essa.
\begin{center}
	\begin{tikzpicture}
		\draw[thick] (0, 0) rectangle (2, 1.5);
		\node (add) at (1, 0.75) {ADD};

		\draw (0.5, 2) node[label=above:$x_1$] {} to[short, o-] (0.5, 1.5);
		\draw (1.5, 2) node[label=above:$x_2$] {} to[short, o-] (1.5, 1.5);
		\draw (2.5, 0.75) node[label=right:$r_0$] {} to[short, o-] (2, 0.75);
		\draw (0, 0.75) -- (-0.5, 0.75) -- (-0.5, -0.75) node[label=below:$r_1$] {};
		\draw (1, 0) -- (1, -0.75) node[label=below:$s$] {};
	\end{tikzpicture}
\end{center}
Le mappe di Karnaugh per $s$ ed $r_1$ risultano le seguenti
\begin{center}
\begin{figure}[!ht] \centering
\begin{subfigure}[b]{0.4\textwidth}
\begin{karnaugh-map}[4][2][1][$c$][$b$][$a$]
\minterms{1,2,4,7}
\maxterms{0,3,5,6}
\implicant{1}{1}
\implicant{2}{2}
\implicant{4}{4}
\implicant{7}{7}
\end{karnaugh-map}
\end{subfigure}
\begin{subfigure}[b]{0.4\textwidth}
\begin{karnaugh-map}[4][2][1][$c$][$b$][$a$]
\minterms{3,5,7,6}
\maxterms{0,1,2,4}
\implicant{3}{7}
\implicant{7}{6}
\implicant{5}{6}
\end{karnaugh-map}
\end{subfigure}
\end{figure}
\end{center}
Da tali mappe di Karnaugh ricaviamo le seguenti formule per $s$ ed $r_1$
\begin{align*}
	s   & = r_0 \bar{x_1} \bar{x_2} + \bar{r_0} \bar{x_1} x_2 + r_0 x_1 x_2 + \bar{r_0} x_1 \bar{x_2} \\
	r_1 & = x_1 x_2 + r_0 x_2 + r_0 x_1
\end{align*}
Di seguito raffiguriamo il circuito ricavato dalla formula per $r_1$.
\begin{center}
	\begin{circuitikz}
		\node[and port] (and1) at (3.5, 1.5) {};
		\node[and port] (and2) at (3.5, 0) {};
		\node[and port] (and3) at (3.5, -1.5) {};
		\node[or port, number inputs=3] (or) at (5.5, 0) {};

		\draw (0, 2) node[label=above:$x_1$] {} to[short, -*] (0, 52 |- and1.in 1) -- (and1.in 1);
		\draw (0, 52 |- and1.in 1) to[short, -*] (0, 52 |- and3.in 2) -- (and3.in 2);

		\draw (0.5, 2) node[label=above:$x_2$] {} to[short, -*] (0.5, 52 |- and1.in 2) -- (and1.in 2);
		\draw (0.5, 52 |- and1.in 2) to[short, -*] (0.5, 52 |- and2.in 2) -- (and2.in 2);

		\draw (1, 2) node[label=above:$r_0$] {} to[short, -*] (1, 52 |- and2.in 1) -- (and2.in 1);
		\draw (1, 52 |- and2.in 1) to[short, -*] (1, 52 |- and3.in 1) -- (and3.in 1);

		\draw (and1.out) -- (or.in 1);
		\draw (and2.out) -- (or.in 2);
		\draw (and3.out) -- (or.in 3);
		\draw (or.out) --++ (0.5, 0) node[label=right:$r_1$] {};
	\end{circuitikz}
\end{center}

\begin{tcolorbox}
	Possiamo quindi usare sia le regole e gli assiomi dell'algebra booleana per semplificare le
	formule ma questo potrebbe portarci sia alla minima forma possibile sia ad un'espressione più
	complessa. Con le mappe di Karnaugh non abbiamo la certezza di ottenere la miglior minimizzazione
	ma ci offrono un modo meccanico per ridurre la complessità.
\end{tcolorbox}

Vogliamo ora implementare un \textbf{moltiplicatore} che moltiplica due sequenze da 2 bit dando
come risultato una sequenza da 4 bit. Per capire come calcolare tale sequenza possiamo procedere
tramite una tabella di verità che però, avendo 4 bit di ingresso risulta avere 16 righe. Cerchiamo
quindi di rappresentare solo le righe significative.
\begin{center}
	\begin{tabular}{c c c c | c c c c}
		$x_1$ & $x_2$ & $y_1$ & $y_2$ & $z_1$ & $z_2$ & $z_3$ & $z_4$ \\ \hline
		0     & 0     & -     & -     & 0     & 0     & 0     & 0     \\
		-     & -     & 0     & 0     & 0     & 0     & 0     & 0     \\ \hline
		0     & 1     & 0     & 1     & 0     & 0     & 0     & 1     \\
		      &       & 1     & 0     & 0     & 0     & 1     & 0     \\
		      &       & 1     & 1     & 0     & 0     & 1     & 1     \\ \hline
		1     & 0     & 0     & 1     & 0     & 0     & 1     & 0     \\
		      &       & 1     & 0     & 0     & 1     & 0     & 0     \\
		      &       & 1     & 1     & 0     & 1     & 1     & 0     \\ \hline
		1     & 1     & 0     & 1     & 0     & 0     & 1     & 1     \\
		      &       & 1     & 0     & 0     & 1     & 1     & 0     \\
		      &       & 1     & 1     & 1     & 0     & 0     & 1     \\
	\end{tabular}
\end{center}

Come possiamo vedere anche dalle formule che seguono, per il calcolo di $z_1$ è sufficiente una
porta \verb|AND| mentre per il calcolo di $z_3$ sono necessarie 4 porte \verb|AND| e 1 porta
\verb|OR|. C'è quindi un ritardo tra il calcolo di $z_1$ e $z_3$ e il ritardo complessivo è dovuto
al passaggio del calcolo da 2 livelli di porte logiche.
\begin{align*}
	z_1 & = x_1 x_2 y_1 y_2                                                                   \\
	z_3 & = x_1 \bar{y_1} y_2 + x_2 y_1 y_2 + \bar{x_1} x_2 y_1 + x_1 \bar{x_2} y_1 \bar{y_2}
\end{align*}
A questo punto possiamo disegnare una mappa di Karnaugh per ogni uscita $z_i$ che abbiamo.
Limitiamoci per il momento a disegnare solo quelle di $z_1$ e $z_3$.

\begin{figure}[!ht]\centering
\resizebox{0.8\textwidth}{!}{
	\begin{subfigure}[b]{0.5\textwidth}
	\begin{karnaugh-map}[4][4][1][$y_2$][$y_1$][$x_2$][$x_1$]
	\minterms{15}
	\maxterms{0,1, 3, 2, 4, 5, 7, 6, 12, 13, 14, 8, 9, 11, 10}
	\implicant{15}{15}
	\end{karnaugh-map}
	\end{subfigure}
	\begin{subfigure}[b]{0.5\textwidth}
	\begin{karnaugh-map}[4][4][1][$y_2$][$y_1$][$x_2$][$x_1$]
	\minterms{6, 7, 15, 13, 9, 10}
	\maxterms{0, 1, 3, 2, 4, 5, 12, 14, 8, 11}
	\implicant{7}{15}
	\implicant{7}{6}
	\implicant{13}{9}
	\implicant{10}{10}
	\end{karnaugh-map}
	\end{subfigure}
}
\end{figure}

In alternativa, considerando che una moltiplicazione tra due sequenze di 2 bit si svolge in questo
modo
\begin{center}
	\begin{tabular}{c c c c}
		  & 1 & 1 & $\times$ \\
		  & 1 & 0 & =        \\ \hline
		  & 0 & 0 & +        \\
		1 & 1 & - & =        \\ \hline
		1 & 1 & 0
	\end{tabular}
\end{center}
possiamo notare che se il bit al moltiplicatore è 0 allora avremo tutti 0 mentre se abbiamo 1 il
risultato sarà esattamente il moltiplicando. Possiamo quindi calcolarci separatamente
$x_1 x_2 \cdot y_1$ e $x_1 x_2 \cdot y_2$ tramite due multiplexer di questo tipo
\begin{center}
	\begin{tikzpicture}
		\draw[thick] (-1.25, 1) -- (1.25, 1) -- (0.75, 0) -- (-0.75, 0) -- cycle;
		\node (mux) at (0, 0.5) {MUX};
		\draw (-0.5, 1.5) node[label=left:$x_1 x_2$] {} to[short, o-] (-0.5, 1);
		\draw (0.5, 1.5) node[label=right:$00$] {} to[short, o-] (0.5, 1);
		\draw (-1.75, 0.5) node[label=left:$y_i$] {} to[short, o-] (-1, 0.5);
		\draw [->, >=Stealth] (0, 0) -- (0, -0.625);
	\end{tikzpicture}
\end{center}
Dove $x_1 x_2$ è un ingresso da 2 bit e dove l'altro ingresso è la costante 00. Questo multiplexer
effettua esattamente la scelta di cui abbiamo parlato prima: se $y_i = 0$ dà come risultato 00, se
invece $y_i = 1$ dà come risultato $x_1 x_2$.

Per implementare un moltiplicare $2 \times 2$ dobbiamo sostanzialmente affiancare due di questi
multiplexer, uno per $y_1$ e uno per $y_2$, aggiungere degli zeri dove necessario e poi effettuare
una somma con un bit di riporto $r$ che viene messo in cima alla sequenza generata. Il circuito che
ne risulta è un qualcosa di questo tipo
\begin{center}
	\begin{tikzpicture}[scale=0.7]
		\draw[thick] (-4.25, 1) -- (-1.75, 1) -- (-2.25, 0) -- (-3.75, 0) -- cycle;
		\node (mux) at (-3, 0.5) {MUX};
		\draw (-3.5, 1.5) node[label=left:$x_1 x_2$] {} to[short, o-] (-3.5, 1);
		\draw (-2.5, 1.5) node[label=right:$00$] {} to[short, o-] (-2.5, 1);
		\draw (-4.75, 0.5) node[label=left:$y_1$] {} to[short, o-] (-4, 0.5);
		\draw (-3, 0) -- (-3, -1) node[label=above left:$z_1 z_2$] {};

		\draw[thick] (4.25, 1) -- (1.75, 1) -- (2.25, 0) -- (3.75, 0) -- cycle;
		\node (mux) at (3, 0.5) {MUX};
		\draw (2.5, 1.5) node[label=left:$x_1 x_2$] {} to[short, o-] (2.5, 1);
		\draw (3.5, 1.5) node[label=right:$00$] {} to[short, o-] (3.5, 1);
		\draw (4.75, 0.5) node[label=right:$y_2$] {} to[short, o-] (4, 0.5);
		\draw (3, 0) -- (3, -1) node[label=above right:$z_3 z_4$] {};

		\draw (-1.5, -0.5) node[label=right:$0$] {} to[short, o-] (-1.5, -1) to[short, -*] (-2.25, -1);
		\draw (-3, -1) -- (-2.25, -1) -- (-2.25, -1.75) -- (-0.5, -1.75) -- (-0.5, -2.5);

		\draw (1.5, -0.5) node[label=left:$0$] {} to[short, o-] (1.5, -1) to[short, -*] (2.25, -1);
		\draw (3, -1) -- (2.25, -1) -- (2.25, -1.75) -- (0.5, -1.75) -- (0.5, -2.5);

		\draw[thick] (-1, -2.5) rectangle (1, -3.75);
		\node (add) at (0, -3.125) {ADD};
		\draw (0, -3.75) -- (0, -4.25);
		\draw (-1, -3.125) -- (-1.5, -3.125) -- node[label=left:$r$] {} (-1.5, -4.25);
	\end{tikzpicture}
\end{center}
In questo modo, dato che, come abbiamo visto in precendenza, sia il multiplexer che l'addizionatore
sono implementati tramite un circuito a due livelli di porte logiche abbiamo in totale un circuito
costituito da quattro livelli di porte logiche.

Questo si traduce in un maggior numero di componenti e in un maggior tempo di elaborazione ma in
compenso facciamo uso di due componenti standard che abbiamo già implementato e non dobbiamo
ricorrere alla costruzione di tabelle di verità, mappe di Karnaugh ecc.
