\chapter{Input/Output}
Parliamo in ultima battuta di I/O e del ruolo che ricopre all'interno del modello Von Neumann che
abbiamo visto all'inizio del corso. Nello specifico andremo a trattare \textbf{Memory Mapped I/O},
\textbf{DMA} e \textbf{interruzioni}.

\section{Memory Mapped I/O}
Con il \textbf{Memory Mapped I/O} riusciamo a vedere, all'interno dello spazio di indirizzamento
della nostra macchina, la memoria delle \textbf{periferiche} come tastiere, mouse, dischi ecc.

Possiamo vedere ognuna delle periferiche come oggetti hardware con una parte di controllo e una
parte operativa. Al loro interno possiamo trovare delle piccole unità di memoria che servono a
comunicare con il calcolatore.

Tramite il Memory Mapped I/O il processore è in grado di leggere o scrivere le locazioni di memoria
di queste periferiche. Nei processori RISC come ad esempio quelli con architettura ARM quello che
viene fatto è avere una parte dello spazio di indirizzamento dedicata alle operazioni di I/O.
\begin{center}
	\includesvg[inkscapelatex=false, scale=0.9] {circuiti/mem_map_IO.svg}
\end{center}
Quello che viene fatto è utilizzare la MMU per ricevere o inviare le informazioni da o verso le
varie periferiche, utilizzanod un bus collegato ad esse. La MMU è dunque incaricata di capire se
l'indirizzo ricevuto appartiene allo spazio di indirizzamento per le operazioni di I/O o meno.

La MMU non decide però con quale delle periferiche comunicare. Per riuscire a recapitare le
informazioni alla periferica corretta è necessario confrontare il prefisso degli indirizzi privati
della periferica con quello dell'informazione in viaggio sul bus. Se il confrontatore ritorna 1
allora abbiamo trovato la periferica desiderata.
\begin{center}
	\includesvg[inkscapelatex=false] {circuiti/bus_periferiche.svg}
\end{center}
Il problema è che l'interazione con le periferiche fatta in questo modo richiede un ciclo in cui
ogni volta interroghiamo la periferica per constatare lo stato della periferica.

Considerando inoltre che il tempo di accesso della CPU è dell'ordine dei nanosecondi, mentre quello
del disco (per esempio) è dell'ordine dei millisecondi, è facile constatare che perdiamo milioni
di cicli di clock per constatare lo stato della periferica.
