\section{Interruzioni}
Vogliamo quindi ottimizzare l'interazione con i dispositivi di I/O e per farlo entrano in gioco le
\textbf{interruzioni}. Un modo per permettere a tali dispositivi di notificare al processore che
sono passati in uno stato di \emph{pronto}.

Fino ad ora abbiamo eseguito le varie operazioni in \textbf{modalità utente} ma ogni processore ha
a disposizione diversi \textbf{stati di esecuzione} che differiscono sostanzialmente per i alcuni
privilegi di cui possono godere o meno. Tra le modalità disponibili in un processore ARM abbiamo:
\begin{itemize}
	\item \textbf{User}: modalità normale.
	\item \textbf{FIQ}: modalità \emph{fast interrupt} per gestire interruzioni brevi.
	\item \textbf{IRQ}: modalità normale per la gestione di interruzioni.
	\item \textbf{Supervisor}: privilegi massimi utilizzato per le routine del sistema operativo.
\end{itemize}
Questi stati del processore sono rappresentati nella \textbf{parola di stato}, ossia il registro
\verb|cpsr| che, se ricordiamo mantiene nei primi 4 bit lo stato dei flag. Negli ultimi 4 bit è
rappresentato lo stato in cui si trova il processore e nei tre bit precedenti abbiamo altre
informazioni come
\begin{itemize}
	\item \verb|I| e \verb|F|: bit per sapere se dobbiamo gestire le interruzioni. Il bit \verb|F|
	      ci fa entrare in modalità FIQ mentre il bit \verb|I| ci fa entrare in modalità IRQ.
	\item \verb|T|: serve ad entrare in modalità THUMB per la gestione di interruzioni che generano
	      codice più compatto.
\end{itemize}
Quando ci troviamo in modalità FIQ i primi 8 registri rimangono inalterati mentre da \verb|r8| fino
a \verb|lr| abbiamo una copia. Memorizziamo inoltre la parola di stato precedente al passaggio in
modalità FIQ in un registro apposito.

In modalità FIQ gli unici registri ad essere copiati sono lo stack pointer e il link register
poiché la gestione di interruzioni più lunghe potrebbe richiedere l'utilizzo di più registri.
Un'interruzione causa innesca dunque un procedimento di gestione in cui
\begin{enumerate}
	\item Vengono salvati program counter e parola di stato.
	\item La parola di stato viene settata opportunamente.
	\item \`E possibile disabilitare ulteriori interruzioni settando la parola di stato.
	\item Esecuzione del codice all'indirizzo dato dall'indirizzo dell'interruzione.
\end{enumerate}
Quando invece terminiamo il trattamento dell'interruzione dobbiamo ripristinare la parola di stato,
se necessario riabilitiamo le interruzioni e ripristiniamo il program counter.

Quando si verifica un'interruzione il processore passa in modalità \textbf{interruzione}, la quale
interrompe ciò che il processore stava facendo fino a quel momento per permettergli di gestire
l'interruzione. In questa modalità è inoltre possibile utilizzare il Memory Mapped I/O.

Viene inoltre messo in esecuzione un \textbf{handler} delle interruzioni che, nel caso la gestione
dell'interruzione sia troppo complessa, delega ad \textbf{processo gestore}.

\subsection{Direct Memory Access}
Quando abbiamo parlato di come avviene la comunicazione dal processore verso i dispositivi di I/O
abbiamo parlato di un bus lungo il quale viaggiano le informazioni. Abbiamo inoltre parlato delle
modalità con cui questi dispositivi sono in grado di generare interruzioni per far sì che il
processore prenda visione di ciò che sta accadendo e gestisca di conseguenza tali interruzioni.

Tramite la tecnologia \textbf{DMA} (Direct Memory Access) riusciamo ad avere una gestione delle
interruzioni più snella andando ad alleggerire il carico del processore.

Tutto quello che dobbiamo fare è aggiungere un bus che collega direttamente le periferiche di I/O
alla memoria.
\begin{center}
	\includesvg[inkscapelatex=false, scale=0.9] {circuiti/dma.svg}
\end{center}
Quando il processore per esempio vuole effettuare una lettura dal disco, invece di gestire
personalmente la lettura delega il lavoro alla periferica. Lo fa specificando l'indirizzo di un
buffer in memoria principale in cui il disco può andare direttamente a scrivere.

In questo modo il processore non prende in carico la lettura dei dati dal disco e continua ad
eseguire le istruzioni con cui era impegnato in precedenza. Una volta che il disco termina di
scrivere quanto richiesto in memoria principale, lo segnala al processore tramite un'interruzione e
questo sarà dunque in grado di leggere direttamente da quest'ultima.
