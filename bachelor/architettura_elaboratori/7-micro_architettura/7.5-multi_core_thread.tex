\section{Multicore e multithreading}
Una nuova classe di processori è quella dei processori \textbf{multithread} e \textbf{multicore}
(o multiprocess), i quali amplificano ancora di più il grado di parallelismo spaziale ma in un modo
differente dai superscalari.

Entrambi i modelli gestiscono flussi di controllo \textbf{indipendenti} ma differiscono per lo
spazio di indirizzamento che vanno ad accedere.
\begin{itemize}
	\item I processori multithread eseguono istruzioni appartenenti allo stesso spazio di
	      indirizzamento \textbf{condiviso}.
	\item I processori multicore eseguono istruzioni appartenenti a spazi di indirizzamento
	      \textbf{differenti}.
\end{itemize}
A differenza del superscalare che provava a eseguire più istruzioni contemporaneamente ma comunque
era legato al flusso di esecuzione del nostro programma, questi modelli si propongono di eseguire
flussi di esecuzioni differenti in modo indipendente.

\subsection{Multithreading}
Il modello multithread cerca di eseguire più flussi in contemporanea che però condividono lo stesso
spazio di indirizzamento. Nella pratica andiamo a replicare lo stato architetturale $n$ volte. Se
per esempio avessimo un livello di multithreading pari a 2 avremmo grosso modo un modello in cui
parti come \verb|PC| e \verb|REG| sono doppie ma in cui la memoria rimane unica.
\begin{center}
	\includesvg[inkscapelatex=false, scale=0.7] {circuiti/multithread.svg}
\end{center}
Quello che viene fatto è alternare, secondo politiche scelte in fase di progettazione, l'esecuzione
di uno o l'altro flusso di esecuzione utilizzando unità di controllo che determinano il
comportamento dei multiplexer.

\subsection{Dipendenze}
Si potrebbe pensare che uno stile architetturale del genere possa essere ancora più pesante dal
punto di vista delle dipendenze ma, in realtà, può addirittura alleggerire il carico sulle unità di
controllo dedicate alla gestione delle dipendenze.

Una prima politica per il cambio del flusso di esecuzione è detta \textbf{Hyper-threading}, la
quale, ad ogni ciclo di clock alterna i vari flussi di esecuzione. In questo modo riusciamo ad
allontanare le istruzioni che hanno delle dipendenze tenendo occupato il processore nell'esecuzione
di istruzioni appartenenti ad un altro flusso d'esecuzione.

Con l'Hyper-threading abbiamo l'effetto di eliminare gli stalli da 1 clock e di dimezzare stalli da
2 cicli clock.

Sebbene l'Hyper-threading sia un buon modo per alleggerire il peso delle dipendenze, esiste un'altra
politica che prevede l'esecuzione continua dello stesso flusso di esecuzione finché non ci si
imbatte in una dipendenza che causerebbe uno stallo. In quel caso cambiamo flusso di esecuzione e
proseguiamo allo stesso modo.

\subsection{Multicore}
Il modello multicore nasce dall'esigenza di eseguire più processi contemporaneamente. Abbiamo
infatti più \textbf{core} che a loro volta posso implementare del multithreading e possono essere
di tipo superscalare, pipeline e così via.

Ognuno di questi core lavora indipendentemente dall'altro e, sebbene tutti i core accedano alla
stessa memoria, lo spazio di indirizzamento accessibile dai singoli core è differente.

L'effetto che si vuole ottenere con un processore multicore è quello di avere un processore per
ogni processo. Ovviamente non è possibile se pensiamo a quanti processi lavorano contemporaneamente
su un computer ma, tramite meccanismi che si occupano di distribuire i vari processi tra i core in
modo più efficace possibile otteniamo una buona simulazione.

\subsubsection{Multicore ARM}
Dato che i processori ARM sono spesso utilizzati su dispositivi mobili si è adottato una politica
chiamata \textbf{big.LITTLE}, che prevede un'eterogeneità tra i core della macchina per ottimizzare
il consumo energetico delle batterie.

Abbiamo infatti core più potenti per calcoli più pesanti e che però consumano più energia, che
vengono spenti quando non sono necessari. Abbiamo poi core meno potenti che invece sono sufficienti
a gestire le funzionalità più semplici di dispositivi come ad esempio smartphone. Se necessario
tutti i core possono lavorare simultaneamente per avere la massima potenza di calcolo.
