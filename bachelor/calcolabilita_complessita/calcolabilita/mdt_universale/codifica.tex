\section{Codifica}
Per la costruzione della nostra MdT universale useremo la
codifica \emph{unaria}, ossia codificheremo i vari stati, simboli
ecc. tramite una sequenza di $\mid$.

Dato che abbiamo tre nastri dobbiamo definire una funzione di
transizione che tiene di conto dello stato della macchina (che
rimane unico) e dei tre simboli su cui si trovano rispettivamente
i tre cursori. In generale possiamo dire che, ad ogni passo di
computazione, la macchina, si basa sullo stato corrente e su
ognuno dei tre simboli indicati dal cursore. Sempre ad ogni passo
(in generale) ogni cursore potrebbe muoversi, e lo stato potrebbe
cambiare.

Per codificare una MdT abbiamo bisogno di due funzioni di
codifica, una per la codifica di una generica macchina $M$ e una
per la codifica del dato $w$ su cui questa opera. Una volta
codificata $M$ e il dato $w$ su cui opera, possiamo concatenare
le due codifiche per ottenere una codifica completa su cui la
MdT universale $U$ che andremo a costruire potrà operare.

\subsection{Codifica per stati, simboli e direzioni}
Supponiamo di avere i seguenti insiemi ausiliari
\begin{gather*}
	Q_* = \{ q_0, q_1, \dots \} \\
	\Sigma_* = \{ \sigma_0, \sigma_1, \dots \}
\end{gather*}
con $h \notin Q_*$ e $L, R, - \notin \Sigma_*$ (che non sono
quelli su cui opera $U$). In questo modo ogni MdT $M_k$ avrà
l'insieme degli stati $Q_k$ e l'insieme dei simboli $\Sigma_k$
inclusi rispettivamente in $Q_*$ e $\Sigma_*$.

Il prossimo passo consiste nel rappresentare gli elementi di
$Q_*$ e $\Sigma_*$ come stringhe generate dalla concatenazione
del simbolo $\mid$. Vogliamo quindi definire una funzione di
codifica
\[
	\kappa : Q_* \cup \{ h \} \cup \Sigma_* \cup \{ L, R, - \}
	\rightarrow \{ \mid \}^*
\]
in grado di codificare stati, simboli e direzioni in questo modo:
\begin{gather*}
	q_i \mapsto \mid^{i+2}      \quad h \mapsto \mid \\
	\sigma_j \mapsto \mid^{j+4} \\
	L \mapsto \mid \quad R \mapsto \mid^2 \quad - \mapsto \mid^3
\end{gather*}
Il motivo per cui $q_i \mapsto \mid^{i+2}$ è che abbiamo già $h$
codificato come $\mid$ e dunque, partendo da $q_0$ abbiamo che
$q_0 \mapsto \mid^2 = \mid \mid$. Discorso analogo per la
codifica dei simboli.

\`E immediato notare però che la funzione $\kappa$ non è
biunivoca in quanto sia $h$ che $L$ vengono codificati in $\mid$
(così come tanti altri stati e simboli che condividono la stessa
codifica). In realtà a noi interessa considerare $\kappa$
ristretta all'insieme di appartenenza dell'oggetto che vogliamo
codificare, ecco che la funzione diventa biunivoca.

\subsection{Codifica delle tuple}
Arrivati a questo punto è lecito chiedersi su quale alfabeto
opera $U$. Sappiamo che fa uso del simbolo "$\mid$" ma da solo
non permette di fare granché. Definiamo quindi il seguente
alfabeto
\[ \{ \mid, c, d, \#, \start \} \]
sotto l'ipotesi che l'intersezione tra questo e
$\Sigma_* \cup Q_*$ sia vuota. Come possiamo notare ci sono due
simboli ($c$ e $d$) sconosciuti. Il loro significato sarà chiaro
a breve. Prima facciamo il punto della situazione: abbiamo una
funzione $\kappa$ che, se ristretta ad un certo insieme, è in
gado di codificare stati, simboli e direzioni in una stringa
composta da un certo numero di "$\mid$".

\subsubsection{Codifica di una MdT}
Quel che vogliamo fare ora è codificare tuple intere. Per farlo
possiamo certamente usare la funzione $\kappa$ e codificare ogni
singolo elemento della tupla. Il primo problema che salta
all'occhio è che una volta codificata una tupla, non c'è modo di
capire dove inizia o dove finisce la codifica di un certo stato,
simbolo o direzione. L'altro problema è che non tutte le tuple
sono definite per la funzione $\delta$. Ecco che vengono
introdotti $c$ e $d$:
\begin{itemize}
	\item Il simbolo $c$ ha la funzione di separare le codifiche
	      di stati, simboli e direzioni. Senza questo, la
	      codifica ci restituirebbe una stringa di "$\mid$"
	      concatenate, che rappresentano l'intera tupla e sarebbe
	      quindi impossibile capire quando inizia o finisce la
	      codifica di ciascun oggetto.
	\item Il simbolo $d$ serve a codificare i casi in cui
	      la funzione $\delta$ non è definita per certi stati
	      iniziali e per certi simboli sul nastro. In quei casi
	      la codifica restituisce il simbolo $d$.
\end{itemize}
Fatte queste premesse, per riuscire a codificare una generica
MdT $M$ definita come segue
\[ M = (Q, \Sigma, \delta, s) \]
con $s$ stato iniziale, dobbiamo prima ordinare l'insieme degli
stati
\[ Q = \{ q_{i_1}, \dots, q_{i_k} \} \]
e l'insieme dei simboli
\[ \Sigma = \{ \sigma_{j_1}, \dots, \sigma_{j_l} \} \]
in modo tale che, presi due numeri $p$ e $p'$ vale che
\[
	p \leq p' \implies i_p \leq i_{p'} \quad
	\land \quad j_p \leq j_{p'}
\]
% digressione sugli indici di Q e Sigma --------------
Non è ben chiaro perché si usino gli indici $i$ e $j$ se poi c'è
un ulteriore indice. Non ci stiamo nemmeno riferendo all'insieme
degli stati della macchina $i$ o $j$ perché appunto gli indici
sono diversi. Forse è solo una questione di notazione che non
conosco. Anche per l'ordinamento di stati e simboli non sembra
esserci bisogno di $i$ e $j$.

Il doppio pedice per stati e simboli potrebbe portare un po' di
confusione. Quando consideriamo ad esempio l'insieme degli stati
$Q = \{ q_{i_1}, \dots, q_{i_k} \}$ probabilmente ci stiamo
riferendo ad un certo insieme $Q_i$ con i vari stati $q_i$ a cui
viene dato un ordine (il secondo pedice) ma non ne sono
assolutamente certo.
% Fine digressione -------------

Ora che abbiamo definito l'ordinamento degli elementi di $Q$ e
$\Sigma$ siamo in grado di codificare le varie tuple, tramite la
seguente funzione $S_{p,q}$ che, dato lo stato $q_{i_p}$ e il
simbolo $\sigma_{j_q}$ si può comportare in due possibili modi:
\begin{itemize}
	\item Nel caso in cui $\delta$ sia definita per lo stato e
	      il simbolo in input, ovvero se
	      \[
		      \delta (q_{i_p}, \sigma_{j_q}) =
		      (q, \sigma, D)
	      \]
	      abbiamo che
	      \[
		      S_{p,q} = c \; \kappa (q_{i_p})
		      \; c \; \kappa (\sigma_{j_q})
		      \; c \; \kappa (q)
		      \; c \; \kappa (\sigma)
		      \; c \; \kappa (D) \; c
	      \]
	\item Altrimenti, se $\delta(q_{i_p}, \sigma_{j_q})$ non è
	      definita, abbiamo
	      \[
		      S_{p,q} = c \; \kappa (q_{i_p})
		      \; c \; \kappa (\sigma_{j_q})
		      \; c \; d \; c \; d \; c \; d \; c
	      \]
\end{itemize}
Siamo quindi in grado di codificare la funzione $\delta$ tramite
le funzioni $\kappa$ e $S_{p,q}$.

\begin{example}
	Consideriamo la MdT $\hat{M}$ definita in questo modo
	\[
		\hat{M} = (\{ q_2 \}, \{ \sigma_1, \sigma_3, \sigma_5 \},
		\delta, q_2)
	\]
	dove
	\begin{gather*}
		\delta(q_2, \sigma_1) = (h, \sigma_5, -) \\
		\delta(q_2, \sigma_3) = (q_2, \sigma_1, R)
	\end{gather*}
	In questo caso abbima un solo stato e dunque $k=1$ implica
	che $i_1=2$. Abbiamo invece tre simboli e quindi $l = 3$
	implica che $j_1 = 1$, $j_2 = 3$ e $j_3 = 5$. Possiamo quindi
	codificare $\hat{M}$ come segue:
	\begin{gather*}
		S_{1,1} = c \mid^4 c \mid^5 c \mid c \mid^9 c \mid^3 c \\
		S_{1,2} = c \mid^4 c \mid^7 c \mid^4 c \mid^5 c \mid^2 c \\
		S_{1,3} = c \mid^4 c \mid^9 c \; d \; c \; d \; c \; d \; c
	\end{gather*}
	Come possiamo vedere nell'ultimo caso la funzione $\delta$
	non è definita su $(q_2, \sigma_5)$ e dunque la codifica
	del risultato sarà una serie di $d$.
\end{example}

Ci manca solo la codifica dello stato iniziale $s$, che andremo
semplicemente a porre davanti alla codifica di $\delta$ per
riuscire a definire la funzione $\rho$ che permette di mandare
una generica MdT $M$ in una stringa fatta di soli caratteri
$\mid$, $c$ e $d$:
\[ \rho(M) = c \; \kappa(s) \; c \; S_{1,1} \dots S_{k,l} \; c \]
Tale funzione è invertibile ed è dunque possibile, tramite una
stringa codificata, tornare alla sua rappresentazione iniziale.

\subsubsection{Codifica del dato in ingresso}
Quello che manca adesso è codificare il dato in ingresso
$w = \sigma'_0 \dots \sigma'_n$. Se ci pensiamo, il dato in
ingresso, altro non è che una stringa di simboli, i quali
sappiamo già codificare tramite la funzione $\kappa$. Possiamo
quindi definire una funzione $\tau$, definita come segue
\[
	\tau(\sigma'_0 \dots \sigma'_n) =
	c \; \kappa(\sigma'_0) \;
	c \; \kappa(\sigma'_n) \; c
\]
che concatena le codifiche dei simboli per comporre la stringa
rappresentate il dato in input. L'ultimo passo della codifica
consiste nel concatenare i risultati delle funzioni $\rho$ e
$\tau$ in questo modo
\[ \rho(M) \tau(w) \]
ottenendo così la codifica completa della macchina e del suo
input. Si noti che tre $c$ separano la codifica di $M$ da quella
$w$, mentre abbiamo due caratteri $c$ che separano tra loro le
tuple della funzione $\delta$.
