\section{Funzioni calcolabili}
D'ora in avanti proveremo a capire
\begin{itemize}
	\item Quali sono le \textbf{funzioni calcolabili} e di
	      quali proprietà godono.
	\item Se esistono funzioni totali o parziali che non sono
	      calcolabili. Ovvero per cui si dimostra che non esiste
	      un algoritmo che le calcoli.
\end{itemize}
Per farlo andremo a focalizzarci più sul concetto di funzione
rispetto al concetto di algoritmo andando quindi ad analizzare
\emph{cosa} si calcola e non \emph{come} lo si calcola.

\begin{example}
	Prendiamo ora come esempio la
	\textbf{congettura di Goldbach}, la quale ci dice che ogni
	numero pari maggiore di 2 è esprimibile come somma di due
	numeri primi. Da questa congettura (mai dimostrata) nasce
	la \textbf{funzione di Goldbach}, definita come segue con
	$gb : \N \to \N$
	\[
		gb(n) = \begin{cases}
			0 & \text{se la congettura è vera} \\
			1 & \text{altrimenti}
		\end{cases}
	\]
	La congettura non è stata ancora dimostrata ma un algoritmo
	per calcolarla esiste, solo che non sappiamo quale sia.

	Se ad esempio volessimo decidere se la funzione è
	T-calcolabile, basterebbe prendere una MdT che ritorna
	sempre 0 se la congettura è vera o una MdT che ritorna
	sempre 1 se è falsa. Il problema è che fin tanto che la
	congettura non è dimostrata, non sappiamo quale delle due
	scegliere.
\end{example}

