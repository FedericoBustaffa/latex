\chapter{Risultati classici}
Arriviamo finalmente al sodo di questa prima parte del corso in
cui abbiamo definito formalismi su formalismi senza capire bene
come, quando o perché applicarli.

Introdurremo quindi alcuni risultati della teoria della
calcolabilità che ci permetteranno di caratterizzare la classe
delle funzioni calcolabili, mediante alcuni teoremi di
\emph{"chiusura"}.

Prima di iniziare chiariamo che, grazie alla
\hyperref[th: church-turing]{tesi di Church-Turing}, possiamo
chiamare \emph{calcolabili} tutte le funzioni che rispettano
le cinque condizioni intuitive poste agli algoritmi definite
nell'\hyperref[sec: algoritmo]{idea intuitiva di algoritmo},
indipendentemente dal loro formalismo.
