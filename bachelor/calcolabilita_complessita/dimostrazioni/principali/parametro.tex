\subsection{Teorema del parametro}

Esiste una funzione $s$ calcolabile totale e iniettiva tale che
$\forall i, x, y$ vale
\[ \lambda y . \varphi_i(x, y) = \varphi_{s(i,x)} (y) \]

\subsubsection{Dimostrazione}

La dimostrazione si basa sul costruire una funzione $s$
calcolabile, totale e iniettiva, tale per cui la funzione
$\varphi_i$ valutata su $x$ e $y$, nel caso in cui il valore
$x$ sia fissato (sia cioè un \textbf{parametro}), restituisca
lo stesso risultato della funzione indicizzata dal valore di
$s(i, x)$.

In altre parole, nel momento in cui ci troviamo una funzione
a 2 parametri e uno di questi sia fissato vogliamo trovare una
funzione che ci ritorna lo stesso risultato ma in cui il
parametro è parte della funzione stessa. Prendiamo l'esempio
della funzione somma
\[ f(x, y) = x + y \]
e supponiamo che $x$ sia sempre fissato a 2. Possiamo sia
calcolare $f(2, y)$ sia andare a cercare una seconda funzione
$g$ che prende un solo parametro $y$ ed è definita in questo
modo:
\[ g(y) = 2 + y \]
In sostanza stiamo definendo una funzione che prende un numero
ed è solo in grado di sommargli 2.

Per svolgere la dimostrazione replichiamo quindi quanto fatto
con l'esempio. I passi che seguono (da quanto ho capito) hanno
come intuizione la definizione di una nuova macchina
"modificata".

Dato che a primo membro abbiamo $\varphi_i (x,y)$ possiamo
seguire lo stesso schema di sempre e dire che la macchina $M_i$
calcola $\varphi_i$. Ma tale macchina prende ancora 2 argomenti
ignoti a priori. Noi ci stiamo ponendo nell'ipotesi in cui $x$
sia fissato a priori e ne conosciamo quindi il valore.

Quello che stiamo cercando di fare è recuperare una macchina
$M_j$, che prende un solo argomento $y$ e la quale (ad esempio)
non ha l'istruzione che legge la variabile $x$ ma $x$ è
direttamente scritta dentro la definizione della macchina $M_j$.
Per trovare tale indice $j$ possiamo
\begin{enumerate}
	\item Recuperare l'$i$-esima macchina $M_i$.
	\item Scrivere sul nastro di tale macchina il parametro $x$
	      e la variabile $y$.
\end{enumerate}
Da notare che $x$ è un parametro e dunque non abbiamo bisogno di
\emph{leggerlo}. Non abbiamo cioè bisogno di controllare quanto
cose come la sua lunghezza o il suo tipo ecc. Il parametro $x$
è dato e viene scritto direttamente sul nastro.

Questa procedura definisce un algoritmo, in particolare
l'algoritmo di indice $j$, il quale è identificato in funzione
di $i$ e $x$ in quanto ci serve $i$ per recuperare l'$i$-esima
macchina e ci serve $x$ il quale essendo noto viene scritto
immediatamente sul suo nastro. A questo punto è facile
convincersi che la procedura algoritmica che termina sempre e
quindi esiste $s$ calcolabile totale tale che
\[ j = s(i, x) \]
In particolare $s$ è
\begin{itemize}
	\item Calcolabile poiché, come abbiamo appena detto, è
	      possibile definire una procedura algoritmica che la
	      calcola e che termina sempre.
	\item Totale poiché è definita per ogni input. Non esistono
	      infatti un indice $i \in \N$ e un input $x$ tali per
	      cui la funzione $s$ non è definita. Siamo sempre in
	      grado di trovare l'$i$-esima macchina e scrivere $x$
	      sul suo nastro.
\end{itemize}
Ci rimane da dimostrare che $s$ è iniettiva. Per essere
iniettiva abbiamo bisogno che non si verifichi mai una
situazione di questo genere
\[ s(i, x) = s(i', x') \]
Per il padding lemma sappiamo però che esiste un'infinità
numerabile di algoritmi che calcolano la stessa funzione o, se
preferiamo, esiste un'infinità numerabile di indici che
identificano una macchina in gradi di calcolare la stessa
funzione. Premesso ciò è dunque possibile costruire una funzione
$s'$ tale che
\[ \varphi_{s(i, x)} = \varphi_{s'(i, x)} \]
che genera indici in modo strettamente crescente. Deve quindi
valere che
\[ s'(i_0, x_0) < s'(i_1, x_1) \]
dove la codifica della coppia $(i_0, x_0)$ è minore della
codifica della coppia $(i_1, x_1)$. Come codifica possiamo ad
esempio prendere la coda di colomba che mappa una coppia di
naturali in un un'unico naturale.