\subsection{TOT si riduce secondo rec a INF}

Per dimostrare che $TOT \leq_{rec} INF$ ricordiamoci prima
come sono definiti i vari insiemi

\begin{gather*}
	TOT = \{ x \mid \dom (\varphi_x) = \N \} \\
	INF = \{ x \mid \dom (\varphi_x) \text{ è infinito} \} \\
	rec = \{ \varphi_x \mid \forall y \in \N . \varphi_x (y) \downarrow \}
\end{gather*}

Dire quindi che $TOT \leq_{rec} INF$ significa che esiste una funzione
$f \in rec$ tale che
\[ \forall x \in TOT \implies f(x) \in INF \]
Dire che $x \in TOT$ equivale a dire che $\dom (\varphi_x)$ è l'insieme
di tutti i naturali, dire invece che $f(x) \in INF$ equivale a dire che
il dominio della funzione $\varphi_{f(x)}$ è infinito. L'obbiettivo della
dimostrazione è quindi quello di trovare la $f$ tale per cui sia vera
quest'ultima cosa. Iniziamo con il definire la funzione
\[
	\psi (x, y) = \begin{cases}
		1                 & \text{se } \forall z < y .
		\varphi_x(z) \downarrow                        \\
		\text{indefinita} & \text{altrimenti}
	\end{cases}
\]
Come prima vogliamo trovarci nella situazione in cui se $x \in TOT$ allora
scegliamo il primo ramo. Ma dire che $x \in TOT$ equivale a dire che il
dominio della funzione $\varphi_x$ è tutto $\N$, ossia che $\varphi_x$ è
totale. Se la funzione è totale vale che per un qualsiasi valore di $y$ e
per ogni $z < y$ vale che $\varphi_x(z)$ converge. In sostanza stiamo dicendo
che preso $y$ tutti i numeri tra $0$ e $y$ possono essere dati in pasto
a $\varphi_x$ e otterremo sempre un risultato. Questo può essere fatto per
un qualsiasi $y$ e dunque la funzione è totale.

Ci chiediamo ora se $\psi$ è calcolabile. Intuitivamente possiamo
prendere l'$x$-esima macchina $M_x$ e calcolare $M_x (z)$ per ogni
$z < y$.
\begin{itemize}
	\item Se converge ogni volta allora cadiamo nel primo ramo e
	      $\psi(x, y) = 1$.
	\item Se diverge anche solo una volta cadiamo nel secondo ramo e
	      $\psi(x,y)$ è indefinita.
\end{itemize}
Abbiamo quindi trovato una procedura che calcola $\psi$ che
termina sempre e dunque la funzione è calcolabile. Dato che la
funzione è calcolabile allora (per Church-Turing) ha un indice
$i$ e possiamo quindi scrivere
\[ \psi(x, y) = \varphi_i (x, y) \]
A questo punto possiamo applicare il teorema del parametro per
ottenere
\[ \psi(x, y) = \varphi_i (x, y) = \varphi_{s(i,x)} (y) \]
Se notiamo che $i$ è costante (è fissato perché la funzione
$\psi$ è fissata), possiamo scrivere
\[
	\psi(x, y) = \varphi_i (x, y) =
	\varphi_{s(i,x)} (y) = \varphi_{f(x)} (y)
\]
Ecco che abbiamo ritrovato la stessa struttura che avevamo messo
in evidenza all'inizio. Avevamo detto che se $x \in TOT$ allora
$f(x) \in INF$. Dire che $f(x) \in INF$ equivale a dire che
$f(x)$ è l'indice di una funzione ($\varphi_{f(x)}$) il cui dominio
è infinito.

Vogliamo quindi dimostrare che se $x \in TOT$, allora $f(x)$ è
l'indice di una funzione con dominio infinito. Vediamo quindi che
succede se $x \in TOT$
\[ x \in TOT \implies \forall z < y . \varphi_x (z) \downarrow \]
e quindi
\[ \psi (x, y) = \varphi_{f(x)} (y) = 1 \]
e quindi come possiamo vedere $f(x)$ è l'indice di una funzione
$\varphi_{f(x)}$ il cui dominio è infinito in quanto per ogni $y$
vale che $\varphi_x$ è una funzione totale e dunque cadiamo sempre
nel primo caso. Notiamo come in questo caso il dominio della funzione
$\psi = \varphi_{f(x)}$ non solo è infinito ma $\N$ stesso. Possiamo
quindi concludere che $f(x) \in INF$. Per terminare dobbiamo dimostrare
che
\[ x \notin TOT \implies f(x) \notin INF \]
Seguiamo la stessa catena di implicazioni:
\[ x \notin TOT \implies \exists z < y . \varphi_x (z) \uparrow \]
e dunque
\[ \psi (x, y) = \varphi_{f(x)} (y) = \text{indefinita} \]
E dunque in questo caso $f(x)$ è l'indice di una funzione
indefinita e dunque non totale (requisito necessario) affinché
$f(x) \in INF$. Concludiamo quindi che
\[ x \notin TOT \implies f(x) \notin INF \]