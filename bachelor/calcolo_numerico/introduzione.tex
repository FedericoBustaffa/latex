\chapter{Introduzione}
Il corso tratta la risoluzione di problemi matematici mediante \textbf{algoritmi numerici} implementati
ed eseguiti su un calcolatore.

Per riuscire a farlo, il procedimento di base è il seguente
\begin{enumerate}
	\item Formulazione del problema matematico.
	\item Costruzione di un algoritmo numerico di risoluzione.
	\item Implementazione dell'algoritmo in un qualche linguaggio di programmazione.
	\item Esecuzione dell'algoritmo tramite un calcolatore.
\end{enumerate}
L'implementazione di un algoritmo attraversa le due fasi principali di \textbf{sintesi} e \textbf{analisi}.
Entrambe necessarie per comprendere come il calcolatore approssima i risultati calcolati tramite l'algoritmo.
Come vedremo, il risultato che un calcolatore fornisce, potrebbe essere diverso dal risultato atteso.

Dato che vogliamo usare \emph{algoritmi numerici}, vogliamo capire
\begin{itemize}
	\item Come la macchina \emph{manipola} i numeri.
	\item Come sono rappresentati i numeri in macchina.
	\item Come è rappresentata l'aritmetica in macchina.
	\item Quali \textbf{errori} si generano nell'esecuzione dell'aritmetica.
\end{itemize}
I problemi che andremo a trattare sono principalmente di tre tipi:
\begin{itemize}
	\item Risoluzione di \textbf{sistemi lineari}.
	\item Risoluzione di \textbf{equazioni non lineari}.
	\item Calcolo di \textbf{autovalori} di una matrice.
\end{itemize}