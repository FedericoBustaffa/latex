\section{Condizionamento di sistemi lineari}
Ora che abbiamo chiarito il concetto di norma riprendiamo il discorso fatto all'inizio del capitolo e poniamoci
il problema di studiare il condizionamento di
\[ A x = b \]
Per semplicità supponiamo di perturbare solo il termine noto ottenendo
\[ A \hat{x} = b + \delta b \]
Come sempre consideriamo che, in generale, quando si va ad inserire una quantità in macchina questa non sarà un
numero di macchina. Se abbiamo quindi il vettore
\[ b = \begin{bmatrix} b_1 \\ \vdots \\ b_n \end{bmatrix} \]
in macchina avremo il vettore
\[
	\hat{b} = \begin{bmatrix} \hat{b}_1 \\ \vdots \\ \hat{b}_n \end{bmatrix} =
	\begin{bmatrix} b_1 (1 + \epsilon_1) \\ \vdots \\ b_n (1 + \epsilon_n)	\end{bmatrix} =
	\begin{bmatrix} b_1 \\ \vdots \\ b_n \end{bmatrix} +
	\begin{bmatrix} b_1 \epsilon_1 \\ \vdots \\ b_n \epsilon_n \end{bmatrix}
\]
Per riuscire a stimare quanto $x$ sia perturbato nel calcolo in macchina calcoliamo
\[
	\frac{\| \hat{x} - x \|}{\| x \|} =
	\frac{\| A^{-1} b + A^{-1} \delta b - A^{-1} b \|}{\| x \|} =
	\frac{\| A^{-1} \delta b \|}{\| x \|}
\]
Dato che siamo interessati ad una maggiorazione, utilizzando la norma matriciale indotta, possiamo scrivere
\[ \frac{\| A^{-1} \delta b \|}{\| x \|} \leq \frac{\| A^{-1} \| \cdot \| \delta b \|}{\| x \|} \]
Ci rimane da sostituire il denominatore. Per farlo teniamo di conto che $A x = b$ e quindi
\[ \| b \| \leq \| A \| \cdot \| x \| \Rightarrow \| x \| \geq \frac{\| b \|}{\| A \|} \]
Otteniamo quindi
\[
	\frac{\| A^{-1} \| \cdot \| \delta b \|}{\| x \|} \leq
	\frac{\| A^{-1} \| \cdot \| \delta b \|}{\frac{\| b \|}{\| A \|}} \leq
	\| A \| \cdot \| A^{-1} \| \cdot \frac{\| \delta b \|}{\| b \|}
\]
Questa relazione ci dice che l'errore relativo sul risultato è dato dall'errore relativo sui dati
($\| \delta b \| / \| b \|$) amplificato dalla quantità $\| A \| \cdot \| A^{-1} \|$ detta
\textbf{numero di condizionamento} del sistema lineare che indichiamo con
\[ \K (A) = \| A \| \cdot \| A^{-1} \| \]
Se questa quantità è grande il problema è \emph{mal condizionato}, in caso contrario il problema è
\emph{ben condizionato}.