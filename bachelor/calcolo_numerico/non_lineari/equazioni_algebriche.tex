\section{Equazioni algebriche}
Un problema di rilevante interesse applicativo è quello del calcolo delle radici di un'equazione algebrica
a coefficienti reali del tipo
\[ f(x) = p(x) = p_n x^n + p_{n-1} x^{n-1} + \dots + p_1 x + p_0 = 0 \]
con $p_i \in \R$ e $p_n \neq 0$. \`E ben noto che l'equazione ammette $n$ radici eventualmente complesse
contate con le loro molteplicità. Per la determinazione di alcune di queste radici reali il metodo Newton
può essere utilizzato e richiede ad ogni iterazione una valutazione del polinomio e della sua derivata.
Il seguente algoritmo di Horner è impiegato per il calcolo
\begin{lstlisting}[language=matlab]
function [px, dx] = horner(p, x0)
	n1 = length(p);
	n = n1 - 1;
	px = p(n + 1);
	dx = 0;
	for k = n : -1 : 1
		dx = px + x0 * dx;
		px = p(k) + x0 * px;
	end
end
\end{lstlisting}
Se vogliamo però un'approssimazione di tutte le soluzioni reali e complesse dell'equazione ci conviene
riformulare il problema in termini di calcolo degli autovalori di un'opportuna matrice, detta
\emph{companion}, associata al polinomio $p(x)$. Tale matrice è formata come segue
\[
	F(p) = \begin{bmatrix}
		0                & 1      &        &                        \\
		                 & \ddots & \ddots &                        \\
		                 &        & 0      & 1                    & \\
		-\frac{p_0}{p_n} & \cdots & \cdots & -\frac{p_{n-1}}{p_n}
	\end{bmatrix}
\]
Per calcolare gli autovalori dobbiamo calcolare il determinante di $\lambda I - F(p)$, il quale risulta
\[ \det(\lambda I - F(p)) = \frac{p(x)}{p_n} \]
Abbiamo dunque ricondotto il calcolo delle radici dell'equazione algebrica al caso del calcolo degli
autovalori della matrice $F(P)$.