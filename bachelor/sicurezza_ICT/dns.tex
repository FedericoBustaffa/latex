\chapter{Sicurezza DNS}
\section{DNS}
Il \textbf{DNS} o \textbf{Domain Name System} è un database gerarchico che associa nomi degli host ad indirizzi IP.

Esso permette ad un utente di trovare un sistema senza conoscere il suo indirizzo IP. Per riuscire a fornire questo
tipo di servizio organizza la rete in \textbf{domini} organizzati logicamente come un albero invertito.

La \textbf{risoluzione} dei nomi è affidata ad un sistema distribuito. I membri del sistema sono detti
\textbf{Name Server} ed ogni applicazione deve contattare un Name Server per risolvere un nome.

Ogni autorità deve avere almeno un Name Server che gestisce la risoluzione dei nomi per quel dominio.

\subsection{Name Server e Resolver}
Un \textbf{Name Server} mantiene un database delle informazioni sugli host per il suo dominio e contatta il Name
Server di altri domini quando deve reperire informazioni (come l'indirizzo IP) sull'host a cui ci si vuole connettere.

Le informazioni devono essere aggiornate quando quelle dell'host cambiano. Questi aggiornamenti dinamici cambiano i
dati del DNS senza la necessità di ricostruire ogni altra parte dell'albero DNS.

\subsubsection{Risoluzione dei nomi}
La comunicazione con il Name Server è gestita dal \textbf{resolver}, il quale permette all'utente di fornire nomi
parziali, i quali vengono estesi per provare ad ottenere nomi completamente qualificati.

Non esiste però un metodo standard per gli host per individuare un Name Server sulla rete locale. In genere il nome
del Name Server è salvato su un file di configurazione oppure è prelevato da un servere DHCP.

Il Name Sever può usare due modalità differenti per rispondere ad una richiesta:
\begin{itemize}
	\item \textbf{Ricorsiva}: se non conosce la risposta interroga altri Name Server e poi passa la risposta al
	      client.
	\item \textbf{Iterativa}: Se non conosce la risposta passa al client l'indirizzo di un altro Name Server.
\end{itemize}
La modalità di risposta è scelta dal client tramite un flag nel messaggio di richiesta.

Uno dei problemi è la mancanza di autenticazione nella risposta che arriva dal Name Server.

\subsection{Cache}
Ogni name server mantiene in una \textbf{cache} tutte le corrispondenze tra nomi e indirizzi di cui viene a
conoscenza, permettendo così una risoluzione dei nomi direttamente dal name server locale.

Ogni corrispondenza viene mantenuta nella cache per un periodo di tempo fissato dal name server (in genere 2 giorni).

\section{Attacchi al DNS}
Il fatto che nei protocolli DNS manchi un sistema di autenticazione permette numerosi attacchi basati sullo scambio
di false informazioni.

Un meccanismo tra i più usati consiste nella ridirezione del traffico ad un nodo malizioso che può rubare informazioni
contando sulla fiducia tra i due nodi coinvolti.

Un possibile attacco è quello dello \textbf{shadow server} il quale si occupa di \emph{sniffare} le richieste e
produrre informazioni false.

Un altro attacco è il \textbf{cache poisoning} che consiste nel fornire dati falsi relativi a determinati nodi, i
quali vengono salvati nella cache e poi utilizzati dal name server.

\subsection{Autenticazione basata su nomi}
Alcune applicazioni utilizzano meccanismi di autenticazione e autorizzazione basate su nomi. Le quali associano
due utenti di due host differenti sulla base dei loro nomi.

L'accesso al computer remoto è garantito se l'utente remoto è dichiarato \emph{equivalente} all'utente locale e se
la richiesta dell'hostname coincide con quella contenuta nella definizione equivalente.

Visto quanto è semplice falsificare le informazioni relative ad un nodo, la prima cosa che si può fare è fornire
informazioni false al name server, spacciandosi così per un nodo \emph{fidato}. A questo punto si può accedere ad
un nodo remoto in maniera molto semplice.

\section{DNSSEC}
Con \textbf{DNSSEC} si provvede ad aggiungere un sistema di firma al sistema precedente. In questo modo si crea
un sistema di autenticazione crittografico in cui ogni name server possiede una coppia di chiavi pubblica e privata.

La chiave privata viene utilizzata dal name server per cifrare le informazioni che trasmette, mentre quella privata
per permettere l'autenticazione di tale name server. In questo modo si può verificare autenticità ed integrità del
messaggio e anche che certi domini (falsi) non esistono.

DNSSEC utilizza la crittografia solo per l'integrità e non per la confidenzialità.

\subsection{Problemi}
DNSSEC non è molto utilizzato per tre motivi principalmente:
\begin{itemize}
	\item C'è controllo sulle risposte ma non sulle domande fatte.
	\item Non c'è sicurezza nel dialogo tra DNS client e DNS server.
	\item La crittografia appesantisce troppo il carico di lavoro su macchine che già devono gestire computazioni
	      molto pesanti.
\end{itemize}