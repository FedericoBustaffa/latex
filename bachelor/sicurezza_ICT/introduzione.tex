\chapter{Introduzione}
L'obbiettivo del corso è quello di progettare sistemi robusti che non crollino al primo attacco o davanti al
primo utente ingenuo che ne fa uso.

Gran parte degli oggetti di uso quotidiano sono stati resi "intelligenti" dall'aggiunta di una componente
informatica al loro interno. Tali oggetti, anche se dall'esterno possono sembrare estremamente limitati, sono
sistemi completi e flessibili in grado di eseguire qualsiasi software.

L'aggiornamento online del software all'interno di tali sistemi è un mezzo per mantenere l'oggetto all'avanguardia
ma implica che esso possa essere manipolato da remoto per eseguire qualsiasi software e dunque per modificare
il comportamento dell'oggetto a proprio piacimento.

\section{Proprietà di sicurezza}
Un sistema informatico è, ad ogni livello di implementazione, formato da un insieme di moduli connessi, ognuno dei
quali offre un certo numero di operazioni.

Queste operazioni permettono di leggere e manipolare informazioni, che hanno poi un impatto sul mondo esterno.

In ogni sistema informatico ci sono regole (\textbf{politica di sicurezza}) che definiscono chi può invocare una
certa operazione e quindi ha il diritto di leggere o manipolare informazioni. Tali regole vengono implementate da
un sottoinsieme dei moduli del sistema informatico.

In questo contesto le tre principali proprietà che ci interessano sono:
\begin{itemize}
	\item \textbf{Confidenzialità}: solo chi ha il diritto di leggere una certa informazione può farlo.
	\item \textbf{Integrità}: solo chi ha il diritto di aggiornare una certa informazione può farlo.
	\item \textbf{Disponibilità}: chi ha un diritto e vuole esercitarlo riesce a farlo in un tempo finito.
\end{itemize}
Da queste tre proprietà fondamentali possiamo derivarne altre secondarie:
\begin{itemize}
	\item \textbf{Tracciabilità}: individuare chi ha invocato un'operazione.
	\item \textbf{Accountability}: addebitare l'uso delle risorse.
	\item \textbf{Auditability}: verificare l'efficacia dei meccanismi di \emph{enforcement} di una politica (come
	      viene realizzata).
	\item \textbf{Forensics}: provare che certe azioni hanno avuto luogo.
	\item \textbf{Privacy/GDPR}: individuare chi, come e se un utente può usare informazioni personali.
\end{itemize}

\section{Politiche di sicurezza}
Una \textbf{politica di sicurezza} è un insieme di regole definite dal proprietario del sistema o del processo
aziendale per decidere gli utenti che possono invocare un'operazione e quando possono farlo.

Esistono diverse categorie di politiche descrivibili come il risultato di due scelte indipendenti:
\begin{itemize}
	\item La prima scelta è relativa al come la politica viene descritta e al come essa gestisce situazioni ambigue:
	      \begin{itemize}
		      \item \textbf{Default allow}: il permesso sull'operazione viene negato negato.
		      \item \textbf{Default deny}: il permesso sull'operazione viene concesso.
	      \end{itemize}
	\item La seconda scelta definisce vincoli sul proprietario del sistema:
	      \begin{itemize}
		      \item \textbf{Discretionary Access Control}: il proprietario del sistema definisce i diritti e chi li
		            possiede.
		      \item \textbf{Mandatory Access Control}: esistono vincoli globali a tutto il sistema che nemmeno il
		            proprietario può violare.
	      \end{itemize}
\end{itemize}

\subsection{Soggetti e oggetti}
Relativamente alla seconda caratteristica che una politica deve avere, si cerca di modellare le risorse condivise
come \textbf{oggetti} e gli utenti come \textbf{soggetti}.

I soggetti invocano le operazioni definite dagli oggetti, se un oggetto invoca le operazioni definite da altri
oggetti allora diventa esso stesso un soggetto. In sintesi
\begin{itemize}
	\item \textbf{Soggetto}: in genere è un utente, un processo, un thread, un'istruzione.
	\item \textbf{Oggetto}: in genere si tratta di tipi di dati astratti, procedure, parametri, risorse logiche o
	      fisiche.
\end{itemize}

\subsection{Discretionary Access Control - DAC}
In questo modello per ogni oggetto esiste un \textbf{proprietario} (del sistema o del processo), il quale decide i
diritti dei vari soggetti mentre lui non ha vincoli di alcun tipo. Questo modello è tipico del mondo industriale.

\subsection{Mandatory Access Control - MAC}
Questo modello prevede la divisione di soggetti (utenti) e oggetti (risorse) in \textbf{classi}. Le classi sono
ordinate parzialmente in \textbf{livelli} (1, 2, 3 e così via).

Il livello di un soggetto esprime il grado di libertà che vogliamo lasciare a tale soggetto. Tanto più alto è il
livello di un soggetto tanto maggiore sarà il livello degli oggetti con cui esso può interagire.

Il livello di un oggetto esprime invece il grado di importanza di tale oggetto. Tanto più alto è il livello di un
oggetto tanto maggiore dev'essere è il livello del soggetto perché esso possa interagire con tale oggetto.

\subsubsection{Information Flow I}
In questo tipo di politica un utente può
\begin{itemize}
	\item Leggere tutti i file che hanno classe minore o uguale alla sua.
	\item Modificare i record dei file che hanno classe uguale alla sua.
	\item Appendere un record ad un file che ha classe maggiore della sua.
\end{itemize}
Per compiere queste operazioni è necessario il permesso dell'\emph{owner} che può solo restringere ulteriormente ciò
che un utente può fare. Si tratta di una politica MAC di tipo \textbf{no write down} e privilegia la confidenzialità.

\subsubsection{Information Flow II}
In questo tipo di politica un utente può
\begin{itemize}
	\item Scrivere tutti i file di una classe minore o uguale alla sua.
	\item Leggere tutti i file di una classe maggiore o uguale.
\end{itemize}
Questo implica che un utente poco affidabile, ossia di basso livello, non può andare a modificare dati critici. Si
tratta di una politica MAC di tipo \textbf{no write up} e privilegia l'integrità.

\subsubsection{Chinese Wall}
Gli oggetti del sistema sono partizionati in sottoinsiemi. L'utente che ha operato su un oggetto di un insieme non
può operare su oggetti di un altro insieme.

Questa politica \textbf{dinamica} permette di gestire conflitti di interesse ed è compatibile con la politica MAC
poiché aggiunge vincoli.

\subsubsection{Watermark}
Questa politica non prevede che un soggetto abbia un livello fissato ma che varia in base alle operazioni che esso
compie sui vari oggetti e dal livello di questi ultimi.

Il livello di un soggetto aumenta dopo che questo legge un oggetto di livello più alto del suo, rimane invece invariato
se il soggetto legge un oggetto di livello più basso.

\section{Matrice di controllo degli accessi}
Si tratta di una matrice con un comportamento molto dinamico che ha tante righe quanti sono i soggetti e tante colonne
quanti sono gli oggetti.

Nella posizione identificata dal soggetto $i$ e dall'oggetto $j$ si trovano i \textbf{diritti} che quel soggetto ha
su quell'oggetto. In generale è bene ai fini di sicurezza che la matrice contenga pochi diritti e che dunque appaia
sostanzialmente vuota.

Una rappresentazione concreta di tale matrice è necessaria ma non sufficiente per il rispetto della politica.

\subsection{Linux}
Nei sistemi operativi Linux vi è una rappresentazione concreta di tale matrice nel file system. Per ogni file, Linux
fornisce una sequenza di bit (bitmask) che indica i diritti che l'utente ha su tali file (lettura, scrittura,
esecuzione e così via).

\section{Trusted Computing Base - TCB}
Una caratteristica dei sistemi informatici è quella di avere al loro interno componenti informatici per implementare
e gestire la loro politica di sicurezza interna.

Se uno di questi componenti ha un errore o c'è un errore nei componenti che esso utilizza, allora l'implementazione
della politica non è corretta e quasi sicuramente un soggetto potrebbe invocare operazioni per le quali non ha i
diritti necessari.

\subsection{Dimensioni del TCB}
Per quanto riguarda le dimensioni del TCB possiamo dire che, tanto minore è il numero delle sue componenti, tanto
maggiore è la sicurezza del sistema.

Un TCB con dimensioni contenute permette anche una dimostrazione matematica relativamente semplice della sua
correttezza.

La dimensione del TCB è anche un criterio che permette di confrontare strategie alternative nella realizzazione della
politica di sicurezza.

\section{Vulnerabilità}
Quando si parla di \textbf{vulnerabilità} si vuole indicare un \emph{difetto}
\begin{itemize}
	\item Hardware
	\item Software
	\item Nell'utente
	\item Nelle regole della politica
\end{itemize}
che permette di violare la politica di sicurezza del sistema permettendo ad un soggetto di compiere un'operazione per
la quale non ha diritti.

L'obbiettivo della sicurezza informatica è quello di costruire sistemi che funzionano anche con delle vulnerabilità
e non quello di costruire sistemi senza vulnerabilità. In genere le varie vulnerabilità dei vari componenti vengono
\emph{compensate} in qualche modo da altri componenti.

La vulnerabilità più frequente, spesso non è nel codice, ma nel fatto che chi implementa il sistema dia per scontato
che l'utente non faccia errori nell'interfacciarsi con esso.

\subsection{Intrusioni}
Un'\textbf{intrusione} è una sequenza di \textbf{azioni} e \textbf{attacchi} per riuscire a controllare gli oggetti
del sistema.

Non tutte le azioni che l'attaccante fa sono \emph{illegali}, alcune di esse possono essere perfettamente lecite ma
sfruttate poi per violare il sistema.

In un'intrusione si sfrutta una o più vulnerabilità, usando anche programmi automatizzati (\textbf{exploit}) per ognuna
di esse, per riuscire a sostituirsi all'\emph{owner} del sistema e dunque avere la possibilità di
\begin{itemize}
	\item Raccogliere informazioni
	\item Modificare informazioni
	\item Impedire ad altri di accedere alle informazioni
\end{itemize}

\subsubsection{Fasi di un'intrusione}
Vediamo nello specifico cosa fa un hacker quando tenta un'intrusione:
\begin{enumerate}
	\item Raccoglie di informazioni iniziali sul sistema.
	\item Individuazione delle vulnerabilità del sistema per compiere un accesso iniziale.
	\item Ripetizione di una sequenza di operazioni finché non ha successo:
	      \begin{itemize}
		      \item Raccolta di informazioni sul sistema.
		      \item Scoperta di vulnerabilità.
		      \item Costruzione di exploit.
		      \item Attacco: si eseguono gli exploit ed eventuali azioni manuali.
	      \end{itemize}
	\item Installazione di strumenti per il controllo.
	\item Cancellazione delle tracce dell'attacco.
	\item Accesso, modifica ecc. ad un \emph{sottoinsieme} delle informazioni del sistema o si compiono altri
	      tipi attacchi:
	      \begin{itemize}
		      \item Furto di informazioni.
		      \item Cifratura di dati per chiedere riscatto.
	      \end{itemize}
\end{enumerate}
Un'intrusione può essere vista come un'\textbf{escalation} nell'acquisizione di diritti e nella raccolta di
informazioni tramite vari attacchi ripetuti.

Un attaccante in genere cerca di acquisire delle informazioni che gli permettano di avere nuovi diritti. Una volta
ottenuti i diritti è in grado di acquisire nuove informazioni e così via finché non raggiunge il proprio obbiettivo.

\subsection{Approcci alla sicurezza}
Le intrusioni sono dunque possibili grazie alle vulnerabilità e questo ci porta a definire due approcci alla sicurezza:
\begin{itemize}
	\item \textbf{Sicurezza incondizionale}: si assume che qualsiasi sia la vulnerabilità nel sistema, esista qualcuno
	      interessato ad usarla ed è quindi necessario eliminarle tutte.
	\item \textbf{Sicurezza condizionale}: in questo tipo di approccio si fa un'analisi in cui si cerca di capire quali
	      siano le reali minacce per il sistema e si eliminano solo le vulnerabilità che tali minacce potrebbero usare
	      per attaccare il sistema.
\end{itemize}