\chapter{Attacchi}
Prima di tutto chiariamo che un'intrusione è un insieme di azioni che si svolgono con fini differenti a seconda dei
casi, tra cui:
\begin{itemize}
	\item Raccolta di informazioni
	\item Persistenza
	\item Acquisizione di diritti
\end{itemize}
Un attacco consiste nel portare a compimento una delle azioni appena elencate. Possiamo quindi affermare che un
attacco è una delle parti di cui si compone un'intrusione. Il difensore deve sempre proteggersi dalle intrusioni e
non dai singoli attacchi.

\section{Classificazione degli attacchi}
Ogni attacco elementare è caratterizzato da
\begin{itemize}
	\item Diritti necessari
	\item Diritti acquisiti
	\item Probabilità di successo
	\item Rumore generato
\end{itemize}
Come già anticipato ogni attacco ha una precondizione (diritti necessari) e una post condizione (diritti acquisiti).

Altre proprietà descrivono in dettaglio le azioni per eseguire un'attacco permettendoci di generare una tassonomia
per individuare attacchi simili.

\section{Esempi di attacchi}
Vediamo ora alcuni esempi concreti di attacco informatico

\subsection{Replay attack}
Supponiamo che un client richieda un servizio ad un server tramite un messaggio su di un canale sicuro. L'attaccante
\begin{enumerate}
	\item \emph{Sniffa} il messaggio e lo salva.
	\item Prima che il canale venga distrutto invia più volte il messaggio al server.
\end{enumerate}
Se tutto va a buon fine il servizio richiesto viene fornito più volte. Viene spesso effettuato in ambiti che
coinvolgono transazioni di denaro, come la richiesta di bonifici ad una banca.

Per difendersi è fondamentale riconoscere il mittente del messaggio e avere un sistema che controlli se un messaggio è
già stato inviato.

\subsection{Man in the middle}
Tipico attacco alla crittografia in cui l'attaccante si frappone tra due utenti che vogliono comunicare.

All'atto pratico quello che succede è che quando utente \verb|A| vuole comunicare con un utente \verb|B|, un attaccante
\verb|C| si frappone fra loro intercettando le comunicazioni e fingendosi \verb|A| agli occhi di \verb|B| e \verb|B|
agli occhi di \verb|A|.

In questo modo \verb|C| non viene scoperto dai due utenti ed entra in possesso di tutti i messaggi scambiati fra i due
utenti.

\subsection{XSS}
Questo attacco è effettuabile quando un sito web permette ad utenti di memorizzare informazioni scaricabili in seguito
da altri utenti.

L'attaccante potrebbe quindi caricare del codice che attacca gli altri utenti e nel caso in cui il sito sia molto
popolare e non faccia alcun controllo sui dati caricati, un gran numero di utenti potrebbe essere soggetto ad attacchi.

\subsection{SQL injection}
Questo tipo di attacco è utilizzato per condurre attacchi a database in possesso di informazioni sensibili come ad
esempio delle password.

L'attaccante invia al database del codice che poi esegue delle \emph{query}. Per esempio quando viene chiesto ad un
utente di autenticarsi si invia del codice al posto dei dati di autenticazione che, se non sottoposto ad opportuni
controlli, potrebbe fornire dati personali all'attaccante o danneggiare il database.

Per difendersi da questi attacchi esistono funzioni built-in del linguaggio che permettono di ovviare al problema.

\subsection{Gerarchia di macchine virtuali}
I sistemi informatici sono implementati come una \textbf{gerarchia} di macchine virtuali, ossia come una serie di
strati, a partire dall'hardware proseguendo con la macchina assembler, il sistema operativo, la rete, le applicazioni
e così via.

Ogni macchina virtuale
\begin{itemize}
	\item Definisce un insieme di meccanismi implementati tramite un linguaggio di programmazione.
	\item Astrae e nasconde i meccanismi della macchina virtuale sottostante.
	\item Può essere standard con tutto quello che implica sulle vulnerabilità.
\end{itemize}
Il problema sta nel fatto che le vulnerabilità non si possono astrarre perché permettono di attaccare le macchine
virtuali che eseguono.

Una vulnerabilità di una certa macchina virtuale permette di attaccare tutte le macchine virtuali soprastanti.

\subsubsection{Blue Pill attack}
Una pratica molto usata è quella di attaccare i livelli più bassi di un sistema in modo da controllare tutti quelli
superiori ed è anche possibili andare ad inserire una nuova macchina virtuale nella gerarchia poiché si tratta di
un'operazione molto difficile da scoprire ma che può avere ripercussioni molto pesanti sulla sicurezza.

Tramite questo attacco il nuovo strato manipola le invocazione che lo strato soprastante vuole fare allo strato
sottostante andando a fornire informazioni false sugli strati sottostanti e inviando loro falsi comandi.