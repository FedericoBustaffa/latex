\section{Immagine di un'applicazione lineare}
La riduzione a scalini per colonna risulta molto utile anche per lo
studio di applicazioni lineari, ed in particolare per la
determinazione di dimensione e base dell'immagine di una applicazione
lineare.

Dati due spazi vettoriali $V$ e $W$ sul campo $\K$ di dimensione $n$
e $m$ rispettivamente, consideriamo una applicazione lineare:
\[
	L : V \to W
\]
Fissiamo una base $\{e_1, e_2, \dots, e_n\}$ di $V$ e una base
$\{\epsilon_1, \epsilon_2, \dots, \epsilon_m\}$ di $W$. Indichiamo con $[L]$
la matrice, di forma $m \times n$, associata a $L$ nelle basi scelte.

Per quanto abbiamo fin qui detto, possiamo, tramite un numero finito $k$ di
operazioni elementari sulle colonne di $[L]$, portarla in forma a scalini
ridotta. Ma c'è di più. Ogni operazione elementare sulle colonne corrisponde
a moltiplicare la matrice iniziale $[L] \in \Mat_{m \times n}(\K)$,
a destra, per una matrice $B$ di dimensioni $n \times n$ invertibile.

\begin{theorem}
	Siano $V, W, U$ spazi vettoriali su $\K$, fissiamo per ciascuno
	una base. Siano $T : V \to W$ e $S : W \to U$ applicazioni lineari. Allora,
	rispetto alle basi fissate, vale:
	\[
		[S \circ T] = [S][T]
	\]
	dove nel membro di destra stiamo considerando il prodotto righe per colonne
	fra matrici.
\end{theorem}

Tornando all'applicazione lineare $L$, sappiamo già che lo span delle colonne
di $[L]$ coincide con lo span delle colonne della matrice ottenuta portando $[L]$
in forma a scalini. Sappiamo cioè che $\Imm(L)$ coincide con l'immagine
dell'applicazione lineare associata alla matrice ottenuta portando i forma a
scalini $[L]$.

\begin{proposition}
	Siano $L$ ed $M$ due applicazioni lineari, vale
	\[ \Imm(L \circ M) = \Imm(L) \]
	ossia, scritto con un'altra notazione,
	\[ (L \circ M)(V) = L(V) \]
\end{proposition}

\begin{proposition}
	Sia $B : V \to V$ un'applicazione lineare invertibile, allora vale
	\[ \Imm(L) = \Imm(L \circ B) \]
	\begin{proof}
		Dato che $B$ è una funzione invertibile, è bigettiva, ossia
		\[ B(V) = V \]. Dunque
		\[
			\Imm(L \circ B) = L(B(V)) = L(V)
		\]
	\end{proof}
\end{proposition}

\begin{definition}
	Data una applicazione lineare $L : V \to W$, dove $V$ e $W$ sono due spazi
	vettoriali di dimensione finita sul campo $\K$ e sia $[L]$ la matrice
	associata ad $L$. Se riduco in forma a scalini $[L]$, il \textbf{rango} equivale
	al numero di pivot della matrice ottenuta.
\end{definition}

\begin{theorem}
	Data una applicazione lineare $L$ come sopra e fissate le basi, vale che
	il rango di $L$ è uguale al numero di colonne non nulle che si trovano
	quando si trasforma $[L]$ in forma a scalini.
\end{theorem}

\begin{observation}
	Osserviamo che il rango di una applicazione lineare $L$ è anche uguale al
	\textbf{massimo numero di colonne linearmente indipendenti} di $[L]$.
	Infatti sappiamo che $\Imm(L)$ è il sottospazio vettoriale di $W$ generato
	dai vettori colonna di $[L]$. Da questi vettori è possibile estrarre una
	base di $\Imm(L)$ e, inoltre, possiamo dire che \[\dim(\Imm(L)) = rango (L)\]
\end{observation}

\begin{observation}
	Possiamo ora definire un algoritmo in 3 passi che, data una
	applicazione lineare tra due spazi vettoriali $V$ e $W$, e fissata una base
	$\{e_1, e_2, \dots, e_n\}$ di $V$ e una base
	$\{\epsilon_1, \epsilon_2, \dots, \epsilon_m\}$ di $W$, ci permette di
	determinare la dimensione ed una base di $\Imm(L)$:
	\begin{enumerate}
		\item Scrivere la matrice $[L]$ associata ad $L$ rispetto alle basi
		      fissate
		\item Ridurre $[L]$ in forma a scalini
		\item Contare il numero di pivot per ottenere la dimensione di $\Imm(L)$.
	\end{enumerate}
\end{observation}

