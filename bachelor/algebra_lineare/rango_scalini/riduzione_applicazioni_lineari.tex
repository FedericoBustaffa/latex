\section{Riduzione a scalini e applicazioni lineari}

\begin{theorem}
	Data una applicazione lineare $L : V \to W$ tra due spazi vettoriali $V$ e
	$W$ sul campo $\K$, e data la matrice $[L]$ associata a $L$ rispetto
	a due basi fissate di $V$ e $W$:
	\begin{enumerate}
		\item Il massimo numero di righe linearmente indipendenti di $[L]$ è
		      uguale al massimo numero di colonne linearmente indipendenti di
		      $[L]$, ossia al rango di $L$.
		\item Se si riduce la matrice $[L]$ a scalini, sia che lo si faccia
		      per righe, sia che lo si faccia per colonne, il numero di scalini
		      sarà sempre uguale al rango di $L$.
	\end{enumerate}
\end{theorem}

\begin{observation}
	Se componiamo $L$ a destra o a sinistra per una	applicazione invertibile,
	il rango non cambia. Dunque, se moltiplichiamo $[L]$ a destra o a sinistra
	per matrici invertibili, anche il rango della matrice prodotto non cambia.
\end{observation}

\begin{proposition}
	Se $A$ è una matrice $m \times n$ a valori su un campo $\K$, ed
	indichiamo con $e_1, \dots, e_n$ le sue colonne, e $B$ è una riduzione
	a scalini per righe di $A$, allora le colonne di $A$ in corrispondenza alla
	posizione dei pivot di $B$ formano una base dello $Span$ delle colonne di $A$
	($Span(e_1, \dots, e_n)$).
\end{proposition}

\begin{observation}
	A partire dalla proposizione precedente possiamo ricavare un algoritmo per
	estrarre una base di uno spazio vettoriale da un insieme di generatori.

	Consideriamo $V$ uno spazio vettoriale di dimensione $n$ e del quale
	conosciamo una base $\{e_1, \dots, e_n\}$. Consideriamo $k$ vettori
	$v_1, \dots, v_k$ di $V$ e il sottospazio di $V$ generato da
	$Span(v_1, \dots, v_k)$. Se vogliamo estrarre una base di
	$Span(v_1, \dots, v_k)$ da $v_1, \dots, v_k$ seguiamo questi passaggi:
	\begin{enumerate}
		\item Scriviamo le coordinate dei $v_i$ rispetto alla base
		      $\{e_1, \dots, e_n\}$ e le	consideriamo come colonne di
		      una matrice $M$.
		\item Porto $M$ in forma a scalini per riga.
		\item I vari vettori $v_i$ con indice $i$ in corrispondenza dei pivot di $M'$
		      ($M$ in forma a scalini) formano una base di $Span(v_1, \dots, v_k)$,
		      estratta dall'insieme di generatori $\{v_1, \dots, v_k\}$.
	\end{enumerate}
\end{observation}