\section{Il teorema della dimensione del nucleo e dell'immagine di una applicazione lineare}
Il teorema del completamento, ha come importante corollario un risultato che stabilisce
una relazione tra la dimensione del nucleo e quella dell'immagine di una applicazione
lineare.

\begin{theorem}
	Considerata una applicazione lineare $L : V \to W$, dove $V$ e $W$ sono spazi
	vettoriali su $\K$, vale
	\[
		\dim(\Ker(L)) + \dim(\Imm(L)) = \dim(V)
	\]
\end{theorem}

\begin{definition}
	Una applicazione lineare bigettiva $L : V \to W$, tra due spazi vettoriali $V$ e $W$
	sul campo $\K$, si dice un \textbf{isomorfismo lineare}.

	Dal teorema precedente segue che:
	\begin{itemize}
		\item Se $L : V \to W$ è una applicazione lineare iniettiva allora
		      \[ \dim(\Imm(L)) = \dim(V) \]
		      Infatti sappiamo che $\Ker(L) = \{O\}$ dunque $\dim(\Ker(L)) = 0$.
		\item Se $L : V \to W$ è un isomorfismo lineare allora \[\dim(V) = \dim(W)\]
		      Infatti se $L$ è bigettiva, in particolare è iniettiva e surgettiva.
		\item Se $L : V \to W$ è una applicazione lineare iniettiva, allora $L$,
		      pensata come applicazione da $V$ ad $\Imm(L)$, è un isomorfismo lineare.
	\end{itemize}
\end{definition}

