\section{Riduzione a scalini per righe}
Data una matrice in $\Mat_{m \times n}(\K)$ è possibile definire le
operazioni elementari di riga in modo analogo alle mosse di colonna in questo
modo:
\begin{enumerate}
	\item Sommare alla riga $i$ la riga $j$ moltiplicata per uno scalare $\lambda$
	\item Moltiplicare la riga $i$ per uno scalare $\lambda$
	\item Permutare fra loro due righe, diciamo $i$ e $j$
\end{enumerate}
A questo punto possiamo definire la forma a scalini per righe di una matrice.
In questo caso chiameremo \emph{profondità} di una riga la posizione occupata,
contata da destra, dal suo coefficiente diverso da zero che sta più a sinistra
nella riga. La riga nulla ha \emph{profondità} 0.

\begin{definition}
	Una matrice $A$ in $\Mat_{m \times n}(\K)$, si dice \textbf{in forma
		a scalini per righe} se:
	\begin{itemize}
		\item Leggendo la matrice dall'alto verso il basso, le righe non nulle si
		      incontrano tutte prima delle righe nulle.
		\item Leggendo la matrice dall'alto verso il basso, le profondità
		      delle sue righe non nulle risultano strettamente decrescenti.
	\end{itemize}
\end{definition}

\begin{theorem}
	Data una matrice $A$ in $\Mat_{m \times n}(\K)$ è sempre possibile,
	usando un numero finito di operazioni elementari sulle righe, ridurre la
	matrice in forma a scalini per righe.
\end{theorem}

In particolare, anche quando abbiamo una matrice in forma a scalini per righe, si
possono definire i \textbf{pivot} della matrice, come i coefficienti più a
sinistra delle righe non nulle.

Inoltre anche in questo caso è possibile definire una forma a scalini
\emph{particolare}: la forma a \textbf{scalini per righe ridotta}.

\begin{example}
	matrice in forma a scalini per righe
	\[
		\begin{pmatrix}
			1 & 0 & 0 & -20 \\
			0 & 1 & 0 & 0   \\
			0 & 0 & 1 & 5
		\end{pmatrix}
	\]
\end{example}

\begin{corollary}
	Ogni matrice $A$ può essere trasformata, attraverso le operazioni elementari
	sulle righe, in una matrice in forma a scalini per righe ridotta.
\end{corollary}

In generale valgono tutte le proprietà già elencate per le operazioni sulle
colonne.

