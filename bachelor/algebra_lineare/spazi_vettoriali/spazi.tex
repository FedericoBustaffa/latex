\chapter{Spazi Vettoriali}

\section{Definizione di spazio vettoriale}
Per fornire la definizione di spazio vettoriale si ha bisogno di un insieme non vuoto $V$ e di un campo
$\K$, dove sia possibile definire le operazioni di \textbf{somma vettoriale} e
\textbf{prodotto per scalare}.

\begin{definition}
	Uno \textbf{spazio vettoriale su un campo} $\K$ è un insieme $V$ su cui sono definite la somma
	fra due elementi di $V$ (il cui risultato è ancora un elemento di V, si dice quindi che $V$ è chiuso
	per la somma), e il prodotto di un elemento di $\K$ per un elemento di $V$ (il cui risultato è
	sempre un elemento di $V$, si dice quindi che V è chiuso per il prodotto con elementi di $\K$)
	che verificano le seguenti \textbf{proprietà}:
	\begin{enumerate}
		\item \textbf{Associatività della somma}: $\forall u, v, w \in V$ vale
		      \[ (u + v) + w = u + (v + w) \]
		\item \textbf{Commutatività della somma}: $\forall v, w \in V$ vale
		      \[ v + w = w + v \]
		\item \textbf{Elemento neutro per la somma}: $\exists O \in V$ tale che $\forall v \in V$ vale
		      \[ v + O = v \]
		\item \textbf{Inverso per la somma}: $\forall v \in V$, $\exists w \in V$ tale che
		      \[ v + w = O \]
		\item \textbf{Distributività del prodotto per uno scalare}: $\forall \lambda, \mu \in \K$ e
		      $\forall v, w \in V$ vale
		      \[ \lambda(v + w) = \lambda v + \lambda w \]
		      e anche
		      \[ (\lambda + \mu)v = \lambda v + \mu v \]
		\item \textbf{Associatività del prodotto per uno scalare}: $\forall \lambda, \mu \in \K$ e
		      $\forall v \in V$ vale
		      \[ (\lambda \mu)v = \lambda(\mu v) \]
		\item \textbf{Invariante moltiplicativo}: $\forall v \in V$ vale
		      \[ 1v = v \]
	\end{enumerate}
\end{definition}

\begin{observation}
	L'elemento neutro della somma $O$ e lo $0$, elemento neutro di $\K$ sono due cose ben distinte, il primo è
	un vettore, il secondo è uno scalare.
\end{observation}

\begin{example}
	Ogni campo $\K$ è uno spazio vettoriale su $\K$ stesso con le operazioni di somma vettoriale e prodotto per
	scalare che sono definite identiche alle operazioni di somma e prodotto sul campo. In particolare $\R$ è uno
	spazio vettoriale su $\R$, così come $\Q$ è uno spazio vettoriale su $\Q$.
\end{example}

\begin{example}
	$\R^2 = \{(a, b) \mid a,b \in \R\}$ è uno spazio vettoriale su $\R$ con le operazioni di somma vettoriale e
	prodotto scalare definite come segue:
	\begin{align*}
		(a,b) + (c,d) =  & (a + c, b + d)         \\
		\lambda (a, b) = & (\lambda a, \lambda b)
	\end{align*}
\end{example}

\begin{example}
	Anche l'insieme dei polinomi $\K[x]$, con la somma tra polinomi e il prodotto tra polinomi e costanti di
	$\K$ definiti come segue:
	\begin{itemize}
		\item Il polinomio somma di $p(x)$ e $q(x)$ è quello il cui coefficiente di grado $n$ è la somma dei
		      coefficienti di grado $n$ dei polinomi $p(x)$ e $q(x)$.
		\item Il polinomio prodotto di $k \in \K$ e $p(x)$ è il polinomio che ha come coefficiente di
		      grado $n$ $k$ volte il coefficiente di grado $n$ di $p(x)$.
	\end{itemize}
	è uno spazio vettoriale su $\K$.
\end{example}
