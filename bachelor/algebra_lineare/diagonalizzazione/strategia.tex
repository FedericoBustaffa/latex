\section{Strategia per scoprire se un endomorfismo è diagonalizzabile}
Con il metodo descritto a breve si potrà scoprire se un endomorfismo
$T : V \to V$, dove $V$ è uno spazio vettoriale su $\K$ di dimensione $n$
è diagonalizzabile, e, in caso lo sia, si potrà trovare una base che lo diagonalizza,
ossia una base di $V$ fatta tutta di autovettori di $T$.

\begin{itemize}
	\item PASSO 1. Troviamo gli autovalori di un endomorfismo lineare $T$ calcolando il polinomio
	      caratteristico e le sue radici in $\K$.
	\item PASSO 2. Supponiamo dunque di aver trovato gli autovalori di $T$. A questo punto
	      vogliamo individuare gli autospazi relativi a tali autovalori.
	      Per questo basterà calcolare il $\Ker([T] - \lambda_i I)$. Prendiamo dunque la
	      matrice $[T] - \lambda_i I$ e risolviamo il sistema lineare omogeneo associato.
	\item PASSO 3. Per prima cosa enunciamo il seguente teorema
	      \begin{theorem}
		      Dato un endomorfismo $T : V \to V$, siano $\lambda_1, \dots, \lambda_k$
		      degli autovalori di $T$ distinti fra loro. Consideriamo ora degli autovettori
		      $v_1 \in V_{\lambda_1}, \dots, v_k \in V_{\lambda_k}$. Allora
		      $v_1, \dots, v_k$ è un insieme di vettori linearmente indipendenti.
	      \end{theorem}

	      Il seguente teorema è un rafforazamento del precedente.
	      \begin{theorem}
		      Dato un endomorfismo $T : V \to V$, siano $\lambda_1, \dots, \lambda_k$ degli
		      autovalori di $T$ distinti fra loro. Allora gli autospazi
		      $V_{\lambda_1}, \dots, V_{\lambda_k}$, sono in somma diretta.
	      \end{theorem}

	      Nelle ipotesi del teorema precedente sappiamo allora, che la dimensione della somma
	      degli autospazi è la massima possibile, ossia
	      \[
		      \dim(V_{\lambda_1} \oplus \cdots \oplus V_{\lambda_k}) = \dim(V_{\lambda_1}) +
		      \cdots + \dim(V_{\lambda_k})
	      \]

	      Osserviamo che abbiamo già un criterio per dire se $T$ è diagonalizzabile o no.
	      Ovvero, se
	      \[
		      \dim(V_{\lambda_1}) + \cdots + \dim(V_{\lambda_k}) = n = \dim(V)
	      \]
	      altrimenti se
	      \[
		      \dim(V_{\lambda_1}) + \cdots + \dim(V_{\lambda_k}) < n = \dim(V)
	      \]
	      $T$ non è diagonalizzabile. Infatti non è possibile trovare una base di
	      autovettori.
	\item PASSO 4. Se l'endomorfismo $T$ è diagonalizzabile, scegliamo allora una base di
	      autovettori nel modo descritto al Passo 3, e avremo una matrice associata $[T]$ che
	      risulterà diagonale. Mantenendo la notazione introdotta al Passo 3 troviamo sulla
	      diagonale $\dim(V_{\lambda_1})$ coefficienti uguali a
	      $\lambda_1$, ... e $\dim(V_{\lambda_k})$ coefficienti uguali a $\lambda_k$.
\end{itemize}
Il rango di $T$ sarà uguale al numero dei coefficienti non nulli che troviamo
sulla diagonale di $[T]$, la dimensione del nucleo sarà uguale al numero dei
coefficienti uguali a zero che troviamo sulla diagonale di $[T]$.

\begin{example}
	Consideriamo l'endomorfismo
	\[
		T \begin{pmatrix} x \\ y \end{pmatrix} =
		\begin{pmatrix}
			x + 2y \\
			-y
		\end{pmatrix}
	\]
	e la sua matrice associata rispetto alle basi standard
	\[
		[T] = \begin{pmatrix}
			1 & 2  \\
			0 & -1
		\end{pmatrix}
	\]
	Per prima cosa calcoliamo la matrice $[T] - tI$ che chiameremo $M$ per comodità
	\[
		M = \begin{pmatrix}
			t - 1 & -2    \\
			0     & t + 1
		\end{pmatrix}
	\]
	Troviamo il polinomio caratteristico calcolando il determinante di $M$ e otteniamo
	\[
		P_T(t) = (t - 1)(t + 1)
	\]
	Le radici di tale polinomio (e quindi gli autovalori di $T$) sono $t = 1$ e $t = -1$.
	Dobbiamo trovare quindi i relativi autospazi.
	\begin{itemize}
		\item Se $t = 1$ dobbiamo calcolare $\Ker([T] - 1I)$. Dobbiamo quindi risolvere il
		      sistema associato alla matrice
		      \[
			      \begin{pmatrix}
				      1 - 1 & -2    \\
				      0     & 1 + 1
			      \end{pmatrix} =
			      \begin{pmatrix}
				      0 & -2 \\
				      0 & 2
			      \end{pmatrix}
		      \]
		      ovvero
		      \[
			      \begin{cases}
				      -2y & = 0 \\
				      2y  & = 0
			      \end{cases} \quad \Rightarrow \quad
			      y = 0
		      \]
		      otteniamo dunque che l'autospazio $V_1$ è definito come segue
		      \[
			      V_1 = < \begin{pmatrix} 1 \\ 0 \end{pmatrix} > \quad \Rightarrow \quad
			      \dim(V_1) = 1
		      \]
		\item Se $t = -1$ procediamo in maniera analoga. Stavolta otteniamo il sistema
		      \[
			      \begin{cases}
				      -2x - 2y & = 0 \\
			      \end{cases} \quad \Rightarrow \quad
			      x = -y
		      \]
		      Ne deduciamo che l'autospazio $V_2$ sarà definito come segue
		      \[
			      V_2 = < \begin{pmatrix} -1 \\ 1 \end{pmatrix} > \quad \Rightarrow \quad
			      \dim(V_2) = 1
		      \]
	\end{itemize}
	Dato che $T$ è definta su $\R^2$ che ha dimensione 2 e dato che
	\[ \dim(V_1) + \dim(V_2) = 2 \] l'endomorfismo è diagonalizzabile.
\end{example}