\chapter{Lavori Correlati}\label{cap: related_works}

Allo stato dell'arte, il metodo \textit{LORE} sfrutta l'implementazione fornita
dalla libreria \textit{DEAP} di algoritmi genetici, la quale offre grande
flessibilità, permettendo all'utente di definire le strutture dati necessarie
a rappresentare cromosomi e popolazione e le varie fasi dell'algoritmo.
Mette inoltre a disposizione un metodo per riuscire a sfruttare il parallelismo
tramite il modulo \verb|multiprocessing|, riuscendo così a mantenere un alto
livello di espressività e massima compatibilità con moduli di terze parti.

\section*{DEAP}

La libreria \textit{DEAP} fornisce un'API molto flessibile ma richiede una fase
di inizializzazione in cui vengono definiti tipi, strutture dati e funzioni
necessarie al corretto funzionamento dell'algoritmo.

Come prima cosa è necessario definire la struttura di un generico cromosoma, il
tipo di ciascun gene e i pesi da assegnare a ciascun parametro da ottimizzare.
A tal fine vengono messi a disposizione il modulo \verb|creator| e la classe
\verb|Toolbox|.

Tramite \verb|creator| è possibile definire nuove classi con caratteristiche
personalizzate e che si adattano alle proprie esigenze.

\begin{lstlisting}[caption={Definizione fitness e strutture dati con DEAP}]
from deap import creator

creator.create("FitnessMin", base.Fitness, weights=(-1.0,))
creator.create("Individual", list, fitness=creator.FitnessMin)
\end{lstlisting}

La classe \verb|Toolbox| serve invece ad immagazzinare funzioni e relativi
parametri, di modo da poter essere utilizzati con facilità in seguito. Tramite
tale classe è infatti possibile definire operatori genetici e metodi di
inizializzazione, selezione e valutazione degli individui, ognuno dei quali
viene registrato e legato ad una stringa che ne definisce il nome.

\begin{lstlisting}[caption={Registrazione metodi e operatori con DEAP}]
import random
from deap import base, tools

toolbox = base.Toolbox()
toolbox.register("indices", random.sample, range(n), n)
toolbox.register("individual", tools.initIterate, creator.Individual, toolbox.indices)
toolbox.register("population", tools.initRepeat, list, toolbox.individual)
toolbox.register("mate", tools.cxOrdered)
toolbox.register("mutate", tools.mutShuffleIndexes)
toolbox.register("select", tools.selTournament, tournsize=2)
toolbox.register("evaluate", evaluate)
\end{lstlisting}

Ad eccezione della funzione di valutazione, che è strettamente legata al
problema di interesse e deve essere quindi definita dall'utente caso per caso,
\textit{DEAP} fornisce implementazioni per tutte le componenti appena citate,
lasciando quindi il solo compito di scegliere quale si addice di più al
problema in questione. Rimane comunque la possibilità di definire ciascuna
delle componenti dell'algoritmo come meglio si crede in caso di esigenze
particolari.

\subsection*{Algoritmo}\label{ssec: rw_algos}

Una volta definite tutte le componenti dell'algoritmo, è possibile invocare
direttamente i metodi registrati nell'istanza della classe \verb|Toolbox|,
andando a definire un comportamento personalizzato per esso. Una seconda opzione
consiste invece nel ricorrere alle implementazioni fornite da \textit{DEAP}
tramite il modulo \verb|algorithms|, che presenta diverse opzioni possibili,
tra cui \verb|eaSimple|, che ha una struttura molto simile a quella definita in
figura \ref{fig:simple_ga} ed è l'algoritmo attualmente impiegato da
\textit{LORE}.

L'algoritmo prende in input l'istanza della classe \verb|Toolbox|
precedentemente definita, una popolazione iniziale, il numero massimo di
iterazioni da compiere e le probabilità di crossover e mutazione. \`E possibile
passare anche parametri opzionali per registrare dati statistici sui valori di
fitness e una struttura dati per memorizzare i migliori $k$ individui mai
prodotti durante le varie generazioni, la \verb|HallOfFame|.

\begin{lstlisting}[caption={DEAP \lstinline|eaSimple|}]
import numpy as np
from deap import algorithms, tools

stats = tools.Statistics(key=lambda ind: ind.fitness.values)
stats.register("min", np.min)
stats.register("mean", np.mean)
stats.register("max", np.max)

hall_of_fame = tools.HallOfFame(20)
population = toolbox.population(n=100)
population, logbook = algorithms.eaSimple(
	population=population, toolbox=toolbox,
	cxpb=0.7, mutpb=0.3, ngen=50,
	stats=stats, halloffame=hall_of_fame
)
\end{lstlisting}

A differenza dello schema \ref{fig:simple_ga}, \verb|eaSimple| effettua una
prima valutazione degli individui subito dopo aver generato la popolazione
iniziale e prima di entrare nel ciclo.

\begin{figure}[H]
	\centering
	\includesvg[width=0.95\linewidth]{immagini/deap_ga.svg}
	\caption{Struttura dell'algoritmo eaSimple implementato da DEAP}
	\label{fig: deap_ga}
\end{figure}

In fase di selezione vengono scelti sempre $n$ individui senza alcun controllo
sui duplicati, rendendo possibile scegliere lo stesso individuo più volte.
L'operatore di crossover viene applicato con una certa probabilità \verb|cxpb|,
provocando la generazione di nuovi individui, nel caso venga applicato e la
clonazione dei genitori in caso contrario.

Segue poi la fase di mutazione che introduce una variazione nel cromosoma di
ciascun individuo con una probabilità \verb|mutpb| e, a seconda del metodo di
mutazione, potrebbe essere necessario specificare la probabilità di mutare
ciascun gene del singolo cromosoma.

I nuovi individui vengono infine valutati tramite la funzione di valutazione
registrata, la \textit{HallOfFame} viene aggiornata e la nuova generazione
prende il posto della vecchia, rimpiazzandola totalmente.

\subsection*{Multi-Processing}

La classe \verb|Toolbox| ha anche dei metodi predefiniti tra cui \verb|map|, la
cui implementazione è data dall'omonima funzione built-in di Python e che viene
utilizzata per applicare la funzione di valutazione a tutti gli individui. Per
riuscire a valutare gli individui in parallelo sarà quindi sufficiente
registrare nella classe \verb|Toolbox|, la funzione \verb|map| della classe
\verb|Pool|, offerta dal modulo \verb|multiprocessing|.

\begin{lstlisting}[caption={DEAP Multiprocessing}]
import multiprocessing as mp
pool = mp.Pool()
toolbox.register("map", pool.map)
\end{lstlisting}

Il metodo \verb|map| della classe \verb|Pool| applica una certa funzione a
tutti gli elementi di una struttura dati iterabile passata come argomento. Se
non vengono specificati ulteriori argomenti, gli elementi dell'iterabile
verranno distribuiti uno per volta ai processi non appena questi sono liberi.
Se invece viene specificato il parametro \verb|chunksize| è possibile definire
il numero di elementi che un processo deve elaborare prima di poterne ricevere
altri.

In questo modo l'algoritmo utilizza quest'ultima per valutare gli individui
sfruttando più worker e modificando quindi la sua struttura dell'algoritmo come
segue:

\begin{figure}[H]
	\centering
	\includesvg[width=0.9\linewidth]{immagini/parallel_eaSimple}
	\caption{Versione parallela di \lstinline|eaSimple| con 4 worker}
	\label{fig: parallel_eaSimple}
\end{figure}

In questo modo si abilita facilmente la computazione parallela ma ci si imbatte
in tutti i problemi di comunicazione tipici del multiprocessing, discussi in
precedenza.
