\chapter{Background}\label{cap: background}

Per comprendere le motivazioni dietro all'implementazione della libreria è
necessario analizzare il metodo \textit{LORE} di eXplainable AI (XAI), come
questo sfrutti un algoritmo genetico per la generazione di dati sintetici e
la natura dietro al problema di prestazioni riscontrato.

Si sono quindi messe a fuoco la struttura e le operazioni tipiche di un
algoritmo genetico, per riuscire a implementare una versione sequenziale che
fosse facilmente parallelizzabile. Durante lo sviluppo della versione parallela
dell'algoritmo ci si è però scontrati con il problema introdotto dal
\textit{GIL}, che in Python limita molto l'esecuzione simultanea di più thread,
almeno per quanto riguarda compiti CPU-bound. Si sono quindi individuate
possibili soluzioni al problema, che comportano però alcuni compromessi in
quanto a prestazioni o epsressività del codice.

\section{Algoritmi Genetici}\label{sec: gen_alg}

Gli algoritmi genetici fanno parte delle possibili tecniche euristiche adottate
nell'ambito dell'ottimizzazione. Costituiscono infatti una valida scelta in
molti casi in cui non si conosce un algoritmo efficiente per un problema
specifico e non si è interessati alla ricerca di soluzioni ottime per esso.

Il loro comportamento si ispira ad un modello evolutivo in cui gli individui
migliori di una popolazione sopravvivono e si riproducono, portando quindi la
popolazione iniziale ad uno \quotes{stadio superiore} con il passare delle
generazioni.

Le possibili strutture sono molto varie ma una delle più comuni prevede la
rappresentazione di ogni individuo tramite un cromosoma. Questo è generalmente
strutturato come un vettore o una stringa, i cui elementi rappresentano una
possibile soluzione al problema o una sua codifica. Una delle strutture più
semplici e comuni degli algoritmi genetici è composta dalle seguenti fasi:
\begin{enumerate}
	\item \textbf{Generazione}: viene generata in modo casuale una popolazione
	      di individui il cui cromosoma rappresenta una possibile soluzione al
	      problema.
	\item \textbf{Selezione}: viene selezionato un certo numero di individui
	      utilizzando metodi basati sul valore di \textit{fitness}, il quale
	      indica la qualità della soluzione.
	\item \textbf{Crossover}: i cromosomi di due o più individui vengono
	      combinati per generarne di nuovi, i quali condividono parte del
	      materiale genetico dei genitori.
	\item \textbf{Mutazione}: vengono applicate variazioni casuali al cromosoma
	      di un individuo, permettendo così l'esplorazione di nuove soluzioni e
	      favorendo così la diversità genetica all'interno della popolazione.
	\item \textbf{Valutazione}: ogni individuo viene valutato e gli viene
	      attribuito un valore di \textit{fitness} in base alla qualità della
	      soluzione che questo rappresenta.
	\item \textbf{Rimpiazzo}: viene attuata una politica di rimpiazzo (in
	      alcuni casi chiamata \quotes{di sopravvivenza}), che prevede la
	      scelta degli individui da mantenere da una generazione all'altra.
\end{enumerate}
Tutte le fasi tranne la prima sono eseguite in un ciclo che termina quando
viene soddisfatto un criterio di convergenza, che può essere definito in
termini dei valori di fitness della popolazione, della sua biodiversità o, più
semplicemente, definendo un limite massimo al numero delle iterazioni.

\begin{figure}[H]
	\centering
	\includesvg[width=0.95\linewidth]{immagini/simple_ga.svg}
	\caption{Struttura di un algoritmo genetico semplice}
	\label{fig:simple_ga}
\end{figure}

Data inoltre la natura stocastica dell'algoritmo, è possibile che le migliori
soluzioni vengano trovate dopo poche iterazioni. Se il metodo di convergenza
si basa semplicemente sul raggiungimento di un numero massimo di iterazioni e
la fase di rimpiazzo sostituisce completamente la vecchia generazione con la
nuova, si rischia di perdere alcune delle soluzioni migliori tra le varie
generazioni. È infatti buona norma avere una struttura dati in grado di
mantenere le migliori soluzioni mai generate, di modo da evitarne la perdita.
