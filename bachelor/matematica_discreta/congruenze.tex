\chapter{Congruenze e criteri di divisibilit\`a}
\section{Lemmi aritmetici}
Di seguito alcuni lemmi che ci saranno utili pi\`u tardi.

\begin{lemma}
	Siano $a, b, c \in \mathbb{Z}$ tali che
	\begin{equation*}
		a \mid bc \wedge (a, b) = 1
	\end{equation*}
	Allora $a \mid c$.
	\begin{proof}
		Se $(a, b) = 1$ allora esistono $x, y \in \mathbb{Z}$ tali che
		\begin{equation*}
			ax + by = 1
		\end{equation*}
		Moltiplico a sinistra e a destra per $c$ ottenendo:
		\begin{equation*}
			acx + bcy = c
		\end{equation*}
		Di sicuro $a \mid ac$ e quindi $a \mid acx$. Sappiamo anche che $a \mid bc$ e quindi
		$a \mid bcy$. Ne concludiamo che $a \mid c$ come voluto.
	\end{proof}
\end{lemma}

\begin{lemma}
	Sia $p$ un numero primo ($p > 1$). Allora
	\begin{equation*}
		\begin{array}{lll}
			p \mid ab & \Rightarrow & p \mid a \text{ oppure } p \mid b
		\end{array}
	\end{equation*}
	\begin{proof}
		Considero il MCD $(p, a)$. Posso avere due casi:
		\begin{enumerate}
			\item $(p, a) = 1$ quindi $p \mid b$.
			\item $(p, a) = p$ quindi $p \mid a$ poich\'e $p$ \`e un divisore comune tra
			      $a$ e $p$.
		\end{enumerate}
	\end{proof}
\end{lemma}

\begin{lemma}
	Se un numero $p > 1$ ha la propriet\`a
	\begin{equation*}
		\begin{array}{lll}
			p \mid ab & \Rightarrow & p \mid a \text{ oppure } p \mid b
		\end{array}
	\end{equation*}
	Allora $p$ \`e un numero primo.
	\begin{proof}
		Devo dimostrare che gli unici divisori positivi di $p$ sono $1$ e $p$.
		Sia $d$ un divisore di $p$. Ho che
		\begin{equation*}
			p = dt
		\end{equation*}
		Questo significa che $p \mid dt$ e quindi, per ipotesi, $p \mid d$ oppure $p \mid t$
		\begin{itemize}
			\item Se $p \mid d$ allora $d = py$, quindi $p = pyt$. Semplificando otteniamo
			      $yt = 1$. Sappiamo per\`o che $t > 0$ quindi $y = t = 1$. Ne ricaviamo
			      quindi che $p = d$.
			\item Se $p \mid t$ allora $t = pz$ quindi $p = dpz$. Semplifico e ottengo
			      $dz = 1$. Sappiamo che $d > 0$ quindi $d = 1$.
		\end{itemize}
	\end{proof}
\end{lemma}

\begin{lemma}
	Siano $a, b, c \in \mathbb{Z}$. Se $a \mid c$, $b \mid c$ e $(a, b) = 1$ allora
	\begin{equation*}
		ab \mid c
	\end{equation*}
	\begin{proof}
		Partiamo col dire che
		\begin{equation*}
			\begin{array}{lll}
				a \mid c & \Rightarrow & c = ax    \\
				b \mid c & \Rightarrow & b \mid ax
			\end{array}
		\end{equation*}
		Per il primo lemma possiamo affermare che $b \mid x$ quindi $x = by$. Quindi
		\begin{equation*}
			c = ax = aby
		\end{equation*}
		quindi $c$ \`e un multiplo di $ab$. Concludiamo dunque che $ab \mid c$.
	\end{proof}
\end{lemma}

\section{Congruenze}
Introduciamo adesso le congruenze
\begin{definition}
	Siano $a, b, c \in \mathbb{Z}$ con $c > 0$
	\begin{equation*}
		a \equiv b \mod{c}
	\end{equation*}
	si dice che $a$ \`e \textbf{congruo} $b$ modulo $c$ se $c \mid a - b$
\end{definition}

\begin{example}
	Consideriamo le seguenti congruenze
	\begin{equation*}
		\begin{array}{l}
			3 \equiv 8 \mod{5}  \\
			3 \equiv 13 \mod{5} \\
			3 \equiv 18 \mod{5}
		\end{array}
	\end{equation*}
	Se considero i numeri congrui a $3 \mod{5}$ vedo che si tratta di una successione
	aritmetica di passo 5.
	\begin{equation*}
		\dots, -7, -2, 3, 8, 13, 18, \dots
	\end{equation*}
\end{example}

\begin{observation}
	Come sappiamo, l'operatore $\mod{}$, restituisce il resto della divisione tra due interi e
	scrivere
	\begin{equation*}
		a \equiv b \mod{c}
	\end{equation*}
	equivale a scrivere
	\begin{equation*}
		a \mod{c} = b \mod{c}
	\end{equation*}
\end{observation}

Stiamo dunque affermando che due numeri $a, b$ sono congrui modulo $c$ se il resto della
divisione euclidea di $a$ per $c$ \`e uguale al resto della divisione euclidea di $b$ per $c$.
Possiamo dunque fare un'ulteriore passaggio. Se scriviamo
\begin{equation*}
	\begin{array}{rl}
		a & = cq + r  \\
		b & = cq' + r
	\end{array}
\end{equation*}
possiamo fare la differenza e ottenere
\begin{equation*}
	\begin{array}{rl}
		a - b & = cq - cq'  \\
		a - b & = c(q - q')
	\end{array}
\end{equation*}
Quindi $a - b$ \`e un multiplo di $c$, ovvero
\begin{equation*}
	\begin{array}{c}
		a - b \equiv 0 \mod{c} \\
		\Downarrow             \\
		a \equiv b \mod{c}
	\end{array}
\end{equation*}

\section{Relazioni e classi di equivalenza}

\begin{definition}
	Preso un insieme $A$ non vuoto, una \textbf{relazione di equivalenza} (indicata con $\sim$) in $A$,
	\`e ci\`o che mette in relazione elementi dell'insieme $A$ che condividono la stessa propriet\`a.
	Due elementi in $A$ che condividono la stessa propriet\`a descritta dalla relazione sono detti
	\textbf{equivalenti}.
\end{definition}

La formula
\begin{equation*}
	a \equiv b \mod{m}
\end{equation*}
ci dice che $a$ e $b$ hanno lo stesso resto nella divisione per $m$ e sono dunque
equivalenti ($a \sim b$) per questa specifica relazione.

\begin{definition}
	Preso un insieme $A$ e un suo elemento qualsiasi $a$. La \textbf{classe di equivalenza} associata
	ad $a$ \`e un sottoinsieme di $A$ (generalmente indicato con $[a]$) di cui fanno parte tutti gli
	elementi di $A$ equivalenti ad $a$ secondo la relazione $\sim$.
\end{definition}

Nel caso delle congruenze l'insieme $[a]_m$ \`e formato da tutti i numeri congrui ad $a$ modulo $m$.
\begin{equation*}
	[a]_m = \{ x \in \mathbb{Z} \mid x \equiv a \mod{m} \}
\end{equation*}
e in questo caso $[a]_m$ \`e chiamata \textbf{classe resto} di $a \mod{m}$.

\begin{definition}
	Dato $m \in \mathbb{Z}$ definisco l'\textbf{anello} degli interi modulo $m$ come segue
	\begin{equation*}
		\mathbb{Z}_m = \{ [0]_m, [1]_m, \dots, [m - 1]_m \}
	\end{equation*}
	In sintesi $m$ possiede $m$ classi resto poich\'e la divisione per $m$ pu\`o portare ad $m$ diversi
	possibili resti (da 0 a $m - 1$).
\end{definition}

\section{Propriet\`a}
La congruenza, essendo una relazione di equivalenza, gode di alcune propriet\`a fondamentali.
\begin{itemize}
	\item \emph{Riflessivit\`a}: $a \equiv a \mod{m}$
	\item \emph{Simmetria}: $a \equiv b \mod{m} \Rightarrow b \equiv a \mod{m}$
	\item \emph{Transitivit\`a}: $a \equiv b \mod{m} \wedge b \equiv c \mod{m} \Rightarrow
		      a \equiv c \mod{m}$
\end{itemize}

Di seguito altre propriet\`a utili
\begin{itemize}
	\item Se $m \mid a$, allora $a \equiv 0 \mod{m}$ e viceversa.
	\item Se $m$ \`e primo, allora $ab \equiv 0 \mod{m}$ se e solo se $a \equiv 0 \mod{m}$
	      oppure $b \equiv 0 \mod{m}$.
\end{itemize}

Se $a \equiv b \mod{m}$ e $c \equiv d \mod{m}$, allora:
\begin{itemize}
	\item $a + c \equiv b + d \mod{m}$
	\item $a - c \equiv b - d \mod{m}$
	\item $ac \equiv bd \mod{m}$
\end{itemize}

Se $a \equiv b \mod{m}$ e $n \in \mathbb{Z}$, allora
\begin{itemize}
	\item $a^n \equiv b^n \mod{m}$ (se $n \neq 0$ vale anche il viceversa).
	\item $na \equiv nb \mod{m}$ (se $(n, m) = 1$ vale anche il viceversa).
\end{itemize}

\begin{example}
	Con le propriet\`a appena elencate proviamo a ricavare qual \`e il resto della divisione
	di $2021^{2020}$ per $3$.
	In questo caso vogliamo sfruttare la propriet\`a delle potenze. Per prima cosa troviamo
	il resto della divisione di 2021 per 3.
	\begin{equation*}
		2021 = 673 \cdot 3 + 2
	\end{equation*}
	quindi il resto \`e 2. Troviamo ora un numero tale che il resto della divisione per 3
	sia 2. Il numero pi\`u vantaggioso in questo caso \`e -1. Possiamo quindi scrivere
	\begin{equation*}
		2021 \equiv -1 \mod{3}
	\end{equation*}
	A questo punto sfruttiamo la propriet\`a e scriviamo
	\begin{equation*}
		\begin{array}{c}
			2021^{2020} \equiv (-1)^{2020} \mod{3} \\
			\Downarrow                             \\
			2021^{2020} \equiv 1 \mod{3}
		\end{array}
	\end{equation*}
	Ne concludiamo che il resto della divisione di $2021^{2020}$ per 3 \`e 1.
\end{example}

\section{Criteri di divisibilit\`a}
In questa sezione mostriamo un metodo per trovare i criteri di divisibilit\`a per un certo
numero $m$.

\begin{example}
	Cerchiamo ad esempio il criterio di divisibilit\`a per 11 (gli altri si trovano in maniera
	analoga).

	Per prima cosa prendiamo un numero intero $a > 0$. Sappiamo che $a$ pu\`o essere scritto
	in base 10 nella seguente forma:
	\begin{equation*}
		\begin{array}{lr}
			a = a_n \cdot 10^n + a_{n-1} \cdot 10^{n - 1} + \cdots + a_1 \cdot 10 + a_0
			 & a_n > 0, n \geq 0
		\end{array}
	\end{equation*}
	Dato che $10 \equiv -1 \mod{11}$ si ha che
	\begin{equation*}
		a \equiv (-1)^n \cdot a_n  + (-1)^{n-1} \cdot a_{n-1} + \cdots + (-1) \cdot a_1 + a_0
		\mod{11}
	\end{equation*}
	Senza perdere di generalit\`a posso supporre che $n$ sia pari, in tal caso la somma delle
	cifre di posto pari sar\`a data da:
	\begin{equation*}
		a_0 + a_2 + \cdots + a_n
	\end{equation*}
	Mentre la somma delle cifre di posto dispari sar\`a data da:
	\begin{equation*}
		a_1 + a_3 + \cdots + a_{n-1}
	\end{equation*}

	Dire che $11 \mid a$ equivale a dire che $a \equiv 0 \mod{11}$ e questo accade se e solo se
	\begin{equation*}
		(a_0 + a_2 + \cdots + a_n) - (a_1 + a_3 + \cdots + a_{n-1}) \equiv 0 \mod{11}
	\end{equation*}
	Precisiamo che $n$ potrebbe essere dispari, semplicemente $a_n$ andrebbe a finire nel
	conteggio delle cifre di posto dispari.
\end{example}

\textbf{Pi\`u in generale}, ci\`o che vogliamo fare \`e definire i numeri divisibili per un
certo $m$ in base $b$.

Per prima cosa prendiamo un intero $a > 0$ e scriviamolo in base $b$.
\begin{equation*}
	a = a_n \cdot b^n + a_{n-1} \cdot b^{n-1} + \cdots + a_1 \cdot b + a_0
\end{equation*}

Ora troviamo un numero $k$ tale che $b \equiv k \mod{m}$. Una volta trovato possiamo scrivere
la congruenza
\begin{equation*}
	a \equiv k^n \cdot a_n + k^{n-1} \cdot a_{n-1} + \cdots + k \cdot a_1 + a_0 \mod{m}
\end{equation*}
Concludiamo dicendo che un generico intero $a$ in base $b$ \`e divisibile per $m$ se
$a \equiv 0 \mod{m}$ ovvero se
\begin{equation*}
	k^n \cdot a_n + k^{n-1} \cdot a_{n-1} + \cdots + k \cdot a_1 + a_0 \equiv 0 \mod{m}
\end{equation*}
Il criterio assume forma diversa in base ai numeri considerati e ai passaggi aritmetici che
si fanno.

\begin{example}
	Proviamo a ricavarci il criterio di divisibilit\`a per 2 (base 10). Sappiamo tutti che un
	numero in base 10 \`e divisibile per 2 se la cifra delle unit\`a \`e pari. Proviamo a
	ricavarlo aritmeticamente.

	Il numero $a$ considerato \`e in base 10 quindi si pu\`o scrivere nella forma
	\begin{equation*}
		a = a_n \cdot 10^n + a_{n-1} \cdot 10^{n - 1} + \cdots + a_1 \cdot 10 + a_0
	\end{equation*}
	Ora troviamo un $k$ tale che $10 \equiv k \mod{2}$. A occhio sappiamo che il resto della
	divisione di 10 per 2 \`e 0. Quindi ricaviamo facilmente che $k = 0$ quindi
	\begin{equation*}
		10 \equiv 0 \mod{2}
	\end{equation*}
	A questo punto possiamo scrivere che
	\begin{equation*}
		\begin{array}{c}
			a \equiv 0^n \cdot a_n + 0^{n-1} \cdot a_{n-1} +
			\cdots + 0 \cdot a_1 + a_0 \mod{2} \\
			\Downarrow                         \\
			a \equiv a_0 \mod{2}
		\end{array}
	\end{equation*}
	Ne concludiamo che $2 \mid a$ se e solo se $a_0 \equiv 0 \mod{2}$ ovvero se l'ultima cifra
	\`e un multiplo di 2.
\end{example}

\section{Cicli}
Facciamo qualche considerazione in pi\`u per vedere altri esempi di come possono essere usate
le congruenze e sperare di capirci qualcosa in pi\`u.

Prima di tutto dobbiamo fare un'osservazione importante. Se $c = a \cdot b$ la cifra delle
unit\`a di $c$ sar\`a la cifra delle unit\`a del prodotto tra $a_0$ e $b_0$ dove $a_0$ e $b_0$
sono le ultime cifre di $a$ e di $b$.

\begin{example}
	Prendiamo ad esempio $12 \cdot 5 = 60$. Come detto in precedenza, abbiamo che l'ultima
	cifra (0) \`e l'ultima cifra del prodotto delle ultime cifre ($2 \cdot 5 = 10$)
	(sembra uno scioglilingua lo so).
\end{example}

\begin{example}
	Vogliamo trovare l'ultima cifra di $12567^{9506}$. Come osservato in precedenza dobbiamo
	considerare solo il 7. Se troviamo l'ultima cifra di $7^{9506}$ troviamo anche l'ultima
	cifra di $12567^{9506}$.

	Proviamo ora a scrivere in ordine l'ultima cifra delle potenze di 7.
	\begin{equation*}
		\begin{array}{rl}
			7^0 & \rightarrow 1 \\
			7^1 & \rightarrow 7 \\
			7^2 & \rightarrow 9 \\
			7^3 & \rightarrow 3 \\
			7^4 & \rightarrow 1 \\
			7^5 & \rightarrow 7 \\
			    & \vdots
		\end{array}
	\end{equation*}
	Come possiamo vedere ci sono 4 cifre ($1, 7, 9, 3$) che si ripetono ciclicamente. Questo
	significa che dobbiamo trovare un numero $x$ tale che
	\begin{equation*}
		9506 \equiv x \mod{4}
	\end{equation*}
	Per farlo possiamo ad esempio dividere 9506 per 4 e otterremo 2376 con il resto di 2.
	Da qui otteniamo facilmente che
	\begin{equation*}
		9506 \equiv 2 \mod{4}
	\end{equation*}
	Possiamo concludere che l'ultima cifra di $7^{9506}$ \`e uguale all'ultima cifra di
	$7^2$, ovvero 9.
\end{example}

\section{Inversi}
Gli inversi sono fondamentali per la risoluzione di equazioni con congruenze. Di seguito
vediamo cosa sono e come trovarli.
\begin{definition}
	Dato $a \in \mathbb{Z}$ si diche che $a$ \`e \textbf{invertibile} modulo $m$ se esiste
	$x \in \mathbb{Z}$ tale che
	\begin{equation*}
		ax \equiv 1 \mod{m}
	\end{equation*}
	Un tale $x$ si chiama \textbf{inverso} di $a$ modulo $m$. Chiariamo che $x$ \`e compreso
	tra 0 e $m - 1$ ed \`e unico.
\end{definition}

Per trovare l'inverso di $a \mod{m}$ posso provare a tentativi oppure posso risolvere la
diofantea associata nella forma
\begin{equation*}
	ax + my = 1
\end{equation*}

\begin{example}
	Troviamo l'inverso di 4 modulo 5
	\begin{equation*}
		4x \equiv 1 \mod{5}
	\end{equation*}
	Posso farlo a tentativi oppure usando l'equazione diofantea
	\begin{equation*}
		4x + 5y = 1
	\end{equation*}
	In ogni caso alla fine dovrei trovare che $x = -1$.
\end{example}

\begin{theorem}
	$a$ \`e invertibile modulo $m$ se e solo se $(a, m) = 1$
\end{theorem}

Abbiamo visto che gli inversi permettono di risolvere congruenze del tipo
\begin{equation*}
	ax \equiv 1 \mod{m}
\end{equation*}
Vediamo ora come si risolve una congruenza del tipo
\begin{equation*}
	ax \equiv b \mod{m}
\end{equation*}

\begin{proposition}
	Se esiste l'inverso di $a \mod{m}$ allora risolve anche
	\begin{equation*}
		ax \equiv b \mod{m}
	\end{equation*}
\end{proposition}

Devo prima trovare $x'$ tale che
\begin{equation*}
	ax' \equiv 1 \mod{m}
\end{equation*}
A questo punto mi basta moltiplicare $x'$ per $b$.
\begin{equation*}
	ax'b \equiv b \mod{m}
\end{equation*}
quindi $x = x'b$.

Posso affermare quindi che
\begin{itemize}
	\item se esiste l'inverso di $a \mod{m}$ allora
	      \begin{equation*}
		      ax \equiv b \mod{m}
	      \end{equation*}
	      ha soluzione.
	\item se non esiste l'inverso di $a \mod{m}$ \emph{dipende}.
\end{itemize}

\begin{theorem}
	\begin{equation*}
		ax \equiv b \mod{m}
	\end{equation*}
	\`e risolvibile se e solo se $(a, m) \mid b$
\end{theorem}

Dunque per trovare tutte le soluzioni delle congruenze del tipo
\begin{equation*}
	ax \equiv b \mod{m}
\end{equation*}
posso quindi procedere in due modi
\begin{itemize}
	\item \textbf{Primo metodo}: usiamo la diofantea $ax + my = b$. Trovo la soluzione $(x, y)$,
	      scarto la $y$ e prendo la $x$ in forma parametrica che appunto risolve la congruenza.
	\item \textbf{Secondo metodo}: risolvo aritmeticamente la congruenza, per esempio usando gli
	      inversi.
\end{itemize}

\section{Sistemi di congruenze}
Consideriamo il generico sistema
\begin{equation*}
	\begin{cases}
		a_1 x \equiv b_1 \mod{m_1} \\
		a_2 x \equiv b_2 \mod{m_2} \\
		\dots                      \\
		a_k x \equiv b_k \mod{m_k}
	\end{cases}
\end{equation*}

Il primo metodo che vedremo far\`a uso del \textbf{teorema cinese del resto}.
\begin{theorem}[Teorema cinese del resto]
	Dato il sistema nella forma
	\begin{equation*}
		\begin{cases}
			x \equiv a_1 \mod{m_1} \\
			x \equiv a_2 \mod{m_2} \\
			\dots                  \\
			x \equiv a_k \mod{m_k}
		\end{cases}
	\end{equation*}
	e $(m_i, m_j) = 1$ con $1 \leq i, j \leq k$ il sistema ammette una soluzione modulo
	$m_1 \cdot m_2 \cdot ... \cdot m_k$.

	Come prima cosa scriviamo $M = m_1 \cdot m_2 \cdot ... \cdot m_k$ e $M_i = \frac{M}{m_i}$.
	Otteniamo cos\`i $k$ nuove congruenze da risolvere in questa forma
	\begin{equation*}
		\begin{array}{l}
			M_1 x_1 \equiv 1 \mod{m_1} \\
			M_2 x_2 \equiv 1 \mod{m_2} \\
			\dots                      \\
			M_k x_k \equiv 1 \mod{m_k}
		\end{array}
	\end{equation*}
	A questo punto dobbiamo risolvere le singole congruenze troviando gli inversi, una volta
	fatto possiamo scrivere la soluzione generale in questa forma:
	\begin{equation*}
		x \equiv a_1 M_1 x_1 + a_2 M_2 x_2 + \cdots + a_k M_k x_k \mod{M}
	\end{equation*}
\end{theorem}

\begin{example}
	Proviamo a risolvere il seguente sistema
	\begin{equation*}
		\begin{cases}
			5x \equiv 25 \mod{30} \\
			2x \equiv 1 \mod{25}
		\end{cases}
	\end{equation*}
	Come notiamo anche a occhio $(30, 25) \neq 1$ pe\`o la prima equazione pu\`o essere
	semplificata dividendo tutto per 5 e ottenendo cos\`i
	\begin{equation*}
		\begin{cases}
			x \equiv 5 \mod{6} \\
			2x \equiv 1 \mod{25}
		\end{cases}
	\end{equation*}
	A questo punto possiamo verificare che $(6, 25) = 1$ per\`o non abbiamo ancora il sistema
	nella forma richiesta dal sistema. Per sistemare la seconda equazione troviamo il suo
	inverso modulo 25. Per farlo risolviamo la diofantea
	\begin{equation*}
		2x + 25y = 1
	\end{equation*}
	Anche a occhio possiamo vedere che la soluzione cercata \`e $x = -12$ e $y = 1$.
	dunque sapendo che $x = -12$ e che $a \equiv a \mod{m}$ per qualsiasi $a$ possiamo scrivere
	\begin{equation*}
		x \equiv -12 \mod{25}
	\end{equation*}
	se al posto di 12 vogliamo un numero positivo dobbiamo trovare qualcos'altro a cui 12 \`e
	congruo modulo 25. Per farlo basta vedere il resto di $-12 / 25$ che \`e 13 quindi
	possiamo ora scrivere il nostro sistema in questo modo
	\begin{equation*}
		\begin{cases}
			x \equiv 5 \mod{6} \\
			x \equiv 13 \mod{25}
		\end{cases}
	\end{equation*}
	Troviamo $M$ moltiplicando fra loro tutti gli $m_i$ e otteniamo
	\begin{equation*}
		M = 6 \cdot 25 = 150
	\end{equation*}
	Ora ricaviamo i vari $M_i$:
	\begin{equation*}
		\begin{array}{l}
			M_1 = 150 / 6 = 25 \\
			M_2 = 150 / 25 = 6
		\end{array}
	\end{equation*}
	Concludiamo scrivendo la soluzione generale
	\begin{equation*}
		\begin{array}{l}
			x \equiv 5 \cdot 25 \cdot 1 + 13 \cdot 6 \cdot (-4) \mod{150} \\
			x \equiv -187 \mod{150}                                       \\
			x \equiv 113 \mod{150}
		\end{array}
	\end{equation*}
	Tutte le soluzioni possibili sono quindi congrue a $113 \mod{150}$.
\end{example}

Studiamo ora come risolvere il sistema nel caso i vari $m_i$ non siano a due a due coprimi.
Dato il sistema nella solita forma standard
\begin{equation*}
	\begin{cases}
		x \equiv a_1 \mod{m_1} \\
		x \equiv a_2 \mod{m_2} \\
		\dots                  \\
		x \equiv a_k \mod{m_k}
	\end{cases}
\end{equation*}
Non si fa altro che risolvere le congruenze in ordine e sostituire la soluzione volta per volta.
Chiariamo meglio il tutto provando a risolvere lo stesso sistema.
\begin{example}
	Risolvere
	\begin{equation*}
		\begin{cases}
			5x \equiv 25 \mod{30} \\
			2x \equiv 1 \mod{25}
		\end{cases}
	\end{equation*}
	Come prima passo alla forma standard
	\begin{equation*}
		\begin{cases}
			x \equiv 5 \mod{6} \\
			x \equiv 13 \mod{25}
		\end{cases}
	\end{equation*}
	Dalla prima congruenza ottengo
	\begin{equation*}
		x = 6k + 5
	\end{equation*}
	che vado a sostituire nella seconda equazione:
	\begin{equation*}
		\begin{array}{l}
			6k + 5 \equiv 13 \mod{25} \\
			6k \equiv 8 \mod{25}
		\end{array}
	\end{equation*}
	Risolvo la congruenza e ottengo
	\begin{equation*}
		\begin{array}{l}
			k \equiv 18 \mod{25} \\
			k = 25y + 18
		\end{array}
	\end{equation*}
	In questo caso ho finito le equazioni e quindi sostituisco a ritroso fino alla prima
	\begin{equation*}
		\begin{array}{ll}
			x & = 6(25y + 18) + 5 \\
			x & = 150y + 113
		\end{array}
	\end{equation*}
	Ho ottenuto dunque lo stesso risultato
	\begin{equation*}
		x \equiv 113 \mod{150}
	\end{equation*}
\end{example}

\begin{example}
	Risolviamo il seguente sistema
	\begin{equation*}
		\begin{cases}
			3x \equiv 1 \mod{14} \\
			x \equiv 1 \mod{8}   \\
			3x \equiv 9 \mod{5}
		\end{cases}
	\end{equation*}
	Passo alla forma standard
	\begin{equation*}
		\begin{cases}
			x \equiv 5 \mod{14} \\
			x \equiv 1 \mod{8}  \\
			x \equiv 3 \mod{5}
		\end{cases}
	\end{equation*}
	Dalla prima equazione ricavo
	\begin{equation*}
		x = 14k + 5
	\end{equation*}
	e la sostituisco nella seconda equazione
	\begin{equation*}
		\begin{array}{l}
			14k + 5 \equiv 1 \mod{8} \\
			14k \equiv -4 \mod{8}    \\
			7k \equiv -2 \mod{4}     \\
			7k \equiv 2 \mod{4}      \\
			3k \equiv 2 \mod{4}      \\
			k \equiv 2 \mod{4}       \\
			k = 4y + 2
		\end{array}
	\end{equation*}
	sostituisco questo $k$ nella prima equazione e ottengo
	\begin{equation*}
		x = 56y + 33
	\end{equation*}
	che vado a sostituire nell'ultima equazione ottenendo
	\begin{equation*}
		\begin{array}{l}
			56y + 33 \equiv 3 \mod{5} \\
			56y \equiv -30 \mod{5}    \\
			56y \equiv 0 \mod{5}      \\
			y \equiv 0 \mod{5}        \\
			y = 5z
		\end{array}
	\end{equation*}
	Sostituisco un ultima volta e ottengo
	\begin{equation*}
		\begin{array}{l}
			x = 56 \cdot 5z + 33 \\
			x = 280z + 33
		\end{array}
	\end{equation*}
	Posso concludere che
	\begin{equation*}
		x \equiv 33 \mod{280}
	\end{equation*}
\end{example}

\begin{theorem}
	Il sistema
	\begin{equation*}
		\begin{cases}
			x \equiv a \mod{m_1} \\
			x \equiv b \mod{m_2}
		\end{cases}
	\end{equation*}
	ha soluzione se e solo se
	\begin{equation*}
		a \equiv b \mod{(m_1, m_2)}
	\end{equation*}
	Ovviamente vale anche il viceversa, quindi se
	\begin{equation*}
		a \not \equiv b \mod{(m_1, m_2)}
	\end{equation*}
	allora il sistema non ha soluzione.
\end{theorem}

\begin{example}
	Vediamo se il seguente sistema ha soluzione
	\begin{equation*}
		\begin{cases}
			x \equiv 5 \mod{12} \\
			x \equiv 6 \mod{20}
		\end{cases}
	\end{equation*}
	Trovo quindi il MCD tra 20 e 12
	\begin{equation*}
		\begin{array}{ll}
			(20, 12) & = (20 - 12, 12) \\
			(8, 12)  & = (8, 12 - 8)   \\
			(8, 4)   & = (8 - 4, 4)    \\
			(4, 4)   & = (4, 4 - 4)    \\
			(4, 0)   & \Rightarrow 4
		\end{array}
	\end{equation*}
	A questo punto verifico che
	\begin{equation*}
		6 \equiv 5 \mod{4}
	\end{equation*}
	Ma come \`e banale verificare, 6 e 5 non sono congrui modulo 4, dunque il sistema non
	\`e risolvibile.
\end{example}

\begin{theorem}
	Trovata una soluzione $x_0$ del sistema
	\begin{equation*}
		\begin{cases}
			x \equiv a \mod{m_1} \\
			x \equiv b \mod{m_2}
		\end{cases}
	\end{equation*}
	per trovare le altre basta aggiungere a $x_0$ multipli di $[m_1, m_2]$.
	Ogni soluzione $x$ del sistema sar\`a
	\begin{equation*}
		x \equiv x_0 \mod{[m_1, m_2]}
	\end{equation*}
\end{theorem}

\section{Congruenze con parametro}
Andiamo a vedere come risolvere congruenze del tipo
\begin{equation*}
	ax \equiv b \mod{mk}
\end{equation*}
con $k \in \mathbb{K}$ come parametro. La congruenza ha soluzione per $k$ tale che
$(a, mk) \mid b$.

\begin{example}
	Per quali valori del parametro $k$ la congruenza
	\begin{equation*}
		168x \equiv 1540 \mod{35k}
	\end{equation*}
	ha soluzione ?
	La soluzione si trova nei $k$ tali che $(168, 35k) \mid 1540$.
	Proviamo ad approfondire:
	\begin{equation*}
		\begin{array}{ll}
			168  & = 2^3 \cdot 3 \cdot 7          \\
			1540 & = 2^2 \cdot 5 \cdot 7 \cdot 11 \\
			35   & = 5 \cdot 7
		\end{array}
	\end{equation*}
	Quindi
	\begin{equation*}
		\begin{array}{ll}
			(168, 25k)                               & = \\
			(2^3 \cdot 3 \cdot 7, 5 \cdot 7 \cdot k) & = \\
			7 (2^3 \cdot 3, 5 \cdot k)               & = \\
			7 (2^3 \cdot 3, k)
		\end{array}
	\end{equation*}
	Possiamo dunque affermare che
	\begin{equation*}
		\begin{array}{c}
			7 (2^3 \cdot 3, k) \mid 2^2 \cdot 5 \cdot 7 \cdot 11 \\
			\Leftrightarrow                                      \\
			(2^3 \cdot 3, k) \mid 2^2 \cdot 5 \cdot 11           \\
			\Leftrightarrow                                      \\
			(2^3, k) \cdot (3, k) \mid 2^2 \cdot 5 \cdot 11      \\
			\Leftrightarrow                                      \\
			2^3 \nmid k \wedge 3 \nmid k
		\end{array}
	\end{equation*}
	Ne concludo che i $k$ che sto cercando sono tali che $8 \nmid k$ e $3 \nmid k$.
\end{example}

\section{Sistemi di congruenze con parametro}
Procediamo ora a trovare soluzione per congruenze con un parametro usando ci\`o che abbiamo studiato
nei paragrafi precedenti.

\begin{example}
	Risolviamo il seguente sistema
	\begin{equation*}
		\begin{cases}
			x \equiv 1 \mod{12} \\
			x \equiv b \mod{20}
		\end{cases}
	\end{equation*}
	Per quali $b$ c'\`e soluzione ?

	Prima calcoliamo il MCD tra 20 e 12 e otteniamo che $(20, 12) = 4$ quindi
	\begin{equation*}
		b \equiv 1 \mod{4}
	\end{equation*}
	Scrivo quindi il sistema in questo modo:
	\begin{equation*}
		\begin{cases}
			x \equiv 1 \mod{4} \\
			x \equiv b \mod{4}
		\end{cases}
	\end{equation*}
	Quindi $1 \equiv b \mod{4}$. Ad esempio per $b = 5$ c'\`e soluzione. Troviamo quindi una
	soluzione per
	\begin{equation*}
		\begin{cases}
			x \equiv 1 \mod{12} \\
			x \equiv 5 \mod{20}
		\end{cases}
	\end{equation*}
	Risolvendo la prima congruenza ottengo
	\begin{equation*}
		x = 1 + 12y
	\end{equation*}
	Sostituisco nella seconda equazione:
	\begin{equation*}
		\begin{array}{l}
			1 + 12y \equiv 5 \mod{20} \\
			12y \equiv 4 \mod{20}     \\
			3y \equiv 1 \mod{5}       \\
			y \equiv 2 \mod{5}
		\end{array}
	\end{equation*}
	Risolvendo la congruenza ottengo
	\begin{equation*}
		y = 2 + 5z
	\end{equation*}
	Sostituendo in $x = 1 + 12y$ ottengo
	\begin{equation*}
		\begin{array}{c}
			x = 1 + 12(2 + 5z) \\
			x = 25 + 60z       \\
			\Downarrow         \\
			x \equiv 25 \mod{60}
		\end{array}
	\end{equation*}
	Come potevamo aspettarci la soluzione trovata \`e modulo $[m_1, m_2]$, infatti
	\begin{equation*}
		[12, 20] = 60
	\end{equation*}
	Quindi 25 \`e una soluzione, le altre si trovano aggiungendo a questa, multipli di 60.
\end{example}

\begin{example}
	Vediamo se c'\`e soluzione per il seguente sistema:
	\begin{equation*}
		\begin{cases}
			x \equiv 3 \mod{20} \\
			x \equiv 7 \mod{12}
		\end{cases}
		\Rightarrow
		\begin{cases}
			x \equiv 3 \mod{4} \\
			x \equiv 3 \mod{5} \\
			x \equiv 7 \mod{4} \\
			x \equiv 7 \mod{3}
		\end{cases}
	\end{equation*}
	Non ho fatto altro che scomporre le due equazioni iniziali in altre due equazioni
	equivalenti.

	In questo caso ho generato due congruenze modulo 4. Verifico che i due termini noti siano
	congrui fra loro modulo 4. Se lo sono il sistema ha soluzione e dato che:
	\begin{equation*}
		3 \equiv 7 \mod{4}
	\end{equation*}
	il sistema ha soluzione.
\end{example}

\begin{theorem}
	Il sistema nella forma
	\begin{equation*}
		\begin{cases}
			A \equiv B \mod{m_1} \\
			A \equiv B \mod{m_2}
		\end{cases}
	\end{equation*}
	\`e corretto se e solo se
	\begin{equation*}
		A \equiv B \mod{[m_1, m_2]}
	\end{equation*}
\end{theorem}

\section{Congruenze polinomiali}
Cerchiamo di capire come trovare tutte le soluzioni di congruenze che si presentano nella forma
\begin{equation*}
	p(x) \equiv 0 \mod{m}
\end{equation*}
dove $p(x)$ \`e un polinomio di grado qualsiasi.

\begin{example}
	Trovare tutte le soluzioni di
	\begin{equation*}
		x^4 \equiv 1 \mod{55}
	\end{equation*}
	Portiamo la congruenza nella forma desiderata. Scriviamo quindi:
	\begin{equation*}
		x^4 - 1 \equiv 0 \mod{55}
	\end{equation*}
	A questo punto cerco di fattorizzare $x^4 - 1$. Posso usare il metodo che preferisco, ad
	esempio Ruffini e ottengo:
	\begin{equation*}
		x^4 - 1 = (x^2 + 1)(x + 1)(x - 1)
	\end{equation*}
	Per il teorema studiato in precedenza abbiamo che
	\begin{equation*}
		\begin{array}{rcl}
			x^4 - 1 \equiv 0 \mod{55} &
			\Rightarrow               &
			\begin{cases}
				x^4 - 1 \equiv 0 \mod{5} \\
				x^4 - 1 \equiv 0 \mod{11}
			\end{cases}
		\end{array}
	\end{equation*}
	Quindi posso scrivere che
	\begin{equation*}
		\begin{cases}
			x^2 + 1 \equiv 0 \mod{5} \vee x + 1 \equiv 0 \mod{5} \vee x - 1 \equiv 0 \mod{5} \\
			x^2 + 1 \equiv 0 \mod{11} \vee x + 1 \equiv 0 \mod{11} \vee x - 1 \equiv 0 \mod{11}
		\end{cases}
	\end{equation*}
	A questo punto risolvo le congruenze una per una:
	\begin{equation*}
		\begin{array}{c}
			x^2 + 1 \equiv 0 \mod{5}                   \\
			\Downarrow                                 \\
			x \equiv 2 \mod{5} \vee x \equiv 3 \mod{5} \\
			\\
			x + 1 \equiv 0 \mod{5}                     \\
			\Downarrow                                 \\
			x \equiv 4 \mod{5}                         \\
			\\
			x - 1 \equiv 0 \mod{5}                     \\
			\Downarrow                                 \\
			x \equiv 1 \mod{5}                         \\
			\\
			x^2 + 1 \equiv 0 \mod{11}                  \\
			\Downarrow                                 \\
			\text{Non ha soluzione}                    \\
			\\
			x + 1 \equiv 0 \mod{11}                    \\
			\Downarrow                                 \\
			x \equiv -1 \mod{11}                       \\
			\\
			x - 1 \equiv 0 \mod{11}                    \\
			\Downarrow                                 \\
			x \equiv 1 \mod{11}
		\end{array}
	\end{equation*}
	Da questo ricavo che
	\begin{equation*}
		\begin{array}{c}
			\begin{cases}
				x \equiv 2 \mod{5} \\
				x \equiv 1 \mod{11}
			\end{cases} \vee
			\begin{cases}
				x \equiv 3 \mod{5} \\
				x \equiv 1 \mod{11}
			\end{cases} \vee  \\
			\\
			\begin{cases}
				x \equiv 4 \mod{5} \\
				x \equiv 1 \mod{11}
			\end{cases} \vee
			\begin{cases}
				x \equiv 1 \mod{5} \\
				x \equiv 1 \mod{11}
			\end{cases} \vee  \\
			\\
			\begin{cases}
				x \equiv 2 \mod{5} \\
				x \equiv -1 \mod{11}
			\end{cases} \vee
			\begin{cases}
				x \equiv 3 \mod{5} \\
				x \equiv -1 \mod{11}
			\end{cases} \vee \\
			\\
			\begin{cases}
				x \equiv 4 \mod{5} \\
				x \equiv -1 \mod{11}
			\end{cases} \vee
			\begin{cases}
				x \equiv 1 \mod{5} \\
				x \equiv -1 \mod{11}
			\end{cases}
		\end{array}
	\end{equation*}
	A questo punto risolvo ogni sistema per ottenere un insieme di soluzioni per la congruenza
	iniziale.
	\begin{equation*}
		\begin{array}{cc}
			x \equiv 12 \mod{55} & \vee \\
			x \equiv 23 \mod{55} & \vee \\
			x \equiv 34 \mod{55} & \vee \\
			x \equiv 1 \mod{55}  & \vee \\
			x \equiv 32 \mod{55} & \vee \\
			x \equiv 43 \mod{55} & \vee \\
			x \equiv 54 \mod{55} & \vee \\
			x \equiv 21 \mod{55}
		\end{array}
	\end{equation*}
	Lungo come non so cosa ma si fa cos\`i.
\end{example}

\section{Congruenze esponenziali}
Vediamo ora come risolvere congruenze del tipo
\begin{equation*}
	a^x \equiv b \mod{m}
\end{equation*}

\begin{example}
	Trovare tutte le soluzioni per la congruenza
	\begin{equation*}
		2^x \equiv 1 \mod{5}
	\end{equation*}
	Per tentativi risolviamo la congruenza. Per esempio per $x = 0$ o $x = 4$ soddisfiamo
	effettivamente la congruenza dato che
	\begin{equation*}
		\begin{array}{ccc}
			2^0 = 1 & \wedge & 1 \equiv 1 \mod{5}
		\end{array}
	\end{equation*}
	e che
	\begin{equation*}
		\begin{array}{ccc}
			2^4 = 16 & \wedge & 16 \equiv 1 \mod{5}
		\end{array}
	\end{equation*}
	Ci potrebbe venire in mente che a questo punto la soluzione sia
	\begin{equation*}
		x \equiv 4 \mod{5}
	\end{equation*}
	in realt\`a se proseguiamo con i tentativi ricaviamo che anche $x = 8$ soddisfa
	l'equazione poich\'e:
	\begin{equation*}
		\begin{array}{ccc}
			2^8 = 256 & \wedge & 256 \equiv 1 \mod{5}
		\end{array}
	\end{equation*}
	A questo punto possiamo notare che gli $x$ trovati sono tutti divisibili per 4
	dunque la soluzione \`e:
	\begin{equation*}
		x \equiv 0 \mod{4}
	\end{equation*}
\end{example}

Ora introduciamo uno strumento utile per lavorare col prossimo teorema. Stiamo parlando del
triangolo di Tartaglia. Il triangolo serve a trovare facilmente i coefficienti delle potenze
del binomio $(x + y)^n$.

Il triangolo \`e formato da $n + 1$ righe. Devo disporre tutti 1 sui 2 bordi mentre i numeri
al centro si ottengono dalla somma dei due elementi soprastanti.

\begin{center}
	\includegraphics[width=0.5\textwidth]{immagini/Tartaglia}
\end{center}

Per muoverci all'interno del triangolo utilizziamo i cosidetti \textbf{coefficienti binomiali}
che sono scritti in questo modo:
\begin{equation*}
	\begin{pmatrix}
		n \\ k
	\end{pmatrix}
\end{equation*}
Questa scrittura significa che stiamo cercando sulla riga $n$ e dobbiamo muoverci da sinistra
verso destra di $k$ passi.

\begin{example}
	Molto banalmente abbiamo che
	\begin{equation*}
		\begin{pmatrix}
			4 \\ 2
		\end{pmatrix} = 6
	\end{equation*}
\end{example}

Per la definizione del triangolo di tartaglia possiamo scrivere che
\begin{itemize}
	\item $
		      \begin{pmatrix}
			      n \\ 0
		      \end{pmatrix} =
		      \begin{pmatrix}
			      n \\ n
		      \end{pmatrix} = 1
	      $
	\item $
		      \begin{pmatrix}
			      n + 1 \\ k
		      \end{pmatrix} =
		      \begin{pmatrix}
			      n \\ k - 1
		      \end{pmatrix} +
		      \begin{pmatrix}
			      n \\ k
		      \end{pmatrix}
	      $ con $k \neq 0 \wedge k \neq n + 1$
\end{itemize}

\begin{proposition}
	Siano $n, k \in \mathbb{Z}$ allora
	\begin{equation*}
		\begin{pmatrix}
			n \\ k
		\end{pmatrix} =
		\frac{n!}{k! (n - k)!}
	\end{equation*}
\end{proposition}

\begin{observation}
	Sia $n \in \mathbb{Z}$ allora
	\begin{equation*}
		\sum_{i = 0}^n \begin{pmatrix}
			n \\ i
		\end{pmatrix} = 2^n
	\end{equation*}
\end{observation}

\begin{theorem}[Teorema del binomio di Newton]
	Siano $x, y, n \in \mathbb{Z}$ allora
	\begin{equation*}
		(x + y)^n = \sum_{i = 0}^n \begin{pmatrix}
			n \\ i
		\end{pmatrix}
		x^{n - i} y^i
	\end{equation*}
\end{theorem}

\begin{observation}
	La somma a segni alterni di una riga del triangolo \`e 0.
	\begin{equation*}
		\sum_{i = 0}^n
		(-1)^i
		\begin{pmatrix}
			n \\ i
		\end{pmatrix} = 0
	\end{equation*}
\end{observation}

\begin{theorem}
	Se $p$ \`e primo tutti i numeri sulla riga $p$ eccetto gli 1 sui bordi, sono multipli
	di $p$. Quindi se $i \neq 0$ e $i \neq p$ allora
	\begin{equation*}
		\begin{pmatrix}
			p \\ i
		\end{pmatrix} \equiv 0 \mod{p}
	\end{equation*}
	\begin{proof}
		Ricordiamo che
		\begin{equation*}
			\begin{pmatrix}
				n \\ k
			\end{pmatrix} =
			\frac{n!}{k! (n - k)!}
		\end{equation*}
		Sapendo questo dobbiamo dimostrare che
		\begin{equation*}
			\begin{pmatrix}
				p \\ i
			\end{pmatrix} \equiv 0 \mod{p}
		\end{equation*}
		Se per\`o pensiamo che
		\begin{equation*}
			\begin{pmatrix}
				p \\ i
			\end{pmatrix} =
			\frac{p!}{i! (p - i)!}
		\end{equation*}
		e $p \mid p!$ quindi
		\begin{equation*}
			\frac{p!}{i! (p - i)!} \equiv 0 \mod{p} \\
		\end{equation*}
		e quindi
		\begin{equation*}
			\begin{pmatrix}
				p \\ i
			\end{pmatrix} \equiv 0 \mod{p}
		\end{equation*}
	\end{proof}
\end{theorem}

\begin{proposition}
	Se $p$ \`e primo allora
	\begin{equation*}
		(x + y)^p \equiv x^p + y^p \mod{p}
	\end{equation*}
	Lo si dimostra tramite il teorema precedente.
\end{proposition}

\begin{theorem}[Piccolo teorema di Fermat]
	Sia $p$ un numero primo e sia $a \in \mathbb{Z}$ un intero qualsiasi vale
	\begin{equation*}
		a^p \equiv a \mod{p}
	\end{equation*}
	\begin{proof}
		Tenendo conto di tutto quel che \`e stato detto fino ad ora possiamo affermare che
		\begin{equation*}
			(x_1 + x_2 + \cdots + x_n)^p \equiv x_1^p + x_2^p + \cdots + x_n^p \mod{p}
		\end{equation*}
		Con $x_1 = x_2 = \cdots = x_n = 1$ ho che
		\begin{equation*}
			(1 + 1 + \cdots + 1)^p \equiv 1^p + 1^p + \cdots + 1^p \mod{p}
		\end{equation*}
		ovvero
		\begin{equation*}
			n^p \equiv n \mod{p}
		\end{equation*}
	\end{proof}
\end{theorem}

\begin{corollary}
	Sia $p$ un numero primo ed $a \in \mathbb{Z}$ allora
	\begin{equation*}
		\begin{array}{ccc}
			a \not \equiv 0 \mod{p} & \Rightarrow & a^{p-1} \equiv 1 \mod{p}
		\end{array}
	\end{equation*}
\end{corollary}

\begin{example}
	La congruenza
	\begin{equation*}
		2^7 \equiv 2 \mod{7}
	\end{equation*}
	sappiamo essere corretta per il teorema appena enunciato.
	Possiamo anche vedere la congruenza in questo modo
	\begin{equation*}
		\begin{array}{c}
			2^6 \cdot 2 \equiv 2 \mod{7} \\
			\Downarrow                   \\
			2^6 \equiv 1 \mod{7}
		\end{array}
	\end{equation*}
\end{example}

\begin{example}
	Ora posso risolvere la congruenza di prima in questo modo (spero):
	\begin{equation*}
		2^x \equiv 1 \mod{5}
	\end{equation*}
	Posso affermare con sicurezza che
	\begin{equation*}
		2^5 \equiv 2 \mod{5}
	\end{equation*}
	per il piccolo teorema di Fermat. Notiamo ora che $2 \not \equiv 0 \mod{5}$ dunque posso
	applicare il corollario e scrivere:
	\begin{equation*}
		2^{4} \equiv 1 \mod{5}
	\end{equation*}
	Quindi
	\begin{equation*}
		x \equiv 0 \mod{4}
	\end{equation*}
	Ho dunque ottenuto lo stesso risultato di prima.
\end{example}

\begin{theorem}
	Una congruenza del tipo
	\begin{equation*}
		a^x \equiv b \mod{p}
	\end{equation*}
	con $p$ primo o non ha soluzione oppure \`e del tipo
	\begin{equation*}
		x \equiv x_0 \mod{m}
	\end{equation*}
	dove $m$ \`e il pi\`u piccolo esponente $> 0$ tale che
	\begin{equation*}
		a^m \equiv 1 \mod{p}
	\end{equation*}
	\`E inoltre dimostrabile che $m \mid p-1$ dunque $m$ \`e dell'ordine di $a \mod{p}$
\end{theorem}

\begin{theorem}
	Una congruenza del tipo
	\begin{equation*}
		a^x \equiv 1 \mod{p}
	\end{equation*}
	con $p$ primo ha sempre soluzione se $a \not \equiv 0 \mod{p}$.
\end{theorem}

\begin{proposition}
	La soluzione di
	\begin{equation*}
		a^x \equiv 1 \mod{p}
	\end{equation*}
	sono gli $x$ multipli del periodo, ovvero
	\begin{equation*}
		x \equiv 0 \mod{\text{periodo}}
	\end{equation*}
\end{proposition}

\begin{example}
	Proviamo a risolvere la congruenza
	\begin{equation*}
		3^x \equiv 5 \mod{7}
	\end{equation*}
	Faccio un po' di tentativi e trovo che $x = 5$ soddisfa la congruenza. Infatti
	\begin{equation*}
		3^5 = 243 \equiv 5 \mod{7}
	\end{equation*}
	Per trovare il periodo risolviamo la congruenza
	\begin{equation*}
		3^x \equiv 1 \mod{7}
	\end{equation*}
	Sicuramente $x = 0$ \`e una soluzione ma non ci aiuta. Inoltre posso usare il piccolo teorema di
	Fermat in questo modo:
	\begin{equation*}
		3^7 \equiv 3 \mod{7}
	\end{equation*}
	poich\'e 7 \`e un numero primo e poich\'e $3 \not \equiv 0 \mod{7}$ abbiamo che:
	\begin{equation*}
		3^6 \equiv 1 \mod{7}
	\end{equation*}
	quindi anche $x = 6$ \`e una soluzione. A noi interessa per\`o il pi\`u piccolo $m > 0$.
	Dobbiamo dunque verificare che 6 sia il pi\`u piccolo. Per alleggerire i calcoli ricordiamoci che
	$m \mid 6$ quindi cerchiamolo solo tra i divisori di 6 ovvero 1, 2, e 3:
	\begin{equation*}
		\begin{array}{c}
			3^1 = 3 \not \equiv 1 \mod{7} \\
			3^2 = 9 \not \equiv 1 \mod{7} \\
			3^3 = 27 \not \equiv 1 \mod{7}
		\end{array}
	\end{equation*}
	Ne deduco che $m = 6$ e posso concludere che
	\begin{equation*}
		x \equiv 5 \mod{6}
	\end{equation*}
	\`e la soluzione cercata.
\end{example}

\begin{definition}
	Sia $n \in \mathbb{Z}$, la funzione $\phi$ \`e detta \textbf{funzione di Eulero} e
	$\phi (m)$ equivale al numero di elementi invertibili modulo $m$. E come sappiamo un intero
	$a$ \`e invertibile modulo $m$ se e solo se $(a, m) = 1$ ovvero se \`e coprimo con $m$.
	Possiamo dunque affermare in alternativa che $\phi (m)$ equivale al numero di elementi
	compresi fra 1 e $m$ coprimi con $m$.
\end{definition}

\begin{theorem}[Teorema di Eulero]
	Siano $a, m \in \mathbb{Z}$. Se $a$ \`e invertibile modulo $m$ allora
	\begin{equation*}
		a^{\phi(m)} \equiv 1 \mod{m}
	\end{equation*}
\end{theorem}

\begin{theorem}
	Sia $p$ primo e $k \in \mathbb{N}$ allora
	\begin{equation*}
		\begin{array}{ll}
			\phi (p)   & = p - 1           \\
			\phi (p^k) & = p^k - p^{k - 1}
		\end{array}
	\end{equation*}
\end{theorem}

\begin{theorem}
	Se $n = ab$ con $(a, b) = 1$ allora
	\begin{equation*}
		\phi (n) = \phi (a) \cdot \phi (b)
	\end{equation*}
\end{theorem}

\begin{example}
	A quanto equivale $\phi (40)$ ?

	Per prima cosa scomponiamo 40 in fattori primi e otteniamo:
	\begin{equation*}
		40 = 2^3 \cdot 5
	\end{equation*}
	Ora calcoliamo quante vale $\phi$ per ciascun fattore
	\begin{equation*}
		\begin{array}{ll}
			\phi (2^3) & = 4 \\
			\phi (5)   & = 4
		\end{array}
	\end{equation*}
	dunque
	\begin{equation*}
		\phi (40) = 4 \cdot 4 = 16
	\end{equation*}
\end{example}

Possiamo usare il teorema anche per risolvere molto pi\`u velocemente le congruenze del tipo
\begin{equation*}
	a^x \equiv 1 \mod{p}
\end{equation*}
Ci basta verificare che $(a, p) = 1$ e a quel punto ci basta calcolare $\phi (p)$. Per il teorema
otteniamo che
\begin{equation*}
	a^{\phi (p)} \equiv 1 \mod{p}
\end{equation*}
A questo punto vediamo quali divisori di $\phi (p)$ soddisfano la congruenza. Nel caso ce ne
fossero, il pi\`u piccolo tra loro (chiamiamolo $k$) sar\`a la nostra nuova $x$.
Otterr\`o dunque:
\begin{equation*}
	\begin{array}{c}
		a^k \equiv 1 \mod{p} \\
		\Downarrow           \\
		x \equiv 0 \mod{k}
	\end{array}
\end{equation*}
Nel caso in cui non ci siano divisori che soddisfino la congruenza la soluzione sar\`a:
\begin{equation*}
	x \equiv 0 \mod{\phi (p)}
\end{equation*}

\begin{example}
	Risolviamo la congruenza
	\begin{equation*}
		2^x \equiv 1 \mod{5}
	\end{equation*}
	Notiamo subito che $(2, 5) = 1$. Calcoliamo $\phi (5) = 4$ e otteniamo dunque che
	\begin{equation*}
		\begin{array}{l}
			2^{\phi (5)} \equiv 1 \mod{5} \\
			2^{4} \equiv \mod{5}
		\end{array}
	\end{equation*}
	Dobbiamo per\`o assicurarci che 4 sia il minimo esponente tale che
	\begin{equation*}
		2^x \equiv 1 \mod{5}
	\end{equation*}
	Nel caso ce ne fosse uno diverso da 4 sarebbe un suo divisore. In questo caso n\'e 2 n\'e 1
	soddisfano la congruenza quindi la soluzione \`e
	\begin{equation*}
		x \equiv 0 \mod{4}
	\end{equation*}
\end{example}
