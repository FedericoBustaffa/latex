\chapter{Introduzione}
\section{Insiemi}
Iniziamo col definire gli i seguenti insiemi
\[
	\begin{array}{lll}
		\mathbb{N} & = \{ 0, 1, 2, \dots \}
		           & \text{insieme dei naturali}                                     \\
		\mathbb{Z} & = \{ -2, -1, 0, 1, 2 \}
		           & \text{insieme degli interi}                                     \\
		\mathbb{Q} & = \{ \frac{a}{b} \mid a, b \in \mathbb{Z} \wedge b \neq 0 \}
		           & \text{insieme dei razionali}                                    \\
		\mathbb{R} & = \{ \dots, \sqrt{2}, \pi, \dots \}
		           & \text{Insieme dei reali}                                        \\
		\mathbb{C} & = \{ a + ib \mid a, b \in \mathbb{R}, i = \sqrt{-1}, i^2 = -1\}
		           & \text{insieme dei complessi}
	\end{array}
\]
Le operazioni di addizione e moltiplicazione sono permesse ma possono essere definite in maniera diversa a seconda
dell'insieme su cui si sta lavorando.

\section{Propriet\`a}
\begin{itemize}
	\item \textbf{Transitivit\`a} per il $<$
	      \[ x < y \wedge y < z \Rightarrow x < z \]

	\item \textbf{Totalit\`a} per il $<$
	      \[ x < y \vee x = y \vee y < x \]

	\item \textbf{Transitivit\`a} per $\leq$
	      \[ x \leq y \wedge y \leq z \Rightarrow x \leq z \]

	\item \textbf{Totalit\`a} per il $\leq$
	      \[ x \leq y \vee y \leq x \]
\end{itemize}
Ovviamente queste propriet\`a sono analoghe per il $>$ e per il $\geq$.

\begin{observation}
	\[
		\begin{array}{lll}
			x < y    & \Leftrightarrow & x \leq y \wedge x \neq y \\
			x \leq y & \Leftrightarrow & x < y \vee x = y
		\end{array}
	\]
\end{observation}

\textbf{NOTA:} Tutte queste propriet\`a valgono per tutti gli insiemi definiti prima tranne che per $\mathbb{C}$. Mentre
in $\mathbb{N}$ vale un'altra proprit\`a, ovvero quella del \emph{minimo}, definita in questo modo: Ogni insieme non
vuoto di numeri naturali ha un minimo.

\begin{example}
	\[ A = \{x \in \mathbb{N} \mid \text{$x$ \`e primo}\} \]
	In questo caso il $\min({A}) = 2$. Pi\`u in generale $n = \min({A})$ dove $A$ \`e un insieme non nullo di naturali
	se e solo se
	\[ n \in A \quad \wedge \quad \forall x \in A \quad n \leq x \]
\end{example}