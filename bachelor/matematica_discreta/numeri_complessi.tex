\chapter{Numeri complessi}
\section{Introduzione}
Nei numeri complessi si passa da una rappresentazione monodimensionale a una bidimensionale
rappresentabile sul piano cartesiano. I cosiddetti numeri complessi sono elementi del campo
$\mathbb{C}$ e sono formati dalla coppia di reali $(a, b)$.

\begin{definition}
	Chiamo \textbf{numero complesso} l'elemento
	\begin{equation*}
		a + bi \in \mathbb{C}
	\end{equation*}
	con $a, b \in \mathbb{R}$.
\end{definition}

Sul campo $\mathbb{C}$ sono definite le operazioni di somma e prodotto come segue:
\begin{itemize}
	\item \emph{Somma}: $(a + bi) + (c + di) = (a + c) + (b + d)i$
	\item \emph{Prodotto}: $(a + bi) \cdot (c + di) = (ac - bd) + (ad + bc)i$
\end{itemize}

\begin{example}
	Siano $(1, 2), (2, 3) \in \mathbb{C}$ la loro somma equivale a
	\begin{equation*}
		(1, 2) + (2, 3) = (3, 5)
	\end{equation*}
	e il loro prodotto equivale a
	\begin{equation*}
		(1, 2) \cdot (2, 3) = (-4, 7)
	\end{equation*}
\end{example}

Per queste operazioni valgono le stesse propriet\`a delle operazioni in $\mathbb{R}$
(commutativa, associativa, distributiva e elemento neutro).

\textbf{NOTA}: l'elemento neutro della somma \`e $(0, 0)$, quello del prodotto \`e $(1, 0)$.

I numeri del tipo $(a, 0)$, se sommati o moltiplicati fra loro, generano numeri dello stesso tipo
e anche l'opposto e il reciproco di tali numeri \`e sempre nella stessa forma.

Per alleggerire la notazione possiamo scrivere direttamente $a$ al posto di $(a, 0)$.

Il numero $(0, 1)$ gode di una propriet\`a interessante:
\begin{equation*}
	(0, 1) \cdot (0, 1) = (-1, 0) = -1
\end{equation*}
Questo numero viene chiamato \textbf{unit\`a immaginaria} e lo si indica con $i$. Come possiamo
vedere, \`e un numero che, elevato al quadrato, diventa negativo dunque \`e una delle soluzione
per
\begin{equation*}
	x^2 = -1
\end{equation*}

\begin{definition}
	Definiamo ora quella che viene chiamata \textbf{forma cartesiana} o \textbf{algebrica}.
	Viene scritta in questo modo
	\begin{equation*}
		z = (a, b) = (a, 0) + (0, 1)(b, 0) = a + ib
	\end{equation*}
	Chiameremo $a$ \textbf{parte reale} del numero complesso $z$ e la si pu\`o indicare anche con
	$Re(z)$, mentre $b$ prende il nome di \textbf{parte immaginaria} del numero complesso $z$ e la
	si pu\`o indicare anche con $Im(z)$.
	Potremmo rappresentare il numero complesso in un sistema cartesiano che prender\`a il nome di
	\textbf{piano di Gauss} e che posso rappresentare in questo modo:

	\includegraphics[width=0.9\textwidth]{immagini/Gauss}

	Pongo sull'asse delle ascisse i valori reali $a$ e sulle ordinate i valori immaginari $b$.
	In particolare i numeri che si trovano sull'asse delle ordinate ($a = 0$) si chiamano
	\textbf{immaginari puri}.
\end{definition}

\section{Operazioni}
Definiamo le operazioni di \emph{addizione}, \emph{sottrazione}, \emph{moltiplicazione} e
\emph{divisione} utilizzando la forma cartesiana appena definita.
Per ognuna delle operazioni consideriamo i due numeri complessi
\begin{gather*}
	z_1 = a + ib \\
	z_2 = c + id
\end{gather*}
\begin{itemize}
	\item \emph{Addizione}:
	      \begin{equation*}
		      z_1 + z_2 = (a + c) + (b + d)i
	      \end{equation*}
	\item \emph{Sottrazione}:
	      \begin{equation*}
		      z_1 + z_2 = (a - c) + (b - d)i
	      \end{equation*}
	\item \emph{Moltiplicazione}:
	      \begin{equation*}
		      z_1 \cdot z_2 = (ac - bd) + (ad + bc)i
	      \end{equation*}
	\item \emph{Divisione}:
	      \begin{equation*}
		      \frac{z_1}{z_2} = \frac{ac + bd}{c^2 + d^2} + \frac{bc - ad}{c^2 + d^2}i
	      \end{equation*}
\end{itemize}

\begin{example}
	Siano $z_1 = 1 + 3i$ e $z_2 = 1 - i$ allora
	\begin{gather*}
		z_1 + z_2 = 2 + 2i  \\
		z_1 - z_2 = 4i      \\
		z_1 \cdot z_2 = 4 + 2i  \\
		z_1 / z_2 = -1 + 2i
	\end{gather*}
\end{example}

\section{Forma trigonometrica}
I punti del piano di Gauss possono essere individuati
\begin{itemize}
	\item dalle coordinate cartesiane $a$ e $b$.
	\item dalle coordinate polari $\rho$ e $\theta$.
\end{itemize}
Le coordinate cartesiane possono essere due reali qualsiasi.
Le coordinate polari invece hanno qualche vincolo. Per esempio $\rho \geq 0$ e $\theta$ \`e
determinato a meno di multipli di $2 \pi$.

Guardando la figura di sopra possiamo facilmente convertire $a$ e $b$ in funzione dei valori
$\rho$ e $\theta$ per ottenere il numero in \textbf{forma trigonometrica}
\begin{equation*}
	z = a + ib = \rho \cos{(\theta)} + i \rho \sin{(\theta)} =
	\rho [\cos{(\theta)} + i \sin{(\theta)}]
\end{equation*}

\begin{example}
	Tramite semplici calcoli di trigonometria posso vedere che il numero
	\begin{equation*}
		z = 2 \left[ \cos{\left( \frac{3}{4} \pi \right)} +
			i \sin{\left( \frac{3}{4} \pi \right)} \right]
	\end{equation*}
	in forma trigonometrica \`e equivalente al numero
	\begin{equation*}
		z = -\sqrt{2} + i\sqrt{2}
	\end{equation*}
	in forma cartesiana.
\end{example}

Se invece volessi fare una conversione da forma cartesiana a forma trigonometrica mi
basta ricordare
\begin{gather*}
	\rho = \sqrt{a^2 + b^2} \\
	\theta =
	\begin{cases}
		\arctan{(b / a)}       & \text{ se } a > 0 \\
		\arctan{(b / a)} + \pi & \text{ se } a < 0
	\end{cases}
\end{gather*}
\textbf{NOTA}: Se $a = 0$ allora
\begin{equation*}
	\theta =
	\begin{cases}
		\frac{\pi}{2}   & \text{ se } b > 0 \\
		\frac{3}{2} \pi & \text{ se } b < 0
	\end{cases}
\end{equation*}
Se anche $b = 0$ quindi $z = 0$ allora $\theta = 0$.

\begin{example}
	Consideriamo il numero complesso in forma cartesiana
	\begin{equation*}
		z = \sqrt{3} + i
	\end{equation*}
	Ottengo che
	\begin{equation*}
		\rho = \sqrt{\left( \sqrt{3} \right)^2 + 1^2} = 2
	\end{equation*}
	e che
	\begin{equation*}
		\theta = \arctan{\left( \frac{1}{\sqrt{3}} \right)} = \frac{\pi}{6}
	\end{equation*}
	Posso quindi scrivere il numero
	\begin{equation*}
		z = 2 \left[ \cos{\left( \frac{\pi}{6} \right)}
			+ i  \sin{\left( \frac{\pi}{6} \right)} \right]
	\end{equation*}
\end{example}

Quando il numero \`e scritto in forma trigonometrica, $\theta$ prende il nome di \textbf{argomento}
di $z$ ed \`e, a volte, indicato con $arg(z)$. Il numero reale $\sqrt{a^2 + b^2}$ si dice invece
\textbf{modulo} di $z$ ed \`e indicato con $|z|$.

Definiamo ora le operazioni di addizione, sottrazione, moltiplicazione e divisione tra due numeri
complessi $z_1$ e $z_2$ scritti in forma trigonometrica.
\begin{gather*}
	z_1 = \rho_1 (\cos{(\theta_1)} + i \sin{(\theta_1)}) \\
	z_2 = \rho_2 (\cos{(\theta_2)} + i \sin{(\theta_2)})
\end{gather*}

\begin{itemize}
	\item \emph{Addizione}:
	      \begin{equation*}
		      z_1 + z_2 = [\rho_1 \cos{(\theta_1)} + \rho_2 \cos{(\theta_2)}] +
		      i [\rho_1 \sin{(\theta_1)} + \rho_2 \sin{(\theta_2)}]
	      \end{equation*}
	\item \emph{Sottrazione}:
	      \begin{equation*}
		      z_1 - z_2 = [\rho_1 \cos{(\theta_1)} - \rho_2 \cos{(\theta_2)}] +
		      i [\rho_1 \sin{(\theta_1)} - \rho_2 \sin{(\theta_2)}]
	      \end{equation*}
	\item \emph{Moltiplicazione}:
	      \begin{equation*}
		      z_1 \cdot z_2 = \rho_1 \rho_2
		      [\cos{(\theta_1  + \theta_2)} + i \sin{(\theta_1 + \theta_2)}]
	      \end{equation*}
	\item \emph{Divisione}:
	      \begin{equation*}
		      \frac{z_1}{z_2} = \frac{\rho_1}{\rho_2}
		      [\cos{(\theta_1 - \theta_2)} + i \sin{(\theta_1 - \theta_2)}]
	      \end{equation*}
\end{itemize}

\begin{example}
	Siano
	\begin{gather*}
		z_1 = \sqrt{2} \left[\cos{\left(\frac{\pi}{4}\right)} +
			i \sin{\left(\frac{\pi}{4}\right)}\right] \\
		z_2 = \left[\cos{\left(\frac{\pi}{2}\right)} +
			i \sin{\left(\frac{\pi}{2}\right)}\right]
	\end{gather*}
	allora
	\begin{gather*}
		|z_1 \cdot z_2| = \sqrt{2}                                             \\
		\arg{(z_1 \cdot z_2)} = \frac{\pi}{4} + \frac{\pi}{2} = \frac{3}{4} \pi      \\
		z_1 \cdot z_2 = \sqrt{2} \left[ \cos{\left( \frac{3}{4} \pi \right)}
			+ i \sin{\left( \frac{3}{4} \pi \right)} \right]
	\end{gather*}
\end{example}

\section{Forma esponenziale}
\`E una notazione equivalente a quella trigonometrica in cui si utilizzano sempre $\rho$ e
$\theta$ e in cui, al posto di coseno e seno, si usa l'esponenziale.
\begin{equation*}
	e^{i \theta } = \cos{(\theta)} + i \sin{(\theta)}
\end{equation*}
I numeri complessi in forma esponenziale si scrivono in questo modo:
\begin{equation*}
	z = a + ib = \rho [\cos{(\theta)} + i \sin{(\theta)}] = \rho e^{i \theta }
\end{equation*}
Per convertire in forma esponenziale un numero in forma cartesiana si calcolano $\rho$ e
$\theta$ come per la conversione in forma trigonometrica. Trovati i valori mi basta sostituire
come nella formula.

Se il numero \`e in forma trigonometrica posso passare direttamente alla forma
esponenziale sostituendo i valori.

\begin{example}
	Sia $z = \sqrt{3} + i$ un numero complesso scritto in forma cartesiana, la sua forma
	trigonometrica sar\`a:
	\begin{equation*}
		z = 2 \left[ \cos{\left( \frac{\pi}{6} \right)} +
			i \sin{\left( \frac{\pi}{6} \right)} \right]
	\end{equation*}
	mentre
	\begin{equation*}
		z = 2 e^{i \frac{\pi}{6}}
	\end{equation*}
	sar\`a la sua forma esponenziale.
\end{example}

\section{Radici e potenze}
Nel caso in cui dovessimo calcolare potenze di numeri complessi ci conviene utilizzare la
forma esponenziale, con la quale possiamo ragionare come con i numeri reali.
\begin{equation*}
	z^n = (\rho e^{i \theta})^n = \rho^n e^{i n \theta}
\end{equation*}
Possiamo passare anche in forma trigonometrica in questo modo:
\begin{equation*}
	\rho^n e^{i n \theta} = \rho^n [\cos{(n \theta)} + i \sin{(n \theta)}]
\end{equation*}

In generale, un procedimento veloce per svolgere le potenze di numeri scritti in forma
cartesiana \`e questo:
\begin{enumerate}
	\item Converto il numero in forma esponenziale.
	\item Elevo alla $n$.
	\item Torno in forma cartesiana.
\end{enumerate}

\begin{example}
	Calcolare $(1 + i)^5$.
	Per farlo seguo i tre passi elencati sopra. Per prima cosa converto il numero in forma
	esponenziale:
	\begin{gather*}
		\rho = \sqrt{1^2 + 1^2} = \sqrt{2}  \\
		\theta = \arctan{(1 / 1)} = \pi / 4 \\
		z = 1 + i = \sqrt{2} e^{i \frac{\pi}{4}}
	\end{gather*}
	Ora elevo il numero in forma esponenziale alla quinta.
	\begin{equation*}
		(1 + i)^5 = 4 \sqrt{2} e^{5 i \frac{\pi}{4}}
	\end{equation*}
	Non mi rimane che ritornare alla forma cartesiana
	\begin{gather*}
		a = 4 \sqrt{2} \cos{\left( \frac{5}{4} \pi \right)} = -4 \\
		b = 4 \sqrt{2} \sin{\left( \frac{5}{4} \pi \right)} = -4
	\end{gather*}
	Quindi otteniamo che
	\begin{equation*}
		(1 + i)^5 = -4 - 4i
	\end{equation*}
	\`e il valore cercato.
\end{example}

\begin{definition}
	Dato un numero complesso $w$, si dice che $z$ \`e una \textbf{radice n-esima complessa} di $w$
	se $z^n = w$.
	Se $w = 0$ l'unica radice \`e $z = 0$. Se invece $z \neq 0$ esistono $n$ numeri reali complessi
	che soddisfano l'equazione, ovvero $n$ radici $n$-esime complesse.
\end{definition}

Se vogliamo trovare tutte le radici complesse di $w$ dobbiamo seguire questi passaggi:
\begin{enumerate}
	\item Come detto in precedenza dobbiamo trovare quel numero $z$ tale che
	      \begin{equation*}
		      z^n = w
	      \end{equation*}
	\item Portiamo $z$ e $w$ in forma esponenziale, ottenendo quindi:
	      \begin{gather*}
		      z = \rho e^{i \theta} \\
		      w = r e^{i \phi}      \\
		      \Downarrow            \\
		      \rho^n e^{i n \theta} = r e^{i \phi}
	      \end{gather*}
	\item Arrivati a questo punto $r$ e $\phi$ saranno noti. Dobbiamo quindi eguagliare i
	      moduli per ottenere $\rho$:
	      \begin{gather*}
		      \rho^n = r \\
		      \Downarrow \\
		      \rho = \sqrt[n]{r}
	      \end{gather*}
	      e in seguito eguagliare gli argomenti per ottenere $\theta$:
	      \begin{gather*}
		      n \theta = \phi \\
		      \Downarrow      \\
		      \theta = \frac{\phi}{n}
	      \end{gather*}
	      Gli argomenti possono differire per multipli di $2 \pi$ e saranno diversi per ciascuna
	      delle $n$ radici. Dunque la formula diventa:
	      \begin{align*}
		      \theta_k = \frac{\phi}{n} + \frac{2k \pi}{n}
		       & \text{ con } k = 0, \dots, n-1
	      \end{align*}
	\item Una volta trovati tutti i $\theta_k$ otteniamo $n$ radici complesse di questo
	      tipo:
	      \begin{align*}
		      z_k = \rho e^{i \theta_k} & \text{ con } k = 0, 1, \dots, n-1
	      \end{align*}
	\item Come ultimo passaggio convertiamo le radici trovate in forma cartesiana.
\end{enumerate}

\begin{example}
	Determinare le radici cubiche complesse di -1.
	Per prima cosa portiamo -1 in forma esponenziale.
	\begin{gather*}
		r = \sqrt{(-1)^2} = 1 \\
		\phi = \arctan(0) + \pi = \pi
	\end{gather*}
	Dunque
	\begin{equation*}
		w = -1 = e^{i \pi}
	\end{equation*}
	Dato che $w = z^n$ e che in questo caso abbiamo $n = 3$ allora posso scrivere
	\begin{equation*}
		e^{i \pi} = \rho^3 e^{i 3 \theta}
	\end{equation*}
	Se eguaglio i moduli ottengo
	\begin{gather*}
		\rho^3 = 1 \\
		\Downarrow \\
		\rho = \sqrt[3]{1} = 1
	\end{gather*}
	Eguagliando invece gli argomenti ottengo
	\begin{gather*}
		3 \theta = \pi \\
		\Downarrow     \\
		\theta = \frac{\pi}{3}
	\end{gather*}
	Ricordiamoci per\`o che devo trovare tre radici e quindi tre argomenti che differiscono
	l'uno dall'altro per multipli di $2 \pi$.
	Dunque avr\`o:
	\begin{gather*}
		\theta_0 = \frac{\pi}{3} + \frac{0}{3} = \frac{\pi}{3} \\
		\theta_1 = \frac{\pi}{3} + \frac{2}{3} \pi = \pi       \\
		\theta_2 = \frac{\pi}{3} + \frac{4}{3} \pi = \frac{5}{3} \pi
	\end{gather*}
	A questo punto non mi rimane che scrivere le radici e convertirle in forma cartesiana.
	\begin{gather*}
		z_0 = e^{i \frac{4}{3} \pi} = \frac{1}{2} + \frac{\sqrt{3}}{2} i \\
		z_1 = e^{i \pi} = -1                                             \\
		z_2 = e^{i \frac{5}{3} \pi} = \frac{1}{2} - \frac{\sqrt{3}}{2} i
	\end{gather*}
\end{example}

\section{Equazioni}
Iniziamo ora a risolvere equazioni coi numeri complessi partendo dalle equazioni di secondo grado,
che \`e anche il motivo per cui, storicamente, i numeri complessi sono stati introdotti.
Se ricordiamo, l'equazione
\begin{equation*}
	x^2 = -1
\end{equation*}
non ammette soluzione in $\mathbb{R}$ mentre in $\mathbb{C}$ ammette due soluzioni complesse
ovvero le radici quadrate complesse di -1 ovvero $\pm i$.

\begin{example}
	Risolvere l'equazione
	\begin{equation*}
		x^2 + 9 = 0
	\end{equation*}
	Se proviamo a risolverla come una normale equazione di secondo grado ottengo che
	\begin{equation*}
		\Delta = -36
	\end{equation*}
	Normalmente non potrei usare la classica formula risolutiva ma in questo caso possiamo
	ragionare in questo modo: sappiamo che $i^2 = -1$ quindi possiamo scrivere
	\begin{equation*}
		-36 = i^2 \cdot 36
	\end{equation*}
	A questo punto possiamo usare la formula risolutiva
	\begin{equation*}
		x = \frac{\pm \sqrt{i^2 \cdot 36}}{2} = \frac{\pm 6i}{2} = \pm 3i
	\end{equation*}
\end{example}

In generale per risolvere equazioni di secondo grado coi numeri complessi si pu\`o utilizzare
la "tradizionale" formula risolutiva
\begin{equation*}
	z = \frac{-b \pm \sqrt{b^2 - 4ac}}{2a}
\end{equation*}
a patto di intendere la radice quadrata nella formula in senso complesso.

\begin{definition}
	Dato un generico numero complesso chiamo \textbf{complesso coniugato} o \textbf{coniugio}
	quel numero che ha la parte immaginaria cambiata di segno. Quindi se considero il generico
	numero complesso
	\begin{equation*}
		z = a - ib
	\end{equation*}
	il suo complesso coniugato, che indicher\`o con $\overline{z}$ il numero
	\begin{equation*}
		\overline{z} = a + ib
	\end{equation*}
	Tale numero nel piano di Gauss \`e simmetrico a $z$ rispetto all'asse delle ascisse.
\end{definition}

\begin{observation}
	In forma esponenziale se ho
	\begin{equation*}
		z = \rho e^{i \theta}
	\end{equation*}
	allora
	\begin{equation*}
		\overline{z} = \rho e^{-i \theta}
	\end{equation*}
	sar\`a il complesso coniugato.
\end{observation}

In generale per le equazioni di secondo grado ho tre possibili casi:
\begin{itemize}
	\item Se $\Delta > 0$ allora ho due soluzioni reali distinte.
	\item Se $\Delta = 0$ allora ho due soluzioni reali coincidenti.
	\item Se $\Delta < 0$ allora ho due soluzioni complesse coniugate.
\end{itemize}

\begin{example}
	Risolvere l'equazione
	\begin{equation*}
		z^2 - 3iz - 2 = 0
	\end{equation*}
	Utilizziamo la formula risolutiva.
	\begin{equation*}
		\Delta = (-3i)^2 + 8 = 9i^2 + 8 = -1
	\end{equation*}
	Quindi abbiamo che
	\begin{equation*}
		z = \frac{3i \pm \sqrt{-1}}{2} = \frac{3i \pm i}{2}
	\end{equation*}
	Da qui ricaviamo le due soluzioni
	\begin{gather*}
		z_1 = 2i \\
		z_2 = i
	\end{gather*}
\end{example}

\begin{example}
	Risolvere l'equazione
	\begin{equation*}
		z^3 - 5iz^2 - 6z = 0
	\end{equation*}
	In questo caso possiamo raccogliere $z$ e riscrivere l'equazione in questo modo:
	\begin{equation*}
		z(z^2 - 5iz - 6) = 0
	\end{equation*}
	Da qui ricaviamo subito che una soluzione \`e $z = 0$.

	Risolviamo ora l'equazione di secondo grado con la formula
	\begin{equation*}
		\Delta = 25 i^2 + 24 = -1
	\end{equation*}
	Quindi
	\begin{equation*}
		z = \frac{5i \pm \sqrt{-1}}{2} = \frac{5i \pm i}{2}
	\end{equation*}
	Le soluzioni sono quindi
	\begin{gather*}
		z_1 = 0  \\
		z_2 = 3i \\
		z_3 = 2i
	\end{gather*}
\end{example}

\begin{example}
	Risolvere l'equazione
	\begin{equation*}
		z^4 - 3iz^2 - 2 = 0
	\end{equation*}
	In questo caso ci conviene porre $y = z^2$ e riscriverci l'equazione come
	\begin{equation*}
		y^2 - 3iy - 2 = 0
	\end{equation*}
	Risolviamo con la solita formula
	\begin{equation*}
		\Delta = (-3i)^2 + 8 = 9i^2 + 8 = -1
	\end{equation*}
	quindi
	\begin{equation*}
		y = \frac{3i \pm \sqrt{-1}}{2} = \frac{3i \pm i}{2}
	\end{equation*}
	Le soluzioni sono quindi
	\begin{gather*}
		y_1 = 2i \\
		y_2 = i
	\end{gather*}
	A questo punto troviamo le soluzioni dell'equazione di partenza
	\begin{gather*}
		y = z^2    \\
		\Downarrow \\
		z = \sqrt{y}
	\end{gather*}
	Per calcolarle possiamo usare il metodo visto precedentemente.
	Iniziamo con il trovare le radici quadrate di $2i$. Per prima cosa convertiamo $2i$ in
	forma esponenziale.
	\begin{gather*}
		r = \sqrt{2^2} = 2                            \\
		\phi = \arctan{\left(\frac{2}{0}\right)} = \frac{\pi}{2} \\
		2i = 2 e^{i \frac{\pi}{2}}
	\end{gather*}
	Da qui otteniamo che
	\begin{gather*}
		\rho = \sqrt{2}          \\
		\theta_0 = \frac{\pi}{4} \\
		\theta_1 = \frac{\pi}{4} + \pi = \frac{5}{4} \pi
	\end{gather*}
	Quindi
	\begin{gather*}
		z_1 = \sqrt{2} e^{i \frac{\pi}{4}} = 1 + i \\
		z_2 = \sqrt{2} e^{i \frac{5}{4} \pi} = 1 - i
	\end{gather*}
	Calcoliamo ora le radici quadrate di $i$.
	\begin{gather*}
		r = \sqrt{1^2} = 1                                         \\
		\phi = \arctan{\left( \frac{1}{0} \right)} = \frac{\pi}{2} \\
		i = e^{i \frac{\pi}{2}}
	\end{gather*}
	Da qui otteniamo
	\begin{gather*}
		\rho = \sqrt{1} = 1      \\
		\theta_0 = \frac{\pi}{4} \\
		\theta_1 = \frac{\pi}{4} + \pi = \frac{5}{4} \pi
	\end{gather*}
	Quindi
	\begin{gather*}
		z_3 = e^{i \frac{\pi}{4}} = \frac{\sqrt{2}}{2} + \frac{\sqrt{2}}{2} i \\
		z_4 = e^{i \frac{5}{4} \pi} = -\frac{\sqrt{2}}{2} - \frac{\sqrt{2}}{2} i
	\end{gather*}
	Ricapitolando
	\begin{gather*}
		z_1 = 1 + i                                     \\
		z_2 = 1 - i                                     \\
		z_3 = \frac{\sqrt{2}}{2} + \frac{\sqrt{2}}{2} i \\
		z_4 = -\frac{\sqrt{2}}{2} - \frac{\sqrt{2}}{2} i
	\end{gather*}
	sono le nostre soluzioni.
\end{example}

\begin{example}
	Risolvere l'equazione
	\begin{equation*}
		z^2 + 2 \overline{z} = -i Im(z)
	\end{equation*}
	Il trucco qui sta nel vedere $z$ in forma cartesiana, scriviamo quindi
	\begin{equation*}
		z = x + iy
	\end{equation*}
	Sappiamo che $\overline{z}$ \`e $z$ ma con la parte immaginaria cambiata di segno mentre
	$Im(z)$ \`e proprio la parte immaginaria, ovvero $y$. Quindi, se sostituiamo otteniamo
	\begin{gather*}
		(x + iy)^2 + 2 (x - iy) = -i y       \\
		x^2 - y^2 + 2ixy + 2x - 2iy + iy = 0 \\
		x^2 - y^2 + 2x + 2ixy - iy = 0		 \\
		x^2 - y^2 + 2x - i(2xy - y) = 0
	\end{gather*}
	Perch\'e un complesso sia 0 sia parte reale che parte immaginaria devono essere 0. Da qui
	ricaviamo il sistema:
	\begin{equation*}
		\begin{cases}
			x^2 - y^2 + 2x = 0 \\
			2xy - y = 0
		\end{cases}
	\end{equation*}
	Nella seconda equazione possiamo raccogliere $y$.
	\begin{equation*}
		\begin{cases}
			y(2x - 1) = 0 \\
			x^2 - y^2 + 2x = 0
		\end{cases}
	\end{equation*}
	Da qui otteniamo due possibilit\`a:
	\begin{gather*}
		y = 0 			\\
		x = \frac{1}{2}
	\end{gather*}
	Se $y = 0$ allora
	\begin{gather*}
		\begin{cases}
			y = 0 \\
			x^2 + 2x = 0
		\end{cases} \\
		\begin{cases}
			y = 0 \\
			x (x + 2) = 0
		\end{cases}
	\end{gather*}
	Da qui otteniamo due coppie di soluzioni
	\begin{align*}
		\begin{cases}
			y = 0 \\
			x = 0
		\end{cases} &
		\begin{cases}
			y = 0 \\
			x = -2
		\end{cases}
	\end{align*}
	Se invece $x = \frac{1}{2}$ allora
	\begin{gather*}
		\begin{cases}
			x = \frac{1}{2} \\
			\frac{1}{4} - y^2 + 1 = 0
		\end{cases} \\
		\begin{cases}
			x = \frac{1}{2} \\
			y^2 = \frac{5}{4}
		\end{cases} \\
		\begin{cases}
			x = \frac{1}{2} \\
			y = \sqrt{\frac{5}{4}}
		\end{cases}
	\end{gather*}
	Da qui otteniamo altre due soluzioni
	\begin{gather*}
		\begin{cases}
			x = \frac{1}{2} \\
			y = \frac{\sqrt{5}}{2}
		\end{cases} \\
		\begin{cases}
			x = \frac{1}{2} \\
			y = -\frac{\sqrt{5}}{2}
		\end{cases}
	\end{gather*}
	Dunque le quattro soluzioni sono
	\begin{gather*}
		z_1 = 0 \\
		z_2 = -2 \\
		z_3 = \frac{1}{2} + i \frac{\sqrt{5}}{2} \\
		z_4 = \frac{1}{2} - i \frac{\sqrt{5}}{2}
	\end{gather*}
\end{example}

\begin{example}
	Risolvere l'equazione
	\begin{equation*}
		z^4 = - |z|
	\end{equation*}
	In questo caso ci conviene usare la forma esponenziale.
	\begin{gather*}
		z = \rho e^{i \theta} \\
		z^4 = \rho^4 e^{4 i \theta} \\
		-|z| = -\rho
	\end{gather*}
	Dunque ottengo
	\begin{equation*}
		\rho^4 e^{4 i \theta} = \rho e^{i \pi}
	\end{equation*}
	Da qui ottengo il sistema
	\begin{gather*}
		\begin{cases}
			\rho^4 = \rho \\
			4 \theta_k = \pi + 2k \pi
		\end{cases} \\
		\begin{cases}
			\rho (\rho^3 - 1) = 0 \\
			\theta_k = \frac{\pi}{4} + \frac{k \pi}{2}
		\end{cases}
	\end{gather*}
	Nel caso in cui $\rho$ sia uguale a 0, qualunque sia il valore di $\theta$ otterr\`o sempre
	\begin{equation*}
		z = 0
	\end{equation*}
	Sviluppando invece $\rho^3 - 1 = 0$ ottengo che $\rho = 1$. Mi rimane solo da calcolare
	i possibili $\theta$.
	\begin{gather*}
		\begin{cases}
			\rho = 1 \\
			\theta_0 = \frac{\pi}{4}
		\end{cases}
		\begin{cases}
			\rho = 1 \\
			\theta_1 = \frac{3}{4} \pi
		\end{cases} \\
		\begin{cases}
			\rho = 1 \\
			\theta_2 = \frac{5}{4} \pi
		\end{cases}
		\begin{cases}
			\rho = 1 \\
			\theta_3 = \frac{7}{4} \pi
		\end{cases}
	\end{gather*}
	Ho dunque trovato cinque soluzioni in totale
	\begin{gather*}
		z_0 = 0 \\
		z_1 = e^{i \frac{\pi}{4}} \\
		z_2 = e^{i \frac{3}{4} \pi} \\
		z_3 = e^{i \frac{5}{4} \pi} \\
		z_4 = e^{i \frac{7}{4} \pi}
	\end{gather*}
\end{example}
