\chapter{Equazioni Diofantee}
Introduciamo per prima cosa la notazione $a \mid b$ per indicare che $a$ divide $b$. Da questo
possiamo dedurre che $b = ac$ con $c \in \mathbb{Z}$. Possiamo dedurne che $b$ \`e un multiplo
di $a$.

\section{Algoritmo di Euclide}
Definiamo l'algoritmo molto semplicemente in questo modo: si applica ripetutamente
il teorema 3.0.3 finch\'e uno dei due numeri diventa $0$, l'altro sar\`a il MCD.
Vediamo qualche esempio per fissare meglio le idee
\begin{example}
	Vogliamo trovare ad esempio il MCD tra 10 e 6. Poniamo quindi $a = 10$ e $b = 6$ e
	procediamo
	\begin{equation*}
		\begin{array}{ll}
			(10, 6) & = (10 - 6, 6) \\
			(4, 6)  & = (4, 6 - 4)  \\
			(4, 2)  & = (4 - 2, 2)  \\
			(2, 2)  & = (2, 2 - 2)  \\
			(2, 0)
		\end{array}
	\end{equation*}
	Abbiamo dunque ottenuto che $(10, 6) = 2$.
\end{example}

\begin{theorem}
	Dati $a, b, q, r \in \mathbb{Z}$ tali che $a = qb + r$, vale
	\begin{equation*}
		\begin{array}{lll}
			(a, b) & = (r, b)      & \text{$r =$ resto della divisione di $a$ per $b$} \\
			(a, b) & = (a - qb, b) & \text{$a = qb + r$}
		\end{array}
	\end{equation*}
\end{theorem}

\section{Risoluzione di equazioni diofantee}
\begin{theorem}[Identit\`a di Bezout]
	Siano $a, b$ interi non entrambi nulli
	\begin{equation*}
		\exists x, y \in \mathbb{Z} \mid (a, b) = ax + by
	\end{equation*}
\end{theorem}

\begin{definition}
	Chiamo \textbf{equazione diofantea} un'equazione del tipo
	\begin{equation*}
		ax + by = c
	\end{equation*}
	di cui cerco le soluzioni intere.
\end{definition}

\begin{theorem}
	Dati $a, b \in \mathbb{Z}$, l'equazione diofantea
	\begin{equation*}
		ax + by = c
	\end{equation*}
	ha soluzione se e solo se $c$ \`e un multiplo di $(a, b)$.
\end{theorem}

Procediamo ora con ordine per trovare una soluzione a questa equazione.
\begin{enumerate}
	\item Per prima cosa troviamo l'MCD tra $a$ e $b$. Verifichiamo che l'equazione abbia
	      soluzione e procediamo con lo studiare l'equazione semplificata dividendola per
	      $(a, b)$.
	\item Troviamo ora una soluzione all'equazione
	      \begin{equation*}
		      ax + by = 1
	      \end{equation*}
	      con l'algoritmo di Euclide.
	\item Adesso moltiplichiamo tutto per $c$ (semplificato) e troviamo quindi una soluzione
	      particolare.
	\item Infine dobbiamo trovare una soluzione generale facendo la differenza tra l'equazione
	      con le incognite e quella con la soluzione particolare.
\end{enumerate}

\begin{example}
	Consideriamo l'equazione diofantea
	\begin{equation*}
		9x + 3y = 6
	\end{equation*}
	Calcoliamo per prima cosa il MCD tra 9 e 3 tramite Euclide.
	\begin{equation*}
		\begin{array}{ll}
			(9, 3) & = (9 - 3, 3) \\
			(6, 3) & = (6 - 3, 3) \\
			(3, 3) & = (3, 3 - 3) \\
			(3, 0)
		\end{array}
	\end{equation*}
	Dunque $(9, 3) = 3$ e 6 \`e un multiplo di 3 quindi l'equazione ha soluzione.

	Adesso divido tutto per 3 e ottengo l'equazione
	\begin{equation*}
		3x + y = 2
	\end{equation*}
	A questo punto cerco una soluzione particolare per l'equazione
	\begin{equation*}
		3x + y = 1
	\end{equation*}
	nello stesso modo di prima, ovvero con Euclide.
	\begin{equation*}
		\begin{array}{ll}
			(3, 1) & = (3 - 1, 1)  \\
			(2, 1) & = (2 - 1, 1)  \\
			(1, 1) & = (1, 1 - 1)  \\
			(1, 0) & \Rightarrow 1
		\end{array}
	\end{equation*}

	Adesso cerco una soluzione particolare ripercorrendo l'algoritmo a ritroso.
	\begin{equation*}
		\begin{array}{ll}
			1 & = 2 - 1         \\
			2 & = 3 - 1         \\
			  & \Downarrow      \\
			1 & = 3 - 2 \cdot 1
		\end{array}
	\end{equation*}
	Ne concludiamo che $x = 1$ e $y = -2$ ma a noi serve una soluzione per $c = 2$ quindi
	moltiplico tutto per 2 e ottengo
	\begin{equation*}
		3 \cdot (2) + 1 \cdot (-4) = 2
	\end{equation*}
	Quindi ho ottenuto una soluzione particolare $x = 2$ e $y = -4$ alla mia equazione iniziale.
	Tuttavia non \`e l'unica e devo quindi trovarne una generale.
	Per farlo devo confrontare l'equazione con le incognite e quella con la soluzione particolare
	per poi farne la differenza. Ottengo cos\`i
	\begin{equation*}
		\begin{array}{l}
			3(x - 2) + (y + 4) = 0 \\
			3(x - 2) = -(y + 4)
		\end{array}
	\end{equation*}
	Quindi possiamo affermare che $3 \mid -(y + 4)$. Sappiamo per\`o che $(3, 1) = 1$, dunque
	possiamo affermare che $3 \mid (y + 4)$ per il primo lemma. Possiamo scrivere adesso che
	\begin{equation*}
		\begin{array}{rl}
			y + 4 & = 3k     \\
			y     & = 3k - 4
		\end{array}
	\end{equation*}
	Se sostituisco nell'equazione ottenuta precedentemente ottengo
	\begin{equation*}
		\begin{array}{rl}
			3(x - 2) + (3k - 4 + 4) & = 0     \\
			3x - 6 + 3k             & = 0     \\
			x                       & = 2 - k
		\end{array}
	\end{equation*}
	Le soluzioni all'equazione devono essere dunque del tipo
	\begin{equation*}
		\begin{cases}
			y & = 3k - 4 \\
			x & = 2 - k
		\end{cases}
	\end{equation*}
	Infatti se sostituisco ottengo
	\begin{equation*}
		\begin{array}{rl}
			3 (2 - k - 2) + (3k - 4 + 4) & = 0 \\
			-3k + 3k                     & = 0 \\
		\end{array}
	\end{equation*}
	che \`e sempre vera per qualsiasi $k \in \mathbb{Z}$.
\end{example}

\section{Studio delle soluzioni di un'equazione diofantea}
Come sappiamo l'equazione diofantea
\begin{equation*}
	ax + by = c
\end{equation*}
ha soluzione se e solo se $c$ \`e un multiplo di $(a, b)$. Le soluzioni dell'equazione si
ottengono sommando ad una particolare soluzione $(x_0, y_0)$ una qualunque soluzione
dell'omogenea
\begin{equation*}
	ax + by = 0
\end{equation*}
Il problema \`e trovare le soluzioni dell'omogenea.
Prendiamo ad esempio
\begin{equation*}
	1020x + 351y = 0
\end{equation*}
Una soluzione \`e $(-351, 1020)$ ma ne ho altre come ad esempio $(-k \cdot 351, k \cdot 1020)$
con $k \in \mathbb{Z}$.
Non sono ancora tutte per\`o. Dato che $(1020, 351) = 3$ divido tutto per 3 e ottengo
\begin{equation*}
	340x + 117y = 0
\end{equation*}
Qui le soluzioni sono del tipo $(k \cdot 117, -k \cdot 340)$ quindi prima trovavo una $x$ ogni
351 passi ora ogni 117. In questo modo ho trovato tutte le possibili soluzioni.

\begin{theorem}
	Se $(a, b) = 1$, allora le soluzioni di
	\begin{equation*}
		ax + by = 0
	\end{equation*}
	sono tutte del tipo $(kb, -ka)$.
\end{theorem}
