\chapter{Cifrari a chiave pubblica}
Nei cifrari simmetrici si ha un grosso problema, ovvero lo \textbf{scambio della chiave} che come sappiamo, deve
essere la stessa per entrambi gli utenti.

Fino ad ora abbiamo sempre assunto che i due utenti fossero gi\`a in possesso della chiave e abbiamo solo parlato del
metodo di cifratura e decifrazione ma come avviene lo scambio della chiave ?
\begin{itemize}
	\item Se avviene di persona allora tanto vale scambiarsi direttamente il messaggio.
	\item Se avviene in chiaro non ha pi\`u senso cifrare il messaggio dato che chiunque potrebbe intercettare la
	      chiave e decifrarlo senza sforzo.
	\item Se inviassimo la chiave cifrata si innescherebbe lo stesso problema all'infinito: il destinatario avrebbe
	      bisogno della chiave di cifratura per decifrare e dunque si dovrebbe inviare un'altra chiave e cos\`i via.
\end{itemize}
I \textbf{cifrari a chiave pubblica} risolvono il problema.

\section{Soluzione ingenua}
Se abbiamo un sistema con $N$ utenti, ogni utente pu\`o memorizzare $N-1$ chiavi diverse e condivise con ciascun altro
utente.

In questo modo abbiamo un numero quadratico di chiavi nel numero di utenti del sistema.

Quando un utente $A$ vuole comunicare con l'utente $B$ manda un messaggio a $B$ cifrandolo con la chiave $k_B$.
L'utente $B$ a questo punto decifra il messaggio con la sua chiave $k_B$ e invia un messaggio ad $A$, cifrandolo con
la chiave $k_A$.

\section{TTP}
Una soluzione migliore \`e rappresentata dal \textbf{TTP} o \textbf{trusted 3rd party}, ossia una terza parte
\emph{fidata}, a cui gli utenti si appoggiano per comunicare.

Ogni utente deve ricordarsi una sola chiave mentre TTP gestisce la creazione e lo scambio delle chiavi condivise tra i
due utenti.

Siano $A$ e $B$ i due utenti che vogliono comunicare, il processo di scambio funziona in questo modo:
\begin{enumerate}
	\item $A$ e $B$ possiedono rispettivamente due chiavi $k_A$ e $k_B$.
	\item $A$ comunica a TTP di voler comunicare con $B$.
	\item TTP genera casualmente una chiave $k_{AB}$ che potranno usare i due utenti per quella comunicazione.
	\item TTP genera due crittogrammi $c_A$ e $c_B$ in questo modo
	      \[
		      \begin{matrix}
			      c_A & = & C(k_{AB}, k_A) \\
			      c_B & = & C(k_{AB}, k_B)
		      \end{matrix}
	      \]
	\item TTP invia $c_A$ e $c_B$ ad $A$.
	\item $A$ decifra $c_A$ con $k_A$ e invia $c_B$ a $B$.
	\item $B$ decifra $c_B$ con $k_B$.
\end{enumerate}
Alla fine di questo processo i due utenti sono in possesso di una chiave $k_{AB}$ che potranno usare per quella
comunicazione in un cifrario simmetrico.

\subsection{Svantaggi}
Le problematiche di questo sistema sono due:
\begin{itemize}
	\item TTP deve essere sempre online.
	\item TTP conosce tutte le chiavi.
\end{itemize}
\`E un approccio utilizzabile solo in un sistema ristretto come un'universit\`a o un'azienda.

\section{Chiave pubblica}
In questo tipo di cifratura si vuole implementare un meccanismo che permette a chiunque di inviare messaggi cifrati a
un certo utente ma permettere solo a quell'utente di decifrarli.

Le operazioni di cifratura e decifrazione sono pubbliche e utilizzano due chiavi diverse:
\begin{itemize}
	\item \textbf{Chiave pubblica}: \`e nota a tutti.
	\item \textbf{Chiave private}: nota solo al destinatario.
\end{itemize}
Questa coppia di chiavi \`e generata dall'utente in veste di destinatario, il quale rende nota la sua chiave pubblica e
mantiene segreta la sua chiave privata.

Se volessimo inviare un messaggio all'utente $i$ si dovrebbe cifrare il messaggio con la sua chiave pubblica. L'utente
$i$ decifra il crittogramma con la sua chiave privata.

In questo sistema l'unica cosa non nota a tutti \`e la chiave privata del destinatario. Le funzioni di cifratura e
decifrazione e la chiave pubblica sono note a qualsiasi utente.

\subsection{Pro e contro}
I cifrari a chiave pubblica hanno due principali vantaggi:
\begin{itemize}
	\item Se ci sono $n$ utenti nel sistema, il numero complessivo di chiavi pubbliche e private \`e $2n$ anzich\'e
	      $\frac{n(n-1)}{2}$.
	\item Non \`e richiesto alcuno scambio segreto di chiavi.
\end{itemize}
Possiedono tuttavia anche due principali svantaggi:
\begin{itemize}
	\item Sono molto pi\`u lenti dei cifrari simmetrici.
	\item Sono esposti ad attacchi di tipo \emph{chosen plain text}.
\end{itemize}

\subsection{Attacchi chosen plain text}
Un crittoanalista quando effettua un'attacco di questo tipo
\begin{enumerate}
	\item Si procura un po' di messaggi in chiaro.
	\item Li cifra con la funzione pubblica $C$ e con la chiave pubblica $k_{\text{pub}}$ del destinatario.
	\item Confronta infine i crittogrammi in suo possesso con i crittogrammi che passano sul canale.
\end{enumerate}
Un attacco di questo tipo \`e molto pericoloso nel caso in cui il crittoanalista sospetti che un certo messaggio debba
transitare sul canale.

Ecco perch\'e questi cifrari non sono usati per comunicare ma solo per scambiarsi la chiave di un cifrario simmetrico
(AES per esempio).

In questo modo risolviamo in un colpo solo tutti i problemi elencati in precedenza:
\begin{itemize}
	\item Il processo \`e "lento" solo per lo scambio della chiave (256 bit) ma la comunicazione viene fatta con un
	      cifrario simmetrico molto veloce.
	\item L'attacco di tipo \emph{chosen plain text} diventa inattuabile dato che la chiave \`e una sequenza di bit
	      casuale e dunque impossibile da prevedere.
\end{itemize}

\subsection{Requisiti}
Perch\'e la cifratura a chiave pubblica sia effettuata correttamente devono essere soddisfatti alcuni requisiti:
\begin{itemize}
	\item Il destinatario deve essere in grado di decifrare qualsiasi messaggio con la propria chiave privata anche se
	      questo \`e stato cifrato con la chiave pubblica:
	      \[ m = D(C(m, k_{\text{pub}}), k_{\text{priv}}) \]
	\item Efficienza e sicurezza del sistema:
	      \begin{itemize}
		      \item La coppia di chiavi \`e \emph{facile} da generare e deve risultare praticamente impossibile che due
		            utenti scelgano la stessa chiave.
		      \item Dati $m$ e $k_{\text{pub}}$ \`e \emph{facile} per il mittente produrre il crittogramma.
		      \item Dati $c$ e $k_{\text{priv}}$ \`e \emph{facile} per il destinatario produrre il messaggio.
		      \item Pur conoscendo la chiave pubblica e le funzioni di cifratura e decifrazione, deve essere
		            \emph{difficile} per il crittoanalista risalire al messaggio in chiaro.
	      \end{itemize}
\end{itemize}
La soluzione risiede nel trovare una funzione di tipo \textbf{one-way trap-door}, ovvero, una funzione facile da
calcolare e difficile da invertire a meno che non si conosca la chiave privata.

\section{RSA}
Questo cifrario usa l'algebra modulare e si basa sulla moltiplicazione di due numeri primi $p$ e $q$ poich\'e,
calcolare
\[ n = p \cdot q \]
richiede tempo quadratico nella lunghezza della loro rappresentazione ma, ricostruire $p$ e $q$ a partire da $n$,
richiede tempo esponenziale se $p$ e $q$ sono primi.

Se si conosce tuttavia uno dei due fattori, risalire all'altro \`e facile (basta fare una divisione).

\subsection{Generazione delle chiavi}
Per la generazione delle due chiavi, il destinatario deve
\begin{enumerate}
	\item Scegliere due numeri primi $p$ e $q$ molto grandi, dove per "molto grandi" intendiamo tali che $p \cdot q$
	      sia un numero di circa 2000 bit per una protezione fino al 2030 o di circa 3000 bit se vogliamo una
	      protezione oltre il 2030.

	      Per farlo dobbiamo generare numeri di circa 1500 bit ed effettuare il test di Miller-Rabin per la
	      primalit\`a finch\'e non otteniamo due numeri primi.
	\item Calcolare
	      \[ n = p \cdot q \]
	      e la relativa funzione di Eulero
	      \[ \phi(n) = (p - 1)(q - 1) \]
	\item Scegliere un intero $e$ tale che
	      \[ e < \phi(n) \quad \wedge \quad (e, \phi(n)) = 1 \]
	\item Calcolare con l'algoritmo di Euclide esteso
	      \[ d = e^{-1} \mod{\phi(n)} \]
	      ossia l'inverso di $e$ modulo $\phi(n)$.
\end{enumerate}

\subsection{Cifratura e decifrazione}
Sia $m < n$ un messaggio scritto come una sequenza binaria e trattato come un intero. Il valore decimale di $m$ deve
essere strettamente minore di $n$, se cos\`i non fosse dobbiamo dividere $m$ in blocchi di $b = \log n$ bit

Questo requisito \`e dovuto al fatto che tutte le operazioni sono fatte modulo $n$ e quindi $m \geq n$ implicherebbe
che
\[ m \neq m \mod{n} \]
Si avrebbe quindi una funzione non iniettiva che mappa messaggi diversi nello stesso crittogramma.

Nella pratica si fissa il valore di $b$ in modo da dare un limite inferiore a $n$, costringendo cos\`i l'utente ad usare
chiavi lunghe
\[ m < 2^b < n \]
Per la \textbf{cifratura} si utilizza la chiave pubblica ed \`e definita come segue
\[ c = m^e \mod{n} \]
Per la \textbf{decifrazione} si utilizza la chiave privata del destinatario ed \`e definita in questo modo
\[ m = c^d \mod{n} \]
Sia cifratura che decifrazione hanno costo polinomiale con l'algoritmo di \emph{esponenziazione veloce}.

\begin{example}
	Consideriamo due numeri primi $p = 5$ e $q = 11$, otteniamo quindi
	\[ n = 55 \]
	Calcoliamo la funzione di Eulero su $n$
	\[ \phi(n) = (p - 1) (q - 1) = 40 \]
	Scegliamo ora un numero minore di 40 e coprimo con esso, per esempio
	\[ e = 7 \]
	La coppia $\langle n, e \rangle$ sar\`a la nostra chiave pubblica. Calcoliamo ora la chiave privata $d$ con
	l'algoritmo di Euclide esteso
	\[ d = 7^{-1} \mod{40} = 23 \]
	Adesso non ci rimane che cifrare o decifrare con le funzioni di cifratura e decifrazione descritte sopra
	\begin{align*}
		c = & m^7 \mod{55}    \\
		m = & c^{23} \mod{55}
	\end{align*}
\end{example}

\subsection{Correttezza}
Dimostriamo che il cifrario \`e corretto.

\begin{proof}
	Per dimostrare la correttezza del cifrario dobbiamo dimostrare che
	\[ m = D(C(m, k_\text{pub}), k_\text{priv}) \]
	per farlo dobbiamo dimostrare che
	\[ m = {\underbrace{(m^e \mod{n})}_c}^d \mod{n} \]
	che possiamo riscrivere come
	\[ m = m^{ed} \mod{n} \]
	Si procede per casi:
	\begin{enumerate}
		\item Nel primo caso supponiamo $m$ e $n$ coprimi. Dato che $m$ e $n$ sono coprimi vale il teorema di Eulero:
		      \[ m^{\phi(n)} \equiv 1 \mod{n} \]
		      Partendo dalla definizione di inverso facciamo anche un'altra considerazione: dato che $e$ e $d$ sono
		      l'uno l'inverso dell'altro modulo $\phi(n)$ vale
		      \[ ed \equiv 1 \mod{\phi(n)} \]
		      che equivale a dire che
		      \[ ed = r \cdot \phi(n) + 1 \]
		      con $r \in \mathbb{N}$. Possiamo quindi scrivere l'ipotesi come
		      \[ m^{ed} \mod{n} = m^{r \phi(n) + 1} \mod{n} \]
		      Per le propriet\`a delle potenze possiamo scrivere
		      \[ m \cdot (m^{\phi(n)})^r \mod{n} \]
		      Per il teorema di Eulero questo equivale a
		      \[ m \cdot 1^r \mod{n} = m \mod{n} = m\]
		      dato che $m < n$.
		\item Nel secondo caso $(n, m) \neq 1$. Il massimo comun divisore tra $n$ ed $m$ deve essere o $p$ o $q$ e
		      vale che
		      \[ p \mid m \quad \text{oppure} \quad q \mid m \]
		      ma non entrambi possono essere divisori di $m$ perch\'e altrimenti $m$ sarebbe maggiore o uguale a $n$.
		      Supponiamo che $p \mid m$ quindi $q \nmid m$, questo ci dice che
		      \[ m \equiv 0 \mod{p} \quad \Rightarrow \quad m^r \equiv 0 \mod{p} \]
		      per ogni $r \in \mathbb{N}$. Vale quindi che
		      \[ m^r - m \equiv 0 \mod{p} \]
		      Se prendiamo $r = ed$, otteniamo
		      \[ m^{ed} - m \equiv 0 \mod{p} \]

		      Consideriamo ora $q$, che ricordiamo, non divide $m$ e consideriamo
		      \[ m^{ed} \mod{q} \]
		      Questa equazione pu\`o essere riscritta come
		      \[ m^{k \phi(n) + 1} \mod{q} \]
		      ma come sappiamo $\phi(n)$ si pu\`o riscrivere come $(p - 1)(q - 1)$
		      \[ m \cdot m^{k (p - 1)(q - 1)} \mod{q} \]
		      come prima, per le propriet\`a delle potenze, possiamo scrivere
		      \[ m \cdot (m^{(q - 1)})^{k (p - 1)} \mod{q} \]
		      Dato che $q$ \`e un numero primo possiamo riscrivere l'equazione in questo modo
		      \[ m \cdot (m^{\phi(q)})^{k (p - 1)} \mod{q} \]
		      Usiamo il teorema di Eulero:
		      \begin{gather*}
			      m \cdot 1^{k (p - 1)} \mod{q} = m \mod{q} \\
			      \Downarrow \\
			      m^{ed} \mod{q} = m \mod{q} \\
			      \Downarrow \\
			      m^{ed} - m \equiv 0 \mod{q}
		      \end{gather*}

		      Possiamo quindi affermare che $m^{ed} - m$ \`e divisibile sia per $p$ che per $q$, quindi vale che
		      \begin{gather*}
			      m^{ed} - m \equiv 0 \mod{n} \\
			      \Downarrow \\
			      m^{ed} \mod{n} = m \mod{n}
		      \end{gather*}
		      ma dato che $m < n$ otteniamo
		      \[ m \mod{n} = m \]
		      Questo dimostra la correttezza anche quando messaggio e modulo non sono coprimi.
	\end{enumerate}
\end{proof}

\subsection{Sicurezza}
La \textbf{sicurezza} del cifrario \`e molto legata alla difficolt\`a nel fattorizzare un numero arbitrario molto
grande.

Riuscire a fattorizzare in tempo polinomiale \`e condizione sufficente per riuscire a forzare l'RSA ma non ci sono
dimostrazione del fatto che sia anche condizione necessaria.

Dato che il crittogramma $c$ \`e ottenuto tramite
\[ c = m^e \mod{n} \]
si potrebbe pensare di calcolare una radice
\[ m = \sqrt[e]{c} \mod{n} \]
ma dato che stiamo lavorando nell'algebra modulare e $n$ \`e un numero composto, l'estrazione della radice \`e un
problema \emph{difficile} almeno quanto la fattorizzazione.

\`E dunque pi\`u conveniente fattorizzare e compiere un attacco all'intero sistema piuttosto che sul singolo
crittogramma.

Anche calcolare $\phi(n)$ direttamente da $n$ \`e un problema \emph{difficile} quanto la fattorizzazione.

Provare a ricavare $d$ direttamente da $n$ ed $e$ \`e un problema che \emph{sembra} costoso quanto la
fattorizzazione.

Il problema della fattorizzazione rimane \emph{difficile} ma non pi\`u come un tempo. Ci sono ad oggi algoritmi
pi\`u raffinati che richiedono un numero di operazioni \textbf{subesponenziale}. Parliamo dell'algoritmo GNFS, il
quale, dato un intero $n$ rappresentato con
\[ b = \lceil \log n \rceil + 1 \]
bit, richiede un numero di operazioni dell'ordine di $O(2^{\sqrt{b \log b}})$ per fattorizzarlo.

\subsubsection{Vincoli sui parametri}
I due parametri $p$ e $q$ devono essere scelti in base ad alcune propriet\`a, senza le quali si riuscirebbe
facilmente forzare il cifrario:
\begin{itemize}
	\item Devono essere numeri di almeno 1024 bit per resistere agli attacchi a forza bruta.
	\item Sia $p-1$ che $q-1$ devono contenere un fattore primo molto grande (altrimenti $n$ si fattorizza
	      velocemente).
	\item Il massimo comun divisore tra $p-1$ e $q-1$ deve essere piccolo. Conviene scegliere $p$ e $q$ tali che
	      \[ \left( \frac{p-1}{2}, \frac{q-1}{2} \right) \]
	\item Non si deve riusare uno dei due primi per altre sessioni di comunicazione.
	      \[
		      \begin{matrix}
			      n_1 = p \cdot q_1 \\
			      n_2 = p \cdot q_2
		      \end{matrix}
		      \quad \Rightarrow \quad
		      p = (n_1, n_2)
	      \]
	\item La differenza tra $p$ e $q$ deve essere grande. Se cos\`i non fosse $p^2$ e $q^2$ sarebbero circa $n$ e
	      quindi $\sqrt{n}$ sar\`a vicino ai numeri primi.

	      Si effettua quindi un attacco a forza bruta che cerca i fattori vicini a $\sqrt{n}$.
\end{itemize}
Anche la scelta di $e$ e $d$ deve rispettare alcune propriet\`a. \`E vero che valori di $e$ e $d$ bassi accelerano
la cifraura e la decifrazione ma ci fa andare incontro ad altri problemi
\begin{itemize}
	\item Se $d$ \`e piccolo l'attacco a forza bruta diventa un'opzione valida.
	\item Se $m$ ed $e$ sono cos\`i piccoli che
	      \[ m^e < n \]
	      allora risulta facile trovare la radice $e$-esima di $c$ poich\'e non interviene la riduzione in modulo.
\end{itemize}
Ecco perch\'e, per avere garanzie sulla sicurezza, \`e importante scegliere anche $e$ e $d$ opportunamente grandi.

\subsection{Attacchi}
Vediamo qualche attacco efficiente nel caso si faccia un uso scorretto dei parametri $e$ ed $n$.

\subsubsection{Scelta di e}
Il parametro $e$, non solo deve essere sufficientemente grande, ma deve anche possedere alcune propriet\`a
\begin{itemize}
	\item Deve essere tale che
	      \[ e \neq \frac{\phi(n)}{2} + 1 \]
	\item Scelto $k \in \mathbb{N}$ tale che $k \mid p-1$ e $k \mid q-1$ allora
	      \[ e \neq \frac{\phi(n)}{k} + 1 \]
\end{itemize}
Se questi due vincoli non sono soddisfatti allora vale
\[ m^e \mod{n} = m \]
quando $m$ ed $n$ sono coprimi.

Supponiamo ora che ci siano almeno $e$ utenti che hanno scelto lo stesso valore di $e$ e supponiamo che tutti questi
utenti ricevano lo stesso messaggio $m$. Si va a creare una situazione di questo tipo
\[
	\begin{matrix}
		U_1    & c_1 = m^e \mod{n_1} \\
		U_2    & c_2 = m^e \mod{n_2} \\
		\vdots & \vdots              \\
		U_e    & c_e = m^e \mod{n_e}
	\end{matrix}
\]
Supponiamo ora che, per ogni coppia $i, j$ con $1 \leq i < j \leq e$, valga
\[ (n_i, n_j) = 1 \]
Per le propriet\`a del cifrario, deve valere, per ogni $i$ con $1 \leq i \leq e$
\[ m < n_i \]
affinch\'e $m$ possa essere cifrato.

In queste condizioni \`e possibile applicare il teorema cinese del resto, il quale ci consente di calcolare in
tempo polinomiale un valore $m' < n$ tale che
\[ m' \equiv m^e \mod{n} \]
dove $n$ \`e il prodotto di tutti gli $n_i$
\[ n = \prod_{i=1}^e n_i \]
Una volta in possesso di $m'$ si possono fare i seguenti passaggi
\begin{enumerate}
	\item Per le propriet\`a di $m'$ descritte sopra vale che
	      \[ m' \mod{n} = m^e \mod{n} \]
	      e, dato che $m' < n$, vale che
	      \[ m' \mod{n} = m' \]
	\item Notiamo ora che
	      \[ m^e = \underbrace{m \cdot m \cdot \text{\dots} \cdot m}_{e \text{ volte}} \]
	      e che
	      \[ n_1 \cdot n_2 \cdot \text{\dots} \cdot n_e = n \]
	      Ma, dato che tutti gli utenti hanno ricevuto lo stesso messaggio e, come abbiamo gi\`a detto, vale
	      \[ m < n_i \]
	      per qualsiasi $1 \leq i \leq e$, allora sappiamo vale
	      \[ m \cdot m \cdot \text{\dots} \cdot m < n_1 \cdot n_2 \cdot \text{\dots} \cdot n_e \]
	      quindi
	      \[ m^e < n \quad \Rightarrow \quad m^e \mod{n} = m^e \]
	\item Si ottiene quindi l'eguaglianza
	      \[ m' = m^e \]
	\item Si calcola
	      \[ m = \sqrt[e]{m'} \]
\end{enumerate}
Si risolve il problema facendo un po' di \emph{padding}.

\subsubsection{Attacco del modulo comune}
Supponiamo ora che due utenti utilizzino lo stesso valore di $n$. Il primo utente ha a disposizione la coppia
$\langle e_1, n \rangle$ e il secondo utente la coppia $\langle e_2, n \rangle$.

Se $e_1$ ed $e_2$ sono coprimi e i due utenti ricevono lo stesso messaggio $m$, vale
\[
	\begin{matrix}
		c_1 = m^{e_1} \mod{n} \\
		c_2 = m^{e_2} \mod{n}
	\end{matrix}
\]
allora si procede in questo modo
\begin{enumerate}
	\item Dato che $e_1$ ed $e_2$ sono coprimi devono esistere $r$ ed $s$ tali che
	      \[ r e_1 + s e_2 = 1 \]
	      e si trovano in tempo polinomiale con l'algoritmo di Euclide esteso. Supponiamo $r < 0$ e $s > 0$.
	\item Dato che $m = m^1$ e 1 pu\`o essere scritto in funzione dei parametri trovati al passo precedente in
	      questo modo
	      \[ m = m^{r e_1 + s e_2} \]
	      e dato che $m < n$ possiamo scrivere
	      \[ m^{r e_1 + s e_2} \mod{n} \]
	\item Per le propriet\`a delle potenze vale la seguente catena di uguaglianze
	      \[
		      \begin{array}{rcl}
			      m^{r e_1 + s e_2} \mod{n} & = & (m^{r e_1} \mod{n}) (m^{s e_2} \mod{n}) \mod{n} \\
			                                & = & (m^{e_1} \mod{n})^r (m^{e_2} \mod{n})^s \mod{n} \\
			                                & = & c_1^r \cdot c_2^s \mod{n}
		      \end{array}
	      \]
	\item Dato che $s > 0$ allora $c_2$ si calcola con l'algoritmo di esponenziazione veloce in tempo polinomiale.
	\item Dato che invece $r < 0$ si deve fare qualche passaggio in pi\`u
	      \begin{itemize}
		      \item Si scrive $c_1^r \mod{n}$ come
		            \[ (c_1^{-1})^{-r} \mod{n} \]
		      \item Adesso abbiamo $-r > 0$.
		      \item Nel caso in cui $c_1$ sia coprimo con $n$ si calcola il suo inverso moltiplicativo con
		            l'algoritmo di Euclide e si eleva alla $-r$ con l'algoritmo di esponenziazione veloce.
		      \item Nel caso in cui $c_1$ non sia coprimo con $n$ allora siamo riusciti a fattorizzare $n$ dato
		            che quest'ultimo \`e un prodotto di due primi.
	      \end{itemize}
\end{enumerate}

\subsubsection{Attacchi a tempo}
Sono attacchi che cercano di stimare $d$ andando a vedere il tempo di decifrazione che impiega l'algoritmo.

Si basa sul fatto che quando si esegue l'esponenziazione modulare si esegue una moltiplicazione ad ogni iterazione
e un'ulteriore moltiplicazione modulare per ciascun bit uguale a 1 in $d$.

Basta inserire ritardi casuali per proteggersi da questo tipo di attacchi.

\subsubsection{Man in the middle}
Questo attacco, di tipo \textbf{attivo}, consiste nella corruzione della chiave pubblica.
\begin{enumerate}
	\item Il crittoanalista intercetta la comunicazione e sostituisce la chiave pubblica del destinatario con la
	      sua.
	\item Il mittente riceve quindi la chiave pubblica del crittoanalista pensando di avere in mano quella del
	      destinatario.
	\item Il mittente invia la chiave simmetrica sul canale cifrandola con la chiave pubblica del crittoanalista.
	\item Il crittoanalista intercetta la chiave simmetrica cifrata con la propria chiave e la decifra.
	\item Per non interrompere la comunicazione, il crittoanalista cifra la chiave simmetrica del mittente con la
	      chiave del destinatario e la invia a quest'ultimo.
\end{enumerate}
Alla fine di questo procedimento i due utenti hanno correttamente instaurato un protocollo di comunicazione (magari
con un leggero ritardo) ma anche il crittoanalista \`e in possesso della chiave simmetrica con la quale pu\`o
decifrare facilemente ogni messaggio che passa sul canale.

Il modo per proteggersi da questi attacchi \`e quello di prendere i valori dei vari parametri pubblici da un
certificato digitale.

\section{Protocollo DH}
Il \textbf{protocollo DH}, dai due crittografi che lo hanno implementato (Diffie ed Hellman), si basa sul problema
del calcolo del logaritmo discreto.

Questo protocollo nasce per dare pari responsabilit\`a sia a mittente che a destinatario nella creazione della
chiave.

Nella pratica non c'\`e uno scambio di chiave ma uno scambio di informazioni necessarie alla costruzione locale
di una chiave simmetrica.

\subsection{Funzionamento}
Affinch\'e il procedimento funzioni, i due utenti $A$ e $B$, che intendono comunicare, devono scegliere
\begin{itemize}
	\item Un numero primo $p$ molto grande.
	\item Un generatore $g$ per $\mathbb{Z}_p^*$, dove $\mathbb{Z}_p^*$ \`e un gruppo ciclico.
\end{itemize}
Questi due dati sono pubblici.

I due utenti, per instaurare una comunicazione su un canale insicuro, devono
\begin{enumerate}
	\item Scegliere rispettivamente due interi $a$ e $b$ casuali, tali che
	      \[ 1 < a, b < p-1 \]
	\item Calcolare rispettivamente
	      \begin{gather*}
		      c_A = g^a \mod{p} \\
		      c_B = g^b \mod{p}
	      \end{gather*}
	      con l'algoritmo di esponenziazione veloce.
	\item Inviare rispettivamente $c_A$ e $c_B$ sul canale.
	\item Calcolare rispettivamente
	      \begin{gather*}
		      k = c_B^a \mod{p} = g^{ba} \mod{p} \\
		      k = c_A^b \mod{p} = g^{ab} \mod{p}
	      \end{gather*}
\end{enumerate}
Alla fine di questo procedimento, gli utenti sono in possesso di una chiave simmetrica $k$ per la comunicazione.

\subsection{Attacchi}
Il crittoanalista conosce $p$, $g$, $c_A$ e $c_B$ ma non possiede ne $a$ ne $b$, fondamentali per il calcolo di $k$
in tempo polinomiale. Senza queste informazioni deve calcolare il logaritmo discreto che, come sappiamo, ha costo
subesponenziale. Il protocollo \`e dunque robusto ad attacchi passivi.

\subsubsection{Man in the middle}
Simuliamo il processo di costruzione della chiave visto in precedenza ma stavolta, il crittoanalista $C$ interviene
nella comunicazione.
\begin{enumerate}
	\item $A$, $B$ e $C$ scelgono rispettivamente tre interi $a$, $b$ e $c$ casuali, tali che
	      \[ 1 < a, b, c < p-1 \]
	\item $A$, $B$ e $C$ calcolano rispettivamente
	      \begin{gather*}
		      c_A = g^a \mod{p} \\
		      c_B = g^b \mod{p} \\
		      c_C = g^c \mod{p}
	      \end{gather*}
	      con l'algoritmo di esponenziazione veloce.
	\item $A$ e $B$ inviano rispettivamente $c_A$ e $c_B$ sul canale.
	\item $C$ intercetta $c_A$ e $c_B$ e li sostituisce con $c_C$.
	\item $A$ e $B$ calcolano rispettivamente
	      \begin{gather*}
		      k_A = c_C^a \mod{p} = g^{ac} \mod{p} \\
		      k_B = c_C^b \mod{p} = g^{bc} \mod{p}
	      \end{gather*}
	\item $C$ calcola
	      \begin{gather*}
		      k_A = c_A^c \mod{p} = g^{ac} \mod{p} \\
		      k_B = c_B^c \mod{p} = g^{bc} \mod{p}
	      \end{gather*}
\end{enumerate}
A questo punto $A$ e $B$ hanno prodotto due chiavi differenti delle quali $C$ \`e in possesso. Se uno dei due utenti
invia un crittogramma all'altro e $C$ non interviene in alcun modo, i due si accorgono immediatamente che c'\`e
stata una corruzione e ripartono da zero.

Il crittoanalista $C$ potrebbe per\`o continuare ad intercettare i crittogrammi sul canale e agire in questo modo
\begin{enumerate}
	\item $A$ invia un messaggio, cifrato con $k_A$.
	\item $C$ lo intercetta e lo decifra con $k_A$.
	\item $C$ cifra nuovamente il messaggio con $k_B$ e lo invia a $B$.
\end{enumerate}
Ovviamente vale lo stesso se \`e $B$ ad inviare un messaggio.

Il risultato \`e che i due utenti non sperimentano alcuna corruzione dei dati poich\'e sono in grado di decifrare
i crittogrammi che arrivano.

Anche in questo caso il modo per proteggersi da questi attacchi \`e quello di prendere i valori dei vari parametri
pubblici da un certificato digitale.

\section{Cifrario di ElGamal}
Anche \textbf{cifrario di ElGamal}, come il protocollo DH, basa la sua sicurezza sulla difficolt\`a nel calcolare il
logaritmo discreto. Al livello di sicurezza non offre niente in pi\`u del protcollo DH o dell'RSA dato che la
risoluzione di questo problema necessita lo stesso numero di operazioni della fattorizzazione.

\subsection{Generazione delle chiavi}
Per creare la coppia di chiavi si segue il seguente procedimento
\begin{enumerate}
	\item Si sceglie un numero primo $p$.
	\item Si sceglie un generatore $g$ 	di $\mathbb{Z}_p^*$.
	\item Si sceglie un intero casuale $x$ tale che
	      \[ 2 \leq x \leq p-2 \]
	\item Si calcola
	      \[ y = g^x \mod{p} \]
\end{enumerate}
Abbiamo cos\`i ottenuto la chiave pubblica
\[ k_\text{pub} = \langle p, g, y \rangle \]
e la chiave privata
\[ k_\text{priv} = x \]

\subsection{Cifratura}
Prima di vedere come si effettua la cifratura ci sono dei requisti che il messaggio $m$ deve avere:
\begin{itemize}
	\item Deve essere codificato come una sequenza binaria, trattata come un intero.
	\item Deve essere scelto tale che
	      \[ m < p \]
	      se cos\`i non fosse si fa una cifratura a blocchi di $\log p$ bit.
\end{itemize}
Passiamo ora al metodo di cifratura.
\begin{itemize}
	\item Si sceglie un valore casuale $r$ tale che
	      \[ 2 \leq r \leq p - 2 \]
	\item Si calcola la coppia di crittogrammi
	      \begin{gather*}
		      c = g^r \mod{p} \\
		      d = y^r \mod{p}
	      \end{gather*}
	      dove $y$ \`e la chiave pubblica del destinatario.
\end{itemize}
Per la decifrazione si calcola
\[ m = d \cdot c^x \mod{p} \]
Sviluppiamo i calcoli
\[
	\begin{matrix}
		d \cdot c^{-x} \mod{p} & = & m \cdot y^r \cdot c^{-x} \mod{p}         \\
		                       & = & m \cdot y^r \cdot (g^r)^{-x} \mod{p}     \\
		                       & = & m \cdot (g^x)^r \cdot (g^r)^{-x} \mod{p} \\
		                       & = & m \mod{p}
	\end{matrix}
\]
e, come sappiamo, dato che $m < p$, l'ultima equazione equivale a $m$.

\subsection{Attacchi}
Per quanto riguarda attacchi di tipo \emph{man in the middle} valgono le considerazione fatte per RSA e protocollo
DH, basta dunque estrarre i valori da alcuni certificati digitali per ovviare al problema.

Per quanto riguarda attacchi di tipo passivo, il crittoanalista potrebbe provare a trovare $r$, con il quale
riuscirebbe a decifrare il crittogramma.

\`E quindi molto importante che $r$ sia tenuto segreto e che non venga mai riutilizzato per sessioni diverse.

\subsubsection{Attacco su r}
Supponiamo che il mittente invii due messaggi $m_1$ ed $m_2$ al destinatario utilizzando lo stesso $r$ per la
cifratura. I crittogrammi generati sono
\[
	\begin{matrix}
		\langle c = g^r \mod{p}, \quad d_1 = y^r \mod{p} \rangle \\
		\langle c = g^r \mod{p}, \quad d_2 = y^r \mod{p} \rangle
	\end{matrix}
\]
Supponiamo ora che il crittoanalista venga a conoscenza di $m_1$. Dato che
\[
	\begin{matrix}
		d_1 = m_1 \cdot y^r \mod{p} \\
		d_2 = m_2 \cdot y^r \mod{p}
	\end{matrix}
\]
allora possiamo dire che
\[ d_1 m_1^{-1} = y^r \mod{p} \]
abbiamo quindi ottenuto $y^r$ e possiamo calcolare
\[ m_2 = d_2 \cdot (-y^r)^{-1} \mod{p} \]