\chapter{Introduzione}
In questo corso andremo a trattare gli aspetti fondamentali delle reti 
come \textbf{Internet}, \textbf{protocolli di rete} e
\textbf{interconnessioni di rete}.

Quando parliamo di \textbf{Internet} facciamo riferimento alla rete 
pubblica che utilizziamo per scambiare informazioni in rete. Per
riuscire a interagire con Internet dobbiamo aderire a certe specifiche 
ben definite tramite i protocolli. La rete Internet è un sistema 
complesso composto da:
\begin{itemize}
	\item \textbf{Host}: dispositivi utente come computer e smartphone.
	\item \textbf{Link di comunicazione}: mezzi trasmissivi sui quali
		viaggiano le informazioni come cavi in rame, fibra ottica e
		onde radio.
	\item \textbf{Dispositivi di interconnessione}: elementi intermedi
		come router o switch che permettono di condividere i link di
		comunicazione.
\end{itemize}
Passiamo ora a qualche definizione preliminare utile per orientarci
nei prossimi argomenti:
\begin{itemize}
	\item \textbf{Rete}: interconnessione di dispositivi, come host, 
		router e switch, in grado di scambiarsi informazioni.
	\item \textbf{Host}: tipicamente si distinguono in macchine 
		utente, se dedicati all'esecuzione di applicazioni, e server,
		adibiti all'erogazione di servizi per le macchine utente.
	\item \textbf{Dispositivi di interconnessione}: sono tipicamente 
		router (interconnessione di reti) e switch (interconnessione 
		di host).
	\item \textbf{Collegamenti}: sono i mezzi trasmissivi come cavi,
		onde radio e fibra ottica.
\end{itemize}

\section{Tipologie di reti}
Le tipologie principali di rete sono due: la \textbf{LAN}, una rete
a copertura locale, e la \textbf{WAN}, a copertura geografica. Queste
due differiscono sia per dimensione che per impiego.

\subsection{LAN}
Come prima tipologia di rete trattiamo la rete \textbf{LAN} (Local
Area Network), ossia una rete \emph{locale} e quindi circoscritta ad
un'area limitata. Sono reti \emph{private} e quindi di proprietà di 
un'organizazzione utilizzate per interconnetere degli host fra di loro 
e si estendono tipicamente fino a qualche chilometro.

All'inizio le reti LAN erano implementate tramite un cavo condiviso
tra gli host. Questo metodo risulta molto lento nel momento in cui il
traffico cresce in quanto ogni pacchetto, viaggiando sul cavo condiviso
arriva ad ogni host che controlla se tale pacchetto è destinato a lui 
oppure no.

La struttura moderna prevede l'utilizzo di uno switch al quale si 
connettono i vari host in modo indipendente. Lo switch si occupa di
reindirizzare i pacchetti che gli arrivano solo verso gli host
destinatari di tali pacchetti.

\subsection{WAN}
La seconda tipologia di rete è la rete \textbf{WAN} (Wide Area 
Network), ossia una rete \emph{geografica} il cui scopo è 
interconnetere LAN o host separati da distanze geografiche. \`E gestita
da un operatore di rete che fornisce servizi ai clienti. Esistono due 
tipi di WAN:
\begin{itemize}
	\item \textbf{Punto-Punto}: collega due dispositivi in modo
		diretto tramite un mezzo trasmissivo.
	\item \textbf{A commutazione}: collega due dispositivi o due reti
		distinte tramite dispositivi di interconnessione intermedi che
		effettuano la \emph{commutazione}.
\end{itemize}
Parleremo di commutazione nel prossimo paragrafo. Per il momento
prendiamo per buona la definizione così com'è.

