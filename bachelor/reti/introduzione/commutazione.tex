\section{Commutazione}
Una internet è data dall'interconnessione di reti composte da link e 
dispositivi capaci di scambiarsi informazioni. In particolare
distinguiamo i dispositivi in host e dispositivi di interconnessione.

Nel momento in cui vogliamo inviare un'informazione si deve definire
il metodo per con cui questa viaggia sulla rete.

Le tecniche di \textbf{commutazione} definiscono il paradigma usato 
per immettere le informazioni in rete. In particolare andiamo a 
distinguere due principali metodi di commutazione: commutazione di 
\textbf{pacchetto} e commutazione di \textbf{circuito}.

Come vedremo, queste due tecniche differiscono per prestazioni, 
servizi offerti e affidabilità.

\subsection{Commutazione di circuito}
La commutazione di circuito instaura un cammino dedicato tra i due
dispositivi che vogliono comunicare tramite una serie di linee di 
trasmissione e dispositivi di commutazione che collegano sorgente e 
destinazione. Ogni comunicazione che fa uso della commutazione di 
circuito si divide in tre fasi:
\begin{enumerate}
	\item \textbf{Inizializzazione}: fase iniziale in cui si richiede 
		di aprire una connessione verso una determinata entità e una 
		prima infrastruttura si occupa di determinare una connessione 
		dedicata tra mittente e destinatario.

		I dispositivi di mezzo sono incaricati di gestire tale 
		collegamento di modo che la connessione non si interrompa.
	\item \textbf{Trasferimento dati}: una volta determinato il 
		percorso si possono trasferire i dati grazie alle risorse
		allocate in fase di inizializzazione.

		Sarà poi compito delle varie strutture intermedie far sì che
		tali risorse rimangano disponibili per tutta la durata della 
		comunicazione.
	\item \textbf{Chiusura}: una volta terminato il trasferimento dati
		si chiude la connessione deallocando tutte le risorse.
\end{enumerate}
Gli svantaggi di questo tipo di commutazione sono principalmente due:
\begin{itemize}
	\item \`E necessaria una fase iniziale di inizializzazione prima
		di iniziare a comunicare.
	\item Le risorse dedicate ad una connessione rimangono inattive 
		se non utilizzate.
\end{itemize}
D'altro canto abbiamo prestazioni molto elevate in quanto, una volta 
stabilita la connessione, tutte le risorse sono allocate e non si 
subiscono interruzioni di alcun tipo dato che la fase iniziale ha 
assicurato che il percorso scelto fosse in grado di sostenere la 
connessione richiesta.

\subsection{Commutazione di pacchetto}
Questo paradigma per la commutazione cerca di risolvere i due problemi 
della commutazione di circuito, minimizzando il tempo speso nella fase
iniziale e migliorando la gestione delle risorse occupate dalle varie 
connessioni.

Il metodo consiste nel dividere l'informazione in pacchetti che vengono
immessi in rete in maniera del tutto indipendente. Una connessione a
commutazione di pacchetto è tipicamente suddivisa in queste quattro 
fasi:
\begin{enumerate}
	\item \textbf{Suddivisione}: l'informazione viene suddivisa in 
		pacchetti, i quali potranno viaggiare in modo indipendente 
		sulla rete (anche su percorsi diversi).
	\item \textbf{Accodamento}: nel caso in cui la rete sia
		congestionata, il router tiene i pacchetti in coda in un 
		\textbf{buffer} fin quando non ci sarà di nuovo la possibilità
		di inviarli.
	\item \textbf{Instradamento}: una volta che la rete è disponibile 
		all'inoltro del pacchetto il router determina, in base a 
		informazioni che gli fornisce il pacchetto (destinazione, 
		lunghezza in bit ecc.), quali sono i possibili nodi a cui 
		inviarlo e valuta quale tra questi è l'opzione migliore 
		valutando parametri come ad esempio lo stato di congestione 
		del traffico.
	\item \textbf{Ricomposizione}: una volta che tutti i pacchetti
		sono giunti a destinazione è necessario ricomporli nell'ordine
		corretto affinché l'informazione sia significativa.
\end{enumerate}
Nel caso in cui i pacchetti siano tenuti nel buffer si introduce un 
\textbf{ritardo} nella trasmissione. Ritardo che con la commutazione 
di circuito non avevamo dato che l'unico momento di attesa era dovuto 
all'inizializzazione.

Teniamo anche di conto che i buffer hanno dimensione finita e quindi,
nel caso in cui la rete sia congestionata e il buffer sia pieno, si è
costretti a scartare i pacchetti in arrivo.

Il vantaggio sta nella maggiore flessibilità. Se ad esempio uno dei 
dispositivi di interconnessione si guasta, i pacchetti possono 
comunque continuare il loro percorso passando da altri nodi.

Il commutatore (router) deve ricevere l'intero pacchetto prima di 
poterlo inoltrare ad un altro commutatore. Questo perché ha bisogno 
di informazioni per riuscire a determinare il canale d'uscita di tale 
pacchetto (\textbf{store and forward}).

C'è inoltre la possibilità di creare \textbf{code di priorità} per
dare la precedenza a tipologie di pacchetto che necessitano una 
trasmissione più rapida.

