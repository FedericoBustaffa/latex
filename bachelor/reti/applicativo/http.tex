\section{HTTP}
L'\textbf{HTTP} (Hyper Text Transfer Protocol) è il protocollo 
generalmente usato per navigare in rete e dunque per fare richieste di 
contenuti sul web. Si tratta di un protocollo \textbf{stateless}, ossia
senza memoria di stato, in quanto le coppie richiesta/risposta sono
indipendenti.

Il protocollo non è limitato all'ipertesto ma supporta lo scambio di 
informazioni di qualsiasi genere. Questo grazie un sistema di 
tipizzazione e negoziazione sulla rappresentazione dei dati.

Il protocollo utilizza TCP come servizio di trasporto in quanto non può
permettersi perdita di dati.

Per il protocollo HTTP possiamo andare ad aggiungere un ulteriore 
parametro alle URL, ossia la \verb|query|, una stringa di informazioni 
che deve essere interpretata dal server.
\begin{center}
	\verb|http://<host>:<port>/<path>?<query>|
\end{center}
In questo modo possiamo andare a manipolare la risorsa richiesta 
(sempre nei limiti di ciò che il server ci rende disponibile).

\subsection{Connessioni}
Una \textbf{connessione} è un circuito logico di livello trasporto 
stabilito tra due programmi applicativi per comunicare. Le connessioni 
possono essere
\begin{itemize}
	\item \textbf{Non persistenti}: ad ogni richiesta del client viene 
		aperta una nuova connessione TCP.
	\item \textbf{Persistenti}: la connessione TCP non viene chiusa fin
		quando non viene conclusa la sessione di scambio delle 
		informazioni.
\end{itemize}
Ad oggi HTTP implementa un tipo di connessione persistente che è 
diventato uno standard de facto.

\subsubsection{Pipelining}
Il \textbf{pipelining} serve per migliorare ulteriormente le 
prestazioni. Nella pratica consiste nell'invio, da parte del client,
di molteplici richieste senza aspettare la ricezioni di nessuna 
risposta con il limite che il server deve inviare le risposte nello 
stesso ordine in cui sono arrivate le richieste.

Il client può inviare in pipeline solo richieste che usano metodi HTTP 
\emph{idempotenti}, ossia operazioni che, anche se fatte più volte,
non variano il risultato finale.

Il problema principale di questo sistema è che il server deve 
rispondere alle richieste nello stesso ordine in cui gli sono arrivate 
e nel caso ci sia una richiesta particolarmente elaborata, questa 
blocca tutte le altre. Tutto questo senza contare il ritardo dovuto
al ripristino di segmenti persi.

\subsection{Messaggi HTTP}
Un \textbf{messaggio} HTTP può essere di due tipi: \textbf{request} o 
\textbf{response} ed è la riga iniziale che distingue la richiesta 
dalla risposta. Seguono una serie di \textbf{header}, ossia coppie 
nome/valore che forniscono informazioni in più sul messaggio. Infine
(anche se opzionale) c'è il corpo del messaggio. In realtà la risposta 
può essere anche solo la pagina html richiesta. La \textbf{request 
line} è composta da tre parti principali:
\begin{itemize}
	\item \textbf{Metodo}: l'operazione che il client richiede venga 
		eseguita.
	\item \textbf{Request URI}: l'oggetto dell'operazione specificata 
		con il metodo.
	\item \textbf{Versione HTTP}: formato del messaggio e capacità di 
		comprendere ulteriori comunicazioni.
\end{itemize}
Come abbiamo detto, nei messaggi HTTP possono essere specificati anche 
una serie di \emph{header}, i quali servono a specificare alcuni 
parametri del messaggio trasmesso o ricevuto. Possiamo avere header di 
vario tipo:
\begin{itemize}
	\item \textbf{Generali}: contengono informazioni sulla trasmissione
		come data, codifica o connessione.
	\item \textbf{Entità}: relativi all'entità trasmessa come la 
		lunghezza e il tipo del contenuto trasmesso.
	\item \textbf{Richiesta}: relativi alla richiesta, come ad esempio
		informazioni su chi fa la richiesta, a chi viene fatta la 
		richiesta, autorizzazioni ecc.
	\item \textbf{Risposta}: si trova nel messaggio di risposta e 
		contiene informazioni sul server come ad esempio 
		l'autorizzazione richiesta.
\end{itemize}
La \textbf{response line} è invece composta in questo modo:
\begin{itemize}
	\item \textbf{Status line}: contiene tre informazioni:
		\begin{itemize}
			\item Versione HTTP.
			\item Lo \emph{status code} (codice di 3 cifre).
			\item Una descrizione testuale del codice di stato.
		\end{itemize}
	\item \textbf{Headers}: header informativi sulla risposta che non 
		possono essere inseriti nella status line.
\end{itemize}
Gli status codes si dividono in 5 categorie:
\begin{itemize}
	\item 1xx: informativi.
	\item 2xx: successo.
	\item 3xx: redirezione ad azioni necessarie per completare 
		correttamente la richiesta.
	\item 4xx: errori del client nella richiesta.
	\item 5xx: errori del server nel gestire una richiesta 
		apparentemente corretta.
\end{itemize}
Sono i codici informativi che vengono inseriti nel messaggio di 
risposta dal server.

\subsubsection{Metodi di richiesta}
Nel messaggio di richiesta, come detto in precedenza, dobbiamo 
specificare un metodo che andrà ad influire anche sulla risposta che 
andremo a ricevere:
\begin{itemize}
	\item \verb|OPTIONS|: la risposta conterrà i metodi disponibili 
		per l'URI corrente.
	\item \verb|GET|: richiede il trasferimento di una risorsa 
		identificata da una URL e operazioni associate alla URL stessa.
		Sono possibili anche delle \emph{conditional get} in cui si
		aggiungono delle condizioni sulla risorsa richiesta.
	\item \verb|HEAD|: simile a \verb|GET| restituisce solo 
		l'intestazione del messaggio.
	\item \verb|POST|: metodo usato dal client per inviare contenuti 
		(nel corpo del messaggio) al server.
	\item \verb|PUT|: richiesta di creare o modificare una risorsa 
		(sono necessari i permessi).
	\item \verb|DELETE|: si chiede di eliminare qualcosa nella risorsa.
\end{itemize}
I metodi si dividono in \textbf{safe} e \textbf{idempotenti}. I metodi 
safe non hanno effetti collaterali in quanto non modificano in alcun 
modo la risorsa.

I metodi idempotenti possono avere effetti collaterali ma, anche se 
eseguiti diverse volte, hanno lo stesso effetto di una richiesta 
singola.

\subsection{Web Caching}
L'obbiettivo del \textbf{web caching} è quello di soddisfare le 
richieste del client senza contattare il server. Per riuscirci il 
client può mantenere delle copie locali delle risorse che aveva 
ottenuto in precedenza da altre richieste.

In alternativa è possibile utilizzare un \textbf{proxy} che intercetta 
il traffico e lo inoltra al server. Una volta ricevuta una risposta ne 
mantiene una copia e sarà quindi lui a fornire la risorsa al client.

\subsection{Cookies}
Come abbiamo detto HTTP è stateless, ossia non mantiene informazioni 
sul client. Con questo tipo di implementazione del protocollo nasce 
però un problema, che è quello di \emph{riconoscere} un utente di 
un'applicazione Web.

Si devono quindi creare applicazioni web \emph{con stato} tenendo di 
conto che ogni utente si connette ogni volta con indirizzo IP e porta 
differenti.

Ecco che viene creato il meccanismo dei \textbf{cookie}, con il quale 
è possibile creare delle \textbf{sessioni} sopra un protocollo 
stateless:
\begin{enumerate}
	\item Il client invia una richiesta al server.
	\item Il server manda un messaggio di risposta con una linea
		\begin{center} \verb|Set-cookie : <id>| \end{center}
	\item Il client memorizza il cookie in un file associandolo al 
		server e lo aggiunge con una linea
		\begin{center}
			\verb|cookie : <id>|
		\end{center}
		a tutte le sue successive richieste a quel sito.
	\item Il server confronta il cookie presentato con l'informazione 
		associata ad esso.
\end{enumerate}
I cookie vengono utilizzati per autenticazione, ricordare il profilo 
utente, scelte precedenti ecc.

\subsection{Prestazioni}
In precedenza abbiamo parlato del tentativo di migliorare le 
prestazioni per il caricamento delle pagine introducendo il meccanismo 
di pipeline. Il problema principale di quel meccanismo è che il server 
è vincolato nel servire le richieste una per una e in ordine di arrivo.

Un altro metodo, attualmente usato dai server HTTP/1.1 è quello della 
gestione di richieste in parallelo. I browser instaurano più 
connessioni in parallelo (massimo 6) per fare richieste. Questo però 
intacca l'equità del protocollo TCP in quanto un singolo processo 
richiede più risorse di altri e inoltre introduce un cattivo controllo 
della congestione.

\subsubsection{HTTP/2}
Con HTTP/2 si cerca di dare maggiore flessibilità lato server per 
l'invio di oggetti al client. Rispetto ad HTTP/1.1 non ci sono grosse 
variazioni per quanto riguarda metodi, codici di stato e
campi d'intestazione.

Con questa versione del protocollo il server può servire il client 
gestendo le richieste in ordine di priorità e può inviare al client 
anche oggetti non richiesti.

Per riuscire a farlo si dividono gli oggetti in \textbf{frame} i quali 
vengono gestiti da un meccanismo di \textbf{scheduling} per riuscire a 
servire prima le richieste ad alta priorità. La connessione TCP è 
sempre unica ma viene usata per inviare più richieste.

In HTTP/2 il frame è l'unità di comunicazione che compone parte del 
messaggio HTTP ed è suddiviso in header e dati.

Si introduce anche il concetto di \textbf{stream} come un flusso 
bidirezionale di frame all'interno di un'unica connessione TCP.
Mediante l'astrazione dello stream è possibile effettuare richieste
multiple sulla stessa connessione TCP.

Queste meccanismo permette di definire una priorità per ogni stream ed
eventuali dipendenze da altri stream tramite un albero delle 
dipendenze.

\`E inoltre possibile avere un compressione delle intestazioni di modo 
che un messaggio di richiesta o risposta comprenda tutte le 
intestazioni necessarie omettendo le informazioni non necessarie.

Si ha comunque il problema che, nel caso di perdita di pacchetti, tutta
la connessione viene rallentata per riuscire a ripristinare il 
pacchetto perso.

\subsubsection{HTTP/3}
Ultima versione del protocollo ancora in via di sviluppo che propone 
l'utilizzo di UDP e di un ulteriore protocollo, chiamato \textbf{QUIC},
che si frappone tra UDP e HTTP.

Il protocollo QUIC comprende meccanismi di sicurezza, controllo degli 
errori e controllo della congestione. La differenza sostanziale che 
introduce QUIC è la possibilità di gestire più flussi su una singola 
connessione astraendo lo stream a livello di trasporto.

Gli stream sono indipendenti l'uno dall'altro permettendo così, in caso
di perdita di pacchetti su uno di essi, di rallentare la connessione 
solo per quel flusso di dati e non per tutta la connessione.
