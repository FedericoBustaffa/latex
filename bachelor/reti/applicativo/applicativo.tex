\chapter{Livello applicativo}
In questo corso vedremo i quattro strati principali del modello 
TCP/IP andando dal livello applicativo a quello di collegamento. 

Iniziamo dall'alto con il livello applicativo, che è quello che
più ci è familiare ed è di uso comune per molti utenti di Internet.

Un'\textbf{applicazione di rete} è formata da processi distribuiti e 
comunicanti dove, per \textbf{processo}, intendiamo un programma 
eseguito dagli host di una rete.

All'interno dello stesso host, due processi possono comunicare 
attraverso la comunicazione \textbf{inter-processo} definita dal 
sistema operativo.

Nella comunicazione a livello applicativo fra due dispositivi terminali
diversi, all'interno della stessa rete, due o più processi girano su 
ciascuno degli host comunicanti e si scambiano messaggi.

I livelli applicazione nei due lati della comunicazione agiscono come 
se esistesse un collegamento diretto attraverso il quale poter 
comunicare.

Il protocollo definisce i tipi di messaggi scambiati a livello 
applicativo andando a definire la \emph{sintassi} dei vari tipi di 
messaggio (campi del messaggio), la \emph{semantica} dei campi 
(significato di ogni campo) e le \emph{regole} per determinare quando 
e come un processo invia messaggi o risponde ai messaggi.

I \textbf{paradigmi} utilizzati dalle applicazioni a questo livello
sono sostanzialmente due:
\begin{itemize}
	\item \textbf{Client-Server}: in cui ci sono dei processi 
		\textbf{server} che forniscono un servizio e dei processi 
		\textbf{client} che ne fanno richiesta.
	\item \textbf{Peer-To-Peer}: tutti i processi fungono sia da 
		client che da server.
\end{itemize}
Noi ci interesseremo principalmente di protocolli di tipo client-server
anche se a fine corso spenderemo qualche parola per alcuni sistemi
peer-to-peer più famosi. Esistono anche delle implementazione miste 
dei due paradigmi.

\section{Comunicazione tra processi}
Per riuscire a comunicare, viene fornita ai programmatori, una
\textbf{API}, ossia il \textbf{Socket}, che, altro non è che
un'astrazione di un endpoint. Tramite il Socket è possibile instaurare 
connessioni di tipo client-server per riuscire a scambiarsi messaggi.

Per ricevere messaggi, un processo deve avere un identificatore. La 
macchina host ha un indirizzo \textbf{IP} di 32 bit. L'indirizzo IP 
però ci è utile solo per identificare l'host, non il processo, per 
identificare il processo è necessario usare, in coppia all'IP, il 
numero di \textbf{porta} (16 bit) sulla quale il processo sta in 
ascolto.

\subsection{TCP vs UDP}
Se volessimo mettere a confronto i due servizi di trasporto 
\textbf{TCP} e \textbf{UDP} possiamo notare che TCP offre molte più 
garanzie e si comporta anche in modo più \emph{intelligente} rispetto 
a UDP.

TCP infatti è un servizio \emph{connection-oriented}, viene quindi 
richiesto un setup iniziale tra client e server, ha un trasporto 
affidabile e compie controlli sul flusso dati e sulla congestione 
della rete.

UDP invece ha dalla sua il fatto di essere un servizio 
\emph{connectionless} e che non richiede quindi un fase iniziale di 
setup. Per il resto non offre alcuna garanzia né sulla buona riuscita
né sulla qualità della comunicazione. \`E tuttavia molto usato per 
vari motivi
\begin{itemize}
	\item \textbf{Throughput}: Alcune applicazioni hanno bisogno di un 
		certo throughput per poter offrire un buon servizio all'utente.
		Per esempio se dobbiamo riprodurre un video online, abbiamo 
		bisogno di un alto throughput di modo che il video non si 
		blocchi di continuo o ci costringa ad abbassare la qualità. Se
		invece dobbiamo inviare del testo non ce ne facciamo niente di
		un alto throughput, esso influirà solo sulla velocità del 
		trasferimento, ma sarà comunque necessario che l'intero 
		pacchetto sia arrivato e ricompattato prima di poterne leggere 
		il contenuto.
	\item \textbf{Perdita dati}: Come nell'esempio di prima, quando 
		riproduciamo un video online, non importa se qualche frame va 
		perso. Mentre se stiamo inviando una mail e perdiamo qualche 
		byte, l'intera mail potrebbe risultare corrotta.
	\item \textbf{Sensibilità ai ritardi}: alcune applicazioni come 
		ad esempio giochi online hanno bisogno di un basso ritardo per
		avere un'interazione fluida.
\end{itemize}
In generale UDP offre maggiori garanzie al livello di utilizzo e di 
esperienza per tutte quelle applicazioni che richiedono un tipo di 
interazione con la rete molto veloce e responsivo.

