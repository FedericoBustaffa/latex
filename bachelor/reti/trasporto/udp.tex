\section{UDP}
Il protocollo \textbf{UDP} si dice un servizio a \emph{massimo sforzo} 
poiché i datagrammi UDP possono essere perduti o consegnati fuori 
sequenza e l'affidabilità può essere aggiunta ma solo a livello 
applicativo.

Si tratta inoltre di un servizio \textbf{orientato al messaggio} dato 
che ogni datagramma è indipendente dall'altro e i processi devono 
inviare messaggi di dimensioni limitate che possono essere incapsulati 
in un datagramma UDP.

In generale UDP è meno complesso di TCP e offre meno servizi ma è più 
indicato in contesti in cui è necessario un completo controllo della 
temporizzazione.

Si tratta infatti di un tipo di servizio \textbf{connectionless}, che 
evita quindi i ritardi dovuti alla gestione della connessione come ad 
esempio la fase di instaurazione e i vari controlli di flusso e 
congestione. Questo fa sì che l'intestazione del datagramma sia anche 
molto più breve (8 byte).

Anche l'operazione per il controllo dell'integrità dei pacchetti, ossia
la \textbf{checksum}, è facoltativa.

\subsection{Intestazione UDP}
Come anticipato, l'intestazione di un datagramma UDP ha una dimensione 
di 8 byte in cui sono riportati
\begin{itemize}
	\item \textbf{Porta sorgente}: la porta utilizzata dal mittente 
		per trasmettere il messaggio.
	\item \textbf{Porta destinatario}: la porta utilizzata dal 
		destinatario in fase di Demultiplexing.
	\item \textbf{Lunghezza messaggio}: lunghezza totale (header + 
		dati) del messaggio in byte.
	\item \textbf{Checksum}: valore per il controllo dell'integrità 
		del messaggio.
\end{itemize}
Fuori dall'intestazione troviamo il corpo del messaggio.

\subsubsection{Checksum}
Il valore di checksum viene calcolato sull'intero datagramma con una 
parte dello \emph{pseudo-header}, ossia dell'header IP (livello rete) 
composto dagli IP di mittente e destinatario, un campo di 8 zeri, il 
campo che indica il protocollo di trasporto e infine la lunghezza del 
datagramma. Per calcolare la checksum il mittente
\begin{enumerate}
	\item Setta il campo checksum a 0.
	\item Tratta il contenuto del datagramma come una sequenza di 
		parole da 16 bit.
	\item Somma le parole di 16 bit in complemento a 1.
	\item Complemento a 1 del risultato.
\end{enumerate}
Il risultato finale viene messo nel campo checksum del datagramma. Una 
volta che il destinatario riceve il datagramma calcola la checksum del 
segmento ricevuto senza settare il campo checksum del datagramma a 0.
Se il valore calcolato è 0 allora il pacchetto è integro.

\subsection{Conclusioni}
In conclusione possiamo dire che UDP offre maggiore velocità a 
discapito dell'affidabilità. \`E però molto utile in tutti i contesti 
in cui c'è una certa tolleranza alla perdita dei dati come ad esempio 
streaming multimediali (ad esempio perdita di qualche frame di un 
video).

Se necessario, è possibile implementare controlli a livello 
applicativo specifici e rigorsi tanto quanto viene richiesto 
dall'applicazione.

