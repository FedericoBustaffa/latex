\section{NAT}
Il servizio \textbf{NAT} (Network Address Translation) permette di 
abilitare il traffico verso/dalla rete Internet pubblica per una rete
privata.

Permette cioè di trasmettere su Internet traffico proveniente da 
sistemi attestati su sottoreti private, in cui sono assegnati indirizzi
IP privati.

L'accesso di una rete privata su Internet è realizzato attraverso un 
router abilitato alla NAT, il quale deve avere almeno un indirizzo IP
pubblico.

\subsection{Tabella NAT}
Tutto il traffico in uscita/ingresso da/verso la sottorete ha come
indirizzo IP sorgente/destinazione l'indirizzo IP pubblico del router.
Questo crea qualche problema soprattutto in fase di ricezione di
messaggi in quanto il router non sa a quale host inoltrare il
pacchetto.

Per risolvere questo problema il router implementa una tabella NAT la
quale contiene delle associazioni che permettono la traduzione. Nello
specifico, i campi della tabella contengono una corrispondenza tra
l'IP privato dell'host e l'IP pubblico del router e tra la porta
utilizzata dall'host e la porta scelta dal router.

Questa tabella traduce in un modo o nell'altro a seconda che il
pacchetto sia in ingresso o in uscita:
\begin{itemize}
	\item Nel caso in cui si voglia inviare un pacchetto, il router
		cambierà l'indirizzo IP privato dell'host sorgente con il
		proprio indirizzo IP pubblico e cambierà la porta utilizzata
		dall'host con una sua porta utilizzata per l'inoltro.
	\item Nel caso in cui si riceva un pacchetto questo avrà come
		indirizzo destinazione l'indirizzo IP pubblico del router 
		e una porta. Sfruttando l'associazione IP porta, tramite la
		tabella NAT è possibile inoltrare il pacchetto su una porta
		usata dall'host che aspetta il pacchetto.
\end{itemize}
Il router gestisce la tabella in modo che le corrispondenze tra
indirizzi IP e porte sia univoci di modo che più host della stessa
sottorete possano usare il servizio di NAT senza entrare in conflitto
tra loro.
