\section{Architettura di un router}
Fino ad ora abbiamo parlato di \emph{inoltro} e \emph{instradamento},
rissumiamo brevemente le differenze tra i due:
\begin{itemize}
	\item \emph{Inoltro}: meccanismo che si occupa di trasferire i
		pacchetti sull'appropriato collegamento in uscita. Per farlo
		si fa uso di una \textbf{tabella di inoltro}
	\item \emph{Instradamento}: processo decisionale di scelta del
		percorso verso una destinazione. In questo caso si determinano
		i valori da inserire nella tabella di inoltro attraverso 
		\textbf{algoritmi di routing}.
\end{itemize}
Gli algoritmi di routing si possono suddividere in due categorie:
\begin{itemize}
	\item \textbf{Decentralizzato}: su ogni router gira 
		indipendentemente un un algortimo di routing.
	\item \textbf{Centralizzato}: un controller remoto si occupa
		dell'esecuzione degli algoritmi di routing e poi fornisce i 
		valori da inserire nelle varie tabelle di inoltro dei router.
\end{itemize}
In questo corso trattaremo solamente algoritmi di routing 
decentralizzati.

\subsection{Struttura}
Un router è composto sostanzialmente da quattro componenti di nostro
interesse:
\begin{itemize}
	\item \textbf{Porte di input}
	\item \textbf{Porte di output}
	\item \textbf{Processore}: per l'esecuzione degli algoritmi di
		routing.
	\item \textbf{Struttura di commutazione}: trasferimento del 
		pacchetto dalle porte di input alle porte di output.
\end{itemize}
Il router mantiene una copia della tabella di inoltro per velocizzare
tale processo ma nel caso in cui la velocità con cui arrivano i 
datagrammi superi la velocità di inoltro della struttura, questi
vengono messi in attesa in una coda.

\subsubsection{Struttura di commutazione}
L'implementazione della struttura di commutazione può essere di vario
tipo
\begin{itemize}
	\item Attraverso la \textbf{memoria} della macchina.
	\item Attraverso un \textbf{bus} condiviso.
	\item Attraverso una \textbf{matrice di commutazione}.
\end{itemize}
La matrice di commutazione è quella più performante in quanto fa uso
di più bus per il trasferimento dei datagrammi dalle porte di input a
quelle di output.

Una volta che il pacchetto ha attraversato la struttura di commutazione
lo si trasferisce sul collegamento fisico. Anche in questa fase abbiamo
un buffer in cui è possibile accodare i datagrammi, nel caso in cui 
la velocità di uscita dalla struttura di commutazione superi quella 
di trasmissione sul collegamento di uscita.

Vi è anche un meccanismo di \textbf{scheduling} che permette di 
definire politiche per l'ordine di instradamento dei datagrammi.
