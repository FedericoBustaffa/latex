\section{ICMP}
Quando si verificano degli errori come ad esempio perdita di pacchetti
o vi sono dei nodi non raggiungibili, il protocollo IP non ha 
meccanismi integrati. Per renderlo noto all'host mittente entra in 
gioco il protocollo \textbf{ICMP} (Internet Control Message Protocol).

Il protocollo permette di effettuare richieste sullo stato di un 
sistema remoto e può essere usato sia da host che da router per 
scambiarsi messaggi di errore o per segnalare altre situazioni che 
richiedono intervento.

I messaggi ICMP vengono incapsulati all'interno di datagrammi IP ma
viene considerato parte integrante dello strato di rete in quanto chi
implementa IP deve implementare anche ICMP.

\subsection{Casi d'uso}
Uno dei casi tipici in cui entra in gioco ICMP è quello in cui un host
di destinazione o un router devono informare il mittente di qualche
errore che si è verificato all'inoltro di un pacchetto.

I pacchetti ICMP vengono in genere instradati prima dei pacchetti IP
ma ne viene generato solo uno per ogni frammento con offset a 0 per
evitarne la proliferazione.

I messaggi ICMP non sono mai inviati in risposta a pacchetti IP con 
indirizzo mittente che non rappresenta in modo univoco un host.
Inoltre non sono mai inviati come risposta IP a pacchetti con IP
destinazione di tipo broadcast o multicast.

Infine non possono essere inviati in risposta a messaggi ICMP di errore
ma solo a messaggi ICMP di interrogazione.

\subsection{Tipologie di messaggi ICMP}
Non andiamo a vedere nel dettaglio il formato dei messaggi ICMP, ci
basti sapere che si dividono in messaggi di errore e messaggi di 
richiesta/risposta.

Nell'header ICMP i campi \verb|TIPO| e \verb|CODICE|, entrambi di 8
bit, indicano tipo e significato dei messaggi che stiamo inviando.

\subsection{Ping}
Uno dei programmi utilizzabili da un host per verificare il 
funzionamento di un altro host è il programma \textbf{ping}. Esso si
basa su una serie di messaggi di richiesta e risposta \textbf{eco}
dell'ICMP:
\begin{enumerate}
	\item Un host invia una richiesta eco ad un altro host.
	\item Se tale host è disponibile invia un messaggio di risposta
		eco.
\end{enumerate}
Possiamo inviare questa richiesta più volte per calcolare un RTT,
facendo una stima.

\subsection{Traceroute}
Il programma \textbf{traceroute} può essere usato per individuare
il percorso di un datagramma da una sorgente ad una destinazione
tramite l'identificazione dell'indirizzo IP di tutti i router che
vengono visitati lungo il percorso.

Solitamente il programma è impostato con un numero di limite di 30
salti (solitamente sufficienti per raggiungere la destinazione).

\subsubsection{Implementazione}
A seconda delle implementazioni, che possono variare tra i vari sistemi
operativi, traceroute può fare uso del protocollo UDP inviando un 
datagramma ad una porta su cui è improbabile che ci sia un processo in 
ascolto, oppure inviando pacchetti ICMP.

I datagrammi sono configurati per scatenare l'invio di messaggi di 
errore di due tipi:
\begin{itemize}
	\item Tempo scaduto.
	\item Porta destinazione non raggiungibile.
\end{itemize}
In generale traceroute per tracciare il percorso invia diversi messaggi
ognuno con TTL diverso e progressivamente maggiore in modo da ricevere
messaggi di errore ogni volta che si incontra un nodo lungo il percorso
così da poterlo tracciare.
