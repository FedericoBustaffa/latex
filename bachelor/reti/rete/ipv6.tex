\section{IPv6}
Si tratta di una nuova versione di indirizzi IP che implementa nuova
funzionalità e che cerca di risolvere alcuni problemi tra cui la 
carenza di indirizzi IPv4.

Si passa infatti da indirizzi a 32 bit a indirizzi a 128 bit e si cerca
di minimizzare il tempo di elaborazione e inoltro dei pacchetti nei
nodi.

Vi è infatti una lunghezza fissa dell'header, viene eliminato il
campo di checksum e non si supporta più la frammentazione che viene 
invece delegata al nodo sorgente.

Si applicano inoltre delle etichette che permettono di identificare 
dei particolari \emph{flussi}, andando così ad implementare politiche
diverse per pacchetti appartenenti a flussi differenti.

\subsection{Tunneling}
Al momento IPv4 e IPv6 convivono e infatti non è raro imbattersi in
dispositivi che implementano entrambi i protocolli.

Nel momento in cui si incontrino però dei nodi che supportano solo 
traffico di tipo IPv4 si ha un problema in quanto si ha un tipo di 
instradamento ed inoltro differente.

Per ovviare al problema si utilizza un meccanismo di \textbf{tunneling}
che va di fatto a incapsulare il datagramma IPv6 aggiungendo un header
IPv4 finché non si esce dalla sottorete IPv4.

