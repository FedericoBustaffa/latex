\section{Utilità}
Per snellire il codice e renderlo più leggibile ho deciso di incapsulare alcune funzionalità in
oggetti specifici.

\subsection{Utente}
La classe \verb|User| incapsula tutto ciò che identifica lo stato di un utente e tutte le
statistiche relative alle partite giocate.

\subsubsection{Calcolo del punteggio}
Per calcolare il punteggio si tiene di conto del numero e della qualità di ogni vittoria. Le
partite perse per un motivo o per un altro non vengono contate al livello di punteggio.

Ogni utente possiede infatti un vettore di interi di lunghezza 12, dove in prima posizione troviamo
il numero di partite vinte al primo tentativo, in seconda posizione il numero di partite vinte al
secondo tentativo e così via fino alla dodicesima posizione.

Ogni volta che si termina una partita con successo il punteggio viene aggiornato tramite una sorta
di media pesata. Si effettua infatti la somma di tutte le vittorie, ognuna moltiplicata per un peso
che varia in base al numero di tentativi effettuati per raggiungerla.

Una vittoria ottenuta al primo tentativo avrà il peso massimo, ossia 12, mentre una vittoria
ottenuta al dodicesimo tentativo avrà il peso minimo, ossia 1.

\subsection{JsonWrapper}
Vorrei soffermarmi infine sulla classe \verb|JsonWrapper| in quanto mi è risultata molto utile per
semplificare la sintassi (troppo verbosa) della libreria \verb|Jackson| per la gestione dei file
json.

La mia intenzione non è quella di implementare uno strumento per la completa gestione dei file
json, ma un tool dall'interfaccia comoda, che tramite semplici metodi, mi permetta di effettuare
operazioni più complesse.

\subsubsection{Implementazione}
Il \verb|JsonWrapper| contiene due costruttori uno a cui è possibile passare un file e uno senza
argomenti.

Nel caso in cui si passi un file è possibile leggere o scrivere un array di \verb|User| in formato
json. Come possiamo vedere, l'argomento passato al metodo \verb|writeArray| è in realtà la tabella
hash degli utenti, così come lo è il valore di ritorno del metodo \verb|readArray|.

Altri metodi come \verb|getContent| sono stati utili per leggere i file di configurazione e le
risposte HTTP ricevute in fase di traduzione delle parole (entrambi in formato json) e salvarli
sottoforma di stringa.

Una volta convertito il file in stringa è possibile usare i metodi \verb|getString|,
\verb|getInteger| e \verb|getLong| per reperire il valore associato ad uno dei campi dell'oggetto
json. Se il campo cercato non è presente si solleva una \verb|NoSuchElementException|.